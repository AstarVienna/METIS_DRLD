%=======================================================================
\section{Summary of the usage of \texttt{molecfit}}\label{app:mf}
This section is intended to be a summary on how the package \texttt{molecfit} will be used for the \ac{METIS} pipeline. As it will be incorporated in the \ac{IFU} and both \ac{LSS} modes in the same way, we describe here the technical part only once for all pipelines. The overall strategy is described in Section~\ref{ssec:tellcorr}.

%=======================================================================
\subsection{Principles}\label{app:mf_principles}
The software \texttt{molecfit} is provided by \ac{ESO} via the \texttt{telluriccorr} package. It incorporates three individual recipes: \texttt{XXX\_mf\_model}, \texttt{XXX\_mf\_calctrans}, and \texttt{XXX\_mf\_correct} (being \texttt{XXX} the prefix of the respective pipelines, e.g. \texttt{met\_lm\_lss}). The first recipe aims at fitting telluric features visible in the spectra with the help of a radiative transfer code, a line database and an atmospheric profile. This best fit is done on small wavelength regions only, where specific features of atmospheric molecules are visible. Having this best fit at hand, the transmission curve is calculated on the whole wavelength regime of the input spectrum (recipe \texttt{XXX\_mf\_calctrans}). The final telluric correction is performed by the last recipe \texttt{XXX\_mf\_correct}. For more details on the actual algorithms in \texttt{molecfit} we refer the user to the respective User Manual \cite{molecfit}.\\

%=======================================================================
\subsection{Input}\label{app:mf_input}
All three recipes require the usual \texttt{sof} file and a \texttt{.rc} file with parameters as input.
\subsubsection{\texttt{XXX\_mf\_model}}
Following the User Manual of \texttt{molecfit}\cite{molecfit} the first recipe requires a 1d-spectrum as input, either a science observation or a telluric standard star. This means that from both, \ac{IFU}-cubes and 2d-\ac{LSS}-data one-dimensional spectra need to be extracted beforehand. For the time being we assume that the best-fit model done on that 1d-spectrum is sufficient to correct the entire \ac{IFU} cube. If it turns out that one of the fitting parameters shows major variations over the \ac{FoV} (especially the the \ac{LSF}), best-fit models need to be created for each spaxel. For the \ac{LSS} spectra we do not assume large \ac{LSF} variations along the slit. \\
%---------------------------------------------------------------------
\paragraph{\texttt{sof}-files}
The \texttt{sof}-files should have the following input:\\
\begin{table}
%\begin{center}
\centering
\begin{tabular}{r|l|l}
input & \texttt{sof}-tag & type\\
    & \FITS{PRO.CATG} & mandatory/optional\\
\hline
\texttt{1d-input\_spectrum_filename} &  & mandatory\\
\hdashline
\texttt{1d-input\_spectrum_filename} &  & mandatory\\
\end{tabular}
\caption{Expected content of the \texttt{sof}-file for the recipe \texttt{XXX\_mf\_model}\label{tab:sof_mf_model}).
%\end{center}
\end{table}

%---------------------------------------------------------------------
\paragraph{\texttt{sof}-files}
Due to the complexity of \mf a large set of parameters are required. It is beyond the scope of this document w
