\section{Summary of the usage of \texttt{molecfit}}\label{app:mf}
This section is intended to be a summary on how the package \texttt{molecfit} will be used for the \ac{METIS} pipeline. As it will be incorporated in the \ac{IFU} and both \ac{LSS} modes in the same way, we describe here the technical part only once for all pipelines. The overall strategy is described in Section~\ref{ssec:tellcorr}.
%---------------------------------------------------------------------
\subsection{Principles}\label{app:mf_principles}
\texttt{molecfit} is provided by \ac{ESO} via the \texttt{telluriccorr} package. It incorporates three individual recipes: \texttt{XXX_mf_model}, \texttt{XXX_mf_calctrans}, and \texttt{XXX_mf_correct} (being \texttt{XXX} the prefix of the respective pipelines, e.g. \texttt{met_lm_lss}). The first recipe aims at achieving a best-fit of the telluric features visible in the spectra by using a radiative transfer code, a lan atmospheric profile, and 

For more details on the actual algorithms in \texttt{molecfit} we refer the user to the respective User Manual \cite{molecfit}.\\

