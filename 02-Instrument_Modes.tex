\section{Instrument Modes and Configurations}
\label{sec:instrument_modes}

The following table lists the instrument modes for METIS
\cite{METIS-operational_concept}. All of these modes will be
supported by the data reduction pipeline, where horizontal lines
delimit groups of modes that can be reduced with the same set of
recipes.  Post-processing recipes may be available for individual
modes within a group (e.g.\ \ac{ADI} post-processing for HCI modes).

Exposures or sequences of exposures can be taken in pupil-tracking
mode or in field-tracking mode. In the first case, the derotator keeps
the pupil fixed with respect to the detector pixel rows and columns,
while the field rotates. In the second case, the field is kept fixed,
while the pupil rotates. For the pipeline, this has consequences when
it comes to combining several exposures. In pupil-tracking mode,
images have to be rotated in software before they can be stacked to
improve signal to noise in scientific data products.

\begin{table}[ht!]
\small
  \caption[METIS instrument modes]{The five main observing modes of METIS. The acronyms stand for: \textbf{CVC} -- Classical Vortex Coronagraph; \textbf{RAVC} -- Ring-apodized Vortex Coronagraph; \textbf{APP} -- Apodized Phase Plate; N/A -- not applicable.; \textbf{P/T} -- pupil tracking; \textbf{F/T} -- field tracking.}\label{tab:instrument_mode}
%  \renewcommand{\arraystretch}{1.15}
  %\begin{tabular}{\textwidth}{|P{0.12\textwidth}|c|c|P{0.12\textwidth}|c|c|c|}
  \begin{tabular}{|P{0.12\textwidth}|c|c|P{0.12\textwidth}|c|c|c|l|}
    \hline
    \multirow{2}{\linewidth}{\centering\textbf{Observing Mode}}                          & \multicolumn{6}{c|}{\textbf{Instrument Configuration}}                                                                                                                                                                    & \\
                                                                                         & \textbf{Subsystem}                                     & \textbf{Band}         & \parbox[c][4ex]{\hsize}{\centering\textbf{IFS Setting}}                         & \textbf{HCI Mask}         & \textbf{P/T} & \textbf{F/T} & \textbf{Code} \\
    \hline\hline
    %%%%%%% Direct Imaging
    \multirow{2}{\hsize}{\centering\textbf{Direct Imaging}}                             & IMG                                                    & L, M                  & \textcolor{black!35}{N/A}                                                       & \textcolor{black!35}{N/A} & $\bullet$ & $\bullet$  & IMG\_LM \\
    \cline{2-8} & IMG & N & \textcolor{black!35}{N/A}                                                       & \textcolor{black!35}{N/A} & $\bullet$ & $\bullet$  & IMG\_N \\
    \hline\hline
    %%%%%%%% High-contrast imaging
    \multirow{4}{\hsize}{\centering\textbf{High Contrast Imaging}}                       &                                                        &                       & \parbox[c][4ex]{\hsize}{\centering \textcolor{black!35}{N/A}}                   & RAVC/CVC                  & $\bullet$ &            & IMG\_LM\_(RA/C)VC \\
    \cline{4-8}    & IMG                                                    & \multirow{1}{*}{L, M} & \parbox[c][4ex]{\hsize}{\centering \textcolor{black!35}{N/A}}                   & APP                       & $\bullet$ &            & IMG\_LM\_APP \\
    \cline{2-8}  & \multirow{1}{*}{IMG}                                   & N                & \parbox[c][4ex]{\hsize}{\centering \textcolor{black!35}{N/A}}                   & CVC                       & $\bullet$ &            & IMG\_N\_CVC \\
    \hline\hline
    %%%%%%%% Longslit spectroscopy
    \multirow{2}{\hsize}{\centering\textbf{Longslit spectroscopy}}                       & IMG                                                    & L, M                  & \textcolor{black!35}{N/A}                                                       & \textcolor{black!35}{N/A} &           & $\bullet$  & SPEC\_LM\\
    \cline{2-8}   & IMG                                                    & N                     & \textcolor{black!35}{N/A}                                                       & \textcolor{black!35}{N/A} &           & $\bullet$  & SPEC\_N\_LOW \\
    \hline\hline
    %%%%%%%% IFU
    \multirow{3}{\hsize}{\parbox[c]{\hsize}{\centering\textbf{IFU spectroscopy}}}        & IFU                                                    & L, M                  & full IFU field                                                                  & \textcolor{black!35}{N/A} & $\bullet$ & $\bullet$  & IFU\_nominal \\
    \cline{2-8}     & IFU                                                    & L, M                  & \parbox[c][7ex]{\hsize}{\centering extended $\Delta\lambda = 300\,\mathrm{nm}$} & \textcolor{black!35}{N/A} & $\bullet$ & $\bullet$  & IFU\_extended \\
    \hline\hline
    %%%%%% IFU + HCI
    \multirow{4}{\hsize}{\parbox[c]{\hsize}{\centering\textbf{IFU + HCI spectroscopy}}}  & \multirow{2}{*}{IFU}                                   & \multirow{2}{*}{L, M} & \multirow{2}{\hsize}{\centering full IFU field}                                            & APP                       & $\bullet$ &            & IFU\_nominal\_APP \\
    \cline{5-8} &                                                        &                       &                                                                                 & RAVC/CVC                  & $\bullet$ &            & IFU\_nominal\_(RA/C)VC \\
    
    \cline{2-8}  & \multirow{2}{*}{IFU}                                   & \multirow{2}{*}{L, M} & \multirow{2}{\hsize}{\centering extended $\Delta\lambda = 300\,\mathrm{nm}$}    & APP                       & $\bullet$ &            & IFU\_extended\_APP \\
    \cline{5-8}   &                                                        &                       &                                                                                 & RAVC/CVC                  & $\bullet$ &            & IFU\_extended\_(RA/C)VC \\
    
    \hline

  \end{tabular}
  % \end{tabularx}
\end{table}

%%% Local Variables:
%%% TeX-master: "METIS_DRLD"
%%% End:
