\subsection{Long-slit spectroscopy, LM band}
\label{ssec:recipes_lss_lm}

A draft of the reduction cascade is shown in
Fig.~\ref{Fig:LMLssAssomap} together with the data processing table
(Table~\ref{Tab:LssDatProc}). The first part concerns the detector
calibrations, which are independent of the observing mode and already
described in Section~\ref{Sec:detector_calibration}. This affects the
dark correction, linearity and gain determination and bad pixel
detection. The second step is the flatfielding, followed by the wavelength
correction. Subsequently, the main reduction is conducted, which applies
the previously created master calibration files to the science frames,
followed by the flux conversion to physical units. Finally, the telluric 
absorption correction is applied. For the time being we keep the option with the telluric standard star although the modelling approach with molecfit is intended to be the standard method. Note that all steps after the \ac{RSRF} correction is based on the science frames only.

Special emphasis has to be drawn to the effects of the Earth's
atmosphere in several respects:
\begin{itemize}
\item Wavelength calibration: Absorption/emission features are to be
  used for the wavelength calibration. Thus, a good knowledge on (
  identification of these features is crucial for the accuracy of the
  wavelength calibration.
\item Telluric correction: In the MIR regime telluric absorption is
  one of the most dominant effects visible in spectra. Modelling
  approaches like \texttt{molecfit} heavily rely on accurate
  atmospheric input profiles, which represent the actual state and
  composition of the Earth's atmosphere. This especially applies to
  the \ac{PWV} content since this is the most
  dominant and most variable species.
\item Atmospheric dispersion: \ac{METIS} will have \ac{ADC}s compensating the
  effect of atmospheric dispersion. However, for technical reasons
  these ADCs are fixed at several positions. This means that the
  compensation is only partially. This leads to two practical effects:
  (a) wavelength-dependent slit losses and (b) distortions in both,
  the spatial and the spectral direction (see \cite{METIS-ADC_study}
  for more details). For both, the pipeline needs to correct
  for. Since airmass, ambient temperature/pressure, slit orientation,
  and the properties of the \ac{ADC}s are known, the only required
  parameter to be determined for the compensation is the \ac{PWV}.
\end{itemize}
It is therefore crucial to have a radiometer (e.g.\ L-HATPRO)
available at the \ac{ELT}, which provides direct measurements of the \ac{PWV}
along the observation direction.

As the flux standard stars are treated very similar to the science
data, we also included them in this recipe. For the observatory
pipeline at the mountain, this recipe most probably has to be split
into dedicated recipes for processing flux standards and a science
data, since the observing templates directly trigger the corresponding
recipes after their execution.

\clearpage

\subsubsection{LM-band spectroscopic flatfield / RSRF}
\label{sssec:LM_LSS_flat}

The main aim of this recipe is to detect the wavelength dependent
sensitivity of the detectors by means of combining a static \ac{RSRF}
with a dynamic one taken with a grid (cf.~\cite{METIS-calibration_plan})

\begin{recipedef}
Name:		& \REC{metis_LM_lss_rsrf} \\
Purpose:	& determine the scpectroscopic flat field of the LM detector to determine\\
		& a pixel sensitivity and a bad pixel map.\\
Type:		& Calibration\\
Requirements: & METIS-6084, METIS-6698  \\
Observing templates: & \TPL{METIS_spec_cal_rsrf}  \\
Input data: 	& WCU Lamp flat field (\FITS{LM_FLAT_RAW})  \\
                & \EXTCALIB{PERSISTENCE_MAP} \\
                & \STATCALIB{REF_RSRF} \\
                & \PROD{BADPIX_MAP_2RG} \\
                & \PROD{MASTER_DARK_2RG} \\
Parameters: 	& Taken from HDRL\\
Algorithm:  	& Standard flat algorithm (see Fig.\,\ref{Fig:rec_lm_lss_flat}):\\
		& Pixel-by-pixel median / mean filtering of several flatfield frames to compute the sensitivity
        of the detector; normalisation of the result (to e.g. median = 1.)\\
Output data:	& Normalised flat field (\FITS{PRO_CATG=LM_LSS_FLAT}) \\
        & Bad pixel map (\FITS{PRO_CATG=LM_LSS_BPM}) \\
        & \PROD{MEDIAN_LM_LSS_RSRF_IMG} (\FITS{PRO.CATG=LM_LSS_MEDIAN_RSRF}): median filtered LM flatfield (for QC) \\
        & \PROD{MEAN_LM_LSS_RSRF_IMG} (\FITS{PRO.CATG=LM_LSS_MEAN_RSRF}): mean filtered LM flatfield (for QC) \\
        & \PROD{MASTER_LM_LSS_RSRF} (\FITS{PRO.CATG=LM_LSS_MASTER_RSRF}): Master flatfield \\
Expected accuracies: & TBD\\
QC1 parameters: & \QC{QC LM LSS RSRF NOISE} : Mean of the spectroscopic noise level\\
        & \QC{QC LM LSS SENS MEAN}: Mean of the spectroscopic sensitivity (TBDef)\\
        & \QC{QC LM LSS SENS MEDIAN}: Median of the spectroscopic sensitivity (TBDef)\\
        & \QC{QC LM LSS SENS SIGMA}: Sigma of the spectroscopic sensitivity (TBDef)\\
        & \QC{QC LM LSS FLXMIN / FLXMAX}: Min/max of flux level\\
        & \QC{QC LM LSS NORM LEVEL}: Level for the normalisation (TBDef)\\
        & (more TBD)\\
\end{recipedef}
\begin{figure}[ht]
  \centering
  \includegraphics[width=0.5\textheight]{figures/metis_lm_lss_flat.pdf}
  \caption[Recipe: \REC{metis_LM_lss_rsrf}]{\REC{metis_LM_lss_rsrf} --
    Creation of the LM LSS master flatfield.}
  \label{Fig:rec_lm_lss_flat}
\end{figure}
\clearpage

\subsubsection{LM-band wavelength correction}
\label{sssec:LM_LSS_wave}

\begin{figure}[ht]
  \centering
  \includegraphics[width=0.5\textheight]{figures/metis_lm_lss_wave_0.62.pdf}
  \caption[Recipe: \REC{metis_LM_lss_wave}]{\REC{metis_LM_lss_wave} --
    Creation of the LM LSS master wavelength correction.}
  \label{Fig:rec_lm_lss_wave}
\end{figure}
\clearpage

\subsubsection{LM-band main science reduction recipe}
\label{sssec:LM_LSS_sci}

This recipe comprises all required calibration steps. Since the flux
standard star calibrations are nearly identical to the science data
calibrations we process these stars in the \REC{metis_LM_lss_sci}
recipe (cf.\ Fig.~\ref{Fig:rec_lm_lss_sci}).

Spectrophotometric standard stars calibration: As first step the
master calibration files derived previously are applied followed by
the wavelength calibration (cf.\
Section~\ref{Sec:critalg_wavecal}). Then the recipe extracts the
standard star spectrum object, removes sky lines, collapses the 2D to
1D spectra and applies a telluric correction in an automated way to
the standard star spectrum (in contrast to the science observations,
which are telluric corrected in a dedicated recipe to achieve the best
correction). The response curve is obtained by comparing the extracted
spectrum with a model and/or another reference spectrum of the
standard star.

Science calibration: The calibration of the science data is done in
the same way as for the spectrophotometric stars, except that the
response curve is additionally applied. The telluric correction on the
science data is done in a dedicated recipe afterwards to achieve best
quality for the correction.

For the observatory pipeline, this recipe most probably has to be
split into dedicated recipes for processing flux standards and science
data, since the observing templates directly trigger the corresponding
recipes after their execution.


\begin{recipedef}
Name:		& \REC{metis_LM_lss_sci} \\
Purpose:	& main science reduction recipe: \\
            & applies master calibration files \\
            & applies wavelength calibration + distortion solution\\
            & - determines sky/object pixel\\
            & - subtracts sky emission\\
            & - performs coaddition of wavelength calibrated spectra
            taken during one OB\\
            & applies the flux calibration\\
Type:		& Science reduction\\
Requirements: & METIS-6084, METIS-6074 \\
Observing templates: & \TPL{METIS_spec_lm_acq}, \\
                & \TPL{METIS_spec_lm_obs_AutoNodOnSlit}, \\
                & \TPL{METIS_spec_lm_obs_GenericOffset} \\
                & \TPL{METIS_spec_lm_cal_lampwave}\\
                & \TPL{METIS_spec_lm_cal_standard}\\
                & \TPL{METIS_spec_lm_cal_slit_adc}\\
Input data: 	& raw SCIENCE data (\FITS{LM_SCI_RAW})\\
                & raw spectrophotometric STANDARD star data (\FITS{LM_FLUX_RAW})\\
                & WCU grid for first guess distortion correction (TBD) \\
                & WCU lamp spectrum for first guess wavelength solution (TBD)\\
                & \EXTCALIB{PERSISTENCE_MAP} \\
                & \PROD{MASTER_LM_LSS_RSRF} \\
                & \PROD{MASTER_DARK_2RG} \\
                & \PROD{BADPIX_MAP_2RG} \\
                & \EXTCALIB{AO_PSF_MODEL} \\
                & \STATCALIB{ATM_LINE_CAT} \\
                & \STATCALIB{REF_AIRG_CAT} \\
Parameters: 	& (TBD)\\
Algorithm:      & Application of master calibration files\\
                & Determination and application of the distortion correction\\
                & Determination and application of the wavelength solution\\
                & Identifying sky/object pixels by applying e.g. Horne 1986, Ritter 2014, or by user \\
                & Removing sky lines: Creation and Subtraction of 2D sky\\
                & Coaddition of individual object spectra of one OB\\
                & Collapsing 2D to 1D spectrum, (see Fig.\,\ref{Fig:rec_lm_lss_sci})\\
                & Determination and application of response curve\\
Output data:	& \PROD{LM_STD_OBJ_MAP}: Pixel map of object pixels\\
            	& \PROD{LM_STD_SKY_MAP}: Pixel map of sky pixels\\
                & \PROD{LM_STD_2D}: coadded, wavelength calibrated 2D spectrum\\
              	& \PROD{LM_STD_1D}: coadded, wavelength calibrated, collapsed 1D spectrum\\
                & \PROD{INSTR_RESPONSE}: response function (TBD)\\
                & \PROD{LM_SCI_SLIT} (\FITS{PRO_CATG=LM_LSS_2D_WAVE}): background corrected 2d spectra \\
                & \PROD{LM_SCI_OBJ_MAP}: Pixel map of object pixels\\
            	& \PROD{LM_SCI_SKY_MAP}: Pixel map of sky pixels\\
                & \PROD{LM_SCI_2D}: coadded, wavelength calibrated 2D spectrum\\
                & (\FITS{PRO_CATG}: \FITS{LM_LSS_2d_coadd_wavecal}) \\
                & (\FITS{PRO_CATG}: \FITS{LM_LSS_1d_coadd_wavecal}) \\
              	& \PROD{LM_SCI_FLUX_2D}: coadded, wavelength and flux calibrated 2D spectrum\\
                & (\FITS{PRO_CATG}: \FITS{LM_LSS_2d_coadd_wavecal_flux}) \\
              	& \PROD{LM_SCI_FLUX_1D}: coadded, wavelength and flux calibrated 1D spectrum\\
                & (\FITS{PRO_CATG}: \FITS{LM_LSS_1d_coadd_wavecal_flux}) \\
                & \PROD{LM_WAVE_SOL} (\FITS{PRO.CATG=LM_WAVE_SOL}): Wavelength solution\\
                & \PROD{LM_DIST_SOL} (\FITS{PRO.CATG=LM_DIST_SOL}): Distortion solution\\
Expected accuracies: & (TBD)\\
QC1 parameters: & \QC{QC LM LSS SCI BACKGD MEAN}: Mean value of background\\
                & \QC{QC LM LSS SCI BACKGD MEDIAN}: Median value of background\\
                & \QC{QC LM LSS SCI BACKGD SIGMA}: Sigma value of background\\
                & \QC{QC LM LSS SCI SNR}: Signal-to-noise ration of flux standard star spectrum\\
                & \QC{QC LM LSS SCI SNRNOISE}: Noise level of flux standard star spectrum\\
                & \QC{QC LM LSS SCI FWHM}: FWHM of flux standard spectrum\\
%                & \QC{LM LSS SCI}: (TBdef) \\
                & \QC{QC LM LSS WAVECAL DEVMEAN}: Mean deviation from the
                   wavelength reference frame (TBDef)\\
                & \QC{QC LM LSS WAVECAL FWHM}: Measured FWHM of lines\\
                & \QC{QC LM LSS WAVECAL NIDENT}: Number of identified lines\\
                & \QC{QC LM LSS WAVECAL NMATCH}: Number of lines matched between
                    catalaogue and spectrum\\
                & \QC{QC LM LSS WAVECAL POLYDEG}: Degree of the polynomial\\
                & \QC{QC LM LSS WAVECAL POLYCOEFF\<n\>}: $n$-th coefficient of the polynomial\\
                & \QC{QC LM LSS FLUXCAL SNR}: Signal-to-noise ration of flux standard star spectrum\\
                & \QC{QC LM LSS FLUXCAL SNRNOISE}: Noise level of flux standard star spectrum\\
                & \QC{QC LM LSS FLUXCAL FWHM}: FWHM of flux standard spectrum\\
                & \QC{QC LM LSS FLUXCAL PSFLOSS}: percentage of AO induced slit losses (TBdef)\\
\end{recipedef}
\subsubsection{LM-band flux calibration recipe}\label{subsubsection:LM_LSS_flux}

\begin{figure}[ht]
  \centering
  \includegraphics[height=0.95\textheight]{figures/metis_lm_lss_flux_v0.62.pdf}
  \caption[Recipe: \REC{metis_LM_lss_flux}]{\REC{metis_LM_lss_flux} --
    Creation of the LM LSS master wavelength correction.}
  \label{Fig:rec_lm_lss_wave}
\end{figure}
\clearpage
