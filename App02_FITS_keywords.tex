% This section contains a description of the format of all FITS keywords
\section{FITS keywords}
\label{app:fits_keywords}

\Large\textbf{\textcolor{red}{CAVEAT: Section~\ref{sec:essentialimagingheaders} TAKEN 1:1 FROM MICADO - TO BE ADAPTED!!!11111}}\normalsize


%\todo[inline]{DONE-HB-210317: limit documentation to FITS keywords needed in processing of imaging and spectroscopic data. See \url{https://gitlab.astro-wise.org/micado/datareductionlibrarydesigndrs/-/issues/50}}

The subsection~\ref{sec:allheaderkeywords} provides a listing of header keywords used in the FITS files of MICADO data items, both raw and processed, for which the values are used as parameters in the processing.
%The subsection~\ref{sec:headerkeywordsperdataitemimagingmode} lists per data item the header keywords that shall be in the header of the FITS file for data which is based on observations taken in the imaging observing modes.

\subsection{List of FITS header keywords used in imaging and spectroscopy recipes}
\label{sec:essentialimagingheaders}


All FITS files with raw exposures will have the same set of header keywords.
The following header keywords are mandatory for the data processing to function:
\begin{itemize}
    \setlength{\itemsep}{0pt}
        \item \hyperref[fits:cdeltn]{\FITS{CDELTn}}
    \item \hyperref[fits:crpixn]{\FITS{CRPIXn}}
    \item \hyperref[fits:crvaln]{\FITS{CRVALn}}
    \item \hyperref[fits:ctypen]{\FITS{CTYPEn}}
    \item \hyperref[fits:cunitn]{\FITS{CUNITn}}
    \item \hyperref[fits:det.dit]{\FITS{DET.DIT}}
    \item \hyperref[fits:det.readout]{\FITS{DET.READOUT}}
    \item \hyperref[fits:dpr.catg]{\FITS{DPR.CATG}}
    \item \hyperref[fits:dpr.tech]{\FITS{DPR.TECH}}
    \item \hyperref[fits:dpr.type]{\FITS{DPR.TYPE}}
    \item \hyperref[fits:exptime]{\FITS{EXPTIME}}
    \item \hyperref[fits:ins.filt.name]{\FITS{INS.FILTi.NAME}}
    \item \hyperref[fits:instrume]{\FITS{INSTRUME}}
    \item \hyperref[fits:mjd-obs]{\FITS{MJD-OBS}}
    \item \hyperref[fits:obs.tplno]{\FITS{OBS.TPLNO}}
    \item \hyperref[fits:ocs.pxscale]{\FITS{OCS.PXSCALE}}
    \item \hyperref[fits:tpl.expno]{\FITS{TPL.EXPNO}}
    \item \hyperref[fits:tpl.start]{\FITS{TPL.START}}

\end{itemize}

%\subsection{FITS header keywords used in spectroscopic recipes}
%\label{sec:FitsKeywordsSpectroscopy}



\subsection{Specification of FITS header keywords used in imaging and spectroscopy recipes}{}\label{sec:allheaderkeywords}

\subsubsection{DET\textit{n}.DIT}\label{fits:det.dit}
\begin{recipedef}
Name & DET\textit{n}.DIT \tabularnewline
Class & header \tabularnewline
Context & Template \tabularnewline
Type & double \tabularnewline
Value & \%.3f \tabularnewline
Unit & s \tabularnewline
Comment & \textit{n}=1: LM IMG/LSS detector (Hawaii2RG)  \tabularnewline
        & \textit{n}=2: N IMG/LSS detector (Geosnap)  \tabularnewline
        & \textit{n}=3: IFU detectors \tabularnewline
Default & 1.0 \tabularnewline
Range & 0.0..3600.0 \tabularnewline
Description & Detector integration time (average when NDIT $>$ 1) \tabularnewline
\end{recipedef}


\subsubsection{DET\textit{n}.NDIT}\label{fits:det.ndit}
\begin{recipedef}
Name & DET\textit{n} NDIT \tabularnewline
Class & header \tabularnewline
Context & Template \tabularnewline
Type & integer \tabularnewline
Value & \%i \tabularnewline
Unit & None \tabularnewline
Comment & \textit{n}=1: LM IMG/LSS detector (Hawaii2RG)  \tabularnewline
        & \textit{n}=2: N IMG/LSS detector (Geosnap)  \tabularnewline
        & \textit{n}=3: IFU detectors \tabularnewline
Default & 1 \tabularnewline
Range & 1..10 \tabularnewline
Description & Number of detector integrations \tabularnewline
\end{recipedef}

\subsubsection{DPR.CATG}\label{fits:dpr.catg}
\begin{recipedef}
Name & DPR CATG \tabularnewline
Class & header|template \tabularnewline
Context & Template \tabularnewline
Type & string \tabularnewline
Value & \%.50s \tabularnewline
Unit & None \tabularnewline
Comment & Set by \ac{ICS}\tabularnewline
Default & None \tabularnewline
Range & CALIB SCIENCE TEST \tabularnewline
Description & Raw data  product category \tabularnewline
\end{recipedef}


\subsubsection{DPR.TECH}\label{fits:dpr.tech}
\begin{recipedef}
Name & DPR TECH \tabularnewline
Class & header|template \tabularnewline
Context & Template \tabularnewline
Type & string \tabularnewline
Value & \%.50s \tabularnewline
Unit & None \tabularnewline
Comment & Set by \ac{ICS} \tabularnewline
Default & None \tabularnewline
Range & Depending on the observing mode \tabularnewline
Description & Raw data product technique \tabularnewline
\end{recipedef}


\subsubsection{DPR.TYPE}\label{fits:dpr.type}
\begin{recipedef}
Name & DPR TYPE \tabularnewline
Class & header|template \tabularnewline
Context & Template \tabularnewline
Type & string \tabularnewline
Value & \%.50s \tabularnewline
Unit & None \tabularnewline
Comment & Set by \ac{ICS} \tabularnewline
Default & None \tabularnewline
Range & Depending on the observing mode \tabularnewline
Description & Raw data  product type \tabularnewline
\end{recipedef}


\subsubsection{EXPTIME}\label{fits:exptime}
\begin{recipedef}
Name & EXPTIME \tabularnewline
Class & header \tabularnewline
Context & FITS \tabularnewline
Type & double \tabularnewline
Value & \%.3f \tabularnewline
Unit & s \tabularnewline
Comment & Integration time \tabularnewline
Default & None \tabularnewline
Range & None \tabularnewline
Description & The integration time for a single observation (in the infrared this corresponds to DIT. Note that this does not represent the photon statistics). \tabularnewline
\end{recipedef}

\subsubsection{INS.OPTI3.NAME}\label{fits:ins.opti3.name}
\begin{recipedef}
Name & INS OPTI3 NAME \tabularnewline
Class & header \tabularnewline
Context & INS \tabularnewline
Type & string \tabularnewline
Value & \%.50s \tabularnewline
Unit & None \tabularnewline
Comment & Name of the LSS slit \tabularnewline
Default & None \tabularnewline
Range & A\_19 B\_29 C\_38 D\_57 E\_114 \tabularnewline
Description & LSS slit name \tabularnewline
\end{recipedef}

\subsubsection{INS.OPTI9.NAME}\label{fits:ins.opti9.name}
\begin{recipedef}
Name & INS OPTI9 NAME \tabularnewline
Class & header \tabularnewline
Context & INS \tabularnewline
Type & string \tabularnewline
Value & \%.50s \tabularnewline
Unit & None \tabularnewline
Comment & LM-LSS Mask/Grism name \tabularnewline
Default & None \tabularnewline
Range & open GRISM-L GRISM-M  \tabularnewline
Description & LM-LSS Mask/Grism name \tabularnewline
\end{recipedef}

\subsubsection{INS.OPTI10.NAME}\label{fits:ins.opti10.name}
\begin{recipedef}
Name & INS OPTI10 NAME \tabularnewline
Class & header \tabularnewline
Context & INS \tabularnewline
Type & string \tabularnewline
Value & \%.50s \tabularnewline
Unit & None \tabularnewline
Comment & LM-LSS Filter name \tabularnewline
Default & None \tabularnewline
Range & None \tabularnewline
Description & LM-LSS Filter name \tabularnewline
\end{recipedef}

\subsubsection{INS.OPTI11.NAME}\label{fits:ins.opti11.name}
\begin{recipedef}
Name & INS OPTI11 NAME \tabularnewline
Class & header \tabularnewline
Context & INS \tabularnewline
Type & string \tabularnewline
Value & \%.50s \tabularnewline
Unit & None \tabularnewline
Comment & Name of the neutral density filter \tabularnewline
Default & None \tabularnewline
Range & OPEN ND1 ND2 ND3 ND4 ND5 \tabularnewline
Description & ND Filter name \tabularnewline
\end{recipedef}

\subsubsection{INS.OPTI12.NAME}\label{fits:ins.opti12.name}
\begin{recipedef}
Name & INS OPTI12 NAME \tabularnewline
Class & header \tabularnewline
Context & INS \tabularnewline
Type & string \tabularnewline
Value & \%.50s \tabularnewline
Unit & None \tabularnewline
Comment & N-LSS Mask/Grism name \tabularnewline
Default & None \tabularnewline
Range & open GRISM-N  \tabularnewline
Description & N-LSS Mask/Grism name \tabularnewline
\end{recipedef}

\subsubsection{INS.OPTI13.NAME}\label{fits:ins.opti13.name}
\begin{recipedef}
Name & INS OPTI13 NAME \tabularnewline
Class & header \tabularnewline
Context & INS \tabularnewline
Type & string \tabularnewline
Value & \%.50s \tabularnewline
Unit & None \tabularnewline
Comment & N-LSS Filter name \tabularnewline
Default & None \tabularnewline
Range & full\_N \tabularnewline
Description & N-LSS Filter name \tabularnewline
\end{recipedef}

\subsubsection{INS.OPTI14.NAME}\label{fits:ins.opti14.name}
\begin{recipedef}
Name & INS OPTI14 NAME \tabularnewline
Class & header \tabularnewline
Context & INS \tabularnewline
Type & string \tabularnewline
Value & \%.50s \tabularnewline
Unit & None \tabularnewline
Comment & ND Filter name \tabularnewline
Default & None \tabularnewline
Range & OPEN ND1 ND2 ND3 ND4 \tabularnewline
Description & ND Filter name \tabularnewline
\end{recipedef}


\subsubsection{INS.OPTI20.NAME}\label{fits:ins.opti20.name}
\begin{recipedef}
Name & INS OPTI20 NAME \tabularnewline
Class & header \tabularnewline
Context & INS \tabularnewline
Type & string \tabularnewline
Value & \%.50s \tabularnewline
Unit & None \tabularnewline
Comment & Pinhole mask inserted? \tabularnewline
Default & None \tabularnewline
Range & LM-pinhole \tabularnewline
Description & WCU FP2.1 mask wheel \tabularnewline
\end{recipedef}

\subsubsection{PRO.CATG}\label{fits:pro.catg}
\begin{recipedef}
Name & PRO CATG \tabularnewline
Class & header \tabularnewline
Context & Pipeline \tabularnewline
Type & string \tabularnewline
Value & \%.50s \tabularnewline
Unit & None \tabularnewline
Comment & Set by recipes \tabularnewline
Default & None \tabularnewline
Range & None \tabularnewline
Description & Processed data product category \tabularnewline
\end{recipedef}

\Large\textbf{\textcolor{red}{CAVEAT: Part below TAKEN 1:1 FROM MICADO - USED AS TEMPLATES TO COPY TO UPPER PART AFTER ADAPTION TO METIS!}}\normalsize

\subsubsection{DET.READOUT}\label{fits:det.readout}
\begin{recipedef}
Name & DET READOUT \tabularnewline
Class & header \tabularnewline
Context & INS \tabularnewline
Type & string \tabularnewline
Value & \%.50s \tabularnewline
Unit & None \tabularnewline
Comment & Readout mode of the detector \tabularnewline
Default & CDS \tabularnewline
Range & CDS TLI RRR \tabularnewline
Description & Readout mode of the detector \tabularnewline
\end{recipedef}


\subsubsection{CDELTn}\label{fits:cdeltn}
\begin{recipedef}
Name & CDELTn \tabularnewline
Class & header \tabularnewline
Context & FITS \tabularnewline
Type & double \tabularnewline
Value & \%.8f \tabularnewline
Unit &  \tabularnewline
Comment & Increment in <axis direction> \tabularnewline
Default & None \tabularnewline
Range & None \tabularnewline
Description & Increment of coordinate specified by CTYPEn for each pixel step at CRPIXn. Possible values for <axis direction> are: rows (1), columns (2), frame(3) For RA and DEC the unit is degree. In this case, the comment field includes the value expressed in seconds of arc. In the proposed WCS system it should be replaced by CDn\_m \tabularnewline
\end{recipedef}


\subsubsection{CDn\_ms}\label{fits:cdnms}
\begin{recipedef}
Name & CDn\_ms \tabularnewline
Class & header \tabularnewline
Context & FITS \tabularnewline
Type & double \tabularnewline
Value & \%f \tabularnewline
Unit &  \tabularnewline
Comment & Coordinate translation matrix element \tabularnewline
Default & None \tabularnewline
Range & None \tabularnewline
Description & Gives the translation from array axis n to coordinate axis m. For images the comment should read SS.ss arcsec per pixel \tabularnewline
\end{recipedef}


\subsubsection{CRPIXn}\label{fits:crpixn}
\begin{recipedef}
Name & CRPIXn \tabularnewline
Class & header \tabularnewline
Context & FITS \tabularnewline
Type & double \tabularnewline
Value & \%.1f \tabularnewline
Unit &  \tabularnewline
Comment & Ref pixel in <axis direction> \tabularnewline
Default & None \tabularnewline
Range & None \tabularnewline
Description & Pixel position of the reference point in axis n. Possible values for <axis direction> are: rows (1), columns (2), frame (3) By convention the center of the pixel is pix.0, pix.5 gives the right edge of the pixel. Reference pixel is also used to identify the pointing centre (with respect to the WCS transformation, i.e. the optical axis). \tabularnewline
\end{recipedef}


\subsubsection{CRVALn}\label{fits:crvaln}
\begin{recipedef}
Name & CRVALn \tabularnewline
Class & header \tabularnewline
Context & FITS \tabularnewline
Type & double \tabularnewline
Value & \%.5f \tabularnewline
Unit &  \tabularnewline
Comment & Coordinate at reference pixel in <axis direction> \tabularnewline
Default & None \tabularnewline
Range & None \tabularnewline
Description & Coordinate value as specified by CTYPEn at reference pixel CRPIXn. Possible values for <axis direction> are: rows (1), columns (2), frame (3) If world coordinates are used (i.e. CTYPEn is either RA---TAN and DEC--TAN), the comment field includes the value expressed in hours, minutes and seconds (RA) or degrees, minutes, and seconds (DEC). The unit has to be degrees, if RA and DEC are used as world coordinates. \tabularnewline
\end{recipedef}


\subsubsection{CTYPEn}\label{fits:ctypen}
\begin{recipedef}
Name & CTYPEn \tabularnewline
Class & header \tabularnewline
Context & FITS \tabularnewline
Type & string \tabularnewline
Value & \%s \tabularnewline
Unit &  \tabularnewline
Comment & Coordinate system of <axis direction> \tabularnewline
Default & None \tabularnewline
Range & None \tabularnewline
Description & Name of the coordinate represented by axis n. Possible values for <axis direction> are: rows (1), columns (2), frame (3) Examples for values are "PIXEL", "RA---TAN", "DEC--TAN" \tabularnewline
\end{recipedef}


\subsubsection{CUNITn}\label{fits:cunitn}
\begin{recipedef}
Name & CUNITn \tabularnewline
Class & header \tabularnewline
Context & FITS \tabularnewline
Type & string \tabularnewline
Value & \%s \tabularnewline
Unit &  \tabularnewline
Comment & Unit of coordinate translation \tabularnewline
Default & None \tabularnewline
Range & None \tabularnewline
Description & Unit of the coordinate in n axis n \tabularnewline
\end{recipedef}




\subsubsection{EXTINCT}\label{fits:extinct}
\begin{recipedef}
Name & EXTINCT \tabularnewline
Class & header \tabularnewline
Context & PRO \tabularnewline
Type & double \tabularnewline
Value & \%.3f \tabularnewline
Unit & mag \tabularnewline
Comment & Extinction of the observation \tabularnewline
Default & 0.0 \tabularnewline
Range & None \tabularnewline
Description & Extinction of the observation \tabularnewline
\end{recipedef}


\subsubsection{GAIN}\label{fits:gain}
\begin{recipedef}
Name & GAIN \tabularnewline
Class & header \tabularnewline
Context & Detector \tabularnewline
Type & double \tabularnewline
Value & \%.3f \tabularnewline
Unit & e/adu \tabularnewline
Comment & Gain of the detector \tabularnewline
Default & 1.0 \tabularnewline
Range & None \tabularnewline
Description & Gain of the detector \tabularnewline
\end{recipedef}


\subsubsection{ICCOEFi}\label{fits:iccoefi}\label{fits:iccoef0}\label{fits:iccoef1}\label{fits:iccoef2}\label{fits:iccoef3}\label{fits:iccoef4}\label{fits:iccoef5}\label{fits:iccoef6}\label{fits:iccoef7}\label{fits:iccoef8}\label{fits:iccoef9}\label{fits:iccoef00}\label{fits:iccoef01}\label{fits:iccoef02}\label{fits:iccoef03}\label{fits:iccoef04}\label{fits:iccoef05}\label{fits:iccoef06}\label{fits:iccoef07}\label{fits:iccoef08}\label{fits:iccoef09}\label{fits:iccoef10}\label{fits:iccoef11}\label{fits:iccoef12}\label{fits:iccoef13}\label{fits:iccoef14}\label{fits:iccoef15}\label{fits:iccoef16}\label{fits:iccoef17}\label{fits:iccoef18}\label{fits:iccoef19}\label{fits:iccoef20}\label{fits:iccoef21}\label{fits:iccoef22}\label{fits:iccoef23}\label{fits:iccoef24}\label{fits:iccoef25}\label{fits:iccoef26}\label{fits:iccoef27}\label{fits:iccoef28}\label{fits:iccoef29}\label{fits:iccoef30}\label{fits:iccoef31}\label{fits:iccoef32}\label{fits:iccoef33}\label{fits:iccoef34}\label{fits:iccoef35}\label{fits:iccoef36}\label{fits:iccoef37}\label{fits:iccoef38}\label{fits:iccoef39}\label{fits:iccoef40}\label{fits:iccoef41}\label{fits:iccoef42}\label{fits:iccoef43}\label{fits:iccoef44}\label{fits:iccoef45}\label{fits:iccoef46}\label{fits:iccoef47}\label{fits:iccoef48}\label{fits:iccoef49}\label{fits:iccoef50}\label{fits:iccoef51}\label{fits:iccoef52}\label{fits:iccoef53}\label{fits:iccoef54}\label{fits:iccoef55}\label{fits:iccoef56}\label{fits:iccoef57}\label{fits:iccoef58}\label{fits:iccoef59}\label{fits:iccoef60}\label{fits:iccoef61}\label{fits:iccoef62}\label{fits:iccoef63}\label{fits:iccoef64}\label{fits:iccoef65}\label{fits:iccoef66}\label{fits:iccoef67}\label{fits:iccoef68}\label{fits:iccoef69}\label{fits:iccoef70}\label{fits:iccoef71}\label{fits:iccoef72}\label{fits:iccoef73}\label{fits:iccoef74}\label{fits:iccoef75}\label{fits:iccoef76}\label{fits:iccoef77}\label{fits:iccoef78}\label{fits:iccoef79}\label{fits:iccoef80}\label{fits:iccoef81}\label{fits:iccoef82}\label{fits:iccoef83}\label{fits:iccoef84}\label{fits:iccoef85}\label{fits:iccoef86}\label{fits:iccoef87}\label{fits:iccoef88}\label{fits:iccoef89}\label{fits:iccoef90}\label{fits:iccoef91}\label{fits:iccoef92}\label{fits:iccoef93}\label{fits:iccoef94}\label{fits:iccoef95}\label{fits:iccoef96}\label{fits:iccoef97}\label{fits:iccoef98}\label{fits:iccoef99}\label{fits:iccoef}
\begin{recipedef}
Name & ICCOEFi \tabularnewline
Class & header \tabularnewline
Context & PRO \tabularnewline
Type & double \tabularnewline
Value & \%.3f \tabularnewline
Unit & None \tabularnewline
Comment & Illumination Correction Coefficient \tabularnewline
Default & 0.0 \tabularnewline
Range & None \tabularnewline
Description & Illumination Correction Coefficient \tabularnewline
\end{recipedef}


\subsubsection{INS.DROT}\label{fits:ins.drot}
\begin{recipedef}
Name & INS DROT \tabularnewline
Class & header \tabularnewline
Context & INS \tabularnewline
Type & double \tabularnewline
Value & \%.3f \tabularnewline
Unit & degree \tabularnewline
Comment & Derotator angle \tabularnewline
Default & 0.0 \tabularnewline
Range & None \tabularnewline
Description & Derotator angle \tabularnewline
\end{recipedef}


\subsubsection{INS.FILT.NAME}\label{fits:ins.filt.name}
\begin{recipedef}
Name & INS FILT NAME \tabularnewline
Class & header \tabularnewline
Context & INS \tabularnewline
Type & string \tabularnewline
Value & \%.50s \tabularnewline
Unit & None \tabularnewline
Comment & Filter name \tabularnewline
Default & None \tabularnewline
Range & None \tabularnewline
Description & Filter name \tabularnewline
\end{recipedef}


\subsubsection{INS.FILTi.NAME}\label{fits:ins.filti.name}
\begin{recipedef}
Name & INS FILTi NAME \tabularnewline
Class & header \tabularnewline
Context & INS \tabularnewline
Type & string \tabularnewline
Value & \%.50s \tabularnewline
Unit & None \tabularnewline
Comment & Filter name \tabularnewline
Default & None \tabularnewline
Range & None \tabularnewline
Description & Filter name \tabularnewline
\end{recipedef}


\subsubsection{INS.MODE}\label{fits:ins.mode}
\begin{recipedef}
Name & INS MODE \tabularnewline
Class & header \tabularnewline
Context & INS \tabularnewline
Type & string \tabularnewline
Value & \%.50s \tabularnewline
Unit & None \tabularnewline
Comment & Instrument Mode \tabularnewline
Default & NODEFAULT \tabularnewline
Range & IMAGING SPEC PUP \tabularnewline
Description & Instrument Mode \tabularnewline
\end{recipedef}





\subsubsection{INS.READMODE}\label{fits:ins.readmode}
\begin{recipedef}
Name & INS READMODE \tabularnewline
Class & header \tabularnewline
Context & INS \tabularnewline
Type & string \tabularnewline
Value & \%.50s \tabularnewline
Unit & None \tabularnewline
Comment & Readout mode of the detector \tabularnewline
Default & CDS \tabularnewline
Range & CDS TLI RRR \tabularnewline
Description & Readout mode of the detector \tabularnewline
\end{recipedef}


\subsubsection{INSTRUME}\label{fits:instrume}
\begin{recipedef}
Name & INSTRUME \tabularnewline
Class & header \tabularnewline
Context & FITS \tabularnewline
Type & string \tabularnewline
Value & \%s \tabularnewline
Unit &  \tabularnewline
Comment & Instrument used \tabularnewline
Default & None \tabularnewline
Range & None \tabularnewline
Description & ESO acronym for the instrument used. \tabularnewline
\end{recipedef}


\subsubsection{MJD-OBS}\label{fits:mjd-obs}
\begin{recipedef}
Name & MJD-OBS \tabularnewline
Class & header \tabularnewline
Context & FITS \tabularnewline
Type & double \tabularnewline
Value & \%.8f \tabularnewline
Unit &  \tabularnewline
Comment & Obs start \tabularnewline
Default & None \tabularnewline
Range & None \tabularnewline
Description & Modified Julian Day of the start of the exposure. The MJD is related to the Julian Day (JD) via the formula: MJD = JD-2400000.5 The comment includes a civil representation of the date and time. 8 decimals are required for a precision of one millisecond, 5 decimals for a precision of one second. \tabularnewline
\end{recipedef}


\subsubsection{OBS.DEC}\label{fits:obs.dec}
\begin{recipedef}
Name & OBS DEC \tabularnewline
Class & header \tabularnewline
Context & Template \tabularnewline
Type & double \tabularnewline
Value & \%.3f \tabularnewline
Unit & deg \tabularnewline
Comment & Declination \tabularnewline
Default & 0.0 \tabularnewline
Range & -90.0..90.0 \tabularnewline
Description & Declination \tabularnewline
\end{recipedef}


\subsubsection{OBS.RA}\label{fits:obs.ra}
\begin{recipedef}
Name & OBS RA \tabularnewline
Class & header \tabularnewline
Context & Template \tabularnewline
Type & double \tabularnewline
Value & \%.3f \tabularnewline
Unit & deg \tabularnewline
Comment & Right Ascension \tabularnewline
Default & 0.0 \tabularnewline
Range & 0.0..360.0 \tabularnewline
Description & Right Ascension \tabularnewline
\end{recipedef}


\subsubsection{OBS.TPLNO}\label{fits:obs.tplno}
\begin{recipedef}
Name & OBS TPLNO \tabularnewline
Class & header \tabularnewline
Context & Template \tabularnewline
Type & integer \tabularnewline
Value & \%i \tabularnewline
Unit & None \tabularnewline
Comment & Number of the template of this exposure in an ObservingBlock \tabularnewline
Default & 0 \tabularnewline
Range & 0..1000 \tabularnewline
Description & Number of the template of this exposure in an ObservingBlock \tabularnewline
\end{recipedef}


\subsubsection{OCS.PXSCALE}\label{fits:ocs.pxscale}
\begin{recipedef}
Name & OCS PXSCALE \tabularnewline
Class & header \tabularnewline
Context & INS \tabularnewline
Type & double \tabularnewline
Value & \%.3f \tabularnewline
Unit & arcsec/pix \tabularnewline
Comment & Pixel scale \tabularnewline
Default & 0.004 \tabularnewline
Range & None \tabularnewline
Description & Pixel scale \tabularnewline
\end{recipedef}


\subsubsection{PVn\_ks}\label{fits:pvnks}\label{fits:pv1i}\label{fits:pv2i}\label{fits:pv1}\label{fits:pv2}\label{fits:pv11}\label{fits:pv22}
\begin{recipedef}
Name & PVn\_ks \tabularnewline
Class & header \tabularnewline
Context & FITS \tabularnewline
Type & double \tabularnewline
Value & \%f \tabularnewline
Unit &  \tabularnewline
Comment & Coordinate projection parameter \tabularnewline
Default & None \tabularnewline
Range & None \tabularnewline
Description & Describes the parameter k for the axis n. Required for certain coordinate types \tabularnewline
\end{recipedef}


\subsubsection{TPL.EXPNO}\label{fits:tpl.expno}\label{fits:obs.tpl.expno}
\begin{recipedef}
Name & TPL EXPNO \tabularnewline
Class & header \tabularnewline
Context & Template \tabularnewline
Type & integer \tabularnewline
Value & \%i \tabularnewline
Unit & None \tabularnewline
Comment & Number of this exposure in a Template \tabularnewline
Default & 0 \tabularnewline
Range & 0..1000 \tabularnewline
Description & Number of this exposure in a Template \tabularnewline
\end{recipedef}


\subsubsection{TPL.START}\label{fits:tpl.start}\label{fits:obs.tpl.start}
\begin{recipedef}
Name & TPL START \tabularnewline
Class & header|template \tabularnewline
Context & Template \tabularnewline
Type & double \tabularnewline
Value & \%.8f \tabularnewline
Unit & None \tabularnewline
Comment & Start time of template \tabularnewline
Default & None \tabularnewline
Range & None \tabularnewline
Description & Start time of template \tabularnewline
\end{recipedef}


\subsubsection{ZEROPNT}\label{fits:zeropnt}
\begin{recipedef}
Name & ZEROPNT \tabularnewline
Class & header \tabularnewline
Context & PRO \tabularnewline
Type & double \tabularnewline
Value & \%.3f \tabularnewline
Unit & mag \tabularnewline
Comment & Zeropoint of the observation \tabularnewline
Default & 0.0 \tabularnewline
Range & None \tabularnewline
Description & Zeropoint of the observation \tabularnewline
\end{recipedef}


\subsubsection{INS.FILT1.NAME}\label{fits:ins.filt1.name}
\begin{recipedef}
Name & INS FILTi NAME \tabularnewline
Class & header \tabularnewline
Context & INS \tabularnewline
Type & string \tabularnewline
Value & \%.50s \tabularnewline
Unit & None \tabularnewline
Comment & Filter name \tabularnewline
Default & None \tabularnewline
Range & None \tabularnewline
Description & Filter name \tabularnewline
\end{recipedef}


\subsubsection{INS.FILT2.NAME}\label{fits:ins.filt2.name}
\begin{recipedef}
Name & INS FILTi NAME \tabularnewline
Class & header \tabularnewline
Context & INS \tabularnewline
Type & string \tabularnewline
Value & \%.50s \tabularnewline
Unit & None \tabularnewline
Comment & Filter name \tabularnewline
Default & None \tabularnewline
Range & None \tabularnewline
Description & Filter name \tabularnewline
\end{recipedef}

\subsubsection{INS.SLIT.NAME}\label{fits:ins.slit.name}
\begin{recipedef}
Name & INS SLIT NAME \tabularnewline
Class & header \tabularnewline
Context & INS \tabularnewline
Type & string \tabularnewline
Value & \%.50s \tabularnewline
Unit & None \tabularnewline
Comment & Slit name \tabularnewline
Default & None \tabularnewline
Range & provided at PAE \tabularnewline
Description & Slit name \tabularnewline
\end{recipedef}

%\subsection{List of header keywords per MICADO data item for imaging observing modes}{}\label{sec:headerkeywordsperdataitemimagingmode}
%
%This section lists per data item the header keywords that shall be in the header of the FITS file for data which is based on observations taken in the imaging observing modes. As a long list of header keywords appears in all data items, we provide that common set only once up front, in the section called DATAITEM.
%\input{generated/keywords_list_dataitems_imaging}


\subsection{Spectroscopic FITS header keyword values}{}\label{sec:specheaderkeywordvalues}

\begin{table}[ht]
\centering
\begin{tabular}{|c | c| c| c|} 
 \hline

  \FITS{INS.SLIT.NAME} & \FITS{INS.FILT.NAME} & Description & Spectral setup \\ [0.5ex] 
 \hline\hline
\PAR{SHORT} & \PAR{SPEC_IJ} &short 16mas$\times$3" slit & IzJ\\
%\PAR{HK_NARROW} & short 16mas$\times$3" slit & HK\\
\PAR{BROAD} & \PAR{SPEC_IJ} &broad 48mas$\times$3" slit & IzJ\\
\PAR{BROAD} & \PAR{SPEC_HK} &broad 48mas$\times$3" slit & HK\\
\PAR{LONG} & \PAR{SPEC_J} &long 20mas$\times$15" slit & J\\
\PAR{LONG} & \PAR{SPEC_HK} &long 20mas$\times$15" slit & HK\\
\PAR{SHORT_SH} & \PAR{SPEC_IJ} &shifted 16mas$\times$3" slit & IzJ\\
%\PAR{HK_NARROW_SHIFTED} & 16mas$\times$3" slit & HK\\
\PAR{BROAD_SH} & \PAR{SPEC_IJ} &shifted 48mas$\times$3" slit & IzJ\\
\PAR{BROAD_SH} & \PAR{SPEC_HK}  &shifted 48mas$\times$3" slit & HK\\
\PAR{LONG_SH} & \PAR{SPEC_J} & shifted 20mas$\times$15" slit & J\\
\PAR{LONG_SH} & \PAR{SPEC_HK} & shifted 20mas$\times$15" slit & HK\\
 \hline
\PAR{PINH} & all & pinhole slit & all\\
\PAR{PINH_SH} & all &shifted pinhole slit & all\\
 \hline
\end{tabular}
\caption{Values for the \ac{FITS} keywords \FITS{INS.FILT.NAME} and \FITS{INS.SLIT.NAME} representing the spectrosopic modes}.
\label{tab:insspecsetup_keyw}
\end{table}