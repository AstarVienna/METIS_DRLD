%% METIS_DRLD.tex
%%
%% METIS data reduction library design document
%%
%% 2020-05-27: initialised from specification document (OC)
%% 2020-09-07: MTR release (OC)

\newcommand{\doctitle}{%
  \parbox[t]{\textwidth}{Data Reduction Library\\Design}}
\newcommand{\docnumber}	{E-REP-AST-MET-1006}     % Document number
\newcommand{\issuenumber}{0-3.5}                % Version
\newcommand{\issuedate}{2023-05-26}              % Date
\newcommand{\workpackage}{8.2}      % Workpackage

% The purpose of texorpdfstring is to provide different
% strings for the title page (tex string) and for the pdf bookmark
% (via hyperref, no tex formatting allowed).
% The two versions need to be kept in sync.
\newcommand{\authorname}{\texorpdfstring{%  % Authors
  O.~Czoske,\\                        % tex string
  W.~Kausch,\\
  N.~Sabha,\\
  W.~Zeilinger}
  {Czoske, Kausch, Sabha, Zeilinger}  % pdf string
  }

\newcommand{\authorsigndate}  {} % date of signature of author

\newcommand{\reviewername}     {%
}       % checked by name
\newcommand{\reviewersigndate} {} % date of signature of checker

\newcommand{\approvername}    {%
}       % name of the approver
\newcommand{\approvalsigndate}{} % date of signature of approver

\newcommand{\releasername}{%
}
\newcommand{\releasesigndate}{}  % date of release

% Title displayed in the header
\newcommand{\shorttitle}{Data Reduction Library Design}


\newcommand{\aj}{The Astronomical Journal}
\newcommand{\aap}{Astronomy \& Astrophysics}
\newcommand{\aaps}{Astronomy \& Astrophysics, Supplement}
\newcommand{\mnras}{Monthly Notices of the Royal Astronomical Society}
\newcommand{\procspie}{Proceedings of the International Society for Optical Engineering}
\newcommand{\pasp}{Publications of the Astronomical Society of the Pacific}
\newcommand{\apjs}{The Astrophysical Journal, Supplement}


%%%%%%%%%%%%%%%%%%%%%%%%%%%%%%%%%%%%%%%%%%%%%%%%%%%%%%%%%%%%%%%%%%%%%%%%%%%%%
\documentclass[11pt,oneside,a4paper]{article}

%% Layout  --- TeX commands
\raggedbottom
\topmargin=-9mm
\headsep=0.5cm
\textheight=246mm
\textwidth=164mm
\hoffset=0cm
\oddsidemargin=-2mm
\parindent=0mm
\parskip=0.3em

\renewcommand{\floatpagefraction}{0.98} % separate float page does not work
\renewcommand{\bottomfraction}{1.0}

%% Fonts
\usepackage{mathptmx}
\usepackage[scaled=0.92]{helvet}

\usepackage[pdftex]{graphicx}
\graphicspath{{./figures/}}

\usepackage{fancyhdr}
\setlength{\headheight}{19pt} % ...at least 18.26802pt
\setlength{\footskip}{31pt}   % ...at least 30.86163pt
\usepackage{titlesec}
\usepackage{booktabs}
\usepackage{tabularx}
\usepackage{array}
\usepackage{longtable}
\usepackage{numprint}
\usepackage{placeins}
\usepackage{enumitem}
\usepackage{acronym}
\usepackage[dvipsnames]{xcolor}
\usepackage{xcolor}
\usepackage{colortbl}
\definecolor{listingbg}{gray}{0.95}
\definecolor{darkgreen}{rgb}{0.0, 0.7, 0.0}
\definecolor{darkblue} {rgb}{0.0, 0.0, 0.7}
\definecolor{darkred}  {rgb}{0.7, 0.0, 0.0}
\definecolor{darkorange}{rgb}{1.0, 0.49, 0.0}

\usepackage[many]{tcolorbox}
\usepackage{multirow}

% does not work with my version of biblatex:
% \usepackage[defernumbers=true,backend=biber,hyperref=true,url=false]{biblatex}
\usepackage[%
   defernumbers=true,
   backend=biber,
   hyperref=true,
   url=false,
   sorting=none]{biblatex}

\usepackage[%
% pagebackref=true,   % does not work with biblatex
   pdfpagelabels=true,
   plainpages=false,
   colorlinks=false]{hyperref}


% Git stuff, must by after hyperref
% The git hash and commit date will be shown when compiling with --shell-escape
% https://tex.stackexchange.com/a/598669/2775
\newif\ifshowgit  % flag to show git info instead of Issue info
% "The control sequence \pdfshellescape is (only?) available in pdftex"
% http://tex.stackexchange.com/a/13253/2595
\ifnum\pdfshellescape=1{ %
    % Check for overleaf https://tex.stackexchange.com/a/598669/2775
    \ifnum\pdfstrcmp{\jobname}{output}=0
        % Apparently overleaf does not compile from their git repo.
        \showgitfalse
    \else
        \usepackage[noheader]{gitver}
        \def\parsegitdate#1-#2-#3 #4\endparse{#3.#2.#1}
        \makeatletter
            \edef\gitdate{\@@input|"git log -1 --format=\@percentchar ci"}
        \makeatother
        \showgittrue
    \fi
\else
    \showgitfalse
\fi


% does not work with my version:
%\usepackage[xindy,nopostdot,nogroupskip]{glossaries}
\usepackage[xindy]{glossaries}
%\makeglossaries

\usepackage{lastpage}
\usepackage{multirow}
\usepackage[%
   font=small,
   format=plain,
   indention=1.5em,
   labelfont={small,bf},
   up,
   justification=justified,
   singlelinecheck=true]{caption}

%% for sidewaysfigure and sidewaystable
\usepackage{rotating}

%% for landscape
\usepackage{pdflscape}

% for newgeometry and restoregeometry
\usepackage{geometry}

%% put recipe names, QC parameters, etc., in \lstinline{}
\usepackage{listings}
\lstloadlanguages{csh, c}

\usepackage{textcomp}

\usepackage{csquotes}

\usepackage[countmax]{subfloat}
% ADDING NEW DEFINITIONS -------------------------------------------- start
\definecolor{listingbg}{gray}{0.95}
\definecolor{darkgreen}{rgb}{0.0, 0.7, 0.0}
\definecolor{darkblue} {rgb}{0.0, 0.0, 0.7}
\definecolor{cyan} {rgb}{0.0, 0.4, 0.4}
\definecolor{darkred}  {rgb}{0.7, 0.0, 0.0}
\definecolor{darkorange}{rgb}{1.0, 0.49, 0.0}
\definecolor{violett}{rgb}{255, 0, 255}
\definecolor{turq}{rgb}{0.0, 0.7, 0.8}
\definecolor{fits}{rgb}{0.4, 0.1, 1}


\makeatletter
\newcommand{\replaceunderscores}[1]{\expandafter\replace@underscores#1_\relax}

\def\replace@underscores#1_#2\relax{%
    \ifx \relax #2\relax
        #1%
    \else
        #1%
        \textunderscore
        \replace@underscores#2\relax
    \fi
}
\makeatother

%\setcounter{tocdepth}{4}
\setcounter{secnumdepth}{4}
%% Write a raw FITS file like this: \hyperref[dataitem:lm_sci_raw]{\RAW{LM_SCI_RAW}}
\makeatletter
\lstdefinestyle{RAWstyle}{%
  basicstyle=\ttfamily\color{fits}%
  \lst@ifdisplaystyle\scriptsize\fi}
\makeatother
\newcommand{\RAW}[1]{\lstinline[style=RAWstyle]!#1!}

%% Write parameters like this: \PAR{DARK_LM}
\makeatletter
\lstdefinestyle{PARstyle}{%
  basicstyle=\ttfamily\color{cyan}%
  \lst@ifdisplaystyle\scriptsize\fi}
\makeatother
\newcommand{\PAR}[1]{\lstinline[style=PARstyle]!#1!}

%% Write DRL functions names like this: \DRL{function}
\makeatletter
\lstdefinestyle{DRLstyle}{%
  basicstyle=\ttfamily\color{violett}%
  \lst@ifdisplaystyle\scriptsize\fi}
\makeatother
\newcommand{\DRL}[1]{\lstinline[style=DRLstyle]!#1!}

%% Write recipe names like this: \REC{metis_do_stuff}
\makeatletter
\lstdefinestyle{RECstyle}{%
  basicstyle=\ttfamily\color{darkgreen}%
  \lst@ifdisplaystyle\scriptsize\fi}
\makeatother
\newcommand{\REC}[1]{\texorpdfstring{\lstinline[style=RECstyle]!#1!}{\replaceunderscores{#1}}}

%% Write QC parameters like this: \QC{QC_SOMETHING_OR_OTHER}
\makeatletter
\lstdefinestyle{QCstyle}{%
  basicstyle=\ttfamily\color{darkblue}%
  \lst@ifdisplaystyle\scriptsize\fi}
\makeatother
\newcommand{\QC}[1]{\lstinline[style=QCstyle]!#1!}

%% Write templates like this: \TPL{DARK_LM}
\makeatletter
\lstdefinestyle{TPLstyle}{%
  basicstyle=\ttfamily\color{darkred}%
  \lst@ifdisplaystyle\scriptsize\fi}
\makeatother
\newcommand{\TPL}[1]{\lstinline[style=TPLstyle]!#1!}

%% Write products like this: \hyperref[dataitem:some_thing]{\PROD{SOME_THING}}
\makeatletter
\lstdefinestyle{PRODstyle}{%
  basicstyle=\ttfamily\color{darkorange}%
  \lst@ifdisplaystyle\scriptsize\fi}
\makeatother
\newcommand{\PROD}[1]{\lstinline[style=PRODstyle]!#1!}

%% Write requirements like this: \REQ{METIS-xxxx}
\newcommand{\REQ}[1]{\href{https://polarion.astron.nl/polarion/\#/project/METIS/workitem?id=#1}{\textcolor{brown}{#1}}}

%% external calib files
\makeatletter
\lstdefinestyle{EXTCALIBstyle}{%
  basicstyle=\ttfamily\color{Turquoise}%
  \lst@ifdisplaystyle\scriptsize\fi}
\makeatother
\newcommand{\EXTCALIB}[1]{\lstinline[style=EXTCALIBstyle]!#1!}

% static calib files
\makeatletter
\lstdefinestyle{STATCALIBstyle}{%
  basicstyle=\ttfamily\color{teal}%
  \lst@ifdisplaystyle\scriptsize\fi}
\makeatother
\newcommand{\STATCALIB}[1]{\lstinline[style=STATCALIBstyle]!#1!}

%% Write FITS keywords (and values) like this: \FITS{EXPTIME}
\newcommand{\FITS}[1]{\lstinline[]!#1!}

%% Write code snippets like this: \CODE{SOMETHING==ELSE}
%% (I know, this is a bit stupid)
\newcommand{\CODE}[1]{\lstinline[]!#1!}

%% Use environment recipedef to define a new recipe.
\newenvironment{recipedef}
{% begin code
  \begin{tcolorbox}[breakable,enhanced jigsaw]%
    \begin{longtable}{p{0.24\hsize}p{0.69\hsize}}}
      {% end code
    \end{longtable}%
  \end{tcolorbox}}

%% Use environment datastructdef to define the data structure of a data item (see e.g. LSS_data_structures.tex)
\newenvironment{datastructdef}
{% begin code
  \begin{tcolorbox}[breakable,enhanced jigsaw]%
    \begin{longtable}{p{0.97\hsize}}}
      {% end code
    \end{longtable}%
  \end{tcolorbox}}


%%%%% This is a definition that has the flow diagram within the box:
%% \newcommand{\theimage}{}  % dummy definition
%% \newenvironment{recipedef}[1]
%% {% begin
%% code
%% \renewcommand{\theimage}{#1}
%% \begin{tcolorbox}%
%%   \begin{tabular}{p{0.24\hsize}p{0.74\hsize}}
%%   }%
%%     {% end code
%%   \end{tabular}%
%%   \begin{center}%
%%     \includegraphics[width=0.6\textwidth]{\theimage}%
%%   \end{center}%
%% \end{tcolorbox}}




%% Proper typography for micrometer
\newcommand{\micron}{\textmu\mathrm{m}}
\newcommand{\mat}[1]{\mathbf{#1}}
\newcommand{\abs}[1]{\lvert#1\rvert}

%% Ion
\DeclareRobustCommand{\ion}[2]{%
  \textup{#1{\mdseries\textsc{#2}}}%
}

%%%% \la...kleiner als ungef"ahr
\def\la{\mathrel{\mathchoice {\vcenter{\offinterlineskip\halign{\hfil
          $\displaystyle##$\hfil\cr<\cr\sim\cr}}}
    {\vcenter{\offinterlineskip\halign{\hfil$\textstyle##$\hfil\cr
          <\cr\sim\cr}}}
    {\vcenter{\offinterlineskip\halign{\hfil$\scriptstyle##$\hfil\cr
          <\cr\sim\cr}}}
    {\vcenter{\offinterlineskip\halign{\hfil$\scriptscriptstyle##$\hfil\cr
          <\cr\sim\cr}}}}}

%%%% \ga...Gr"o"ser als ungef"ahr
\def\ga{\mathrel{\mathchoice {\vcenter{\offinterlineskip\halign{\hfil
          $\displaystyle##$\hfil\cr>\cr\sim\cr}}}
    {\vcenter{\offinterlineskip\halign{\hfil$\textstyle##$\hfil\cr
          >\cr\sim\cr}}}
    {\vcenter{\offinterlineskip\halign{\hfil$\scriptstyle##$\hfil\cr
          >\cr\sim\cr}}}
    {\vcenter{\offinterlineskip\halign{\hfil$\scriptscriptstyle##$\hfil\cr
          >\cr\sim\cr}}}}}

%%%% Degrees, arcmin, arcsec
\def\degr{\hbox{$^\circ$}}

\def\arcmin{\hbox{$^\prime$}}

\def\arcsec{\hbox{$^{\prime\prime}$}}

%% Bright red for things that still need to be looked at. None of
%% these should appear in the release version
\newcommand{\TODO}[1]{\textcolor{red}{\bfseries TODO: #1}}
\newcommand{\TBD}{\textcolor{red}{\bfseries TBD}}

\newcommand{\intentblankpage}{}

%% paragraph column, centred text
\newcolumntype{P}[1]{>{\centering\arraybackslash}p{#1}}

%=== Header / Footer definition ===============================================
\newcommand{\headerformat}{%
  \lhead{\small\sffamily
    \begin{tabularx}{\textwidth}{Xlll}
      \ifshowgit
      \shorttitle & \docnumber & GIT \ \gitVer & \gitdate\\[0.7ex]
      \else
      \shorttitle & \docnumber & \issuenumber & \issuedate\\[0.7ex]
      \fi
    \end{tabularx}}
  \rhead{}
  \cfoot{\small\bfseries\sffamily
    Page {\textbf{\thepage}}  of {\textbf{\pageref{LastPage}}} }
  \rfoot{\raisebox{-0.3\height}{\includegraphics[width=25mm]{metis_noText.pdf}} }
}

\fancypagestyle{plain}{\fancyhf{} \headerformat}
\headerformat
\definecolor{brn}{RGB}{99,36,35}
\definecolor{rd1}{RGB}{229,184,183}
\definecolor{rd2}{RGB}{242,219,219}

\renewcommand\headrule{%
  \begingroup
  \color{brn}
  \hrule height 0.5pt width\headwidth
  \endgroup
}

\renewcommand{\contentsname}{Table of Contents}

%% ---- Format of section titles --------------------------------
% Combining sffamily and scshape works only for some fonts
\titleformat{\section}{\LARGE\sffamily\scshape}{\thesection}{1em}{\textsc{}}
\titleformat{\subsection}{\Large\sffamily}{\thesubsection}{1em}{}
\titleformat{\subsubsection}{\large\sffamily}{\thesubsubsection}{1em}{}

\setlength\heavyrulewidth{1.5pt}


% === PDF Definition ==========================================================
\hypersetup{%
  setpagesize,
  bookmarksnumbered=true,
  pdfborder= 0 0 0,
  pdftitle={\shorttitle},
  pdfauthor={\authorname}}
%\pdfinfo{ /CreationDate (D:\issuedate)}


% === biblatex Definitions ====================================================
\bibstyle{apj}
\addbibresource{references.bib}
\newbibmacro{string+doiurlisbn}[1]{\iffieldundef{url}{#1}{\href{\thefield{url}}{#1}}}
\DeclareFieldFormat{title}{\usebibmacro{string+doiurlisbn}{\mkbibemph{#1}}}


% === glossaries definitions  =================================================
% \input{../common/metis_glossary.tex}     % the common glossary file
\renewcommand{\glossarysection}[2][]{}   % Avoid the heading 'Glossary'
\setlength{\LTleft}{0pt}                 % make all longtables aligned left
\setlength{\glsdescwidth}{0.8\textwidth}

% === Headers for notebook rendering ==========================================
\input{JuPyter_header}

%%%%%%%%%%%%%%%%%%%%%%%%%%%%%%%%%%%%%%%%%%%%%%%%%%%%%%%%%%%%%%%%%%%%%%%%%%%%%
%
% Start of document
%
%%%%%%%%%%%%%%%%%%%%%%%%%%%%%%%%%%%%%%%%%%%%%%%%%%%%%%%%%%%%%%%%%%%%%%%%%%%%%
%
\begin{document}

% listings style
\lstset{basicstyle=\ttfamily\lst@ifdisplaystyle\scriptsize\fi,
  columns=flexible,
  frame=single,
  backgroundcolor=\color{listingbg},
  captionpos=b,
  showspaces=false}


%%%%% titlepage
\thispagestyle{empty}

% Logo
\vspace*{0cm}
\includegraphics[width=6.69cm]{metis_logo.pdf}

\vspace*{\fill}

% Document title
{\color{brn}\rule[1.9ex]{\textwidth}{1.5pt}}
\scalebox{1.44}{\Huge\textsf{\doctitle}}\\
{\color{brn}\rule{\textwidth}{1.5pt}}\\ [0.5ex]

% Document info
{\Large\textsf{\docnumber}}  \\ [1ex]
\ifshowgit
{\Large\textsf{Git revision: \ \gitVer}} \\ [1ex]
{\Large\textsf{\gitdate}} \\[1ex]
\else
{\Large\textsf{Issue \issuenumber}} \\ [1ex]
{\Large\textsf{\issuedate}}  \\ [1ex]
\fi
{\Large\textsf{Work package: \workpackage}}  \\[1ex]

\vspace*{\fill}

% Signature table
\begin{center}
  \renewcommand{\arraystretch}{0.75}
%  \begin{tabularx}{\textwidth}{XXXX}
  \begin{tabular}{p{0.25\textwidth}p{0.25\textwidth}p{0.25\textwidth}p{0.25\textwidth}}
    \arrayrulecolor{brn}
    \toprule
    & \multicolumn{3}{l}{\scriptsize\textsf{Signature and Approval}} \\
    \midrule
    & {\scriptsize\textsf Name}
    & {\scriptsize\textsf Date}
    & {\scriptsize\textsf Signature} \\
    \cline{2-4}
    \\
    \textsf{Prepared} & \parbox[c]{\hsize}{\raggedright \authorname} & \authorsigndate & \\
    \\
    \midrule
    \\
    \textsf{Reviewed} & \parbox[c]{\hsize}{\raggedright \reviewername} & \reviewersigndate & \\
    \\
    \midrule
    \\
    \textsf{Approved} & \parbox[c]{\hsize}{\raggedright \approvername} & \approvalsigndate & \\
    \\
    \midrule
    \\
    \textsf{Released} & \parbox[c]{\hsize}{\raggedright \releasername} & \releasesigndate & \\
    \\
    \bottomrule
     %    \end{tabularx}
  \end{tabular}
\end{center}

%%% Local Variables:
%%% TeX-master: "METIS_DRLD"
%%% End:


% === MAIN TEXT================================================================

\clearpage
\pagestyle{fancy}

% === Revision History =========================================================
\section*{Revision History}

\renewcommand{\arraystretch}{1.2}
\arrayrulecolor{brn}
\begin{tabularx}{\textwidth}{|l|l|l|X|}
  \hline
  \rowcolor{rd1}
  \textbf{Issue} & \textbf{Date} & \textbf{Owner} & \textbf{Changes} \\
  \hline
                 & 2020-05-27    & Czoske         & initial draft    \\
  0.1            & 2020-09-27    & all            & MTR release      \\
  0.2            & 2022-06-30    & all            & preliminary internal release      \\
  0.3            & 2022-09-07    & LSS Sections   & Internal review \\
  0.35           & 2023-05-26    & telluric correction revision   & Internal review \\
  \hline
\end{tabularx}


%=== Indexes ==================================================================
\newpage
\tableofcontents
\clearpage
\listoffigures
\clearpage
\listoftables

\clearpage
\phantom{a}
\vfill
\begin{center}
  This page intentionally left blank
\end{center}
\vfill
\clearpage

%%%%%%%%%%%%%%%%%%%%%%%%%%%%%%%%%%%%%%%%%%%%%%%%%%%%%%%%%%%%%%%%%%%%%%%%%%%%%


% INTRODUCTION
\section{Introduction}
\label{sec:intro}

\subsection{Scope}

This document describes the design of the data reduction software for
METIS, the Mid-infrared ELT Imager and Spectrograph. It builds upon
the Data Reduction Library Specifications document \cite{DRLS}
presented for \ac{PDR} and supersedes that document for \ac{FDR}.

\subsection{Applicable documents}

\begin{refcontext}[labelprefix=AD]
  \printbibliography[keyword=applicable, heading=none]
\end{refcontext}


\subsection{Reference documents}

\begin{refcontext}[labelprefix=RD]
  \printbibliography[keyword=reference, heading=none]
\end{refcontext}

\subsection{Acronyms}
\label{ssec:acronyms}
%%% Acronym list

This document employs several abbreviations and acronyms to refer concisely to an item, after it has been introduced.
The following list is aimed to help the reader in recalling the extended meaning of each short expression:

\begin{acronym}[AAAAAA]
   \acro{ADC}{Atmospheric Dispersion Corrector}
%   \acro{AGB}{Asymptotic Giant Branch}                    
%   \acro{AIT}{Assembly Integration and Test}
%   \acro{AO }{Adaptive Optics}                            
%   \acro{APP}{Apodizing Phase Plate}                      
%   \acro{CFO}{Common Fore Optics}                         
%   \acro{CLC}{Classical Lyot Coronagraph}                 
%   \acro{CPL}{Common Pipeline Library}                    
%   \acro{CVC}{Classical Vortex Coronagraph}               
%   \acro{DFS}{Data Flow System}                           
%   \acro{DIT}{Detector Integration Time}                  
%   \acro{DRL}{Data Reduction Library}                     
%   \acro{DRS}{Data Reduction Software}                    
%   \acro{ELFN}{Excess Low Frequency Noise}                 
   \acro{ELT}{Extremely Large Telescope}                  
%   \acro{ESO}{European Southern Observatory}              
%   \acro{FDR}{Final Design Review}                        
%   \acro{FITS}{Flexible Image Transport System}            
%   \acro{FP}{Focal Plane}                                
%   \acro{HDRL}{High-Level Data-Reduction Library}          
%   \acro{HDU}{Header Data Unit}                           
%   \acro{IFS}{Integral Field Spectrograph}                
%   \acro{IFU}{Integral Field Unit}                        
%   \acro{IR}{Infrared}                                   
%   \acro{LSF}{Line Spread Function}                       
%   \acro{LSS}{Long-slit spectroscopy}                     
%   \acro{MAIT}{Manufacture, Assembly, Integration and Test} 
   \acro{METIS}{Mid-infrared E-ELT Imager and Spectrograph} 
%   \acro{MIR}{Mid-Infrared}                               
%   \acro{NIR}{Near-Infrared}                              
%   \acro{PP}{Pupil Plane}                                
%   \acro{PSF}{Point Spread Function}                      
   \acro{PWV}{Precipitable Water Vapour}                  
%   \acro{QC}{Quality Control}                            
%   \acro{QCL}{Quantum Cascade Laser}                      
%   \acro{RAVC}{Ring Apodized Vortex Coronagraph}           
%   \acro{RMS}{Root Mean Square}                           
   \acro{RSRF}{Relative Spectral Response Function}        
%   \acro{SCAO}{Single Conjugate Adaptive Optics}           
%   \acro{SED}{Spectral Energy Distribution}               
%   \acro{TBC}{To Be Confirmed}                            
%   \acro{TBD}{To Be Determined}                           
%   \acro{VLT}{Very Large Telescope}                       
%   \acro{WCS}{World Coordinate System}                    
%  \acro{ WCU}{Warm Calibration Unit}                      
\end{acronym}

%%%%%%%%%%%%%%%%%%%%%%%%%%%%%%%%%%%%%%%%%%%%%%%%%%%%%%%

%%% Local Variables:
%%% TeX-master: "METIS_DRLS_PDR"
%%% End:


\subsection{Requirements}
\label{ssec:requirements}

The following table connects the requirements (Polarion-links) to the relevant
place in this document. All pipeline subsystem requirements are listed; an entry
stating \emph{N/A} in the last column indicates that this document does not
provide evidence to verify the requirement in question. See \cite{DRLVT} for information about their verification.

Requirements marked with $\star$ are commented on individually below the table.

\begin{longtable}[c]{|l|l|l|}
	\caption{Requirements compliance}
	% \endfirsthead
 \hline
 \textbf{Req. ID} & \textbf{Requirement Title} & \textbf{Section} \\
 \hline
    \endhead
    \hline
		\REQ{METIS-5945} & DRS for science-grade data reduction & \ref{sec:pipeline_recipes} \\
		\REQ{METIS-5989} & ADI post-processing &  \ref{ssec:ADI_postprocessing}\\
		\REQ{METIS-5997} & Detector linearity characterisation & \ref{rec:metis_det_lingain} \\
		\REQ{METIS-6058} & Data reduction package & \ref{sec:pipeline_recipes} \\
		\REQ{METIS-6059} & DRS for Quick-Look and Operational Data QC & \ref{sec:pipeline_recipes} \\
		\REQ{METIS-6060} & Interactive data reduction system &  \ref{sec:pipeline_recipes}\\
		\REQ{METIS-6061} & Reduced simulated data products & \ref{ssec:reduced_data_format} \\
		\REQ{METIS-6062} & Software Assurance Requirements & N/A \\
		\REQ{METIS-6063} & Master dark frames & \ref{rec:metis_det_dark} \\
		\REQ{METIS-6065} & System Engineering Management Plan & N/A \\
		\REQ{METIS-6067} & PIP Requirement Precedence & N/A  \\
		\REQ{METIS-6069} & METIS Standards & N/A \\
		\REQ{METIS-6070} & METIS coordinate systems definition & \ref{ssec:instrument_data_format} \\
		\REQ{METIS-6071} & External Interfaces to METIS &  N/A \\
		\REQ{METIS-6072} & Parallel observing mode IMG-LM and LMS & \ref{sssec:parallellmifu} \\
		\REQ{METIS-6073} & LMS Spectral Resolution & \ref{qc:qc_ifu_wavecal_line_width} \\
		\REQ{METIS-6074} & Wavelength calibration & \ref{rec:metis_lm_lss_wave}, \ref{rec:metis_n_lss_std} \ref{rec:metis_ifu_wavecal}\\
		\REQ{METIS-6075} & Burst mode data rates & \ref{ssec:criticalifudatarate} \\
		\REQ{METIS-6077} & Operational use cases &  $\star$ \\
		\REQ{METIS-6078} & Electrical power consumption of DRS work station &  N/A \\
		\REQ{METIS-6080} & FITS format and meta-data of generated data & \ref{ssec:reduced_data_format} \\
		\REQ{METIS-6081} & FITS keywords & \ref{ssec:instrument_data_format}  \\
		\REQ{METIS-6082} & Data products conformity to ESO standards &  \ref{ssec:reduced_data_format}\\
		\REQ{METIS-6083} & Material usage & N/A \\
		\REQ{METIS-6084} & Low-resolution slit spectroscopy & \ref{ssec:recipes_lss_lm}, \ref{ssec:recipes_lss_n} \\
		\REQ{METIS-6085} & Background subtraction in IMG-LM with dedicated sky observations & \ref{rec:metis_lm_img_background} \\
		\REQ{METIS-6086} & Background subtraction in IMG-LM from dithered images & \ref{rec:metis_lm_img_background} \\
		\REQ{METIS-6087} & Distortion/plate-scale calibration & \ref{rec:metis_lm_img_distortion}, \ref{rec:metis_n_img_distortion} \\
		\REQ{METIS-6088} & Distortion correction & \ref{rec:metis_n_img_calibrate}, \ref{rec:metis_lm_img_calibrate} \\
		\REQ{METIS-6089} & Masked region of the N-band detector & \ref{rec:metis_n_img_chopnod} \\
		\REQ{METIS-6090} & Masked region of the IMG\_LM detector & \ref{rec:metis_lm_img_basic_reduce} \\
		\REQ{METIS-6091} & Telluric calibration recipe & \ref{rec:metis_lm_lss_mf_model}, \ref{rec:metis_n_lss_mf_correct} \\
		\REQ{METIS-6092} & METIS Internal Interfaces & N/A \\
		\REQ{METIS-6093} & Format and content of the raw data & \ref{ssec:instrument_data_format} \\
		\REQ{METIS-6094} & Background subtraction in chopping-nodding mode & \ref{rec:metis_n_img_chopnod} \\
		\REQ{METIS-6096} & LM-band imaging flatfield & \ref{rec:metis_lm_img_flat} \\
		\REQ{METIS-6098} & N-band imaging flatfield & \ref{rec:metis_n_img_flat} \\
		\REQ{METIS-6104} & IMG-LM data products & \ref{rec:lm_img_calibrate}, \ref{rec:lm_img_postprocess} \\
		\REQ{METIS-6105} & IMG-N data products & \ref{rec:n_img_calibrate}, \ref{rec:n_img_restoration} \\
		\REQ{METIS-6112} & SPEC-LM data products & \ref{ssec:recipes_lss_lm} \\
		\REQ{METIS-6113} & SPEC-N data products & \ref{ssec:recipes_lss_n} \\
		\REQ{METIS-6131} & LMS data products & \ref{rec:metis_ifu_sci_process} \\
		\REQ{METIS-6265} & LMS straylight & \ref{rec:metis_ifu_sci_process}  \\
		\REQ{METIS-6267} & Design Life & N/A \\
		\REQ{METIS-6309} & Masked regions of the LMS detectors & \ref{rec:metis_ifu_sci_process} \\
		\REQ{METIS-6681} & Error propagation & \ref{Sec:critalg_errorprop} \\
		\REQ{METIS-6698} & Characterisation and correction of relative spectral response function & \ref{rec:metis_lm_lss_rsrf}, \ref{rec:metis_n_lss_rsrf}, \ref{rec:metis_ifu_rsrf} \\
		\REQ{METIS-6733} & Auxiliary Parameters in Data Products & \ref{ssec:reduced_data_format} \\
		\REQ{METIS-6923} & QC parameters for permanent monitoring &  \ref{sec:qc_parameters}\\
		\REQ{METIS-7244} & Parallel observing mode IMG-LM and IMG-N & \ref{sssec:parallellmnimg} \\
		\REQ{METIS-7245} & Parallel observing mode LSS-LM and LSS-N & \ref{sssec:parallellmnspec} \\
		\REQ{METIS-9145} & Persistence Correction & \ref{rec:metis_det_persistence} \\
		\REQ{METIS-9150} & Calibration of atmospheric dispersion correction and slit losses & \ref{sssec:adc_slitlosses} \\
		\REQ{METIS-9151} & Fringing correction & \ref{rec:metis_fringing_correction} \\
		\REQ{METIS-9212} & IMG-LM\_(RA/C)VC data products & \ref{sssec:adi_img_vc} \\
		\REQ{METIS-9213} & IMG-LM\_APP data products & \ref{sssec:adi_img_app} \\
		\REQ{METIS-9214} & IMG-N\_CVC data products & \ref{sssec:adi_img_vc} \\
		\REQ{METIS-9215} & LMS\_(RA/C)VC data products & \ref{sssec:adi_ifu} \\
		\REQ{METIS-9216} & LMS\_APP data products & \ref{sssec:adi_ifu} \\
		\REQ{METIS-9355} & Code documentation & \ref{sec:drl_functions} \\
		\REQ{METIS-9626} & AO Telemetry in Science Data Products &  \ref{ssec:reduced_data_format}\\
		\REQ{METIS-9627} & Data provenance & \ref{ssec:reduced_data_format} \\
		\REQ{METIS-10300} & Wavelength-to-pixel calibration uncertainty during AIT  & \ref{rec:metis_lm_lss_wave}, \ref{rec:metis_ifu_wavecal}  \\ 
    \hline
\end{longtable}

Notes on individual requirements:

\noindent\REQ{METIS-6077} refers to the operational use cases
\emph{observation}, \emph{calibration} and \emph{maintenance}. As such, there is
no single point in this document to point to for verification evidence. The recipes in Chapter \ref{sec:pipeline_recipes} cover all of
these use cases.

\clearpage


\section{Instrument Modes and Configurations}
\label{sec:instrument_modes}

The following table lists the instrument modes for METIS
\cite{METIS-operational_concept}. All of these modes will be
supported by the data reduction pipeline, where horizontal lines
delimit groups of modes that can be reduced with the same set of
recipes.  Post-processing recipes may be available for individual
modes within a group (e.g.\ ADI post-processing for HCI modes).

\begin{center}\small
  \captionof{table}[METIS instrument modes]{The five main observing modes of METIS. The acronyms stand for: \textbf{CLC} -- Classical Lyot Coronagraph; \textbf{CVC} -- Classical Vortex Coronagraph; \textbf{RAVC} -- Ring-apodized Vortex Coronagraph; \textbf{APP} -- Apodized Phase Plate; N/A -- not applicable.; \textbf{P/T} -- pupil tracking; \textbf{F/T} -- field tracking.}\label{tab:instrument_mode}
%  \renewcommand{\arraystretch}{1.15}
  %\begin{tabular}{\textwidth}{|P{0.12\textwidth}|c|c|P{0.12\textwidth}|c|c|c|}
  \begin{tabular}{|P{0.12\textwidth}|c|c|P{0.12\textwidth}|c|c|c|l|}
    \hline
    \multirow{2}{\linewidth}{\centering\textbf{Observing Mode}}                          & \multicolumn{6}{c|}{\textbf{Instrument Configuration}}                                                                                                                                                                    & \\
                                                                                         & \textbf{Subsystem}                                     & \textbf{Band}         & \parbox[c][4ex]{\hsize}{\centering\textbf{IFS Setting}}                         & \textbf{HCI Mask}         & \textbf{P/T} & \textbf{F/T} & \textbf{Code} \\
    \hline\hline
    %%%%%%% Direct Imaging
    \multirow{2}{\hsize}{\centering\textbf{Direct Imaging}}                             & IMG                                                    & L, M                  & \textcolor{black!35}{N/A}                                                       & \textcolor{black!35}{N/A} & $\bullet$ & $\bullet$  & IMG\_LM \\
    \cline{2-8}
                                                                                         & IMG                                                    & N                     & \textcolor{black!35}{N/A}                                                       & \textcolor{black!35}{N/A} & $\bullet$ & $\bullet$  & IMG\_N \\
    \hline\hline
    %%%%%%%% High-contrast imaging
    \multirow{6}{\hsize}{\centering\textbf{High Contrast Imaging}}                       &                                                        &                       & \parbox[c][4ex]{\hsize}{\centering \textcolor{black!35}{N/A}}                   & RAVC/CVC                  & $\bullet$ &            & IMG\_LM\_(RA/C)VC \\
    \cline{4-8}
                                                                                         & IMG                                                    & \multirow{1}{*}{L, M} & \parbox[c][4ex]{\hsize}{\centering \textcolor{black!35}{N/A}}                   & APP                       & $\bullet$ &            & IMG\_LM\_APP \\
    \cline{4-8}
                                                                                         &                                                        &                       & \parbox[c][4ex]{\hsize}{\centering \textcolor{black!35}{N/A}}                   & CLC                       & $\bullet$ &            & IMG\_LM\_CLC \\
    \cline{2-8}
                                                                                         & \multirow{2}{*}{IMG}                                   & N1, N2                & \parbox[c][4ex]{\hsize}{\centering \textcolor{black!35}{N/A}}                   & CVC                       & $\bullet$ &            & IMG\_N\_CVC \\
    \cline{3-8}
                                                                                         &                                                        & N                     & \parbox[c][4ex]{\hsize}{\centering \textcolor{black!35}{N/A}}                   & CLC                       & $\bullet$ &            & IMG\_N\_CLC \\
    \hline\hline
    %%%%%%%% Longslit spectroscopy
    \multirow{2}{\hsize}{\centering\textbf{Longslit spectroscopy}}                       & IMG                                                    & L, M                  & \textcolor{black!35}{N/A}                                                       & \textcolor{black!35}{N/A} &           & $\bullet$  & SPEC\_LM\\
    \cline{2-8}
                                                                                         & IMG                                                    & N                     & \textcolor{black!35}{N/A}                                                       & \textcolor{black!35}{N/A} &           & $\bullet$  & SPEC\_N\_LOW \\
    \hline\hline
    %%%%%%%% IFU
    \multirow{3}{\hsize}{\parbox[c]{\hsize}{\centering\textbf{IFU spectroscopy}}}        & LMS                                                    & L, M                  & full IFU field                                                                  & \textcolor{black!35}{N/A} & $\bullet$ & $\bullet$  & IFU\_nominal \\
    \cline{2-8}
                                                                                         & LMS                                                    & L, M                  & \parbox[c][7ex]{\hsize}{\centering extended $\Delta\lambda = 300\,\mathrm{nm}$} & \textcolor{black!35}{N/A} & $\bullet$ & $\bullet$  & IFU\_extended \\
    \hline\hline
    %%%%%% IFU + HCI
    \multirow{6}{\hsize}{\parbox[c]{\hsize}{\centering\textbf{IFU + HCI spectroscopy}}}  & \multirow{3}{*}{LMS}                                   & \multirow{3}{*}{L, M} & \multirow{3}{\hsize}{full IFU field}                                            & APP                       & $\bullet$ &            & IFU\_nominal\_APP \\
    \cline{5-8}
                                                                                         &                                                        &                       &                                                                                 & RAVC/CVC                  & $\bullet$ &            & IFU\_nominal\_(RA/C)VC \\
    \cline{5-8}
                                                                                         &                                                        &                       &                                                                                 & CLC                       & $\bullet$ &            & IFU\_nominal\_CLC\\
    \cline{2-8}
                                                                                         & \multirow{3}{*}{LMS}                                   & \multirow{3}{*}{L, M} & \multirow{3}{\hsize}{\centering extended $\Delta\lambda = 300\,\mathrm{nm}$}    & APP                       & $\bullet$ &            & IFU\_extended\_APP \\
    \cline{5-8}
                                                                                         &                                                        &                       &                                                                                 & RAVC/CVC                  & $\bullet$ &            & IFU\_extended\_(RA/C)VC \\
    \cline{5-8}
                                                                                         &                                                        &                       &                                                                                 & CLC                       & $\bullet$ &            & IFU\_extended\_CLC\\
    \hline

  \end{tabular}
  % \end{tabularx}
\end{center}

For each observing mode, one or more templates exist, each of which
triggers a pipeline recipe at the observatory. Templates and recipes
cover the observational use cases ``Observation'', ``Calibration'' and
``Maintenance'' (\REQ{METIS-6077}).

%\clearpage

%%%% OLD INSTRUMENT MODE TABLE
%%\begin{center}\normalsize
%%%  \caption[Instrument modes]{}
%% % \label{tab:instrument_modes}
%%\captionof{table}{Instrument modes}\label{tab:instrument_modes}
%%\begin{tabularx}{\textwidth}{|l|l|X|}
%%    \hline
%%    \textbf{Instrument mode code} & \textbf{HCI mode code} & \textbf{Instrument mode name}                      \\
%%    \hline\hline
%%    \CODE{IMG_LM}                 & --                     & LM band imaging                                    \\
%%    \CODE{IMG_LM_(RA/C)VC}        & \CODE{(RA/C)VC-L/M}    & Ring-apodized or classical vortex coronagraphy L/M \\
%%    \CODE{IMG_LM_CLC}             & \CODE{CLC-LM}          & Classical Lyot coronagraphy L or M                 \\
%%    \CODE{IMG_LM_APP}             & \CODE{APP-IMG-LM}      & Classical APP LM                                   \\
%%    \hline
%%    \CODE{IMG_NQ}                 & --                     & NQ band imaging                                    \\
%%    \CODE{IMG_N_CVC}              & \CODE{CVC-N1/N2}       & Classical vortex coronagraphy N1 or N2             \\
%%    \CODE{IMG_N_CLC}              & \CODE{CLC-N}           & Classical Lyot coronagraphy in N-band              \\
%%    \hline
%%    \CODE{SPEC_LM}                & --                     & full L or M band low-resolution spectroscopy       \\
%%    \hline
%%    \CODE{SPEC_N_LOW}             & --                     & full N-band low-resolution spectroscopy            \\
%%    \hline
%%    \CODE{IFU_nominal}            & --                     & nominal IFU spectroscopy                           \\
%%    \CODE{IFU_nominal_APP}        & \CODE{APP-LMS}         & nominal IFU spectroscopy with APP                  \\
%%    \CODE{IFU_nominal_(RA/C)VC}   & \CODE{(RA/C)VC-LMS}    & nominal IFU spectroscopy with (RA/C)VC in L or M   \\
%%    \CODE{IFU_nominal_CLC}        & \CODE{CLC-LMS}         & nominal IFU spectroscopy with CLC in L or M band   \\
%%    \CODE{IFU_extended}           & --                     & extended IFU spectroscopy                          \\
%%    \CODE{IFU_extended_APP}       & \CODE{APP-LMS}         & extended IFU spectroscopy with APP                 \\
%%    \CODE{IFU_extended_(RA/C)VC}  & \CODE{(RA/C)VC-LMS}    & extended IFU spectroscopy with (RA/C)VC in L or M  \\
%%    \CODE{IFU_extended_CLC}       & \CODE{CLC-LMS}         & extended IFU spectroscopy with CLC in L or M band  \\
%%    \hline
%%%    \CODE{PUP_L}                  & --                     & Pupil imaging L band (engineering mode)            \\
%%%    \CODE{PUP_N}                  & --                     & Pupil imaging N band (engineering mode)            \\
%%%    \hline
%%  \end{tabularx}
%%\end{center}


%%% Local Variables:
%%% TeX-master: "METIS_DRLD"
%%% End:

%% 03-Instrument_Data_Description.tex

\clearpage
\section{Instrument Data Description}
\label{sec:instrument_data_description}

METIS data uses the FITS format for both raw and product data
files. Raw frames are the unprocessed output of METIS instrument
observations, while product frames are the result of pipeline
processing.

All data files can be classified on the basis of sets of keywords
stored in the FITS headers. Association of raw frames to calibration
files is achieved by comparing keyword values.

\TODO{The following is taken verbatim from the PDR document,
  \cite{DRLS}. Updates are currently being discussed in the interface
  control document between ICS and PIP and will be ported here once
  finished. In particular the format for the N-band data from the new
  GeoSnap detector needs to be updated.}

\subsection{Raw data format}
\label{ssec:instrument_data_format}

All raw data files produced by METIS will be in FITS format and follow
ESO standards (\REQ{METIS-6081}, \REQ{METIS-6093}), in particular as
laid out in \cite{ESO-DICD}.

Data from the different subsystems (LM imager/spectrograph, NQ
imager/spectrograph, LM IFU) are always stored in separate files.
When more than one subsystem is recording data simultaneously, then
more than one FITS file is created by one observing template.

The LM and NQ imagers are single-chip sub-instruments, hence the
imaging data will appear in the primary HDU of the FITS file. The LMS
integral-field spectrograph consists of four chips, hence the FITS
file will consist of an empty primary data unit, whose header holds
information pertaining to the exposure as a whole, and four extension
units that hold the data from the four chips along with information
pertaining to each chip (such as world-coordinate system).

A template may produce several exposures, each of which consists of
NDIT subexposures (DITs). The exact definition of what constitutes an
``exposure'' or which and how many DITs will be save in a FITS file,
are still under discussion. The goal will be for a file to contain the
``smallest calibratable subset'' of DITs, for instance one cycle of
the chopping patters, or data taken in one nod position. Keeping files
small in this way allows the pipeline to process data even for cases
where observing blocks are only partially executed.

Where applicable, all the DITs belonging to an exposure may be saved
in a 3D-cube format. This is described in more detail for the various
observing modes in the following sections.

The raw data should include World Coordinate Systems based on the
telescope pointing, derotator angle and the design characteristics of
the telescope and instrument. Instrument-internal coordinates
(e.g.~angles of movable components inside METIS) will follow the
definition in \cite{METIS-coordinates} (\REQ{METIS-6070}).

The list of these FITS header keywords used by METIS is kept in
\cite{METIS-DID}.



%%%%%%%%%%%%%%%%%%%%%%%%%%%%%%%%%%%%%%%%%%%%%%%%%%%%%%%%%%%%%%%%%%%%%%%%%%%%%%%%

\subsection{LM-band Imager and Spectrograph}
\label{ssec:instrument_data_LM-IMG}

The LM-band imager is equipped with a $2\mathrm{k}\times2\mathrm{k}$
HAWAII2RG detector with a pixel scale of $5.47\,\mathrm{mas}$. A
number of rows and columns along the edges of the detector will be
masked (see Sect.~\ref{Sec:detector_masks}).

Grisms in the imager pupil wheel in conjunction with a number of slit
masks in the CFO focal plane provide low-resolution spectroscopy
covering the entire L~and M~bands, respectively, with $R\ga 1400$.

The basic templates for LM-band imaging are
(\cite{METIS-operational_concept}, \cite{METIS-template_manual}):
\begin{itemize}
\item \lstinline{METIS_img_lm_obs_AutoJitter}
\item \lstinline{METIS_img_lm_obs_GenericOffset}
\item \lstinline{METIS_img_lm_obs_FixedSkyOffset}
\end{itemize}

Each of these templates moves the target position between exposures,
either using the internal (CFO-PP2) chopper or telescope offsets, and
takes an exposure at each position, consisting of subintegrations
defined by DIT and NDIT. Depending on the setting of a Boolean
template parameter (\FITS{DET CUBE MODE}), all DITs are stored as
layers of 3D cube, or a co-added frame is stored as a 2D image. The 3D
cube or 2D image are written out in the primary HDU of the FITS file.

The header of the FITS file will contain all information that pertains
to the exposure. This should include information about the position of
the internal chopper (\FITS{SEQ.CFO.CHOP.POSANG},
\FITS{SEQ.CFO.CHOP.THROW}), the telescope offsets (\FITS{SEQ OFFSET1},
\FITS{SEQ OFFSET2}) and the characterisation of the observation as
\FITS{OBJECT} or \FITS{SKY} (in \FITS{DPR.TYPE}).

Similar considerations apply to the spectroscopic and coronagraphic
templates, where pointing offsets are however restricted by the slit
or coronagraphic masks.

%%
%%\begin{center}
%%\begin{table}
%%  \caption[LM\_IMG data keywords]{Columns foreseen for the LM
%%    imager/spectrograph FITS files.}
%%\label{tab:lm_img_colums}
%%\begin{tabular}{|l|l|p{10cm}|}
%%  \hline
%%  \textbf{Column} & \textbf{Format}      & \textbf{Description}                                       \\
%%  \hline\hline
%%  TIME            & long integer         & Time since start of this observation in milliseconds (TBD) \\
%%  \hline
%%  OBSTYPE         & OBJ or SKY           & Exposure on target or on sky?                              \\
%%  \hline
%%  NODDIST         & float (arcsec)       & Nodding distance from reference position                   \\
%%  \hline
%%  CHOPPOS         & $0\dots 360$         & Chopper position angle                                     \\
%%  \hline
%%  CHOPTHROW       & float (mas)          & Chopper distance from reference position                   \\
%%  \hline
%%  CHOPPER         & string               & Flag specifying the position of the internal chopper       \\
%%  \hline
%%  IMAGE           & 2k$\times$2k integer & Raw detector image                                         \\
%%  \hline
%%\end{tabular}
%%\end{table}
%%\end{center}
%%%%%%%%%%%%%%%%%%%%%%%%%%%%%%%%%%%%%%%%%%%%%%%%%%%%%%%%%%%%%%%%%%%%%%%%%%%%%%%%

\subsection{NQ-band Imager and Spectrograph}
\label{ssec:instrument_data_NQ-IMG}

The NQ-band imager is equipped with a $1\mathrm{k}\times 1\mathrm{k}$
AQUARIUS detector with a pixel scale of
$11.3\,\mathrm{mas}$.\footnote{The AQUARIUS may be replaced with a
  $2\mathrm{k}\times2\mathrm{k}$ GEOSNAP detector, which yields the
  same pixel scale when using $2\times2$ binned pixels. With the
  GEOSNAP detector, no Q-band observations will be possible.} A number
of rows and columns along the edges of the detector will be masked
(see Sect.\ \ref{Sec:detector_masks}).

A grism in the imager pupil wheel in conjunction with a number of slit
masks in the CFO focal plane provide low-resolution spectroscopy in
the N~band with $R\ga 400$.

Subtraction of the thermal background in the N~and Q~bands will be
done using fast chopping ($\sim 10\,\mathrm{Hz}$) using the internal
chopper and regular nodding using telescope offsets. The basic
templates for NQ~imaging are
\begin{itemize}
\item \lstinline{METIS_img_nq_obs_AutoChopNod}
\item \lstinline{METIS_img_nq_obs_GenericChopNod}
\end{itemize}

At each nod position, a series of exposures are taken at alternating
chop positions. Depending on the setting of the template parameter
\FITS{DET CUBE MODE}, all exposures are stored as layers of a 3D cube
or they are precombined in a 2D image. Another chop cycle is then
taken at another nod position, and so on. Whether separate nod
positions are stored as individual FITS files or as extension units of
a single FITS file remains to be determined. As explained in
Sect.~\ref{ssec:instrument_data_format}, there is a preference for
small FITS files containing minimal calibratable subsets of DITs.

The primary header of the FITS file will contain all information that
pertains to the chop cycle or the sequence of chop/nod cycle. The
headers of the individual files or extensions will allow unique
determination of the chop/nod positions.

%%\begin{center}
%%\begin{table}
%%\caption[NQ\_IMG data keywords]{Columns foreseen for the NQ imager/spectrograph FITS files.}
%%\label{tab:nq_img_colums}
%%\begin{tabular}{|l|l|p{10cm}|}
%%  \hline
%%  \textbf{Column} & \textbf{Format}      & \textbf{Description}                                       \\
%%  \hline\hline
%%  TIME            & long integer         & Time since start of this observation in milliseconds (TBD) \\
%%  \hline
%%  OBSTYPE         & OBJ or SKY           & Exposure on target or on sky?                              \\
%%  \hline
%%  NODDIST         & float (arcsec)       & Nodding distance from reference position                   \\
%%  \hline
%%  CHOPPOS         & $0\dots 360$         & Chopper position angle                                     \\
%%  \hline
%%  CHOPTHROW       & float (mas)          & Chopper distance from reference position                   \\
%%  \hline
%%  CHOPPER         & string               & Flag specifying the position of the internal chopper       \\
%%  \hline
%%  IMAGE           & 1k$\times$1k integer & Raw detector image                                         \\
%%  \hline
%%\end{tabular}
%%\end{table}
%%\end{center}

%%%%%%%%%%%%%%%%%%%%%%%%%%%%%%%%%%%%%%%%%%%%%%%%%%%%%%%%%%%%%%%%%%%%%%%%%%%%%%%%

\subsection{LM-band Integral Field Unit}
\label{ssec:instrument_data_LMS}

METIS will contain an image slicing integral-field unit (IFU) for
high-resolution spectroscopy in L- and M-band ($R\approx
100,000$). This will be referred to as ``LMS'' (LM spectrograph).

The LMS is equipped with a $2\times2$ mosaic of HAWAII2RG detectors
separated by small gaps. A number of rows and columns along the edges
of the detector will be masked (see Sect.\ \ref{Sec:detector_masks}).

In the nominal mode, the field will be resolved into 28 spatial slices
covering a field of view of $1.00\times0.58\,\mathrm{arcsec^{2}}$.
The slit width (i.e.\ the across-slice pixel scale) is
$20.7\,\mathrm{mas}$; the along-slice pixel scale is
$8.2\,\mathrm{mas}$. Spectrally, the slices span a short wavelength
range of width $37\,\mathrm{nm}$ and $70\,\mathrm{nm}$, depending on
the central wavelength setting, which is selected by rotation of the
echelle grating.

The extended mode provides a larger wavelength range coverage at the
expense of a reduced field of view. The dispersed two-dimensional
image from the pre-disperser is spectrally sliced, resulting in six
non-overlapping wavelength subranges (selected by the echelle angle)
onto the detector, each with three spatial slices.

The layout of spectra on the detector array in the nominal and
extended modes is shown in the left and right panels of
Fig.~\ref{fig:IFU_detector_layout}, respectively.

\begin{figure}[ht]
  \centering
  \resizebox{0.495\textwidth}{!}{%
    \includegraphics{LMS_detector_layout_normal_tikz}}\hfill
  \resizebox{0.495\textwidth}{!}{%
    \includegraphics{LMS_detector_layout_extended_tikz}}
  \caption[LMS detector layout]{Layout of the LMS/IFU detector mosaic
    in the nominal (left) and extended (right) modes. The dispersion
    direction is horizontal. In the nominal mode, all 28 slices cover
    the same wavelength range. In the extended mode, slices marked by
    the same colour cover the same wavelength ranges, while the
    wavelength ranges of the six groups of slices each are not
    contiguous. The position and curvature of the slices are
    indicative only. }
  \label{fig:IFU_detector_layout}
\end{figure}

The basic templates for LM integral-field spectroscopy are:
\begin{itemize}
\item \lstinline{METIS_ifu_obs_FixedSkyOffset}
\item \lstinline{METIS_ifu_obs_GenericOffset}
\end{itemize}
Both templates move the target position between exposures, using the
internal chopper or the telescope. At a given position a number of
exposures with DIT/NDIT are taken. Depending on the setting of the
template parameter \FITS{DET CUBE MODE} all the DITs may be stored as
layers of a 3D cube (burst mode) or co-added into a 2D image.

As the LMS contains an array of four detectors, there will be four
image or cube extensions in the FITS file for one exposure.

The primary header of the FITS file will contain all information that
pertains to the exposure. This will include information on the
position of the internal chopper, the telescope offsets, and the type
of observation (\FITS{DPR.TYPE=OBJECT} or \FITS{SKY}).

A typical sequence of LMS exposures will go through a series of
dispersion grating settings in order to fill gaps in the instantaneous
wavelength coverage (``spectral dithering''). Finally, the field will be
rotated by 90~degrees and the observing sequence repeated to permit
full image reconstruction
(cf.~Sect.~\ref{ssec:image_reconstruction}). The FITS header
information will ensure that the position of each exposure within this
complex sequence can be uniquely identified.

%%\begin{center}
%%\begin{table}
%%\caption[LMS data keywords]{Columns foreseen for the LMS FITS files.}
%%\label{tab:lms_colums}
%%\begin{tabular}{|l|l|p{10cm}|}
%%  \hline
%%  \textbf{Column} & \textbf{Format}      & \textbf{Description}                                       \\
%%  \hline
%%  TIME            & long integer         & Time since start of this observation in milliseconds (TBD) \\
%%  \hline
%%  OBSTYPE         & OBJ or SKY           & Exposure on target or on sky?                              \\
%%  \hline
%%  NODDIST         & float (arcsec)       & Nodding distance from reference position                   \\
%%  \hline
%%  CHOPPOS         & $0\dots 360$         & Chopper position angle                                     \\
%%  \hline
%%  CHOPTHROW       & float (mas)          & Chopper distance from reference position                   \\
%%  \hline
%%  CHOPPER         & string               & Flag specifying the position of the internal chopper       \\
%%  \hline
%%  IMAGE1          & 2k$\times$2k integer & raw image from detector 1                                  \\
%%  IMAGE2          & 2k$\times$2k integer & raw image from detector 2                                  \\
%%  IMAGE3          & 2k$\times$2k integer & raw image from detector 3                                  \\
%%  IMAGE4          & 2k$\times$2k integer & raw image from detector 4                                  \\
%%  \hline
%%\end{tabular}
%%\end{table}
%%\end{center}


\subsection{File classification keywords}
\label{ssec:file_classification_keywords}

\TODO{This section will contain the list of files with the values of
  kewords \FITS{DPR.CATG}, \FITS{DPR.TYPE}, \FITS{DPR.TECH}, \FITS{DO
    Category} in accordance with \cite{ESO-DICD}.}

%%%%%%%%%%%%%%%%%%%%%%%%%%%%%%%%%%%%%%%%%%%%%%%%%%%%%%%%%%%%%%%%%%%%%%%%%%%%%%%%
\subsection{Reduced data format}
\label{ssec:reduced_data_format}

All reduced data will be provided in FITS format compliant with
\cite{ESO-products_standard}.

% THE END
%%%%%%%%%%%%%%%%%%%%%%%%%%%%%%%%%%%%%%%%%%%%%%%%%%%%%%%%%%%%%%%%%%%%%%%%%%%%%%%%

%%% Local Variables:
%%% TeX-master: "METIS_DRLD"
%%% End:


% General part on pipeline and DFS
\section{Data Processing Overview}
\label{sec:data_processing_overview}

The METIS data reduction system runs in different environments and
serves various purposes.  According to the setting, the following
pipeline levels are distinguished \cite{1618}:

\begin{description}
\item[Quality Control Level 0 (QC0):] The QC0 pipeline runs
  automatically in real time on a dedicated pipeline workstation in
  the instrument control room at the observatory. Its purpose is to
  analyse every FITS file created by the instrument and produce
  quality control parameters that allow assessment of whether the
  observation and instrument performance were within specifications.
  The appropriate reduction recipe is triggered either when a single
  FITS file is delivered to the workstation or when a template is
  finished. The files are classified based on header keywords, grouped
  and associated to the necessary standard calibration files.

\item[Quality Control Level 1 (QC1):] The goal of the QC1 pipeline is
  to produce certified calibration products from calibration
  observations as well as to produce QC parameters that are used to
  check the quality of observations and to monitor observing
  conditions and instrument health. Calibration products and QC
  parameters are ingested into the ESO Science Archive.

\item[Quality Control Level 2 (QC2):] The QC2 pipeline produces
  Science Data Products compliant with \cite{ESO-products_standard} as
  well as QC parameters derived from science exposures. It runs
  offline in an automatic way and uses the best calibration products
  for the night of observation (produced by the QC1 pipeline). Science
  data products and QC parameters are ingested into the ESO Science
  Archive.

\item[Science-Grade Desktop Environment:] The pipeline recipes are
  delivered to the astronomical community to enable users to reduce
  data in an optimal and interactive way. Recipes can be run from the
  command line using the \lstinline{esorex} front-end or in the
  context of a \lstinline{Reflex} workflow. While the desktop recipes
  are identical to those used in the QC2 pipeline, the user can change
  recipe parameters to optimise the reduction. Within the
  \lstinline{Reflex} environment, interactive tools are provided that
  allow the user to assess the quality of individual reduction steps
  and to repeat them with different parameters. The products of this
  pipeline are compliant with \cite{ESO-products_standard}.

\end{description}


The following sections describe the recipes used in the QC1, QC2 and
desktop pipelines. Recipes used in the QC0 environment may need to be
streamlined to allow them to run in real time.


\subsection{Required calibrations}
\label{ssec:calibrations}

Table~\ref{tab:calibrations_per_mode} (taken from
\cite{METIS-calibration_plan}) lists the main calibration steps that
are required for each instrument mode.

\TODO{Do we apply NCPA + PSF to HCI data? For ADI a simple recipe is foreseen.}

\begin{table}
  \newcommand{\yes}{\tikz\fill[scale=0.35,color=green!50!black](0,.35) -- (.25,0) -- (0.9,.7) -- (.25,.15) -- cycle;}
  \newcommand{\no}{\textcolor{red!50!black}{---}}
  \caption{Overview of required calibrations per instrument mode. The IFU modes refer to both the nominal configuration and to the extended wavelength configuration. From \cite{METIS-calibration_plan}.}
  \label{tab:calibrations_per_mode}
  \centering\scriptsize
  \begin{tabularx}{\textwidth}{lcccXccccc}
    \hline
                           & Dark & Flat & Wave & Background subtraction & Telluric & Flux & Distortion & NCPA + PSF & RSRF \\
    \hline\hline
    \CODE{IMG_LM}          & \yes & \yes & \no  & Dither         & \no      & \yes & \yes       & \no        & \no  \\
    \CODE{IMG_LM_(RA/C)VC} & \yes & \yes & \no  & ADI            & \no      & \yes & \yes       & \yes       & \no  \\
    \CODE{IMG_LM_CLC}      & \yes & \yes & \no  & ADI            & \no      & \yes & \yes       & \yes       & \no  \\
    \CODE{IMG_LM_APP}      & \yes & \yes & \no  & Dither + ADI   & \no      & \yes & \yes       & \yes       & \no  \\
    \CODE{SPEC_LM}         & \yes & \no  & \yes & Dither along slit & \yes  & \yes & \yes       & \no        & \yes \\
    \CODE{IFU}             & \yes & \no  & \yes & Dither         & \yes     & \yes & \yes       & \no        & \yes \\
    \CODE{IFU_APP}         & \yes & \no  & \yes & Dither + ADI   & \yes     & \yes & \yes       & \yes       & \yes \\
    \CODE{IFU_(RA/C)VC}    & \yes & \no  & \yes & ADI            & \yes     & \yes & \yes       & \yes       & \yes \\
    \CODE{IFU_CLD}         & \yes & \no  & \yes & Dither + ADI   & \yes     & \yes & \yes       & \yes       & \yes \\
    \hline
    \CODE{IMG_N}           & \no  & \yes & \no  & chop/nod       & \no      & \yes & \yes       & \no        & \no  \\
    \CODE{IMG_N_CVC}       & \no  & \yes & \no  & three-point chopping & \no & \yes & \no       & \yes       & \no  \\
    \CODE{IMG_N_CLC}       & \no  & \yes & \no  & out-of-field chopping & \no & \yes & \no      & \yes       & \no  \\
    \CODE{SPEC_N_LOW}      & \no  & \no  & \yes & chop/nod along slit & \yes & \yes & \yes      & \no        & \yes \\
    \hline
  \end{tabularx}
\end{table}

\FloatBarrier

%%%
\subsection{Imaging in LM and N}
\label{ssec:overview_lm_imaging}

\textbf{Note: The pipeline layout has been modified compared to the
  PDR design in order to achieve better modularity. Basic reduction
  and background subtraction have been split into two recipes that now
  are applied to both standard calibration and science data. ADI recipes have been added since PDR however integration of \ac{HCI}
  into this workflow requires more work: \ac{HCI} images will be treated
  the same way at least through basic reduction, possibly through
  background subtraction. \ac{ADI} combination may require a separate
  recipe, at least for some \ac{HCI} configurations.}

The purpose of the pipeline is to correct or remove contributions from
the instrument, telescope, and atmosphere and produce science-grade
data products.  In the case of the METIS imaging modes the main
contributions to correct or remove are dark current, flatfield, bad
pixels, and, most importantly, thermal background emission from the
sky and the telescope. Further effects include persistence,
cross-talk, geometric distortions, etc. The final product of the
imaging pipeline is one or more image(s) that are flux-calibrated in units of
photons/s/pixel against a standard star.
Several images can be stacked into a single possibly mosaiced image.

Due to the differences in characteristics between the HAWAII2RG
detector used for imaging in the L and M bands and the GeoSnap
detector used for the N band, the operational concept for the two
imager subsystems are quite different. This induces differences in the
way the data have to be reduced.

The GeoSnap detector has more stable gain than AQUARIUS detector,
which was still in the baseline at PDR.  Chopping is still necessary,
albeit at a lower frequency of a few Hz, and the standard chop/nod
technique will be employed for background subtraction.  As the dark
signal is automatically removed when the exposures from the different
chop and nod positions are combined no master dark is required for the
reduction of science data. Flat fielding may be possible, pending
further investigation of the detector stability

Observations and reduction of LM band data with the HAWAII2RG detector
can proceed as in the near infrared. After dark subtraction and
flat-fielding, the background is estimated from a series of dithered
science exposures or from exposures on a nearby blank patch of sky.

The association maps for the current designs of the imaging pipelines
in~LM and~N are shown in Figs.~\ref{fig:IMG_LM_Assomap}
and~\ref{fig:IMG_N_Assomap}, respectively.

%\TODO{For \ac{HCI} data, \ac{ADI} may need to be part of reduction recipe if
%  individual background subtracted images are the goal?} We provide ADI recipes since PDR.

\begin{landscape}
  \begin{figure}
    \centering
    \includegraphics{IMG_LM_assomap_tikz}
    \caption[Reduction cascade and association map for imaing in L and
      M]{Association map for imaging in the LM band. The figure shows only
      the primary product created from each recipe; for a full list of
      products refer to the recipe descriptions in
      Sect.~\ref{ssec:recipes_img_lm}. The dashed line separates
      calibration tasks that are done at AIT or infrequently during
      operations (left) from daily tasks (right). The prefix ``\REC{metis_}'' has been
      omitted from the recipe names to improve clarity. The product
      names omit ``\FITS{LM_}''.}
    \label{fig:IMG_LM_Assomap}
  \end{figure}
\end{landscape}

\begin{landscape}
\begin{figure}
  \centering
    \includegraphics{IMG_N_assomap_tikz}
    \caption[Reduction cascade and association map for imaing in N]{%
      Association map for imaging in the N band. The figure shows
      only the primary product created from each recipe; for a full
      list of products refer to the recipe descriptions in
      Sect.~\ref{ssec:recipes_img_n}. The dashed line separates
      calibration tasks that are done at AIT or infrequently during
      operations (left) from daily tasks (right). The prefix ``\REC{metis_}'' has
      been omitted from the recipe names to improve clarity. The
      product names omit ``\PROD{N_}''.}
    \label{fig:IMG_N_Assomap}
  \end{figure}
\end{landscape}

%%%%%%%%%%%%%%%%%%%%%%%%%%%%%%%%%%%%%%%%%%%%%%%%%%%%%%
%%% Local Variables:
%%% TeX-master: "METIS_DRLD"
%%% End:


%%%
\subsection{Long-Slit Spectroscopy in L/M- and N-bands}

The purpose of the pipeline is to correct or remove contributions from
the instrument, telescope, and atmosphere and produce science-grade
data products for the L/M- and N-band long-slit spectroscopy
mode. Since the type of the detectors are not yet defined, we assume
the same reduction cascade for both spectral ranges LM and
N. However, to keep flexibility and independence of both branches, we
define different recipes for the time being, although they will be
mostly based on the same algorithms.

Figures~\ref{Fig:LMLssAssomap} and \ref{Fig:NQLssAssomap} show the reduction cascade and the
association map for the recipes handling L/M- and N-band long-slit
spectroscopy data.  Table~\ref{Tab:LssDatProc} contains the data processing table for these
modes. For the time being it is not clear whether a geometric
distortion correction will be needed. We therefore consider to investigate
the geometry of atmospheric lines for this purpose. These lines also will
serve as reference frame for the wavelength calibration.

\begin{sidewaysfigure}[ht]
  \centering
  \includegraphics[width=0.9\textheight]{figures/LM_LSS_pipeline_wf_draft_latest_v0.62.pdf}
  \caption[Reduction cascade and association map for LM long-slit
  spectroscopy]{Reduction cascade and association map for long-slit
    spectroscopy in the LM bands.  }
  \label{Fig:LMLssAssomap}
\end{sidewaysfigure}

% \begin{sidewaysfigure}[ht]
%   \centering
%   \includegraphics[width=0.9\textheight]{figures/NQ_LSS_pipeline_wf_draft_latest.png}
%   \caption[Reduction cascade and association map for N long-slit
%   spectroscopy]{Reduction cascade and association map for long-slit
%     spectroscopy in the N band.  }
%   \label{Fig:NQLssAssomap}
% \end{sidewaysfigure}


%% ---- Table: LM long-slit spectroscopy
\begin{sidewaystable}
  \footnotesize
  \begin{center}
    \caption[Data Processing table for LM/N long-slit spectroscopy]{%
      Data Processing table for LM/N long-slit spectroscopy
      calibration modes}\bigskip
    \label{Tab:LssDatProc}
    \begin{tabular}{|l|l|l|l|l|l|}
      \hline
      Data Type   & Classification & Recipe (Level)	& FITS Keywords & CalibDB & Products\\
    (Templates) & Keywords	 & Processing steps	&		&	  &	\\
    \hline
    \TPL{DARK}	& \CODE{DPR.CATG==CALIB} & \REC{metis_det_dark} & Exposure time	&	& Averaged dark frame\\
    		& \CODE{DPR.TYPE==DARK}  &			&		&	& Bad pixel map\\
    		& \CODE{DPR.TECH==IMAGE}  &			&		&	& \\
    \hline
    \TPL{FLAT}	& \CODE{DPR.CATG==CALIB} & \REC{metis_LM_lss_rsrf} & Exposure time	& dark	& Averaged, normalized flatfield\\
    		& \CODE{DPR.TYPE==FLAT}  &			&		&	& Bad pixel map\\
    		& \CODE{DPR.TECH==SPECTRUM}  &			&		&	& \\
    \hline
    \TPL{SCIENCE} & \CODE{DPR.CATG==SCIENCE} & \REC{metis_LM_lss_wave} & Object name & 	 & Science grade spectrum\\
    		& \CODE{DPR.TYPE==LSS}   &			   & Exposure time & &\\
    		& \CODE{DPR.TECH==SPECTRUM}  &			&		&	& \\
    		& \CODE{PRO.CATG==SPECTRUM}   &  &  & & \\
    \hline
    \TPL{SCIENCE} & \CODE{DPR.CATG==SCIENCE} & \REC{metis_LM_lss_sci} & Object name & 	 & Science grade spectrum\\
    		& \CODE{DPR.TYPE==LSS}   &			   & Exposure time & &\\
    		& \CODE{DPR.TECH==SPECTRUM}  &			&		&	& \\
    		& \CODE{PRO.CATG==SPECTRUM}   &  &  & & \\
    \hline
    \TPL{SCIENCE} & \CODE{DPR.CATG==SCIENCE} & \REC{metis_LM_lss_flux} & Object name & 	 & Science grade spectrum\\
    		& \CODE{DPR.TYPE==LSS}   &			   & Exposure time & &\\
    		& \CODE{DPR.TECH==SPECTRUM}  &			&		&	& \\
    		& \CODE{PRO.CATG==SPECTRUM}   &  &  & & \\
    \hline
    \TPL{SCIENCE} & \CODE{DPR.CATG==SCIENCE} & \REC{metis_LM_lss_tac} & Object name & 	 & Science grade telluric\\
    		& \CODE{DPR.TYPE==LSS}   &			   & Transmission curve & &Absorption corrected spectrum\\
    		& \CODE{DPR.TECH==SPECTRUM}  &			&		&	& \\
    		& \CODE{PRO.CATG==SPECTRUM}   &  &  & & \\
    \hline
%     \hline
% %     \TPL{DARK}	& \CODE{DPR.CATG==CALIB} & \REC{metis_det_dark} & Exposure time	&	& Averaged dark frame\\
% %     		& \CODE{DPR.TYPE==DARK}  &			&		&	& \\
% %     		& \CODE{DPR.TECH==IMAGE}  &			&		&	& \\
% %     \hline
%     \TPL{FLAT}	& \CODE{DPR.CATG==CALIB} & \REC{metis_N_lss_rsrf} & Exposure time	& dark	& Averaged, normalized flatfield\\
%     		& \CODE{DPR.TYPE==FLAT}  &			&		&	& Bad pixel map\\
%     		& \CODE{DPR.TECH==SPECTRUM}  &			&		&	& \\
%     \hline
%      \TPL{SCIENCE} & \CODE{DPR.CATG==SCIENCE} & \REC{metis_N_lss_sci} & Object name & 	 & Science grade scpectrum\\
%     		& \CODE{DPR.TYPE==LSS}   &			   & Exposure time & &\\
%     		& \CODE{DPR.TECH==SPECTRUM}  &			&		&	& \\
%     		& \CODE{PRO.CATG==SPECTRUM}   &  &  & & \\
%     \hline
%       \TPL{SCIENCE} & \CODE{DPR.CATG==SCIENCE} & \REC{metis_N_lss_tac} & Object name & 	 & Science grade telluric\\
%     		& \CODE{DPR.TYPE==LSS}   &			   & Transmission curve & &Absorption corrected spectrum\\
%     		& \CODE{DPR.TECH==SPECTRUM}  &			&		&	& \\
%     		& \CODE{PRO.CATG==SPECTRUM}   &  &  & & \\
%     \hline
    \end{tabular}
  \end{center}
\end{sidewaystable}

%%%
\subsection{LM IFU: integral-field spectroscopy}
\label{ssec:overview_ifu}

In general, the workflow is similar to the \ac{LSS} mode,
except the extensive post-processing stage.
The main difference arises from the need to co-add multiple exposures
to achieve the full resolution, since pixel scales are different:
\begin{itemize}
    \item in the along-slice direction, the sampling is sufficient, ie. above the Nyquist rate;
        at $8.2$ mas per pixel;
    \item in the across-slice direction, the sampling is below the Nyquist rate at
        $20.7$ mas per pixel, and dithering/co-adding of multiple exposures is needed.
\end{itemize}

The ratio of these resolutions shows that at least three exposures shifted by one third of the
pixel size are required. The image is then reconstructed on a square pixel grid of
$8.2 \times 8.2 \text{mas}^2$.

The exposures are taken in two perpendicular field rotations,
so that full resolution is obtained naturally in the along-slice direction
and by dithering in the across-slice direction; in the other sequence of three exposures
these directions are swapped.

The association map is shown in Fig.~\ref{Fig:IfuAssomap}.

\newgeometry{bottom=0.1cm, top=0.1cm}
% This geometry makes the tables/figures fit, but messes up the header a bit.
\begin{landscape}
\begin{figure}[ht]
  \centering
  \resizebox{\linewidth}{!}{%%%%%%%%%%%%%%%%%% BEGIN DOCUMENT %%%%%%%%%%%%%%%%%%%%%%%%%%%%%%%%%%%%%%%%
\sffamily

\input{black_style}
\input{recipe_config}

%%% Picture: flow chart
\begin{tikzpicture}[on grid=false, node distance=0.8cm]

  \matrix (recipes) [column sep=1mm, row sep=1cm]{

    % Row *_raw

    \node[above] (REClin_raw){\recipebox{\RAW{DETLIN_IFU_RAW}}{\REC{metis_det_lingain}}}; &
% TODO: Put back in once we actually include the persistence recipe in the DRLD
%    \node[above] (RECpers_raw){\recipebox{\RAW{PERSISTENCE}}{\REC{metis_det_persistence}}}; &
    \node[above] (RECpers_raw)[empty]{}; &
    \node[above] (RECdark_raw){\recipebox{\RAW{DARK_IFU_RAW}}{\REC{metis_det_dark}}}; &
    \node[above] (RECgeom_raw){\recipebox{\RAW{IFU_DISTORTION_RAW}}{\REC{metis_ifu_distortion}}}; &
    \node[above] (RECrsrf_raw){\recipebox{\RAW{IFU_RSRF_RAW}}{\REC{metis_ifu_rsrf}}}; &
    \node[above] (RECwcal_raw){\recipebox{\RAW{IFU_WAVE_RAW}}{\REC{metis_ifu_wavecal}}}; &
    \node[above] (RECstdreduce_raw){\recipebox{\RAW{IFU_STD_RAW}}{\REC{metis_ifu_reduce}}}; &
    \node[above] (RECscireduce_raw){\recipebox{\RAW{IFU_SCI_RAW}}{\REC{metis_ifu_reduce}}}; &
%      \recipebox{\RAW{IFU_STD_RAW}}}{\REC{metis_ifu_std_process}}}
%      \recipenotitlebox{\REC{metis_ifu_std_process}}}
%      so actually this is now just the same as telluric
    \node[above] (RECstd_raw){\recipenotitlebox{\REC{metis_ifu_telluric}}}; &
    \node[above] (RECtac_raw){\recipenotitlebox{\REC{metis_ifu_telluric}}}; &
%      \recipebox{\RAW{IFU_SCI_RAW}}}{\REC{metis_ifu_sci_process}}}
%      \recipenotitlebox{\REC{metis_ifu_sci_process}}}
    \node[above] (RECsci1_raw){\recipenotitlebox{\REC{metis_ifu_calibrate}}}; &
    \node[above] (RECsci2_raw){\recipenotitlebox{\REC{metis_ifu_postprocess}}}; \\
%    &
%    \node[above] (adi_raw){%
%      \recipenotitlebox{\REC{metis_ifu_adi_cgrph}}}
%    };
    \node (REClin_DIlin)[statcalfile]{\STATCALIB{LINEARITY_IFU}}; &
    \node (RECpers_DIlin)[empty]{}; &
    \node (RECdark_DIlin)[empty]{}; &
    \node (RECgeom_DIlin)[empty]{}; &
    \node (RECrsrf_DIlin)[empty]{}; &
    \node (RECwcal_DIlin)[empty]{}; &
    \node (RECstdreduce_DIlin)[connection]{}; &
    \node (RECscireduce_DIlin)[connection]{}; &
    \node (RECtac_DIlin)[empty]{}; &
    \node (RECstd_DIlin)[empty]{}; &
    \node (RECsci1_DIlin)[empty]{}; &
    \node (RECsci2_DIlin)[empty]{}; \\

    \node (REClin_DIpers)[empty]{}; &
    \node (RECpers_DIpers)[extcalfile]{\STATCALIB{PERSISTENCE_MAP}}; &
    \node (RECdark_DIpers)[empty]{}; &
    \node (RECgeom_DIpers)[empty]{}; &
    \node (RECrsrf_DIpers)[empty]{}; &
    \node (RECwcal_DIpers)[empty]{}; &
    \node (RECstdreduce_DIpers)[connection]{}; &
    \node (RECscireduce_DIpers)[connection]{}; &
    \node (RECtac_DIpers)[empty]{}; &
    \node (RECstd_DIpers)[empty]{}; &
    \node (RECsci1_DIpers)[empty]{}; &
    \node (RECsci2_DIpers)[empty]{}; \\

    % Row *_dark
    \node (REClin_DIdark)[empty]{}; &
    \node (RECpers_DIdark)[empty]{}; &
    \node (RECdark_DIdark)[calibproduct]{\PROD{MASTER_DARK_IFU}}; &
    \node (RECgeom_DIdark)[connection]{}; &
    \node (RECrsrf_DIdark)[connection]{}; &
    \node (RECwcal_DIdark)[connection]{}; &
    \node (RECstdreduce_DIdark)[connection]{}; &
    \node (RECscireduce_DIdark)[connection]{}; &
    \node (RECtac_DIdark)[empty]{}; &
    \node (RECstd_DIdark)[empty]{}; &
    \node (RECsci1_DIdark)[empty]{}; &
    \node (RECsci2_DIdark)[empty]{}; \\
%    &
%    \node (adi_DIdark)[empty]{};

    % Row *_geom
    \node (REClin_DIgeom)[empty]{}; &
    \node (RECpers_DIgeom)[empty]{}; &
    \node (RECdark_DIgeom)[empty]{}; &
    \node (RECgeom_DIgeom)[statcalfile]{\PROD{IFU_DISTORTION_TABLE}}; &
    \node (RECrsrf_DIgeom)[empty]{}; &
    \node (RECwcal_DIgeom)[connection]{}; &
    \node (RECstdreduce_DIgeom)[connection]{}; &
    \node (RECscireduce_DIgeom)[connection]{}; &
    \node (RECstd_DIgeom)[empty]{}; &
    \node (RECtac_DIgeom)[empty]{}; &
    \node (RECsci1_DIgeom)[empty]{}; &
    \node (RECsci2_DIgeom)[empty]{}; \\
%    &
%    \node (adi_DIgeom)[empty]{};

% Row *_rsrf
    \node (REClin_DIrsrf)[empty]{}; &
    \node (RECpers_DIrsrf)[empty]{}; &
    \node (RECdark_DIrsrf)[empty]{}; &
    \node (RECgeom_DIrsrf)[empty]{}; &
    \node (RECrsrf_DIrsrf)[statcalfile]{\PROD{RSRF_IFU}}; &
    \node (RECwcal_DIrsrf)[empty]{}; &
    \node (RECstdreduce_DIrsrf)[connection]{}; &
    \node (RECscireduce_DIrsrf)[connection]{}; &
    \node (RECstd_DIrsrf)[empty]{}; &
    \node (RECtac_DIrsrf)[empty]{}; &
    \node (RECsci1_DIrsrf)[empty]{}; &
    \node (RECsci2_DIrsrf)[empty]{}; \\
%    & \node (adi_DIrsrf)[empty]{};

% Row *_wcal
    \node (REClin_DIwcal)[empty]{}; &
    \node (RECpers_DIwcal)[empty]{}; &
    \node (RECdark_DIwcal)[empty]{}; &
    \node (RECgeom_DIwcal)[empty]{}; &
    \node (RECrsrf_DIwcal)[empty]{}; &
    \node (RECwcal_DIwcal) [calibproduct]{\PROD{IFU_WAVECAL}}; &
    \node (RECstdreduce_DIwcal)[connection]{}; &
    \node (RECscireduce_DIwcal)[connection]{}; &
    \node (RECstd_DIwcal)[empty]{}; &
    \node (RECsci1_DIwcal)[empty]{}; &
    \node (RECtac_DIwcal)[empty]{}; &
    \node (RECsci2_DIwcal)[empty]{}; \\
%    \node (adi_DIwcal)[empty]{};


    % Row *_scireduced
    \node (REClin_DIscireduced)[empty]{}; &
    \node (RECpers_DIscireduced)[empty]{}; &
    \node (RECdark_DIstdreduced)[empty]{}; &
    \node (RECgeom_DIscireduced)[empty]{}; &
    \node (RECrsrf_DIscireduced)[empty]{}; &
    \node (RECwcal_DIscireduced)[empty]{}; &
    \node (RECstdreduce_DIscireduced)[empty]{}; &
    \node (RECscireduce_DIscireduced)[scienceproduct]{\EXTCALIB{IFU_SCI_REDUCED}};
    \node (RECscireduce_DIscibackground)[scienceproduct,below=0.5cm]{\EXTCALIB{IFU_SCI_BACKGROUND}};
    \node (RECscireduce_DIscireducedcube)[scienceproduct,below=1.25cm]{\EXTCALIB{IFU_SCI_REDUCED_CUBE}};
    \node (RECscireduce_DIscicombined)[scienceproduct,below=2.0cm]{\EXTCALIB{IFU_SCI_COMBINED}};
    \draw [-] (RECscireduce_DIscireduced) -- (RECscireduce_DIscibackground) -- (RECscireduce_DIscireducedcube) -- (RECscireduce_DIscicombined); &
    \node (RECstd_DIscireduced)[empty]{}; &
    \node (RECtac_DIscireduced)[empty]{}; &
    \node (RECsci1_DIscireduced)[empty]{}; &
    \node (RECsci2_DIscireduced)[empty]{}; \\
%    \node (adi_DIscireduced)[empty]{};

    % Row *_stdreduced
    \node (REClin_DIstdreduced)[empty]{}; &
    \node (RECpers_DIstdreduced)[empty]{}; &
    \node (RECdark_DIstdreduced)[empty]{}; &
    \node (RECgeom_DIstdreduced)[empty]{}; &
    \node (RECrsrf_DIstdreduced)[empty]{}; &
    \node (RECwcal_DIstdreduced)[empty]{}; &
    \node (RECstdreduce_DIstdreduced)[scienceproduct]{\PROD{IFU_STD_REDUCED}};
    \node (RECstdreduce_DIstdbackground)[scienceproduct,below=0.5cm]{\PROD{IFU_STD_BACKGROUND}};
    \node (RECstdreduce_DIstdreducedcube)[scienceproduct,below=1.25cm]{\EXTCALIB{IFU_STD_REDUCED_CUBE}};
    \node (RECstdreduce_DIstdcombined)[scienceproduct,below=2.cm]{\EXTCALIB{IFU_STD_COMBINED}};
    \draw [-] (RECstdreduce_DIstdreduced) -- (RECstdreduce_DIstdbackground) -- (RECstdreduce_DIstdreducedcube) -- (RECstdreduce_DIstdcombined); &
    \node (RECscireduce_DIstdreduced)[empty]{}; &
    \node (RECstd_DIstdreduced)[empty]{}; &
    \node (RECtac_DIstdreduced)[empty]{}; &
    \node (RECsci1_DIstdreduced)[empty]{}; &
    \node (RECsci2_DIstdreduced)[empty]{}; \\
    %    &
%    \node (adi_DIstdreduced)[empty]{};

    % Row *_basicstd
    \node (REClin_DIfluxstd)[empty]{}; &
    \node (RECpers_DIfluxstd)[extcalfile]{\EXTCALIB{FLUXSTD_CATALOG}}; &
    \node (RECdark_DIfluxstd)[empty]{}; &
    \node (RECgeom_DIfluxstd)[empty]{}; &
    \node (RECrsrf_DIfluxstd)[empty]{}; &
    \node (RECwcal_DIfluxstd)[empty]{}; &
    \node (RECstdreduce_DIfluxstd)[empty]{}; &
    \node (RECscireduce_DIfluxstd)[empty]{}; &
    \node (RECstd_DIfluxstd)[connection]{}; &
    \node (RECtac_DIfluxstd)[empty]{}; &
    \node (RECsci1_DIfluxstd)[empty]{}; &
    \node (RECsci2_DIfluxstd)[empty]{}; \\
%    &
%    \node (adi_DIfluxstd)[empty]{};

    % Row *_tac
    \node (REClin_DItac)[empty]{}; &
    \node (RECpers_DItac)[empty]{}; &
    \node (RECdark_DItac)[empty]{}; &
    \node (RECgeom_DItac)[empty]{}; &
    \node (RECrsrf_DItac)[empty]{}; &
    \node (RECwcal_DItac)[empty]{}; &
    \node (RECstdreduce_DItac)[empty]{}; &
    \node (RECscireduce_DItac)[empty]{}; &
    \node (RECstd_DItac)[empty]{}; &
    \node (RECtac_DItac)[calibproduct]{\PROD{IFU_TELLURIC}}; &
    \node (RECsci1_DItac)[connection]{}; &
    \node (RECsci2_DItac)[empty]{}; \\
%    &
%    \node (adi_DItac)[empty]{};

    % Row *_fcal
    \node (REClin_DIfcal)[empty]{}; &
    \node (RECpers_DIfcal)[empty]{}; &
    \node (RECdark_DIfcal)[empty]{}; &
    \node (RECgeom_DIfcal)[empty]{}; &
    \node (RECrsrf_DIfcal)[empty]{}; &
    \node (RECwcal_DIfcal)[empty]{}; &
    \node (RECstdreduce_DIfcal)[empty]{}; &
    \node (RECscireduce_DIfcal)[empty]{}; &
    \node (RECstd_DIfcal)[calibproduct]{\PROD{FLUXCAL_TAB}};
    \node (RECstd_DItelluricstd)[calibproduct,below=.5cm]{\PROD{IFU_TELLURIC}};
    \draw [-] (RECstd_DIfcal) -- (RECstd_DItelluricstd); &
    \node (RECtac_DIfcal)[empty]{}; &
    \node (RECsci1_DIfcal)[connection]{};
    % No clue how to get this .70 nicely done
    \node (RECsci1_DItelluricstd)[connection,below=.70cm]{}; &
    \node (RECsci2_DIfcal)[empty]{}; \\
%    &
%    \node (adi_DIfcal)[empty]{};

    % Row *_sci1
    \node (REClin_DIsci1)[empty]{}; &
    \node (RECpers_DIsci1)[empty]{}; &
    \node (RECdark_DIsci1)[empty]{}; &
    \node (RECgeom_DIsci1)[empty]{}; &
    \node (RECrsrf_DIsci1)[empty]{}; &
    \node (RECwcal_DIsci1)[empty]{}; &
    \node (RECstdreduce_DIsci1)[empty]{}; &
    \node (RECscireduce_DIsci1)[empty]{}; &
    \node (RECstd_DIsci1)[empty]{}; &
%    \node (RECtac_DIsci1)[scienceproduct]{\PROD{IFU_SCI_CALIBRATED_TAC}}};
    \node (RECtac_DIsci1)[empty]{}; &
    \node (RECsci1_DIsci1)[scienceproduct]{\PROD{IFU_SCI_CUBE_CALIBRATED}}; &
    \node (RECsci2_DIsci1)[connection]{}; \\
%    &
%    \node (adi_DIsci1)[empty]{};

    % Row *_sci2
    \node (REClin_DIsci2)[empty]{}; &
    \node (RECpers_DIsci2)[empty]{}; &
    \node (RECdark_DIsci2)[empty]{}; &
    \node (RECgeom_DIsci2)[empty]{}; &
    \node (RECrsrf_DIsci2)[empty]{}; &
    \node (RECwcal_DIsci2)[empty]{}; &
    \node (RECstdreduce_DIsci2)[empty]{}; &
    \node (RECscireduce_DIsci2)[empty]{}; &
    \node (RECstd_DIsci2)[empty]{}; &
%    \node (RECtac_DIsci2)[scienceproduct]{\PROD{IFU_SCI_COMBINED_TAC}}};
    \node (RECtac_DIsci2)[empty]{}; &
%    \node (RECsci1_DIsci2)[scienceproduct]{\PROD{IFU_SCI_COMBINED}}};
    \node (RECsci1_DIsci2)[empty]{}; &
    \node (RECsci2_DIsci2) [empty]{}; \\
%    &
%    \node (adi_DIsci2) [connection]{};

    % Row *_adi
    \node (REClin_adi)[empty]{}; &
    \node (RECpers_adi)[empty]{}; &
    \node (RECdark_adi)[empty]{}; &
    \node (RECgeom_adi)[empty]{}; &
    \node (RECrsrf_adi)[empty]{}; &
    \node (RECwcal_adi)[empty]{}; &
    \node (RECstdreduce_adi)[empty]{}; &
    \node (RECscireduce_adi)[empty]{}; &
    \node (RECstd_adi)[empty]{}; &
    \node (RECtac_adi)[empty]{}; &
    \node (RECsci1_adi)[empty]{}; &
    \node (RECsci2_adi)[scienceproduct]{\PROD{IFU_SCI_COADD}}; \\
%    &
%    \node (adi_adi)[scienceproduct]{ADI\_SCI\_COADD};
  };    % end matrix


%  Dashed line separating daily procedure
%  Commented out: right now there is no clear separation
%  \node (t1) at ($(RECrsrf_raw.east)!0.5!(RECdark_raw.west)$){};
%  \node (t2) at ($(RECrsrf_adi.east)!0.5!(RECdark_adi.west)$){} ;
%  \draw [thick,dashed] ([yshift=4ex]t1.north) -- ([yshift=-0ex]t2.south);


  %% Connections
  \draw [arrow] (REClin_raw) -- (REClin_DIlin);
% TODO: Put back in once we actually include the persistence recipe in the DRLD
%  \draw [arrow] (RECpers_raw) -- (RECpers_DIpers);
  \draw [arrow] (RECgeom_raw) -- (RECgeom_DIgeom);
  \draw [arrow] (RECrsrf_raw) -- (RECrsrf_DIrsrf);
  \draw [arrow] (RECwcal_raw) -- (RECwcal_DIwcal);
  \draw [arrow] (RECdark_raw) -- (RECdark_DIdark);
  \draw [arrow] (RECscireduce_raw)  -- (RECscireduce_DIscireduced);
  \draw [arrow] (RECstdreduce_raw)  -- (RECstdreduce_DIstdreduced);

  \draw [match] (REClin_DIlin) --
% TODO: Put back in once we actually include the persistence recipe in the DRLD
%        (RECpers_DIlin) [xshift=-0.15cm] arc [start angle=180, end angle=0, radius=.15cm] --
        (RECdark_DIlin) [xshift=-0.15cm] arc [start angle=180, end angle=0, radius=.15cm] --
        (RECgeom_DIlin) [xshift=-0.15cm] arc [start angle=180, end angle=0, radius=.15cm] --
        (RECrsrf_DIlin) [xshift=-0.15cm] arc [start angle=180, end angle=0, radius=.15cm] --
        (RECwcal_DIlin) [xshift=-0.15cm] arc [start angle=180, end angle=0, radius=.15cm] --
        (RECscireduce_DIlin);

% Persistence
  \draw [match] (RECpers_DIpers) -- 
        (RECdark_DIpers) [xshift=-0.15cm] arc [start angle=180, end angle=0, radius=.15cm] --
        (RECgeom_DIpers) [xshift=-0.15cm] arc [start angle=180, end angle=0, radius=.15cm] --
        (RECrsrf_DIpers) [xshift=-0.15cm] arc [start angle=180, end angle=0, radius=.15cm] --
        (RECwcal_DIpers) [xshift=-0.15cm] arc [start angle=180, end angle=0, radius=.15cm] --
        (RECscireduce_DIpers);
  \draw [match] (RECdark_DIdark) -- (RECscireduce_DIdark);

  \draw [match] (RECrsrf_DIrsrf) -- (RECscireduce_DIrsrf);

  \draw [match] (RECwcal_DIwcal)  -- (RECscireduce_DIwcal);
  \draw [match] (RECgeom_DIgeom) -- (RECrsrf_DIgeom) [xshift=-0.15cm] arc [start angle=180, end angle=0, radius=.15cm] -- (RECscireduce_DIgeom);
  \draw [match] (RECpers_DIfluxstd)   -- (RECstd_DIfluxstd);   % Line from FLUXSTD_CATALOG to further
  \draw [match] (RECstd_DIfcal)  -- (RECsci1_DIfcal);
  \draw [match,dashed] (RECstd_DItelluricstd)  -- (RECsci1_DItelluricstd);
  \draw [match] (RECtac_DItac)  -- (RECsci1_DItac);

% -| means first horizontal, then vertical
  \draw [arrow] (RECstdreduce_DIstdcombined) -| (RECstd_DIfcal);
  \draw [arrow] (RECscireduce_DIscicombined) -| (RECtac_DItac);
  \draw [arrow] (RECscireduce_DIscireduced) -| (RECsci1_DIsci1);
  \draw [arrow] (RECsci1_DIsci1) -| (RECsci2_adi);



  %\draw [very thick,dashed] ($(raw_geometry.north)!0.5!(raw_dark.north)$) -- ++(270:15cm);

  %% Legend
  \matrix (legend) [draw, fill=gray!15, above right, row sep=0.3cm,
    column 1/.style={anchor=base},
    column 2/.style={anchor=base west}]
  at ([yshift=0cm]current bounding box.south west){%
    \node (leg_recipe) [recipe]{ifu\_sci\_process};
    & \node {recipe}; \\
    \node (leg_calproduct) [calibproduct]{MASTER\_DARK};
    & \node{calib.\ product}; \\
    \node (leg_sciproduct)[scienceproduct]{SCI\_REDUCED};
    & \node {science product}; \\
    \node (leg_statcalfile)[statcalfile]{MASTER\_RSRF};
    & \node {static calib.\ file};\\
    \node (leg_calfile)[extcalfile]{FLUXSTD\_CATALOG};
    & \node {external file}; \\

    \draw [arrow,fill=black] (0,0.4) -- (0,-0.3);  %% should be centred relative to column
    & \node {processing step}; \\

    \draw [connection_arrow] (-1, 0.5ex) -- (1,0.5ex) node [connection,yshift=0cm]{};
    & \node {product match}; \\
  };    %% end matrix (legend)

\end{tikzpicture}

\input{normal_style}
}
  \caption[Reduction cascade and association map for IFU spectroscopy]{%
    Association map for \ac{IFU} spectroscopy in L- and M-band. The
    figure shows only the primary products created by each recipe; for
    a full list of products refer to the recipe descriptions in
    Sect.~\ref{ssec:IFU_recipes}. The dashed line separates
    calibration tasks that are done at AIT or infrequently during
    operations from tasks done daily.}
  \label{Fig:IfuAssomap}
\end{figure}
\end{landscape}
\restoregeometry



%%%%%%%%%%%%%%%%%%%%%%%%%%%%%%%%%%%%%%%%%%%%%%%%%%%%%%

%%% Local Variables:
%%% TeX-master: "METIS_DRLD"
%%% End:


%%% Local Variables:
%%% TeX-master: "METIS_DRLD"
%%% End:


% Imaging-mode
% \input{05_1-Overview_Imaging}

% LSS-mode
% \subsection{Long-Slit Spectroscopy in L/M- and N-bands}

The purpose of the pipeline is to correct or remove contributions from
the instrument, telescope, and atmosphere and produce science-grade
data products for the L/M- and N-band long-slit spectroscopy
mode. Since the type of the detectors are not yet defined, we assume
the same reduction cascade for both spectral ranges LM and
N. However, to keep flexibility and independence of both branches, we
define different recipes for the time being, although they will be
mostly based on the same algorithms.

Figures~\ref{Fig:LMLssAssomap} and \ref{Fig:NQLssAssomap} show the reduction cascade and the
association map for the recipes handling L/M- and N-band long-slit
spectroscopy data.  Table~\ref{Tab:LssDatProc} contains the data processing table for these
modes. For the time being it is not clear whether a geometric
distortion correction will be needed. We therefore consider to investigate
the geometry of atmospheric lines for this purpose. These lines also will
serve as reference frame for the wavelength calibration.

\begin{sidewaysfigure}[ht]
  \centering
  \includegraphics[width=0.9\textheight]{figures/LM_LSS_pipeline_wf_draft_latest_v0.62.pdf}
  \caption[Reduction cascade and association map for LM long-slit
  spectroscopy]{Reduction cascade and association map for long-slit
    spectroscopy in the LM bands.  }
  \label{Fig:LMLssAssomap}
\end{sidewaysfigure}

% \begin{sidewaysfigure}[ht]
%   \centering
%   \includegraphics[width=0.9\textheight]{figures/NQ_LSS_pipeline_wf_draft_latest.png}
%   \caption[Reduction cascade and association map for N long-slit
%   spectroscopy]{Reduction cascade and association map for long-slit
%     spectroscopy in the N band.  }
%   \label{Fig:NQLssAssomap}
% \end{sidewaysfigure}


%% ---- Table: LM long-slit spectroscopy
\begin{sidewaystable}
  \footnotesize
  \begin{center}
    \caption[Data Processing table for LM/N long-slit spectroscopy]{%
      Data Processing table for LM/N long-slit spectroscopy
      calibration modes}\bigskip
    \label{Tab:LssDatProc}
    \begin{tabular}{|l|l|l|l|l|l|}
      \hline
      Data Type   & Classification & Recipe (Level)	& FITS Keywords & CalibDB & Products\\
    (Templates) & Keywords	 & Processing steps	&		&	  &	\\
    \hline
    \TPL{DARK}	& \CODE{DPR.CATG==CALIB} & \REC{metis_det_dark} & Exposure time	&	& Averaged dark frame\\
    		& \CODE{DPR.TYPE==DARK}  &			&		&	& Bad pixel map\\
    		& \CODE{DPR.TECH==IMAGE}  &			&		&	& \\
    \hline
    \TPL{FLAT}	& \CODE{DPR.CATG==CALIB} & \REC{metis_LM_lss_rsrf} & Exposure time	& dark	& Averaged, normalized flatfield\\
    		& \CODE{DPR.TYPE==FLAT}  &			&		&	& Bad pixel map\\
    		& \CODE{DPR.TECH==SPECTRUM}  &			&		&	& \\
    \hline
    \TPL{SCIENCE} & \CODE{DPR.CATG==SCIENCE} & \REC{metis_LM_lss_wave} & Object name & 	 & Science grade spectrum\\
    		& \CODE{DPR.TYPE==LSS}   &			   & Exposure time & &\\
    		& \CODE{DPR.TECH==SPECTRUM}  &			&		&	& \\
    		& \CODE{PRO.CATG==SPECTRUM}   &  &  & & \\
    \hline
    \TPL{SCIENCE} & \CODE{DPR.CATG==SCIENCE} & \REC{metis_LM_lss_sci} & Object name & 	 & Science grade spectrum\\
    		& \CODE{DPR.TYPE==LSS}   &			   & Exposure time & &\\
    		& \CODE{DPR.TECH==SPECTRUM}  &			&		&	& \\
    		& \CODE{PRO.CATG==SPECTRUM}   &  &  & & \\
    \hline
    \TPL{SCIENCE} & \CODE{DPR.CATG==SCIENCE} & \REC{metis_LM_lss_flux} & Object name & 	 & Science grade spectrum\\
    		& \CODE{DPR.TYPE==LSS}   &			   & Exposure time & &\\
    		& \CODE{DPR.TECH==SPECTRUM}  &			&		&	& \\
    		& \CODE{PRO.CATG==SPECTRUM}   &  &  & & \\
    \hline
    \TPL{SCIENCE} & \CODE{DPR.CATG==SCIENCE} & \REC{metis_LM_lss_tac} & Object name & 	 & Science grade telluric\\
    		& \CODE{DPR.TYPE==LSS}   &			   & Transmission curve & &Absorption corrected spectrum\\
    		& \CODE{DPR.TECH==SPECTRUM}  &			&		&	& \\
    		& \CODE{PRO.CATG==SPECTRUM}   &  &  & & \\
    \hline
%     \hline
% %     \TPL{DARK}	& \CODE{DPR.CATG==CALIB} & \REC{metis_det_dark} & Exposure time	&	& Averaged dark frame\\
% %     		& \CODE{DPR.TYPE==DARK}  &			&		&	& \\
% %     		& \CODE{DPR.TECH==IMAGE}  &			&		&	& \\
% %     \hline
%     \TPL{FLAT}	& \CODE{DPR.CATG==CALIB} & \REC{metis_N_lss_rsrf} & Exposure time	& dark	& Averaged, normalized flatfield\\
%     		& \CODE{DPR.TYPE==FLAT}  &			&		&	& Bad pixel map\\
%     		& \CODE{DPR.TECH==SPECTRUM}  &			&		&	& \\
%     \hline
%      \TPL{SCIENCE} & \CODE{DPR.CATG==SCIENCE} & \REC{metis_N_lss_sci} & Object name & 	 & Science grade scpectrum\\
%     		& \CODE{DPR.TYPE==LSS}   &			   & Exposure time & &\\
%     		& \CODE{DPR.TECH==SPECTRUM}  &			&		&	& \\
%     		& \CODE{PRO.CATG==SPECTRUM}   &  &  & & \\
%     \hline
%       \TPL{SCIENCE} & \CODE{DPR.CATG==SCIENCE} & \REC{metis_N_lss_tac} & Object name & 	 & Science grade telluric\\
%     		& \CODE{DPR.TYPE==LSS}   &			   & Transmission curve & &Absorption corrected spectrum\\
%     		& \CODE{DPR.TECH==SPECTRUM}  &			&		&	& \\
%     		& \CODE{PRO.CATG==SPECTRUM}   &  &  & & \\
%     \hline
    \end{tabular}
  \end{center}
\end{sidewaystable}  % TO BE EDITED BY INNSBRUCK ONLY!!!!!

\clearpage
% IFU-mode
% \input{05_3-Overview_IFU}

\clearpage
\clearpage

\section{Algorithms / Mathematical description}

This section provides a mathematical description for the advanced processing techniques needed to the data reduction library.
\label{sec:algorithms}
% This section is dedicated to algorithm which are used in all spectroscopic modes

%-----------------------------------------------------------------------------------------
\subsection{Telluric absorption correction}\label{ssec:tellcorr}
Due to the dense molecular absorptions arising from the Earth's atmosphere, nearly every \ac{MIR} regime spectrum requires a correction for these telluric features. The required atmospheric transmission curve can be achieved either by specific observations of a telluric standard star (\ac{TSS}, the "classical" way) or by a modelling approach. \\
%------------------------------------------------------------------------------
\subsubsection{Classic approach with a standard star:}\label{sssec:tecllcorrclassic}
A telluric standard star (\ac{TSS}) spectrum is taken ideally directly before/after the science observations near the science target position (or at least at the same airmass) to probe the same pathway through the Earth's atmosphere. The most simplest approach can be described by
\begin{equation}
    F_\textrm{sci}=F_{0,\textrm{cal}}*\left(S_\textrm{sci}/S_\textrm{cal}\right)*\left(T_\textrm{SkyCalc\_cal} / T_\textrm{SkyCalc\_sci}\right)
\end{equation}
where $S_\textrm{sci}$ and $S_\textrm{cal}$ are the measured fluxes (e.g. in [\ac{ADU}]) of the science target and the standard star and $F_{0,\textrm{cal}}$ is a model of the standard star in physical flux units. Thus, the standard star is used for the flux calibration, i.e. the conversion between \ac{ADU} and physical units to correct for the instrumental throughput. In case there's only a marginal difference in airmass between the science and the calibrator target, the airmass compensation can be achieved with synthetic model spectra of the transmission of the science target $T_\textrm{SkyCalc\_sci}$ and the calibrator $T_\textrm{SkyCalc\_cal}$, e.g. with the tool \texttt{SkyCalc}\footnote{\url{https://www.eso.org/observing/etc/bin/gen/form?INS.MODE=swspectr+INS.NAME=SKYCALC}}. This approach should be sufficient even for the \ac{LSS} mode as the wide wavelength range and the low resolving power does not allow to resolve individual telluric lines anyway. The high-resolution mode (\ac{LMS}) allows a much better determination of the telluric features and is therefore less prone to problems arising with \texttt{molecfit}. For the time being we assume no wavelength-dependent slit-losses e.g. by different peformances of the \ac{AO} in the science and the calibrator observations.  \\
%For the \ac{LMS} mode a more sophisticated approach could be considered: This \ac{TSS}-spectrum is processed in the same way as the science spectrum (except the absolute flux calibration). To remove intrinsic stellar features this spectrum is corrected with a model spectrum of this \ac{TSS}. Finally its continuum is normalised to unity. The resulting normalised spectrum (ideally) only contains the fingerprint of the Earth's atmospheric absorptions and can be used for the telluric correction.\\
There are several sources for model spectra available:
\begin{itemize}
    \item Cohen set (\cite{coh99}): Set of 422 stellar model templates, mainly K and M giants. One of the standard sets in \ac{MIR}.
    \item The SPEX \ac{IRTF} Spectral library\footnote{\url{http://irtfweb.ifa.hawaii.edu/~spex/IRTF_Spectral_Library/}}: Set of observed stellar spectra (F to M-type, some carbon and S-type stars and L and T dwarfs) in the range $0.8...5.0\mu$m mostly at a resolving power of $R\equiv\lambda/\Delta\lambda\sim2,000$.
    \item Phoenix library\footnote{\url{https://phoenix.astro.physik.uni-goettingen.de/}}\cite{phoenix}: Library of synthetic medium and high resolution spectra between $0.5...5.5\mu$m covering a wide stellar parameter space ($2,300\textrm{K}\leq T_\textrm{eff}\leq12,000\textrm{K}$; $0.0\leq\log g\leq+6.0$; $-4.0\leq$[Fe/H]$\leq+1.0$; $0.2\leq$[$\alpha$/Fe]$\leq+1.2$). \\
\end{itemize}
The \ac{METIS} consortium will soon look into that topic and to assemble a set of appropriate \ac{TSS} stars for each observing mode.

%------------------------------------------------------------------------------
\subsubsection{Modelling synthetic transmission spectra:}
In the last years the modelling method has evolved. It is based on radiative transfer modelling of the Earth's atmosphere. A height model of the Earth's atmosphere containing information of pressure, temperature and the concentration of molecules in combination with a radiative transfer model and a molecular line list is used to calculate a transmission function of the Earth's atmosphere at the time of observations. By fitting specific atmospheric absorption features in the science spectra and varying the input height profile allows to determine the state of the Earth's atmosphere at the time of observation. The best-fit transmission function is finally used for the telluric correction.\\
In the past years the approach of modelling transmission curves of Earth's atmosphere has made significant progress leading to versatile and mature software packages for the telluric correction. One of these packages is \texttt{molecfit}\footnote{\url{http://www.eso.org/sci/software/pipelines/skytools/molecfit}} (\cite{mf1, mf2, molecfit}). This software is optimised for the ESO framework and ESO instruments and is also foreseen to be used for \ac{METIS}.\\
The outcome of the \ac{ELT} working group meeting of 2021-03-15 was that future instrument pipelines should include dedicated recipes based on the telluric correction \texttt{telluriccorr} library \cite{telluriccorr}. This package is based on \texttt{molecfit} and will be provided and maintained by \ac{ESO}. The telluric correction will be performed in three dedicated recipes as post-correction   and closely follow the approach as implemented in the \ac{KMOS} pipeline. The three steps comprise the fit of the telluirc features (e.g. \REC{metis_LM_lss_mf_model} for the LM range), the calculation of the transmission curve (e.g. \REC{metis_LM_lss_mf_calctrans}) and the application of the actual correction (e.g. \REC{metis_LM_lss_mf_correct}). For the determination of the \ac{LSF} we primarily rely on the possibilities as offered by the \texttt{telluricorr}/\texttt{molecfit} package, which is based on a fitting of the \ac{LSF} by a combination of a boxcar, Gaussian or Lorentzian. On basis of the commissioning data we will establish a parameter set providing a good starting point for the fits.\\
We also intend to enable the user to include a dedicated line kernel instead, in case a reliable kernel model can be determined with other (external) tools (e.g. a model-based convolution of an internal \ac{LSF}, the slit-widths and/or an \ac{AO} component). In addition, also external supplementary meteorological data (e.g. provided by an \ac{LHATPRO} radiometer) can be included if the \texttt{telluricorr}/\texttt{molecfit} package offers that possibility.
%------------------------------------------------------------------------------
\begin{figure}[ht]
  \centering
  \includegraphics[width=0.9\textwidth]{figures/tell_corr_methods.pdf}
  \caption{Methods for the telluric correction to be included in the \ac{METIS} pipeline.}
  \label{Fig:tellcorrmethods}
\end{figure}
\subsubsection{Approach for METIS}
The modelling approach has become the standard way for the telluric correction in several ESO pipelines as it avoids to spend valuable observing time on taking \ac{TSS} spectra. Since the synthetic transmission function is noise-free, it also conserves the \ac{SNR} of the science spectrum. It is therefore  foreseen as default method for the \ac{METIS} pipeline.\\
However, there might be situations where the classical way becomes the better option. For example, in case the science object's continuum is too weak to be used for fitting telluric absorption features and not enough sky emission is available. In addition, \texttt{molecfit} relies on a number of fitting parameters, which might not lead to the best minimum. In particular, the quality of the fit is very sensitive to the incorporated \ac{LSF}-Kernel, whereas the \ac{LSF} of a \ac{TSS} spectrum is naturally (almost) identical.\\
We therefore will include three different approaches for the telluric correction in the \ac{METIS} pipeline (cf. Fig.~\ref{Fig:tellcorrmethods}):
\begin{itemize}
    \item "\textit{molecfit-on-science}": This is the usual way of using \texttt{molecfit}, i.e. the science spectrum ("\texttt{1D SCIENCE SPECTRUM}") is used to determine the state of the Earth's atmosphere and to calculate a synthetic transmission (left branch in Fig.~\ref{Fig:tellcorrmethods})
    \item "classical" approach: The transmission of the Earth's atmosphere is determined in the classical way with the help of a \ac{TSS} as described above (right branch in Fig.~\ref{Fig:tellcorrmethods})
    \item "\textit{molecfit-on-star}": This is a combination of the both methods in the sense that \texttt{molecfit} is applied to \ac{TSS} observations, and the resulting synthetic transmission spectrum is used for the telluric correction of the science target  (middle branch in Fig.~\ref{Fig:tellcorrmethods})
\end{itemize}
\textit{Notes:}\\
Following the development within the \ac{ELT} working group "Telluric Correction", the newest release\footnote{\color{red}RELEASE EXPECTED IN MAY 2023\color{black}} of \texttt{molecfit} will contain some new features, which will be used in the \ac{METIS} pipelines: (a) a routine to correct transmission spectra for airmass. This allows the usage of \ac{TSS} which are at a different airmass than the science target; (b) a method to quantitatively estimate the quality of the telluric correction on the science frames, which will be used for \ac{QC}; (c) a routine to use the wavelength fit as wavelength calibration for the science observations, i.e. to use telluric features as reference frame. For more details on that new features we refer to the \texttt{molecfit} documentation of the upcoming release and the documents in the working group\footnote{\url{https://eso.org/wiki/pub/ELTScience/Telluric_correction/TelluricWG_20230425.pdf}}.\\
As mentioned above, \texttt{molecfit} will be the default method. It is therefore the users sole responsibility to decide whether a \ac{TSS} is required/desired. However, the selection of the \ac{TSS} should not be arbitrary, but only possible from a provided catalogue.\\
If possible, these stars will also be used for the absolute flux calibration \color{red}(TBChecked!!!)\color{black}

%------------------------------------------------------------------------------
\subsubsection{Other algorithms included in the telluric correction approach}\label{ssec:otheralgstellcorr}
\paragraph{Normalisation of \ac{TSS} continua\newline}\label{ssec:spec_normalisation}
The normalisation of spectra can be a tricky issue especially for cool stars with plenty intrinsic spectral features. A general approach is therefore not possible. However, we can restrict our routines to stars, which are (a) well-known and (b) model spectra are available. We therefore use the following approach: 
\begin{itemize}
    \item For each star we determine spectral regions, which are known to belong to the continuum, i.e. do not contain absorption/emission features, neither intrinsic nor atmospheric.
    \item These spectral regions are then fit by a polynomial (or alternatively the Rayleigh-Jeans approximation is used). The degree of the polynomial will be determined when the set of \ac{TSS} is established. 
    \item The \ac{TSS} spectrum is finally divided by this polynomial leading to a spectrum normalised to unity.
\end{itemize} 

\paragraph{Airmass correction of transmission functions\newline}\label{ssec:airmass_corr}
\textit{TBWritten; see also approach in latest mf release (and footnote below)}

\paragraph{Quality control parameters for \texttt{molecfit}\newline}\label{ssec_tellcorr_qc_params}
To estimate the quality of the telluric correction we follow the approach incorporated with the new \texttt{molecfit} version \textit{(***add Ref to new mf version doc***)}:
\begin{itemize}
    \item Smoothing the corrected spectrum with a Savitzky-Golay filter\footnote{\url{https://en.wikipedia.org/wiki/Savitzky\%E2\%80\%93Golay_filter}}
    \item dividing the corrected spectrum by the smoothed one; This leads to a normalisation to unity and a crude removal of slopes and intrinsic features
    \item Determine the mean absolute difference of the resulting spectrum, excluding \texttt{NaN} values on the user provided spectral ranges (e.g., only spectral ranges with telluric features). 
\end{itemize} % This section is dedicated to algorithm which are used in all spectroscopic modes
\subsection{High-contrast imaging with the apodizing phase plate}
\label{ssec:algo_app_imaging}

LM-band imaging data taken with the apodizing phase plate undergo the
same basic reduction as standard imaging data. \TODO{Does this include
  background subtraction or is the background subtracted during the
  ADI processing?}



\begin{enumerate}
\item Locate the positions of the PSFs with respect to each other
  (dither correction).
\item Center the leakage PSF in the frames.
\item Extract the coronagraphic PSFs and combine them into a single
  cube (see Fig.~\ref{fig:app_psf_combine}). This has almost
  360\degr\ dark zones.
\item Median combine the cube to obtain the reference PSF image.
\item Subtract PSF image from all layers of the cube.
\item Derotate all layers to correct for field rotation, taking into
  account a static mask to reject noisy border regions of the dark
  zones.
\item Sum the layers to obtain final image.
\end{enumerate}

\begin{figure}
  \centering
  \resizebox{0.8\textwidth}{!}{\includegraphics{vAPP_data_reduction}}
  \caption{Combination of the coronagraphic PSFs of APP observations
    into a single PSF with 360\degr\ dark zones. \TODO{This is a figure
      by David Doelman. The layout of the PSFs is not what we expect
      for METIS, therefore needs to be updated.}}
  \label{fig:app_psf_combine}
\end{figure}

%%% Local Variables:
%%% TeX-master: "METIS_DRLD"
%%% End:

\subsection{Long-slit spectroscopy mode}
\label{ssec:algo_lss_spectroscopy}

%-----------------------------------------------------------------------------------------
\subsubsection{Order background contamination removal}\label{ssec:orderbg}
Order background contamination may arise from internal straylight probably covering larger areas of the detector. Since it is expected to be low frequency only, its removal can be achieved by a low-order 2D polynomial fit 
\begin{equation}
    z = (a_0 + a_1x + a_2y + a_3x^2 + a_4x^2y + a_5x^2y^2 + a_6y^2 + a_7xy^2 + a_8xy ...)
\end{equation}
and a subsequent subtraction. The fitting points/regions must be chosen to be outside the \ac{LSS} order (cf. e.g. Figs~\ref{fig:ff}/\ref{fig:pinh} for simulated data). Whereas these fitting points/regions can be chosen on fairly regular basis due to the straight order geometry in the \ac{LSS} mode, the degree of the polynomial depends on the actual straylight. This can be determined only during the testing phase when first real data are available. Note that this algorithm will only be implemented in case such a contamination is visible.

%-----------------------------------------------------------------------------------------
\subsubsection{Order detection and rectification}\label{ssec:orderhandling}
The algorithms for order detection and rectification are adopted from~\cite{pis02,pis21}.
In brief, the selection of pixels that may belong to spectral \ac{LSS} order is done by first smoothing each column and then selecting pixels above the median of the difference between the original
and the smoothed column, i.e. pixel $(x,y)$ is selected if
\begin{equation}
    I(x,y) > \bar{I}(x,y) + \mathrm{Median} ( I(x,y) - \bar{I}(x,y) ) .
\end{equation}
In the following a clustering analysis is performed, which associates connected groups of pixels. This is done scanning rows and columns and identifying neighbouring pixels selected in the previous step as belonging to the same cluster if $\delta x$ and $\delta y$ differ by at most 1. As spectral orders may be partitioned into different clusters because of e.g. detector defects, polynomial fits to the clusters are performed and the pairwise extensions of the fits to consecutive clusters are compared to identify which clusters are to be merged according to predefined criteria for the goodness of match. For each order, the detection algorithm yields a polynomial description of order location on the detector (the order center and its edges), an uncertainty estimate
for the fitted polynomial, and the first and last columns to be used during spectrum extraction. 
%The upper and lower edges of the orders are also traced and fitted by the order tracing algorithm using the pinhole frames with the flatfield lamp. 
Order rectification is achieved using the PyReduce algorithm described by~\cite{pis21} that can account for both tilt and curvature of the slit image.   

%-----------------------------------------------------------------------------------------
\subsubsection{Wavelength calibration strategy}\label{ssec:wavecal}
The wavelength calibration (i.e. the pixel-to-wavelength relation) of the \ac{LSS} modes is done in a two-step approach:
\begin{itemize}
    \item First guess: The first guess is based on laser sources in the \ac{WCU} and during commissioning.  In the LM range, two fixed-frequency lasers ($@3.39$µm and $@5.26$µm) and one tuneable ($4.68....4.78$µm) is foreseen in the \ac{WCU} to be taken on daily basis (cf. \cite{METIS-calibration_plan}). In the N-band, a laser source will only be available during \ac{AIT}. As we assume the instrument to be very stable, that approach should be sufficient to achieve a first guess solution for the low-resolution N-band spectroscopy.
    \item For the final calibration atmospheric lines are used in the recipes \hyperref[rec:metis_lm_lss_sci]{\REC{metis_LM_lss_sci}} and \hyperref[rec:metis_n_lss_sci]{\REC{metis_N_lss_sci}}, which are used in the \texttt{pyreduce} package as reference frame.
    \item Optionally: the package \texttt{molecfit} in its newest version has a new function implemented, which allows a fine-tuning of the wavelength calibration by means of fitting atmospheric features. It is to be tested during commissioning, whether this step is required.
\end{itemize}


%\paragraph{Spectral resolving power}
%We first estimate the spectral resolving power of the \ac{LSS} mode for the different slits incorporated in the spectrograph. More fancy text to follow \textit{Maybe that part should go somewhere else....?}\cite{METIS-system_analysis}
%\begin{table}
%\begin{center}
%\begin{tabular}{c|cccc|cc|cc}
%Band & $\lambda_\textrm{min}$ & $\lambda_\textrm{cent}$ & $\lambda_\textrm{max}$ & $\Delta\lambda$  & $\lambda_\textrm{min}$ y-Pos & %$\lambda_\textrm{min}$ y-Pos & Dispersion & Dispersion \\
% & [$\mu$m] & [$\mu$m] & [$\mu$m]& on chip [$\mu$m] & on chip [mm] & on chip [mm] & [$\mu$m/mm] & [$\mu$m/pixel]\\
%\hline
%L & 2.9 & 3.55 & 4.2 & 1.300 & 17.856 & -16.704 & 3.762e-02 & 6.771e-04 \\
%M & 4.5 & 4.85 & 5.2 & 0.700 & 17.856 & -16.704 & 2.025e-02 & 3.646e-04 \\
%N & 7.5 & 10.5 & 13.5 & 6.000 & -18.801 & 16.911 & 1.680e-01 & 3.024e-03 \\
%\end{tabular}
%\caption{Spectral dispersion of the long-slit spectrograph in all three bands\label{tab:bands_specres1}}
%\end{center}
%\end{table}

%\begin{table}
%\begin{center}
%\begin{tabular}{c|cc|ccc|ccc}
% & & & proj. Slit & & & & & \\
%Slit & Width$^a$ & Length$^a$ & $L$-band & $M$-band & $N$-band & $L$-band & $M$-band & $N$-band \\
%ID & [mas] & [mas] & [pixel] & [pixel] & [pixel] & Resolution$^b$ & Resolution$^b$ & Resolution$^b$ \\
%\hline
%Slit A & 19.0 & 8000.0 & 2.352e-03 & 1.266e-03 & 8.621e-03 & 1509 & 3830 & 1218 \\
%Slit B & 28.6 & 8000.0 & 3.540e-03 & 1.906e-03 & 1.298e-02 & 1003 & 2544 & 809 \\
%Slit C & 38.1 & 8000.0 & 4.716e-03 & 2.539e-03 & 1.729e-02 & 752 & 1910 & 608 \\
%Slit D & 57.1 & 8000.0 & 7.068e-03 & 3.806e-03 & 2.591e-02 & 502 & 1274 & 405 \\
%Slit E & 114.2 & 8000.0 & 1.414e-02 & 7.612e-03 & 5.182e-02 & 251 & 637 & 203 \\
%\end{tabular}
%\caption{Estimate of the LSS spectral resolving power in all three bands with respect to the different slits;\newline $^a$ from% Tab.~4-2 in~\cite{METIS-system_analysis} \newline $^b$ at $\lambda_\textrm{cent}$ (see Tab.~\ref{tab:bands_specres1})%\label{tab:bands_specres2}}
%\end{center}
%\end{table}


%-----------------------------------------------------------------------------------------
%\subsubsection{Spectroscopic flux calibration strategy for the LSS}\label{ssec:fluxcal}


%\textcolor{red}{TBD: slit flux losses due to PSF (cf. MICADO)?????}




%-----------------------------------------------------------------------------------------
\subsubsection{Object extraction and faint object spectroscopy}\label{sec:fospectro}
The object spectra will be extracted using the optimal extraction described by~\cite{pis21} in order to maximized the \ac{SNR}. 
Faint objects will lead to additional requirements for the observation and the data reduction. For the target acquisition a blind offset from a reference source might become necessary in case the actual object cannot be detected directly. For the data reduction, manual interaction with the user is expected to be necessary to define the target position along the slit since automatic object detection algorithms (e.g. optimal extraction %\cite{hor86}) 
rely on a certain \ac{SNR}. This will be implemented as interactive actor in Reflex (or ESO-DPS).
\subsection{Miscellanea}
\label{ssec:miscellanea}

\subsubsection{Parallactic angle}
\label{sssec:parallactic_angle}

The parallactic angle is the angle between the meridian (direction to the celestial north pole) and the hour circle (direction to the zenith) through the target at the time of an observation. Given hour angle $h$ and declination $\delta$ of the target, as well as the latitude $\phi$ of the observatory, the parallactic angle $\eta$ is given by
\begin{equation}
  \label{eq:parallactic_angle}
  \tan\eta = \frac{\cos\phi\sin h}{\sin\phi \cos\delta - \cos\phi \sin\delta \cos h}
\end{equation}
The hour angle is computed from the telescope's pointing altitude and azimuth, or from the target's right ascension and the sidereal time stamp of the exposure.

\subsubsection{Gain, non-linearity}
\label{sssec:gain}

The variance $N_{c}^{2}$ is related to the signal $S_{c}$ via \cite[Section 9.1]{McLean2008}:
\begin{equation}
  \label{eq:signal-variance}
  N_{c}^{2} = \frac{1}{g} S_{c} + R_{c}^{2},
\end{equation}
where $g$ is the gain and $R_{c}$ the readout noise. All quantities with subscript $c$ are in counts (ADU).

\subsubsection{Conversion of wavelengths between vacuum and air regime}\label{ssec:vacair}
For the conversion of $\lambda_\textrm{vac}$ to $\lambda_\textrm{air}$ the \ac{IAU} standard formula provided by Donald Morton \cite{mor00} $\lambda_\textrm{air}=\lambda_\textrm{vac}/n$ is used, where

\begin{eqnarray}\label{eq:air2vac}
\left.\begin{aligned}
    s &=10^4 / \lambda_{\textrm{vac}}\\
    n &= 1+0.0000834254 + \frac{0.02406147}{(130 - s^2)} + \frac{0.00015998}{(38.9 - s^2)}
\end{aligned}\right.
\end{eqnarray}

The reverse transform $\lambda_\textrm{air}$ to $\lambda_\textrm{vac}$ was derived by N. Piskunov\footnote{\url{https://www.astro.uu.se/valdwiki/Air-to-vacuum\%20conversion}}, being $\lambda_\textrm{vac}=\lambda_\textrm{air}*n$ with 

\begin{eqnarray}\label{eq:vac2air}
\left.\begin{aligned}
    s&=10^4 / \lambda_{\textrm{air}}\\
    n&=1 + 0.000083366242 + \frac{0.0240892687}{(130.106592452 - s^2)} + \frac{0.00015997409}{(38.925687933 - s^2)}
\end{aligned}\right.
\end{eqnarray}

All wavelengths in the \ac{LSS} spectroscopic pipeline are given in the vacuum regime. \textcolor{red}{TBD: Also in LMS?}

\subsubsection{Centroid fit}
\label{sssec:centroid}
TBWritten

\subsubsection{Line shape fit}
\label{sssec:linefit}
TBWritten about Gaussian ine fitting


%%% Local Variables:
%%% TeX-master: "METIS_DRLD"
%%% End:


%%% Local Variables:
%%% TeX-master: "METIS_DRLD"
%%% End:



\clearpage

\section{Pipeline Recipes / CPL Plugins}
\label{sec:pipeline_recipes}

All functionality of the METIS pipeline is provided through \emph{recipes}, i.e.~CPL plugins that run in the contexts that ESO tools provide. This is in accordance with \cite{1618} and fulfills \REQ{METIS-6059} and \REQ{METIS-5945}.

Throughout this document we use the following color scheme for the
recipe workflows:
\begin{center}
  \includegraphics[width=0.3\textheight]{colour_legend}
\end{center}

%% Recipes_Detector.tex
%% Created:     Tue Apr  4 16:37:17 2017 by Koehler@I-Mac
%% Last change: 2020-09-04
%%
%% subsection for Detector Recipes
%%
%%%%%%%%%%%%%%%%%%%%%%%%%%%%%%%%%%%%%%%%%%%%%%%%%%%%%%%%%%%%%%%%%%%%%%%%%%%%%
\subsection{Detector calibration recipes}
\label{Sec:detector_calibration}

METIS will have three focal plane detector arrays:
\begin{itemize}
\item One $2\mathrm{k}\times 2\mathrm{k}$ HAWAII2RG detector used for
  LM-band imaging and slit spectroscopy.
\item One $2\mathrm{k}\times 2\mathrm{k}$ GeoSnap (Teledyne) detector
  used for N-band imaging and slit spectroscopy.
\item An array of four $2\mathrm{k}\times 2\mathrm{k}$ HAWAII2RG
  detectors used for LM-band integral-field spectroscopy.
\end{itemize}
This section lists recipes that calibrate detector characteristics
independent of a specific instrument mode. Where \FITS{_det} appears
in FITS keywords of input or product files, it is taken to mean
\FITS{_LM}, \FITS{_N} or \FITS{_IFU} according to the detector
array for which data are being processed.

\subsubsection{Detector linearity and gain determination recipe \REC{metis\_det\_lingain}}
\label{sssec:metis_det_lingain}
\label{rec:metis_det_lingain}
\label{rec:metisdetlingain}

The recipe \hyperref[rec:metis_det_lingain]{\REC{metis_det_lingain}} determines detector (non-)linearity and absolute detector
gain from a set of flat-field frames taken with the broad-band lamp
over a range of detector exposure times (DITs) and flux levels. The
recipe structure will be similar as for \CODE{detmon_ir_lg} % Not a \REC because it is not our recipe
\cite{detmon-manual}; however, further insight into detector behaviour
(in particular of GeoSnap) may necessitate development of more complex
procedures.

The linearity curve is given by the measured background level as a
function of exposure time for constant illumination. For each pixel
the coefficients of a polynomial fit will be recorded in a
coefficient cube, which can in turn be used to correct for
non-linearity in other recipes. Pixels whose coefficients differ
significantly from the majority of pixels will be marked as bad.

Detector gain is typically computed pixelwise as the slope of a linear
fit of the variance against the mean (or median) values over a set of
frames taken over a range of DITs and illumination levels.  For
mid-infrared detectors that suffer from \ac{ELFN}, e.g.\ the AQUARIUS
detector, this approach does not work.  The GeoSnap is not expected to
show \ac{ELFN}, hence gain determination is probably possible.

The set of calibration frames used for this recipes will include
exposures with WCU window closed (\CODE{LAMP OFF}), which will be used
as `dark' frames that captur thermal emission within the
instrument. This is subtracted from all other exposures in the
sequence.

This satisfies \REQ{METIS-5997}.

\newpage
\begin{recipedef}
  Name:                & \hyperref[rec:metis_det_lingain]{\REC{metis_det_lingain}}                                                             \\
  Purpose:             & determine non-linearity and gain of the detectors                                   \\
  Requirements:        & \REQ{METIS-5997}                                                                    \\
  Type:                & Calibration                                                                         \\
  Templates:           & \TPL{METIS_img_lm_cal_DetLin}                                                       \\
                       & \TPL{METIS_img_n_cal_DetLin}                                                        \\
                       & \TPL{METIS_ifu_cal_DetLin}                                                          \\
  Input data:          & \hyperref[dataitem:detlin_det_raw]{\RAW{DETLIN_det_RAW}}: (set of \FITS{FLAT,LAMP} frames taken with increasing DIT) \\
                       & \hyperref[dataitem:det_wcu_off_raw]{\RAW{det_WCU_OFF_RAW}}: (set of internal darks taken at the start of the template) \\
 % Matched keywords:    & Subsystem ID \TODO{TBD}                                                             \\
  Algorithm:           & Subtract instrument dark (\CODE{hdrl_imagelist_sub_image}).                         \\
                       & Compute mean and variance for each frame (\CODE{TBD}).                              \\
                       & Gain is determined as the slope of variance against mean (\hyperref[drl:metis_derive_gain]{\DRL{metis_derive_gain}}) \\
                       & Fit polynomial of value as a function of DIT and illumination level for each pixel (\hyperref[drl:metis_derive_nonlinearity]{\CODE{metis_derive_nonlinearity}}). \\
                       & Flag pixels with coefficients significantly different from the mean of all pixels. (\CODE{hdrl_bpm_fit_compute}) \\
  Output data:         & \hyperref[dataitem:gain_map_det]{\PROD{GAIN_MAP_det}}                                    \\
                       & \hyperref[dataitem:linearity_det]{\PROD{LINEARITY_det}}                                 \\
                       & \hyperref[dataitem:badpix_map_det]{\PROD{BADPIX_MAP_det}}                                \\
  Expected accuracies: & 0.5\% background subtraction (cf. \cite{METIS_calerrbudget})                             \\
                       & 0.1\% non-linearity measurement (cf. \cite{METIS_calerrbudget})                          \\
  QC1 parameters:      & \hyperref[qc:qc_lin_gain_mean]{\QC{QC LIN GAIN MEAN}}                                    \\
                       & \hyperref[qc:qc_lin_gain_rms]{\QC{QC LIN GAIN RMS}}                                      \\
                       & \hyperref[qc:qc_lin_num_badpix]{\QC{QC LIN NUM BADPIX}}                                  \\
  hdrl functions:      & \CODE{hdrl_imagelist_sub_image}                                                     \\
                       & \CODE{hdrl_bpm_fit_compute}                                                         \\
\end{recipedef}

\begin{figure}[hb]
  \centering
    \def \globalscale {0.700000}
    \fontsize{10}{12}\selectfont
    % % Document preamble. Comment out for final figure! Footer too!
% \documentclass[tikz, margin=5mm, dvipsnames]{standalone}
% \usepackage{listings}
% \input{black_style}
% \input{styles_data}
% \begin{document}

\input{black_style}
\input{recipe_config}


\begin{tikzpicture}
  [x=1cm,
  y=-1cm,
  align=center,
  node distance=2cm and 3.5cm]
  \sffamily

%   % Grid for orientation. Comment out for final figure!
%   \draw[help lines, green](-5, 0) grid (8, 11);

  %%% Put workflow commands here:
  %% Main reduction workflow

  \node (template) [template]{%
    \TPL{METIS_img_lm_cal_DetLin}\\
    \TPL{METIS_img_n_cal_DetLin}\\
    \TPL{METIS_ifu_cal_DetLin}};

  \pic (start) [below=0.75cm of template] {start};

  \node (input) [below=0.75cm of start-m, input] {%
    \textsl{N} \RAW{DETLIN_det_RAW}\\
    \RAW{det_WCU_OFF_RAW}};

  \node (step_persistence) [below=2.0cm of input, redstep] {%
    apply persistence correction};

  \node (step_dark) [below of=step_persistence, redstep]{%
    subtract dark};

  \node (step_gain) [below of=step_dark, redstep]{%
    compute gain};

  \node (step_linearity) [below of=step_gain, redstep]{%
    linearity check};

  \node (step_thresholding) [below of=step_linearity, redstep]{%
    thresholding};

  \pic (stop) [below=2.5cm of step_thresholding]{stop};

  %% Connections
  \draw (template) -- (input);
  \draw (input) -- (step_persistence);
  \draw (step_persistence) -- (step_dark);
  \draw (step_dark) -- (step_gain);
  \draw (step_gain) -- (step_linearity);
  \draw (step_linearity) -- (step_thresholding);
  \draw (step_thresholding) -- (stop-t);

  %% Other input
  \node (connect_persistence) [connection] at ($(input)!0.65!(step_persistence)$){};
  \node (persistence) [left=of connect_persistence, external]{\EXTCALIB{PERSISTENCE_MAP}};
  \draw (persistence) -- (connect_persistence);

  \node (params) [left=of step_thresholding.center, params]{%
    Recipe params:\\
    THRESH\_LOWLIM\\
    THRESH\_UPLIM};
  \draw (params) -- (step_thresholding);

  %% Output
  \node (connectgain) [connection] at
  ($(step_gain)!0.5!(step_linearity)$){};
  \node (gainmap) [right=of connectgain, calproduct]{%
    \STATCALIB{GAIN_MAP_det}};
  \draw (connectgain) -- (gainmap);

  \node (connectlin) [connection] at
  ($(step_linearity)!0.5!(step_thresholding)$) {};
  \node (linearity) [right=of connectlin, calproduct]{%
    \STATCALIB{LINEARITY_det}};
  \draw (connectlin) -- (linearity);

  \node (connectbpm) [connection] at
  ($(step_thresholding)!0.35!(stop-t)$) {};
  \node (bpm) [right=of connectbpm, calproduct]{%
    \STATCALIB{BADPIX_MAP_det}};
  \draw (connectbpm) -- (bpm);

  %% Frame around recipe
  \draw [frame] ($(input)!0.4!(step_persistence) - (2.75cm,0)$) rectangle ($(step_thresholding)!0.5!(stop-t) + (2.75cm, 0)$);
  \node [framecolor, anchor=north west] at
  ($(input)!0.4!(step_persistence) - (2.75cm,0)$){\textsl{metis\_det\_lingain}};
\end{tikzpicture}
\input{normal_style}


% % Document footer. Comment out for final figure! Header too!
% \end{document}

  \caption[Recipe: \REC{metis_det_lingain}]{\REC{metis_det_lingain} --
    determination of linearity and gain of the detectors.}
  \label{Fig:rec_det_lingain}
\end{figure}


\clearpage

\subsubsection{Master dark recipe \REC{metis\_det\_dark}}
\label{sssec:metis_det_dark}
\label{rec:det_dark}
\label{rec:metis_det_dark}

Darks are taken in daytime for all science detectors
\cite{METIS-calibration_plan}. The data will be classified by detector
(e.g.~\FITS{DET.ID} and \FITS{DET.CHIP.ID}) and integration time
(\FITS{DET.DIT}).\footnote{The dark current is not expected to depend on the readout mode of the detectors. Should hardware tests reveal such a dependence, the recipe will be amended to classify on readout mode as well.} There will be ``METIS-dark''
(with the CLOSED position of the CFO-PP1 wheel) and ``Imager-dark''
(with the CLOSED position in the subsystem PP1), to be distinguished
by keyword \TBD. The former will be used for pipeline processing, the
latter for monitoring purposes.

Each set of raw dark frames is processed into a master dark. For the
IFU, both raw frames and master dark have four extensions
corresponding to the four detectors in the focal-plane array. The
recipe also produces bad pixel masks by identifying hot pixels whose
dark current differs significantly (by more than $\pm 5\sigma$) from
the average over the detector.

This fulfills \REQ{METIS-6063}.

\begin{recipedef}
  Name:                & \hyperref[rec:metis_det_dark]{\REC{metis_det_dark}}                                                        \\
  Purpose:             & determine the dark current of the detectors                                 \\
  Requirements:        & \REQ{METIS-6063}                                                            \\
  Type:                & Calibration                                                                 \\
  Templates:           & \TPL{METIS_gen_cal_dark}                                                    \\
                       & \TPL{METIS_gen_cal_InsDark}                                                 \\
  Input data:          & \hyperref[dataitem:dark_det_raw]{\RAW{DARK_det_RAW}}  \\
                       & \hyperref[dataitem:linearity_det]{\STATCALIB{LINEARITY_det}}  \\
% TODO: Remove persistence, because if the raw darks have persistence, we are doing something wrong
                       & \hyperref[dataitem:persistence_map]{\EXTCALIB{PERSISTENCE_MAP}}  \\
  Parameters:          & Combination method (\texttt{median}, \texttt{mean},
                         \texttt{sigclip},\dots)                                                  \\
                       & Parameters for combination methods                                          \\
                       & Thresholds for deviant-pixel identification                                      \\
  Algorithm:           & Group files by detector and \texttt{DIT}, based on header keywords           \\
                       & Call function \DRL{metis_determine_dark} for each set of files\\
                       & Compute median or average of input frames to improve statistics.            \\  % separate routine, or part of determine dark
                       & call \DRL{metis_update_dark_mask} to flag deviant pixels \\
  Output data:         & \hyperref[dataitem:master_dark_det]{\PROD{MASTER_DARK_det}}                                                      \\
% The BPM_COLD_det and BPM_HOT_det do not seem to add value that BADPIX_MAP_det
% does not already provide. Furthermore, the COLD/HOT specific items are not
% otherwise used in the design, so it seems simpler to just remove them.
%                       & \hyperref[dataitem:bpm_cold_det]{\PROD{BPM_COLD_det}}                                                         \\
%                       & \hyperref[dataitem:bpm_hot_det]{\PROD{BPM_HOT_det}}                                                          \\
                       & \hyperref[dataitem:badpix_map_det]{\PROD{BADPIX_MAP_det}}                                                          \\
  Expected accuracies: & 0.1\% (cf. \cite{METIS_calerrbudget})                                          \\
  QC1 parameters:      & \QC{QC DARK MEAN}                                                              \\
                       & \QC{QC DARK MEDIAN}                                                            \\
                       & \QC{QC DARK RMS}                                                               \\
                       & \QC{QC DARK NBADPIX}                                                             \\
                       & \QC{QC DARK NCOLDPIX}                                                               \\
                       & \QC{QC DARK NHOTPIX}                                                                \\
                       & (more \TBD)                                                                  \\
  hdrl functions:      & \CODE{hdrl_bpm_3d_compute}                                 \\
                       & \CODE{hdrl_imagelist_collapse}                             \\
\end{recipedef}

\begin{figure}[hb]
  \centering
        \def \globalscale {0.700000}
        \fontsize{10}{12}\selectfont
        \documentclass[tikz, margin=5mm]{standalone}

\input{recipe_config}

\begin{document}

\begin{tikzpicture}
  [x=1cm,
  y=-1cm,
  align=center,
  node distance=2cm and 3cm]
  \sffamily

  %% Grid for orientation. Comment out for final figure!
  %\draw[help lines, green](-5, 0) grid (8, 11);

  %% Main reduction flow
  \node (template) [template] {METIS\_all\_cal\_dark};
  \pic (start) [below=0.75cm of template] {start};
  \node (input) [below=0.75cm of start-m, input] {\textsl{N} DARK\_RAW};
  \node (step1) [below of=input, redstep] {%
    group by detector and DIT};
  \node (step2) [below of=step1, redstep] {%
    per pixel median/ mean filtering};
  \node (step3) [below of=step2, redstep] {thresholding};
  \pic (stop) [below=3.5cm of step3] {stop};

  %% Connections
  \draw (template) -- (input);
  \draw (input) -- (step1);
  \draw (step1) -- (step2);
  \draw (step2) -- (step3);
  \draw (step3) -- (stop-t);

  %% Output
  \node (connectdark) [connection] at ($(step2)!0.5!(step3)$){};
  \node (masterdark) [right=of connectdark, calproduct] {MASTER\_DARK\_det\_dit};
  \draw (connectdark) -- (masterdark);

  \node (connecthot) [connection] at ($(step3)!0.5!(stop-t)$){};
  \node (hot) [right=of connecthot, calproduct] {BPM\_HOT\_det};
  \draw (connecthot) -- (hot);

  \node (connectcold) [connection, above=0.06cm of connecthot] {};
  \node (cold) [above=0.1cm of hot, calproduct] {BPM\_COLD\_det};
  \draw (connectcold) -- ++(2cm,0) |- (cold);

  \node (connectbpm) [connection, below=0.06cm of connecthot] {};
  \node (bpm) [below=0.1cm of hot, calproduct] {BADPIX\_MAP\_det};
  \draw (connectbpm) -- ++(2cm,0) |- (bpm);

  %% frame around recipe
  \draw [frame] ($(input)!0.4!(step1) - (2.5cm,0)$)
  rectangle ($(step3)!0.8!(stop-t) + (2.5cm, 0)$);
  \node [framecolor, anchor=north west] at
  ($(input)!0.4!(step1) - (2.5cm,0)$) {\textsl{metis\_det\_dark}};

\end{tikzpicture}
\end{document}

  \caption[Recipe: \REC{metis_det_dark}]{\REC{metis_det_dark} -- creation of master
    dark and bad pixel maps}
  \label{Fig:rec_det_dark}
\end{figure}
\clearpage

\subsubsection{Persistence map creation recipe \REC{metis\_det\_persistence}}
\label{sssec:metis_det_persistence}
\label{rec:metis_det_persistence}

Infrared detectors are prone to persistence due to charges trapped on a variety of timescales. The correction for a given science or
calibration exposure is built from a sequence of exposures preceding the exposure in question. As these may include exposures taken for another proprietary programme, the recipe is run by ESO on data taken from the science archive and its products are again ingested into the archive. The recipe will make use of the \ac{HDRL} function for the creation of the persistence map, which will be provided by \ac{ESO} in the future.\\
We will update this Section as soon as more information is available.

%\begin{recipedef}
%  Name:                & \hyperref[rec:metis_det_persistence]{\REC{metis_det_persistence}}           \\
%  Purpose:             & compute persistence correction maps        \\
%  Requirements:        & \REQ{METIS-9145}                      \\
%  Type:                & Calibration                           \\
%  Templates:           & --                                    \\
%  Parameters:          & \TBD                                  \\
%  Algorithm:           & see hdrl functions:                   \\
%  Output data:         & \hyperref[dataitem:persistence_map]{\PROD{PERSISTENCE_MAP}}                \\
%  Expected accuracies: & \TBD                                  \\
%  QC1 parameters:      & see hdrl functions:                   \\
%  hdrl functions:      & \TBD (\CODE{hdrl_persistence_compute} \\
%\end{recipedef}

%\begin{figure}[hb]
%  \centering
%  \resizebox{0.6\textwidth}{0.1\textwidth}{\TODO{\fbox{Figure to be done}}}
%  \caption[Recipe:
%  \REC{metis_det_persistence}]{\REC{metis_det_persistence} -- creation
%    of persistence correction frames.}
%  \label{Fig:rec_det_persistence}
%\end{figure}

%%%%%%%%%%%%%%%%%%%%%%%%%%%%%%%%%%%%%%%%%%%%%%%%%%%%%%%



%%% Local Variables:
%%% TeX-master: "METIS_DRLD"
%%% End:


\subsection{LM-band imaging recipes}
\label{ssec:recipes_img_lm}
%------------------------------------------------------------------------------------------------------------------
\subsubsection{\REC*{metis_lm_img_flat}:  Flatfielding}
\label{lm_img_flatfield}
\label{rec:lm_img_flatfield}
\label{sssec:lm_img_flatfield}
\label{metis_lm_img_flat}
\label{rec:metis_lm_img_flat}
\label{sssec:metis_lm_img_flat}

The purpose of the flat-field calibration is to determine
pixel-to-pixel gain variations and large scale illumination variations
(due to inhomogeneities of optical elements in the telescope or
instrument). Calibration frames are obtained either during day time
using the black-body lamp of the \ac{WCU} (internal flats) or by taken
images of the twilight sky (twilight flats). Advantages and
disadvantages of the two types of flat are discussed in
\cite{METIS-calibration_plan}. Since the operational concept for
twilight flats needs to be refined during commissioning at the
telescope, the current recipe design is primarily valid for internal
flats.

This recipe creates a master flat for the HAWAII2RG detector (LM-band
imaging) from lamp or sky images matched by various setup parameters
as detailed below.  A set of internal flats includes a number of
exposures with \CODE{LAMP OFF}, which will be used for dark
subtraction. For twilight flats a master dark will be subtracted. The
master flat is obtained by the slope of a linear fit of the pixel
values against the illumination level of the exposures.

The quality control parameters give various statistics for each input
frame (mean, standard deviation, etc.), the standard deviation of the
normalised master flat and the number of bad pixels identified by the
recipe. If a bad-pixel map is provided on input, it is updated,
otherwise a new one is created.

\begin{recipedef}
  Name:                & \REC{metis_lm_img_flat}                                        \\
  Purpose:             & Create master flat field for the LM-band imaging detector.     \\
  Requirements:        & \REQ{METIS-6096}                                               \\
  Type:                & Calibration                                                    \\
  Templates:           & \TPL{METIS_img_lm_cal_InternalFlat}                            \\
                       & \TPL{METIS_img_lm_cal_TwilightFlat}                               \\
  Input data:          & \RAW{LM_FLAT_LAMP_RAW} \\
                       & or \RAW{LM_FLAT_TWILIGHT_RAW} \\
                       & \EXTCALIB{BADPIX_MAP_2RG} (optional) \\
                       & \STATCALIB{LINEARITY_2RG} \\
                       & \STATCALIB{GAIN_MAP_2RG} \\
                       & \EXTCALIB{PERSISTENCE_MAP} \\
                       & \PROD{MASTER_DARK_2RG} (for twilight flats) \\
  %                       & \EXTCALIB{BADPIX_MAP_2RG} \\
%                       & \STATCALIB{GAIN_MAP_2RG}\\
  Matched keywords:    & \FITS{DET.DIT}                                                   \\
                       & \FITS{DET.NDIT}                                                  \\
                       & \FITS{DRS.FILTER}                                                     \\
  Parameters:          & Combination method (\texttt{mean}, \texttt{median},
                         \texttt{sigclip}, \dots)                                       \\
                       & Parameters for combination methods                             \\
                         & Threshold(s) for deviant-pixel identification                  \\
 Algorithm:          %  & Call \DRL{metis_apply_persistence_correction} to apply the persistence correction \\
                         & For internal flats: call \REC{metis_det_dark} with \CODE{LAMP OFF} images to create dark frame. \\
 & Subtract internal dark or master dark from flat exposures.     \\
  & call \REC{metis_lm_img_flat} to fit slope of pixel values against illumination level. Frames
  with the same exposure time will be averaged.\\
                       & Compute median or average of input frames to improve statistics.\\
                       & Call \DRL{metis_update_lm_flat_mask} to flag deviant pixels. \\
  Output data:         & \PROD{MASTER_IMG_FLAT_LAMP_LM} \\
                       & or \PROD{MASTER_IMG_FLAT_TWILIGHT_LM} \\
                       & \EXTCALIB{BADPIX_MAP_2RG}                                           \\
  Expected accuracies: & 0.5\% (cf.~\cite{METIS_calerrbudget})                                                           \\
  QC1 parameters:      & \QC{QC LM MFLAT RMS}                                      \\
                       & \QC{QC LM MFLAT NBADPIX}                                        \\
                       & \QC{QC LM FLAT MEAN}                                           \\
                       & \QC{QC LM FLAT RMS}                                            \\
                       & \QC{QC LM FLAT MEDIAN MIN}                                             \\
                       & \QC{QC LM FLAT MEDIAN MAX}                                             \\
                       & \QC{QC LM FLAT MEDIAN RMS}                                             \\

  hdrl functions:      & \CODE{hdrl_bpm_fit_compute}                                    \\
                       & \CODE{hdrl_imagelist_collapse}                                 \\
                       & \CODE{hdrl_imagelist_sub_image}                                \\
\end{recipedef}

\begin{figure}[hb]
  \centering
    \def \globalscale {0.700000}
    \fontsize{10}{12}\selectfont
    % % Document preamble. Comment out for final figure! Footer too!
% \documentclass[tikz, margin=5mm, dvipsnames]{standalone}
% \usepackage{hyperref}
% \usepackage{listings}
% \input{black_style}
% \input{styles_data}
% \begin{document}


\input{black_style}
\input{recipe_config}


\begin{tikzpicture}
  [x=1cm,
  y=-1cm,
  align=center,
  node distance=2cm and 3.5cm]
  \sffamily

%   % Grid for orientation. Comment out for final figure!
%   \draw[help lines, green](-5, 0) grid (8, 11);

  %%% Put workflow commands here:
  %% Main reduction workflow

  \node (template) [template] {%
    \TPL{METIS_img_lm_cal_InternalFlat}\\
    \TPL{METIS_img_lm_cal_TwilightFlat}};

  \pic (start)[below=0.75cm of template]{start};

  \node (input) [below=0.75cm of start-m, input] {%
    \textsl{N} \RAW{LM_FLAT_LAMP_RAW}\\
    or \textsl{N} \RAW{LM_FLAT_TWILIGHT_RAW}};

%  \node (step_persistence) [below=2.0cm of input, redstep] {%
%    apply persistence correction};

  \node (step_signature) [below=2.0cm of input, redstep]{%
    detector signature\\ removal};

  \node (step_fit) [below of=step_signature, redstep]{%
    linear fit (slope)};

  \pic (stop) [below=2.5cm of step_fit] {stop};

  %% Connections
  \draw (template) -- (input);
%  \draw (input) -- (step_persistence);
  \draw (input) -- (step_signature);
  \draw (step_signature) -- (step_fit);
  \draw (step_fit) -- (stop-t);

%  \node (connectpers) [connection] at
%  ($(input)!0.65!(step_persistence)$){};
%  \node (persistence) [left=of connectpers, external]{%
%    \EXTCALIB{PERSISTENCE_MAP}};
%  \draw (persistence) -- (connectpers);

%   \node (connectparams) [connection] at
%   ($(input)!0.7!(step_signature.north)$){};
%   \node (params) [left=of connectparams, params]{%
%     Recipe params:\\
%     INS.IMG.SETUP};
%   \draw (params) -- (connectparams);

%  \node (connectbpm) [connection] at ($(step_persistence)!0.3!(step_signature)$){};
%  \node (bpmin) [left=of connectbpm, calproduct]{\STATCALIB{BADPIX_MAP_2RG}};
%  \draw (bpmin) -- (connectbpm);

  \node (connection_masterdark) [connection] at
  ($(input)!0.7!(step_signature)$){};
  \node (darkin) [left=of connection_masterdark, calproduct]{%
    \STATCALIB{MASTER_DARK_2RG}};
  \draw (darkin) -- (connection_masterdark);

  %% Output
  \node (connectflat) [connection] at
  ($(step_fit)!0.3!(stop-t)$){};
  \node (flatout) [right=of connectflat,calproduct]{%
    \STATCALIB{MASTER_IMG_FLAT_LAMP_LM}\\
    or \STATCALIB{MASTER_IMG_FLAT_TWILIGHT_LM}};
  \draw (connectflat) -- (flatout);

  \node (connectbpm) [below=0.65cm of connectflat, connection]{};
  \node (bpmout) [right=of connectbpm, calproduct]{%
    \STATCALIB{BADPIX_MAP_2RG}};
  \draw (connectbpm) -- (bpmout);

  %% Frame around recipe
  \draw [frame] ($(input)!0.35!(step_signature) -(2.85,0)$) rectangle
  ($(step_fit)!0.75!(stop-t) + (2.85,0)$);
  \node [framecolor, anchor=north west] at
  ($(input)!0.35!(step_signature) - (2.85,0)$){%
    \REC{metis_lm_img_flat}};

\end{tikzpicture}
\input{normal_style}


% % Document footer. Comment out for final figure! Header too!
% \end{document}

  \caption[Recipe: \REC*{metis_lm_img_flat}]{\REC*{metis_lm_img_flat} --
    creation of \CODE{IMG_LM} master flatfield.}
  \label{fig:metis_lm_img_flat}
\end{figure}
%    \TODO{Include averaging of frames at same illumination in figure}

%------------------------------------------------------------------------------------------------------------------
\clearpage
\subsubsection{\REC*{metis_lm_img_basic_reduce}:  Basic reduction}
\label{lm_img_basic}
\label{rec:lm_img_basic}
\label{sssec:lm_img_basic}
\label{metis_lm_img_basic_reduce}
\label{rec:metis_lm_img_basic_reduce}
\label{sssec:metis_lm_img_basic_reduce}

%\TODO{New recipe -- this may be too basic and could be joined with the background subtraction.}

This recipe performs the basic reduction of raw exposures from the
LM-band imager, i.e.\ dark subtraction, flat fielding and removing
other instrumental signals. It is used for both standard and science exposures.

This recipe analyses the masked detector regions for channel offset correction, crosstalk (Section~\ref{ssec:criticaldetetctormasks}) and removal of detector artefacts (electronic ghosts).
In the horizontal dimension, the masked pixels are used to correct for each channel offset.
In the vertical dimension, the masked pixels are used to correct for cross-talk.
% From Rof https://polarion.astron.nl/polarion/#/project/METIS/workitem?id=METIS-6089
% I am not sure whether the masked region should be removed from pipeline products (at least not as in this very general statement). For some products, it can be useful to inspect the masked regions and that would be much more cumbersome if they were removed from products that have undergone any form or processing. I would remove the last sentence from the requirement, or at least rephrase it. For some or even most of the high-level products, the masked region would be a nuisance if present so for most pipeline products one would indeed like to have it removed

Basic statistics of the images can be used to screen for saturation.

\begin{recipedef}
  Name:             & \REC{metis_lm_img_basic_reduce}   \\
  Purpose:          & apply basic reduction of images   \\
  Requirements:     & \REQ{METIS-6090} \\
  Type:             & Calibration, Science              \\
  Templates:        & \TPL{METIS_img_lm_cal_standard}  \\
%                     & \TPL{METIS_img_lm_*_obs_*}       \\
                    & \TPL{METIS_img_lm_obs_AutoJitter} \\
                    & \TPL{METIS_img_lm_obs_GenericOffset} \\
                    & \TPL{METIS_img_lm_obs_FixedSkyOffset} \\
                    & \TPL{METIS_img_lm_app_obs_FixedOffset} \\
                    & \TPL{METIS_img_lm_vc_obs_FixedSkyOffset} \\
                    & \TPL{METIS_img_lm_cal_psf}             \\
                    % LMN combined templates:
                    & \TPL{METIS_img_lmn_obs_AutoChopNod} \\
                    & \TPL{METIS_img_lmn_obs_GenericChopNod} \\
                    % All IFU templates also create an LM image:
                    & \TPL{METIS_ifu_obs_FixedSkyOffset}                                                       \\
                    & \TPL{METIS_ifu_obs_GenericOffset}                                                        \\
                    & \TPL{METIS_ifu_ext_obs_FixedSkyOffset}                                                   \\
                    & \TPL{METIS_ifu_ext_obs_GenericOffset}                                                    \\
                    & \TPL{METIS_ifu_vc_obs_FixedSkyOffset}                                                    \\
                    & \TPL{METIS_ifu_ext_vc_obs_FixedSkyOffset}                                                \\
                    & \TPL{METIS_ifu_app_obs_Stare}                                                            \\
                    & \TPL{METIS_ifu_ext_app_obs_Stare}                                                        \\
                    & \TPL{METIS_ifu_cal_psf}                                                                  \\
  Input data:       & \RAW{LM_IMAGE_SCI_RAW} \\
                    & or \RAW{LM_IMAGE_STD_RAW} \\
                    & or \RAW{LM_IMAGE_SKY_RAW} \\
                    & \EXTCALIB{BADPIX_MAP_2RG} (optional) \\
%                     & \RAW{IFU_SKY_RAW} (Blank sky images, if available.) \\
%                     & \EXTCALIB{LM_DETECTOR_MASK} (if available)  \\
                    & \STATCALIB{LINEARITY_2RG} \\
                    & \STATCALIB{GAIN_MAP_2RG} \\
                    & \EXTCALIB{PERSISTENCE_MAP} \\
                    & \PROD{MASTER_DARK_2RG} \\
                    & \PROD{MASTER_IMG_FLAT_LAMP_LM} \\
                   & or \PROD{MASTER_IMG_FLAT_TWILIGHT_LM} \\
    Matched keywords:    & \FITS{DET.DIT}                                                   \\
                       & \FITS{DET.NDIT}                                                   \\
                       & \FITS{DRS.FILTER}                                                     \\
  Algorithm:        & Remove crosstalk, correct non-linearity \\
                    & Analyse and optionally remove masked regions  \\
                    & Subtract dark, divide by flat       \\
                    & Remove blank sky pattern                \\
  Output data:      & \PROD{LM_SCI_BASIC_REDUCED}       \\
                    & \PROD{LM_STD_BASIC_REDUCED}       \\
  QC1 parameters:   & \QC{QC LM IMG MEDIAN}             \\
                    & \QC{QC LM IMG STANDARD DEVIATION} \\
                    & \QC{QC LM IMG PEAK}               \\
  hdrl functions:   & \CODE{hdrl_imagelist_sub_image}   \\
                    & \CODE{hdrl_imagelist_div_image}   \\
\end{recipedef}

\newgeometry{bottom=0.1cm, right=0.1cm, left=0.1cm, top=0.1cm}
\begin{figure}[hb]
  \centering
    \def \globalscale {0.700000}
    \fontsize{10}{12}\selectfont
    \input{black_style}
\input{recipe_config}


\begin{tikzpicture}
  [x=1cm,
  y=-1cm,
  align=center,
  node distance=2cm and 3.5cm]
  \sffamily


  %% template names
  \node (template) [template] {%
%    METIS\_img\_lm\_cal\_standard\\
%    METIS\_img\_lm\_*\_obs\_*
    \TPL{METIS_img_lm_cal_standard}  \\
    \TPL{METIS_img_lm_obs_AutoJitter} \\
    \TPL{METIS_img_lm_obs_GenericOffset} \\
    \TPL{METIS_img_lm_obs_FixedSkyOffset} \\
    \TPL{METIS_img_lm_app_obs_FixedOffset} \\
    \TPL{METIS_img_lm_vc_obs_FixedSkyOffset} \\
    \TPL{METIS_img_lm_cal_psf}             \\
    % LMN combined templates:
    \TPL{METIS_img_lmn_obs_AutoChopNod} \\
    \TPL{METIS_img_lmn_obs_GenericChopNod} \\
    % All IFU templates also create an LM image:
    \TPL{METIS_ifu_obs_FixedSkyOffset}                                                       \\
    \TPL{METIS_ifu_obs_GenericOffset}                                                        \\
    \TPL{METIS_ifu_ext_obs_FixedSkyOffset}                                                   \\
    \TPL{METIS_ifu_ext_obs_GenericOffset}                                                    \\
    \TPL{METIS_ifu_vc_obs_FixedSkyOffset}                                                    \\
    \TPL{METIS_ifu_ext_vc_obs_FixedSkyOffset}                                                \\
    \TPL{METIS_ifu_app_obs_Stare}                                                            \\
    \TPL{METIS_ifu_ext_app_obs_Stare}                                                        \\
    \TPL{METIS_ifu_cal_psf}                                                                  \\
  };

  \pic (start)[below=0.75cm of template]{start};


  %% input box
  \node (input) [below=0.75cm of start-m, input] {%
    \textsl{N} \RAW{LM_IMAGE_SCI_RAW} or\\
    \textsl{N} \RAW{LM_IMAGE_STD_RAW} or\\
    \textsl{N} \RAW{LM_IMAGE_SKY_RAW}};

  %% algorithm steps
  \node (step_linearity) [below=3.8cm of input, redstep]{%
    Correct non-linearity};

  \node (step_persistence) [below of=step_linearity, redstep]{%
    Correct persistence};

  \node (step_dark) [below of=step_persistence, redstep]{%
    subtract dark};

  \node (step_flat) [below of=step_dark, redstep]{%
    divide by flat};

  \node (step_masks) [below of=step_flat, redstep]{%
    analyse and remove masked regions};

  \pic (stop) [below=2.5cm of step_masks] {stop};


  %% Connections
  \draw [connection_arrow] (template) -- (input);
  \draw [connection_arrow] (input) -- (step_linearity);
  \draw [connection_arrow] (step_linearity) -- (step_persistence);
  \draw [connection_arrow] (step_persistence) -- (step_dark);
  \draw [connection_arrow] (step_dark) -- (step_flat);
  \draw [connection_arrow] (step_flat) -- (step_masks);
  \draw [connection_arrow] (step_masks) -- (stop-t);


  %% External data

  % External input
  \node (connect_bpm) [connection] at ($(input)!0.35!(step_linearity)$) {};
  \node (bpm) [left=of connect_bpm, external] {\EXTCALIB{BADPIX_MAP_2RG}};
  \draw [connection_arrow, dashed] (bpm) -- (connect_bpm);

  \node (connect_gain) [connection] at ($(input)!0.55!(step_linearity)$) {};
  \node (gain) [left=of connect_gain, external] {\EXTCALIB{GAIN_MAP_2RG}};
  \draw [connection_arrow] (gain) -- (connect_gain);

  \node (connect_linearity) [connection] at ($(input)!0.75!(step_linearity)$){};
  \node (linearity) [left=of connect_linearity, external]{\EXTCALIB{LINEARITY_det}};
  \draw [connection_arrow] (linearity) -- (connect_linearity);

  \node (connect_persistence) [connection] at ($(step_linearity)!0.5!(step_persistence)$){};
  \node (persistence) [left=of connect_persistence, external]{\EXTCALIB{PERSISTENCE_MAP}};
  \draw [connection_arrow] (persistence) -- (connect_persistence);

%   \node (connectparams) [connection] at
%   ($(input)!0.7!(step_dark.north)$){};
%   \node (params) [left=of connectparams, params]{%
%     Recipe params:\\
%     INS.IMG.SETUP};
%   \draw [connection_arrow] (params) -- (connectparams);

  \node (connect_masterdark) [connection] at
  ($(step_persistence)!0.5!(step_dark)$){};
  \node (masterdark) [left=of connect_masterdark, calproduct]{%
    \STATCALIB{MASTER_DARK_2RG}};
  \draw [connection_arrow] (masterdark) -- (connect_masterdark);

  \node (connect_masterflat) [connection] at
  ($(step_dark)!0.5!(step_flat)$){};
  \node (flatin) [left=of connect_masterflat, calproduct]{%
    \STATCALIB{MASTER_IMG_FLAT_LAMP_LM}\\
    or \STATCALIB{MASTER_IMG_FLAT_TWILIGHT_LM}};
  \draw [connection_arrow] (flatin) -- (connect_masterflat);

%  \node (connect_bpm) [connection] at
%  ($(step_flat)!0.5!(step_masks)$){};
%  \node (bpmin) [left=of connect_bpm, calproduct]{%
%    \STATCALIB{BADPIX_MAP_2RG}};
%  \draw [connection_arrow] (bpmin) -- (connect_bpm);


  % External output
  \node (connect_scired) [connection] at
    ($(step_masks)!0.3!(stop-t)$){};
  \node (scired) [right=of connect_scired, external]{%
    \PROD{LM_SCI_BASIC_REDUCED}};
  \draw [connection_arrow] (connect_scired) -- (scired);

  \node (connect_stdred) [connection] at
    ($(connect_scired)!0.5!(stop-t)$){};
  \node (stdred) [right=of connect_stdred, external]{%
    \PROD{LM_STD_BASIC_REDUCED}};
  \draw [connection_arrow] (connect_stdred) -- (stdred);


  %% Frame around recipe
  \draw [frame] ($(input)!0.15!(step_linearity) -(2.85,0)$) rectangle
  ($(step_masks)!0.75!(stop-t) + (2.85,0)$);
  \node [framecolor, anchor=north west] at
  ($(input)!0.15!(step_linearity) - (2.85,0)$){%
    \REC{metis_lm_img_basic_reduce}};

    
\end{tikzpicture}
\input{normal_style}

  \caption[Recipe: \REC*{metis_lm_img_basic_reduce}]{\REC*{metis_lm_img_basic_reduce} --
    basic reduction of \CODE{IMG_LM} data.}
  \label{fig:metis_lm_img_basic_reduce}
\end{figure}
\restoregeometry

%------------------------------------------------------------------------------------------------------------------
\clearpage
\subsubsection{\REC*{metis_lm_img_background}: Background subtraction}
\label{lm_img_background}
\label{rec:lm_img_background}
\label{sssec:lm_img_background}
\label{metis_lm_img_background}
\label{rec:metis_lm_img_background}
\label{sssec:metis_lm_img_background}

This recipe estimates and subtracts the background from LM-band
imaging data. Thermal background emission from the atmosphere,
telescope and warm parts of the instrument dominate the photon count
in mid-infrared observations. Accurate determination and removal of
background counts is therefore crucial to make MIR data scientifically
usable.

A set of observations will consist of a number of exposures
of the field, where the offsets are achieved by either using the internal
chopper of METIS or by telescope nodding. For extended objects, the
telescope will be used to perform ``out-of-field nodding'', i.e.\
observe nearby blank patches of sky interlaced with the target
observations. Imaging observations are performed in pupil-tracking
mode, hence angular offsetting of the field is automatic.

For in-field-offset exposures, all offset exposures will be
averaged to obtain the background estimate. In order to only average
the background contribution, an iterative procedure of object
detection and masking will be employed. Averaging will be done using a
robust estimator of the mean (e.g.\ median).

For extended objects, all out-of-field exposures will be averaged
(with object rejection) and subtracted off the in-field exposures.

For more information see Section~\ref{sssec:lmbandsbackgroundsubtracion}.

% Moved to https://github.com/AstarVienna/METIS_DRLD/issues/99
% \TODO{Object catalogues of the target exposures could be created within this
% recipe or in a separate recipe. The catalogue should contain for each
% object: pixel coordinates ($x$, $y$), world coordinates ($\alpha$,
% $\delta$) based on telescope pointing and derotator information, total
% counts within an aperture.}
%
% \TODO{Is this good enough for HCI images or do we need more?}

\begin{recipedef}
  Name:             & \REC{metis_lm_img_background}                             \\
  Purpose:          & estimate and subtract background                          \\
  Requirements:     & \REQ{METIS-6085} and \REQ{METIS-6086} \\
  Type:             & Calibration                                               \\
% HB 20230710: I've commented these all out because this recipe is not triggered directly by a template.
%  Templates:        & \TPL{METIS_img_lm_cal_standard}                           \\
%%                     & \TPL{METIS_img_lm_*_obs_*}                                \\
%                    & \TPL{METIS_img_lm_obs_AutoJitter} \\
%                    & \TPL{METIS_img_lm_obs_GenericOffset} \\
%                    & \TPL{METIS_img_lm_obs_FixedSkyOffset} \\
%                    & \TPL{METIS_img_lm_app_obs_FixedOffset} \\
%                    & \TPL{METIS_img_lm_vc_obs_FixedSkyOffset} \\
%                    & \TPL{METIS_img_lmn_obs_AutoChopNod} \\
%                    & \TPL{METIS_img_lmn_obs_GenericChopNod} \\
  Input data:       & \PROD{LM_SCI_BASIC_REDUCED}                               \\
                    & \PROD{LM_STD_BASIC_REDUCED}                               \\
  Matched keywords: & \FITS{DRS.FILTER} \\
  Algorithm:        & Average all or \CODE{SKY} exposures with object rejection \\
                    & Subtract background                                       \\
  Output data:      & \PROD{LM_SCI_BKG}                                         \\
                    & \PROD{LM_STD_BKG}                                         \\
                    & \PROD{LM_SCI_BKG_SUBTRACTED}                              \\
                    & \PROD{LM_STD_BKG_SUBTRACTED}                              \\
                    & \PROD{LM_SCI_OBJECT_CAT}                                  \\
                    & \PROD{LM_STD_OBJECT_CAT}                                  \\
  QC1 parameters:   & \QC{QC LM IMG BKG MEDIAN}                                 \\
                    & \QC{QC LM IMG BKG MEDIAN DEVIATION}                       \\
  hdrl functions:   & \CODE{hdrl_imagelist_sub_image}                           \\
                    & \CODE{hdrl_imagelist_div_image}                           \\
                    & \CODE{hdrl_catalogue_compute}                             \\
\end{recipedef}

\begin{figure}[hb]
    \centering
    \def \globalscale {0.700000}
    \fontsize{10}{12}\selectfont
    \documentclass[tikz, margin=5mm]{standalone}

\input{recipe_config}

\begin{document}

\begin{tikzpicture}
  [x=1cm,
  y=-1cm,
  align=center,
  node distance=2cm and 3.5cm]
  \sffamily

  %% template names
  \node (template) [template] {%
    METIS\_img\_lm\_cal\_standard\\
    METIS\_img\_lm\_*\_obs\_*};

  \pic (start)[below=0.75cm of template]{start};


  %% input box
  \node (input) [below=0.75cm of start-m, input] {%
    \textsl{N} N\_SCI\_BASIC\_REDUCED\\
    \textsl{N} N\_STD\_BASIC\_REDUCED};


  %% algorithm steps
  \node (step1) [below=2.5cm of input, redstep]{%
    Average all or SKY exposures with object rejection};

  \node (step2) [below=2.5cm of step1, redstep]{%
    Subtract background};

  \pic (stop) [below=4.5cm of step2] {stop};


  %% Connections
  \draw (template) -- (input);
  \draw (input) -- (step1);
  \draw (step1) -- (step2);
  \draw (step2) -- (stop-t);


  %% External data

  % External input

  % External output
  \node (dot_sci_bg) [connection] at
    ($(step1)!0.4!(step2)$){};
  \node (sci_bg) [right=of dot_sci_bg, external]{%
    LM\_SCI\_BKG};
  \draw (dot_sci_bg) -- (sci_bg);

  \node (dot_std_bg) [connection] at
    ($(step1)!0.6!(step2)$){};
  \node (std_bg) [right=of dot_std_bg, external]{%
    LM\_STD\_BKG};
  \draw (dot_std_bg) -- (std_bg);


  \node (dot_sci_bg_sub) [connection] at
    ($(step2)!0.25!(stop-t)$){};
  \node (sci_bg_sub) [right=of dot_sci_bg_sub, external]{%
    LM\_SCI\_BKG\_SUBTRACTED};
  \draw (dot_sci_bg_sub) -- (sci_bg_sub);

  \node (dot_std_bg_sub) [connection] at
    ($(step2)!0.4!(stop-t)$){};
  \node (std_bg_sub) [right=of dot_std_bg_sub, external]{%
    LM\_STD\_BKG\_SUBTRACTED};
  \draw (dot_std_bg_sub) -- (std_bg_sub);


  \node (dot_sci_obj_cat) [connection] at
    ($(step2)!0.6!(stop-t)$){};
  \node (sci_obj_cat) [right=of dot_sci_obj_cat, external]{%
    LM\_SCI\_OBJECT\_CAT};
  \draw (dot_sci_obj_cat) -- (sci_obj_cat);

  \node (dot_std_obj_cat) [connection] at
    ($(step2)!0.75!(stop-t)$){};
  \node (std_obj_cat) [right=of dot_std_obj_cat, external]{%
    LM\_STD\_OBJECT\_CAT};
  \draw (dot_std_obj_cat) -- (std_obj_cat);


  %% Frame around recipe
  \draw [frame] ($(input)!0.35!(step1) -(2.85,0)$) rectangle
  ($(step2)!0.85!(stop-t) + (2.85,0)$);
  \node [framecolor, anchor=north west] at
  ($(input)!0.35!(step1) - (2.85,0)$){%
    \textsl{metis\_lm\_img\_background}};


\end{tikzpicture}

\end{document}

    \caption[Recipe: \REC*{metis_lm_img_background}]{\REC*{metis_lm_img_background} --
    background estimation and subtraction of offset \CODE{IMG_LM} exposures.}
    \label{fig:metis_lm_img_background}
\end{figure}

%------------------------------------------------------------------------------------------------------------------
\clearpage

% \subsubsection{LM-band imaging astrometry calibration}
%
% HB: I've commented the lm_img_astrometry_calib recipe out as it seems
%     not necessary. The two outputs as defined are:
%     - LM_STD_AST_CALIB: this is raw data produced by the
%       template METIS_img_lm_cal_standard, and thus should not be the
%       output of a recipe.
%     - ASTROMETRY_TAB: this seems equivalent to LM_DISTORTION_TABLE, or
%       at least very similar.
% \REC replaced with \CODE to placate test scripts.
%
% \label{lm_img_astrometry_calib}
% \label{rec:lm_img_astrometry_calib}
% \label{sssec:lm_img_astrometry_calib}
% \label{metis_lm_img_astrometry_process}
% \label{rec:metis_lm_img_astrometry_process}
% \label{sssec:metis_lm_img_astrometry_process}
% This recipe is the conversion of the pixel coordinates (X,Y) of objects in an image
% into their corresponding celestial coordinates of right ascension (RA) and
% declination (Dec). The recipe for astrometry calibration involves several
% steps that must be carried out precisely to achieve accurate results.
%
% The first step of the recipe involves identifying a set of reference stars
% in the image, whose coordinates are well known and can be obtained from a
% catalog. This requires careful selection and verification of the reference stars,
% as their accuracy determines the accuracy of the final calibration.
%
% The next step is to use these reference stars to establish a mapping between
% the image's pixel coordinates and their corresponding celestial coordinates. This
% is achieved through a process called plate solving, which involves solving a set
% of mathematical equations to determine the transformation between pixel and celestial
% coordinates. This process requires careful selection of the appropriate transform function
% and calibration parameters.
%
% \begin{recipedef}
%   Name:                & \REC*{metis_lm_img_astrometry_process}                                       \\
%   Purpose:             & Determine conversion factor between pixel and world coordinates              \\
%   Type:                & Calibration                                                                  \\
%   Templates:           & \TPL{METIS_img_lm_cal_standard}                                              \\
%   Input data:          & \PROD{LM_STD_BKG_SUBTRACTED}                                                 \\
%                        & photometric standard catalogue                                               \\
%   Matched keywords:    & OBJECT ID                                                                    \\
%                        & FILTER ID                                                                    \\
%   Parameters:          & Distortion correction functions                                              \\
%   Algorithm:           & Measure position of stars in pixel coordinates                               \\
%                        & Compute conversion factor to world coordinates                               \\
%                        & Measure and evaluate the error.                                              \\
%   Output data:         & \PROD{LM_STD_AST_CALIB}                                                      \\
%                        & \PROD{ASTROMETRY_TAB}                                                        \\
%   Expected accuracies: &  $>$0.1$\arcsec$                                                                    \\
%   QC1 parameters:      & \QC{QC LM AST X POS ERR}                                                     \\
%                        & \QC{QC LM AST Y POS ERR}                                                     \\
%                        & \QC{QC LM AST RA POS ERR}                                                    \\
%                        & \QC{QC LM AST DEC POS ERR}                                                    \\
%   hdrl function:       & \CODE{hdrl_strehl_compute}                                                   \\
%                        & \CODE{hdrl_catalogue_compute}                                                \\
%                        & \CODE{hdrl_efficiency_compute}                                               \\
%                        & \CODE{hdrl_imagelist_collapse}                                               \\
% \end{recipedef}
%
% \begin{figure}[hb]
%   \centering
%    \includegraphics[width=1.0\textwidth]{metis_lm_img_astrometry_process}
%   %\resizebox{0.6\textwidth}{0.1\textwidth}{\TODO{\fbox{Figure to be done}}}
%   \caption[Recipe: \CODE{metis_lm_img_astrometry_process}]{\CODE{metis_lm_img_astrometry_process} --
%     compute conversion between pixel and sky coordinate}
%   \label{fig:metis_lm_img_astrometry_process}
% \end{figure}
%
%
% \clearpage

\subsubsection{\REC*{metis_lm_img_std_process}:  Photometric standard analysis}
\label{lm_img_photstd}
\label{rec:lm_img_photstd}
\label{sssec:lm_img_photstd}
\label{rec:metis_lm_img_std_process}

This recipe determines the conversion from ADU to physical units from
a set of reduced exposures of a photometric standard star. The flux of
the star is measured in each exposure in ADU, normalised to an
exposure time of 1~second and averaged over all exposures. In
addition, the exposures are stacked (after recentering on the standard
star, but without derotation) and the flux is measured in the combined
image. Comparison to the tabulated brightness of the star in the
observing filter yields the conversion factor from
$\mathrm{ADU\,s^{-1}}$ to $\mathrm{photons\,\,s^{-1}\,cm^{-2}}$.

QC parameter will include estimates of the sensitivity for the
detection of point sources and surface brightness sensitivity
following~\cite{visir_manual}.

\begin{recipedef}
  Name:                & \REC{metis_lm_img_std_process}                                               \\
  Purpose:             & Determine conversion factor between detector counts and physical source flux \\
  Type:                & Calibration                                                                  \\
% HB 20230710: I've commented the template out because this recipe is not triggered directly by a template.
%  Templates:           & \TPL{METIS_img_lm_cal_standard}                                              \\
  Input data:          & \PROD{LM_STD_BKG_SUBTRACTED}                                                 \\
                       & \EXTCALIB{FLUXSTD_CATALOG} (photometric standard catalogue) \\
  Matched keywords:    & \FITS{DRS.FILTER}                                                                 \\
  Parameters:          & None                                                                         \\
  Algorithm:           & Call \DRL{metis_lm_calculate_std_flux} to measure flux in input images                      \\
%                       & call \DRL{recentre_img} to recentre and stack images                         \\
                       & call \CODE{hdrl_resample_compute} to recenter the images \\
                       & call \CODE{hdrl_imagelist_collapse} to stack the images \\
                       & call \DRL{metis_lm_calculate_std_flux} on the stacked image to get flux of the star in detector units\\
                       & call \DRL{metis_calculate_std_fluxcal} to calculate the conversion factor to physical units    \\
                       & call \DRL{metis_calculate_detection_limits} to compute measure background noise (std,rms) and compute detection limits \\
  Output data:         & \PROD{LM_STD_COMBINED}                                                       \\
                       & \PROD{FLUXCAL_TAB}                                                           \\
  Expected accuracies: & 3\% (cf.~\cite{METIS_calerrbudget})                                          \\
  QC1 parameters:      & \QC{QC LM IMG STD BACKGD RMS}                                                \\
                       & \QC{QC LM STD PEAK CNTS}                                                     \\
                       & \QC{QC LM STD APERTURE CNTS}                                                 \\
                       & \QC{QC LM STD STREHL}                                                        \\
                       & \QC{QC LM STD FLUXCONV}                                                      \\
                       & \QC{QC LM STD AIRMASS}                                                       \\
                       & \QC{QC LM SENS}                                                       \\
                       & \QC{QC LM AREA SENS}                                                  \\
  hdrl functions:      & \CODE{hdrl_strehl_compute}                                                   \\
                       & \CODE{hdrl_catalogue_compute}                                                \\
                       & \CODE{hdrl_efficiency_compute}                                               \\
                       & \CODE{hdrl_imagelist_collapse}                                               \\
\end{recipedef}

\begin{figure}[hb]
    \centering
    \def \globalscale {0.700000}
    \fontsize{10}{12}\selectfont
    \input{tikz/metis_lm_img_std_process}
  \caption[Recipe: \REC*{metis_lm_img_std_process}]{\REC*{metis_lm_img_std_process} --
    compute conversion between ADU and physical flux units}
  \label{fig:metis_lm_img_std_process}
\end{figure}

%%%%%%%%%%%%%%%%%%%%%%%%%%%%%%%%%%%%%%%%%%%%%%%%%
%------------------------------------------------------------------------------------------------------------------
\clearpage
\subsubsection{\REC*{metis_lm_img_calibrate}:  Image calibration}
\label{lm_img_calibrate}
\label{rec:metis_lm_img_calibrate}
\label{rec:lm_img_calibrate}
\label{sssec:lm_img_calibrate}

This recipe applies the flux calibration to the reduced science
images and adds geometric calibration data to the FITS header. The
products of this recipe are fully calibrated individual exposures.

Each image is multiplied by the conversion factor such that pixel
values are in units of photons per second per centimetre squared. The
header of each file receives keyword \FITS*{BUNIT} with value %
\CODE{'photon.s**(-1).cm**(-2)'}.

% Moved to https://github.com/AstarVienna/METIS_DRLD/issues/98
% \TODO{Other units may be possible, although additional information is
%   needed. For instance,\\ \CODE{photon.s**(-1).cm**(-2).arcsec**(-2)} makes
%   values independent of the pixel scale, but requires a distortion map
%   (variation of pixel scale across the detector). Energy units (erg
%   instead of photons) require knowledge of the spectral energy
%   distribution of the sources, in particular for broad-band filters.}

LM-band imaging observations will be performed in pupil-tracking mode
\cite{METIS-operational_concept}, which means that the field rotates
from exposure to exposure.  The information about the field
orientation along with target coordinates, pixel scale and
higher-order polynomial distortion coefficients is written to the FITS
header. The images are not resampled by this recipe, this is left to
 \REC{metis_lm_img_sci_postprocess}.



\begin{recipedef}
  Name:              & \REC{metis_lm_img_calibrate}                     \\
  Purpose:           & Convert science images to physical units         \\
                     & Add distortion information                       \\
  Type:              & Calibration                                      \\
  Templates          & None                                             \\
  Input data:        & \PROD{LM_SCI_BKG_SUBTRACTED}                     \\
                     & \PROD{FLUXCAL_TAB}                               \\
                     & \PROD{LM_DISTORTION_TABLE}                       \\
  Matched keywords:  & \FITS{DRS.FILTER} \\
  Parameters:        & None                                             \\
  Algorithm:         & call \DRL{metis_lm_scale_image_flux} to Scale image data to ph/s \\
                     & add header information (\FITS*{BUNIT}, WCS, etc.) \\
  Output data:       & \PROD{LM_SCI_CALIBRATED}                         \\
  QC1 parameters:    & None                                             \\
  hdrl functions:    & \CODE{hdrl_imagelist_mult_scalar}                \\
\end{recipedef}

\begin{figure}[hb]
  \centering
  \def \globalscale {0.700000}
  \fontsize{10}{12}\selectfont
  \input{tikz/metis_lm_img_calibrate}
  \caption[Recipe: \REC*{metis_lm_img_calibrate}]{\REC*{metis_lm_img_calibrate} --
    Convert images to physical flux units and update FITS header}
  \label{fig:metis_lm_img_calibrate}
\end{figure}


%%%%%%%%%%%%%%%%%%%%%%%%%%%%
%------------------------------------------------------------------------------------------------------------------
\clearpage
\subsubsection{\REC*{metis_lm_img_sci_postprocess}:  Image post-processing}
\label{lm_img_postprocess}
\label{rec:lm_img_postprocess}
\label{sssec:lm_img_postprocess}
\label{rec:metis_lm_img_sci_postprocess}

This recipe coadds a sequence of flux-calibrated,
background-subtracted images (possibly from several observing blocks)
after resampling the images on a common pixel grid defined by a
standard sky projection.
% The alignment of the images (\FITS{CRVAL} keywords, rotation) may have to be checked and refined through cross-correlation of the overlapping images (TBC).
The alignment of the images is done through by using the position of the \ac{WFS-FS} mirror.
% TODO: Decide how the WFS-FS data is stored in the raw data.
% See https://github.com/AstarVienna/METIS_DRLD/issues/162
The number of input
images contributing to any pixel in the output image (variable due to
offsets and bad pixels) will be documented in a contribution
map.

The output files fulfill the \ac{SDP} criteria and are compliant with \REQ{METIS-6104}.

\begin{recipedef}
  Name:                & \REC{metis_lm_img_sci_postprocess}                         \\
  Purpose:             & Coadd reduced images.                                      \\
  Requirements:        & \REQ{METIS-6104}                                           \\
  Templates:           & None                                                       \\
  Type:                & Science                                                    \\
  Input data:          & \PROD{LM_SCI_CALIBRATED} (Calibrated science images)       \\
%                       & \EXTCALIB{BADPIX_MAP_2RG} (Associated bad-pixel maps)           \\
  Matched keywords:  & \FITS{DRS.FILTER} \\
  Parameters:          & None                                                       \\
  Algorithm:           & Check and refine WCS of input images by using the \ac{WFS-FS} data. \\
                       & Determine output pixel grid encompassing all input images. \\
                       & Call \CODE{hdrl_resample_compute} to recenter the images. \\
                       & Call \CODE{hdrl_imagelist_collapse} to stack the images. \\
  Output data:         & \PROD{LM_SCI_COADD} (coadded, mosaiced image)              \\
% TheLM_SCI_COADD_ERROR and LM_SCI_COADD_CONTRIB can be layers in LM_SCI_COADD
%                        & \PROD{LM_SCI_COADD_ERROR} (coadded, mosaiced error image)  \\
%                        & \PROD{LM_SCI_COADD_CONTRIB} (contribution map)             \\
  Expected accuracies: & n/a                                                       \\
  QC1 parameters:      & \QC{QC LM SCI NEXP}                                   \\
                       &  \QC{QC LM SCI POSTPROC GRIDRNG}     \\
                       &  \QC{QC LM SCI POSTPROC MEDMEAN}     \\
                       &  \QC{QC LM SCI POSTPROC MEDRMS}     \\
                       &  \QC{QC LM SCI POSTPROC MEDMED}     \\
                       & \QC{QC LM SCI POSTPROC DELTAC}            \\
\end{recipedef}

\begin{figure}[hb]
    \centering
    \def \globalscale {0.700000}
    \fontsize{10}{12}\selectfont
    \documentclass[tikz, margin=5mm]{standalone}

\input{recipe_config}

\begin{document}

\begin{tikzpicture}
  [x=1cm,
  y=-1cm,
  align=center,
  node distance=2cm and 3cm]
  \sffamily

  %% Grid for orientation. Comment out for final figure!
  %\draw[help lines, green](-5, 0) grid (8, 11);

  %%% Put workflow commands here:
  %% Main reduction workflow
  \pic (start) {start};

  \node (input) [below=0.75cm of start-m, input, text width=4cm, fill=sciproductcolor]{%
    \textsl{N}~LM\_SCI\_REDUCED\\
    \textsl{N}~LM\_SCI\_BADPIX};

  \node (step1) [below=1.5cm of input, redstep]{%
    determine output grid};

  \node (step2) [below=1.5cm of step1, redstep]{%
    resample images};

  \node (step3) [below=1.5cm of step2, redstep]{%
    coadd images};

  \pic (stop) [below=3cm of step3] {stop};

  %% Connections
  \draw (start-m) -- (input);
  \draw (input) -- (step1);
  \draw (step1) -- (step2);
  \draw (step2) -- (step3);
  \draw (step3) -- (stop-t);

  %% Output
  \node (connectcoadd) [connection] at
  ($(step3)!0.33!(stop-t)$) {};
  \node (coadd) [right=of connectcoadd, sciproduct,
  text width=4.5cm]{%
    LM\_SCI\_COADD};
  \draw (connectcoadd) -- (coadd);

  \node (connectcontrib) [connection] at
  ($(step3)!0.6!(stop-t)$) {};
  \node (contrib) [right=of connectcontrib, sciproduct,
  text width=4.5cm]{%
    LM\_SCI\_COADD\_CONTRIB};
  \draw (connectcontrib) -- (contrib);


  %% Frame around recipe
  \draw [frame]
  ($(input)!0.5!(step1) - (4.5, 0)$) rectangle
  ($(step3)!0.85!(stop-t) + (2.5, 0)$);
  \node [framecolor, anchor=north west] at
  ($(input)!0.5!(step1) - (4.5, 0)$){%
    \textsl{metis\_lm\_img\_sci\_postprocess}};

\end{tikzpicture}

\end{document}

  \caption[Recipe: \REC*{metis_lm_img_sci_postprocess}]{%
    \REC{metis_lm_img_sci_postprocess} -- post-processing (coaddition)
    of reduced \CODE{IMG_LM} science frames.}
  \label{fig:metis_lm_img_sci_postprocess}
\end{figure}

%%%%%%%%%%%%%%%%%%%%%%%%%%%%%%%%%%%%%%%%%%%%
%------------------------------------------------------------------------------------------------------------------
\clearpage
\subsubsection{\REC*{metis_lm_img_distortion}:  Distortion calibration}
\label{rec:metis_lm_img_distortion}
\label{lm_img_distortion}
\label{rec:lm_img_distortion}
\label{sssec:lm_img_distortion}

Calibration of the imaging distortion is done on an image of a
pin-hole grid mask located in the \ac{WCU}. The
distortion is described in terms of a polynomial model whose
coefficients can be transformed to WCS keywords and applied to any
other pipeline product. In addition to the distortion table, a map of
pixel scale across the detector will be created.

\begin{recipedef}
  Name:                & \REC{metis_lm_img_distortion}                                   \\
  Purpose:             & Determine optical distortion coefficients for the LM imager.    \\
  Requirements:        & \REQ{METIS-6087}                                                \\
  Templates:           & \TPL{METIS_img_lm_cal_distortion}                               \\
  Type:                & Calibration                                                     \\
  Input data:          & \RAW{LM_DISTORTION_RAW} (Images of grid mask in WCU-FP2 or CFO-FP2.)\\
                       & \RAW{LM_WCU_OFF_RAW} \\
                       & \EXTCALIB{BADPIX_MAP_2RG} (optional) \\
                       & \EXTCALIB{PERSISTENCE_MAP} \\
                       & \STATCALIB{LINEARITY_2RG} \\
                       & \STATCALIB{GAIN_MAP_2RG} \\
                       & \EXTCALIB{PINHOLE_TABLE} (Grid of pinhole mask positions) \\
  Matched keywords:  & \FITS{DRS.FILTER} \\
%                       & \EXTCALIB{BADPIX_MAP_2RG} \\
  Parameters:          & Parameters for fitting routine      \\
%                       & \TBD \\
  Algorithm:           & Subtract background image.    (\CODE{hdrl_imagelist_sub_image})                                  \\
                       & Measure location of point source images in frames (\CODE{hdrl_catalogue_create})             \\
                       & call \DRL{metis_fit_distortion} to fit polynomial coefficients to deviations from grid positions.  \\
  Output data:         & \PROD{LM_DISTORTION_TABLE} \\
                       & \PROD{LM_DISTORTION_MAP}        \\
                       & \PROD{LM_DIST_REDUCED}               \\
  Expected accuracies: & $10^{-3}$ (cf.~\cite{METIS_calerrbudget})                                                    \\
  QC1 parameters:      & \QC{QC LM DISTORT RMS}                                          \\
                       & \QC{QC LM DISTORT NSOURCE}  \\
  hdrl functions:      & \CODE{hdrl_catalogue_create}                                    \\
                       & \CODE{hdrl_imagelist_sub_image}                                \\
\end{recipedef}


\newgeometry{bottom=0.5cm, right=0.5cm, left=0.5cm}
\begin{figure}[hb]
    \centering
    \def \globalscale {0.700000}
    \fontsize{10}{12}\selectfont
    \input{black_style}
\input{recipe_config}

\begin{tikzpicture}
  [x=1cm,
  y=-1cm,
  align=center,
  node distance=2cm and 3cm]
  \sffamily

  %% Grid for orientation. Comment out for final figure!
  % \draw[help lines, green](-5, 0) grid (8, 11);

  %%% Put workflow commands here:
  %% Main reduction workflow

  %% template names
  \node (template) [template] {\TPL{METIS_img_lm_cal_distortion}};

  \pic (start)[below=0.75cm of template]{start};

  \node (input) [below=0.75cm of start-m, input] {%
    \RAW{LM_DISTORTION_RAW}
  };

  \node (step_subtract) [below=2.5cm of input, redstep]{%
    subtract WCU OFF dark
  };

  \node (step_locate) [below=2.5cm of step_subtract, redstep]{%
    locate images};

  \node (step_fit) [below=1.5cm of step_locate, redstep]{%
    fit polynomial};

  \pic (stop) [below=3cm of step_fit]{stop};

  %% Input
  \node (connect_wcuoff) [connection] at
  ($(input)!0.65!(step_subtract)$) {};
  \node (wcuoff) [left=of connect_wcuoff, input] {\RAW{LM_WCU_OFF_RAW}};
  \draw (wcuoff) -- (connect_wcuoff);

%  \node (connect_bpmin) [connection] at ($(step_subtract)!0.3!(step_locate)$) {};
%  \node (bpmin) [left=of connect_bpmin, calproduct] {\STATCALIB{BADPIX_MAP_2RG}};
%  \draw (bpmin) -- (connect_bpmin);

  \node (connect_pinhole) [connection] at
  ($(step_subtract)!0.7!(step_locate)$) {};
  \node (pinhole) [left=of connect_pinhole, external] {\EXTCALIB{PINHOLE_TABLE}};
  \draw (pinhole) -- (connect_pinhole);

  %% Connections
  \draw (start-m) -- (input);
  \draw (input) -- (step_subtract);
  \draw (step_subtract) -- (step_locate);
  \draw (step_locate) -- (step_fit);
  \draw (step_fit) -- (stop-t);

  %% Output
  \node (connectdisttable) [connection] at
  ($(step_fit)!0.25!(stop-t)$) {};
  \node (disttable) [right=of connectdisttable, calproduct, minimum width=4cm]{%
    \STATCALIB{LM_DISTORTION_TABLE}};
  \draw (connectdisttable) -- (disttable);

  \node (connectdistmap) [connection] at
  ($(step_fit)!0.5!(stop-t)$) {};
  \node (distmap) [right=of connectdistmap, calproduct, minimum width=4cm]{%
    \STATCALIB{LM_DISTORTION_MAP}};
  \draw (connectdistmap) -- (distmap);

  \node (connectdistreduced) [connection] at
  ($(step_fit)!0.75!(stop-t)$) {};
  \node (distreduced) [right=of connectdistreduced, calproduct, minimum width=4cm]{%
    \PROD{LM_DIST_REDUCED}};
  \draw (connectdistreduced) -- (distreduced);

  %% Frame around recipe
  \draw [frame]
  ($(input)!0.25!(step_subtract) - (2.75,0)$) rectangle
  ($(step_fit)!0.85!(stop-t) + (2.5,0)$);
  \node [framecolor, anchor=north west] at
  ($(input)!0.25!(step_subtract) - (2.75, 0)$){%
    \REC{metis_lm_img_distortion}};

\end{tikzpicture}
\input{normal_style}

    \caption[Recipe: \REC*{metis_lm_img_distortion}]{%
    \REC{metis_lm_img_distortion} -- LM IMG distortion calibration}
  \label{fig:metis_lm_img_distortion}
\end{figure}
\restoregeometry

\FloatBarrier

%%% Local Variables:
%%% TeX-master: "METIS_DRLD"
%%% End:


\subsection{N-band imaging}
\label{ssec:recipes_img_n}

\subsubsection{N-band imaging flatfield}
\label{sssec:n_img_flatfield}

The purpose of the flat-field calibration is to determine
pixel-to-pixel gain variations and large scale illumination variations
(due to inhomogeneities of optical elements in the telescope or
instrument). Calibration frames are obtained either during day time
using the black-body lamp of the \ac{WCU} (internal flats) or by taken
images of the twilight sky (twilight flats). Advantages and
disadvantages of the two types of flat are discussed in
\cite{METIS-calibration_plan}.

MIR detectors are typically unstable in that they show gain
fluctuations on rather short time scales, hence science exposures may
have a different flat-field structure from those captured by the
calibration flats.  While the GeoSnap detector is expected to be more
stable than the AQUARIUS detector, its stability properties need to be
studied further in order to assess whether science images can be flat
fielded.  N-band flat fields will be taken in any case for quality
control and monitoring purposes.

Since the operational concept for twilight flats needs to be refined
during commissioning at the telescope, the current recipe design is
primarily valid for internal flats.

This recipe creates a master flat for the GeoSnap detector (N-band
imaging) from lamp or sky images matched by various setup parameters
as detailed below.  A set of internal flats includes a number of
exposures with \CODE{LAMP OFF}, which will be used for dark
subtraction. For twilight flats a master dark will be subtracted. The
master flat is obtained by the slope of a linear fit of the pixel
values against the illumination level of the exposures.

The quality control parameters give various statistics for each input
frame (mean, standard deviation, etc.), the standard deviation of the
normalised master flat and the number of bad pixels identified by the
recipe. If a bad-pixel map is provided on input, it is updated,
otherwise a new one is created.

\begin{recipedef}
  Name:                & \REC{metis_n_img_flat}                                         \\
  Purpose:             & Create master flat field for the N-band imaging detector.      \\
  Requirements:        & \REQ{METIS-6098}                                               \\
  Type:                & Calibration                                                    \\
  Templates:           & \TPL{METIS_img_n_cal_InternalFlat}                             \\
                       & \TPL{METIS_all_cal_TwilightFlat}                               \\
  Input data:          & Flat field images taken with lamp or sky.                      \\
                       & Master dark (for twilight flats)                               \\
                       & Bad pixel map                                                  \\
  Matched keywords:    & Detector ID                                                    \\
                       & Filter ID                                                      \\
                       & ADC ID                                                         \\
                       & possibly others (e.g.\ coronagraphic mask, \TBD)               \\
  Parameters:          & Combination method (\texttt{mean}, \texttt{median},
                         \texttt{sigclip}, \dots)                                       \\
                       & Parameters for combination methods                             \\
                       & Threshold for bad-pixel identification                         \\
  Algorithm:           & For internal flats: combine \CODE{LAMP OFF} exposures to dark. \\
                       & Subtract internal dark or master dark from flat exposures.     \\
                       & Fit slope of pixel values against illumination level.          \\
                       & Add pixels with significant deviations to bad pixel map.       \\
  Output data:         & \PROD{MASTER_IMG_FLAT_GEO}                                     \\
                       & \PROD{BADPIX_MAP_GEO}                                          \\
  Expected accuracies: & \TBD                                                           \\
  QC1 parameters:      & \QC{QC N MASTERFLAT RMS}                                       \\
                       & \QC{QC N FLAT NBADPIX}                                         \\
                       & \QC{QC N FLAT MEAN ##}                                         \\
                       & \QC{QC N FLAT RMS ##}                                          \\
  hdrl functions:      & \CODE{hdrl_bpm_fit_compute}                                    \\
                       & \CODE{hdrl_imagelist_collapse}                                 \\
                       & \CODE{hdrl_imagelist_sub_image}                                \\
\end{recipedef}

\begin{figure}[hb]
  \centering
  \includegraphics[width=0.6\textwidth]{metis_n_img_flat}
  \caption[Recipe: \REC{metis_n_img_flat}]{\REC{metis_n_img_flat} --
    creation of \CODE{IMG_N} master flatfield}
  \label{fig:metis_n_img_flat}
\end{figure}

%%%%%%%

\subsubsection{N-band imaging chop-nod combination}
\label{sssec:img_n_chopnod}

This recipe combines a set of exposures taken at all positions of a
defined chop-nod pattern and adds/subtracts them into a single
chop/nod difference image. Depending on the actual chop-nod pattern,
this image will contain one or more positive and negative beams.

If flat fielding proves feasible and useful for the GeoSnap detector
the master flat can be applied. If no jitter is applied, i.e.\ if the
beam is at the same detector position for all exposures taken at a
given chop position, then the master flat can be divided into the
final chop-nod difference image. Otherwise, the master flat will have
to be divided into the chop half-cycle images before the jitter
correction is applied.

\begin{recipedef}
  Name:              & \REC{metis_n_img_chopnod}                                    \\
  Purpose:           & chop/nod combination of exposures for background subtraction \\
  Type:              & Calibration, Science                                         \\
 Templates:          & \TPL{METIS_img_n_cal_standard}                              \\
                     & \TPL{METIS_img_n_obs_AutoChopNod}                            \\
                     & \TPL{METIS_img_n_obs_GenericChopNod}                         \\
                     & \TPL{METIS_img_n_cvc_obs_AutoChop}                           \\
%                     & \TPL{METIS_img_n_clc_obs_FixedSkyOffset}                     \\
                     & \TPL{METIS_img_n_cal_psf}                                    \\
  Input data:        & Chopped/nodded science or standard images                    \\
                     & Bad-pixel map                                                \\
  Matched keywords:  & Filter ID                                                    \\
                     & Chop position                                                \\
                     & Nod position                                                 \\
  Parameters:        & TBD                                                          \\
  Algorithm:         & Add/subtract images to subtract background                   \\
  Output data:       & \PROD{N_SCI_BKG_SUBTRACTED}                                  \\
                     & \PROD{N_STD_BKG_SUBTRACTED}                                  \\
  QC1 parameters:    & \QC{N IMG PEAK CNTS}                                         \\
  hdrl functions:    & \CODE{hdrl_imagelist_collapse}                               \\
\end{recipedef}

\begin{figure}[hb]
  \centering
  % \includegraphics[width=0.6\textwidth]{metis_lm_img_std_process}
  \resizebox{0.6\textwidth}{0.1\textwidth}{\TODO{\fbox{Figure to be done}}}
  \caption[Recipe: \REC{metis_n_img_chopnod}]{\REC{metis_n_img_chopnod} --
    Combination of chop/nodded images.}
  \label{fig:metis_n_img_chopnod}
\end{figure}

%%%%%%%%%%%%%%%%%%%
\clearpage
\subsubsection{N-band imaging photometric standard analysis}
\label{n_img_std_process}

This recipe determines the conversion from ADU to physical units from
a chop-nod difference image of a photometric standard star.  The flux
of the standard star is measured in each of the beams of the chop-nod
difference image, averaged and normalised to an exposure time of
1~second. Comparison to the tabulated brightness of the star in the
observing filter yields the conversion factor from
$\mathrm{ADU}\,\mathrm{s}^{-1}$ to
$\mathrm{photons}\,\mathrm{s}^{-1}\,\mathrm{cm}^{-2}$.

QC parameters will include estimates of the sensitivity for the
detection of point sources and surface brightness sensitivity
following \cite{visir_manual}.

\begin{recipedef}
  Name:                & \REC{metis_n_img_std_process}                                                 \\
  Purpose:             & Determine conversion factor between detector counts and physical source flux. \\
  Type:                & Calibration                                                                   \\
  Templates:           & \TPL{METIS_img_n_cal_standard}                                                \\
  Input data:          & \CODE{N_STD_BKG_SUBTRACTED}                                                   \\
                       & photometric standard catalogue                                                \\
  Matched keywords:    & Object ID                                                                     \\
                       & Filter ID                                                                     \\
  Parameters:          & None (TBD)                                                                    \\
  Algorithm:           & Create object catalogue, identify standard star                               \\
                       & Measure flux from star in all beams                                           \\
                       & Compute conversion factor to physical units                                   \\
                       & Measure background noise (rms) and compute detection limits.                  \\
  Output data:         & \PROD{FLUXCAL_TAB}                                                            \\
  Expected accuracies: & \TBD                                                                          \\
  QC1 parameters:      & \QC{QC N STD PEAK CNTS}                                                       \\
                       & \QC{QC N STD APERTURE CNTS}                                                   \\
                       & \QC{QC N STD STREHL}                                                          \\
                       & \QC{QC N STD FLUXCONV}                                                        \\
                       & \QC{QC N STD AIRMASS}                                                         \\
                       & \QC{QC N SENSITIVITY}                                                         \\
                       & \QC{QC N AREA SENSITIVITY}                                                    \\
  hdrl functions:      & \CODE{hdrl_catalogue_create}                                                  \\
                       & \CODE{hdrl_strehl_compute}                                                    \\
\end{recipedef}

\begin{figure}[hb]
  \centering
  % \includegraphics[width=0.6\textwidth]{metis_n_img_std_process}
  \resizebox{0.6\textwidth}{0.1\textwidth}{\TODO{\fbox{Figure to be done}}}
  \caption[Recipe: \REC{metis_n_img_std_process}]{\REC{metis_n_img_std_process} --
    compute conversion between ADU and physical flux units.}
  \label{fig:metis_n_img_std_process}
\end{figure}


%%%%%%%%%%%%%%%%%%%%%
\clearpage

\subsubsection{N-band imaging calibration}
\label{sssec:n_img_calibrate}

This recipe applies the flux calibration to the chop-nod difference
image. A unique geometric calibration is not possible at this point,
although one could take one of the beams (e.g.\ the positive beam in a
parallel two-point chop-nod pattern) as reference for a
WCS. Distortion information can be added without a reference point as
it pertains to the detector/focal plane, not to the field.

The products of this recipe is the fully calibrated chop-nod
difference image.

The image is multiplied by the conversion factor such that pixel
values are in units of photons per second per centimetre squared. The
header receives the keyword \FITS{BUNIT} with value %
\CODE{'photon.s**(-1).cm**(-2)'}.

\begin{recipedef}
  Name:              & \REC{metis_n_img_calibrate}                      \\
  Purpose:           & Convert science image to physical units          \\
                     & Add distortion information                       \\
  Type:              & Calibration                                      \\
  Templates:         &                                                  \\
  Input data:        & \CODE{N_SCI_BKG_SUBTRACTED}                      \\
                     & \CODE{FLUXCAL_TAB}                               \\
                     & \CODE{N_DISTORTION_TABLE}                        \\
  Matched keywords:  & Filter ID                                        \\
  Parameters:        & TBD                                              \\
  Algorithm:         & Scale image data to ph/s                         \\
                     & Add header information (\FITS{BUNIT}, WCS, etc.) \\
  Output data:       & \PROD{N_SCI_CALIBRATED}                          \\
  QC1 parameters:    & None                                             \\
\end{recipedef}

\begin{figure}[hb]
  \centering
  % \includegraphics[width=0.6\textwidth]{metis_n_img_std_process}
  \resizebox{0.6\textwidth}{0.1\textwidth}{\TODO{\fbox{Figure to be done}}}
  \caption[Recipe: \REC{metis_n_img_calibrate}]{\REC{metis_n_img_calibrate} --
    convert image to physical flux units}
  \label{fig:metis_n_img_calibrate}
\end{figure}

%%%%%%%%%%%%%
\clearpage

\subsubsection{N-band imaging restoration}
\label{sssec:n_img_restoration}

This recipe attempts to combine the positive and negative beams of the
chop-nod difference image into a single positive image of the
source. For compact sources with a size smaller than half the distance
between the beams, it suffices to cut out small regions around the
source images and add the with the appropriate signs to obtain a
single image.

Algorithms for image restoration of extended sources exist but it
remains \TBD\ whether these are sufficiently simple and robust to be
included in the pipeline (cf.\ Sect.~8.8 of \cite{DRLS}).

\begin{recipedef}
  Name:              & \REC{metis_n_img_restore}                                     \\
  Purpose:           & Restore a single positive beam from chop-nod difference image \\
  Type:              & Science                                                       \\
  Input data:        & \CODE{N_SCI_CALIBRATED}                                       \\
  Parameters:        & size of cutout region                                         \\
  Algorithm:         & Cut regions around beams                                      \\
                     & Add regions with appropriate signs                            \\
  Output data:       & \PROD{N_SCI_RESTORED}                                         \\
  QC1 parameters:    & None                                                          \\
  hdrl functions:    & \CODE{hdrl_imagelist_collapse}                                \\
\end{recipedef}

\begin{figure}[hb]
  \centering
  % \includegraphics[width=0.6\textwidth]{metis_n_img_std_process}
  \resizebox{0.6\textwidth}{0.1\textwidth}{\TODO{\fbox{Figure to be done}}}
  \caption[Recipe: \REC{metis_n_img_restore}]{\REC{metis_n_img_restore} --
    Create a single positive image from chop-nod difference image}
  \label{fig:metis_n_img_restore}
\end{figure}

%%%%%%%%%%%%%%
\clearpage
\subsubsection{N-band imaging distortion calibration}
\label{sssec:n_img_distortion}

Calibration of the imaging distortion is done on an image of a pin
hole mask located in a focal plane within the instrument. The
distortion is described in terms of a polynomial model whose
coefficients can be transformed to WCS keywords and applied to any
other pipeline product. In addition to the distortion table, a map of
pixel scale across the detector will be created.

\begin{recipedef}
  Name:                & \REC{metis_n_img_distortion}                                   \\
  Purpose:             & Determine optical distortion coefficients for the N imager.    \\
  Templates:           & \TPL{METIS_img_n_cal_distortion}                               \\
  Type:                & Calibration                                                    \\
  Input data:          & Images of grid mask in WCU-FP2 or CFO-FP2.                     \\
                       & Image with WCU window closed (background).                     \\
                       & Bad pixel map                                                  \\
  Parameters:          & TBD                                                            \\
  Algorithm:           & Subtract background image.                                     \\
                       & Measure location of point source images in frames.             \\
                       & Fit polynomial coefficients to deviations from grid positions. \\
  Output data:         & \PROD{N_DISTORTION_TABLE} (table with polynomial coefficients) \\
                       & \PROD{N_DISTORTION_MAP} (pixel scale across detector)          \\
                       & \PROD{N_DIST_REDUCED} (reduced grid mask images)               \\
  Expected accuracies: & TBD                                                            \\
  QC1 parameters:      & \QC{QC N DISTORT RMS}                                          \\
  hdrl functions:      & \CODE{hdrl_catalogue_create}                                   \\
\end{recipedef}

\begin{figure}[hb]
  \centering
  \includegraphics[width=0.6\textwidth]{metis_n_img_distortion}
  \caption[Recipe: \REC{metis_n_img_distortion}]{%
    \REC{metis_n_img_distortion} -- \CODE{IMG_N} distortion calibration}
  \label{fig:metis_n_img_distortion}
\end{figure}

%%% Local Variables:
%%% TeX-master: "METIS_DRLD"
%%% End:

\clearpage
\subsection{Long-slit spectroscopy, LM band}
\label{ssec:recipes_lss_lm}

A draft of the reduction cascade is shown in
Fig.~\ref{Fig:LMLssAssomap} together with the data processing table
(Table~\ref{Tab:LMLssDatProc}). The first part aims to update the static calibration database, in particular the creation of the gain map (\hyperref[Sec:detector_calibration]{\REC{metis_det_lingain}}) and the determination of the \ac{ADC} slitlosses (\hyperref[rec:metislmadcmslitloss]{\REC{metis_lm_adc_slitloss}}). These are executed only when an update is required, e.g. after a major instrument interention or on yearly basis. The second part comprises the basic calibrations, e.g. the dark correction and the spectroscopic flatfielding via \ac{RSRF}, followed by the third part, the main calibration steps, incorporating the determination of the first guess wavelength solution by means of the laser sources in the \ac{WCU} and the determination of the response curve for the flux calibration. Subsequently, the main reduction is conducted, which applies the previously created master calibration files to the science frames. Both, the flux standard and the science observations are wavelength calibrated with the help of the atmospheric lines visible in the respective spectra. Therefore the main step of the wavelength calibration is carried out in the recipes \hyperref[rec:lsslmflux]{\REC{metis_LM_lss_flux}} and \hyperref[rec:lsslmsci]{\REC{metis_LM_lss_sci}}. Finally, the telluric absorption correction is applied using the modelling approach with \texttt{molecfit}.


%------------------------------------------------------------------------------------------------------------------
\subsubsection{Recipes \REC{metis\_det\_lingain} and \REC{metis\_det\_dark}}
These recipes are described in Section~\ref{Sec:detector_calibration}.

%------------------------------------------------------------------------------------------------------------------
\subsubsection{Recipe \REC{metis\_LM\_adc\_slitloss}}
The recipe \hyperref[sssec:adc_slitlosses]{\REC{metis_lm_adc_slitloss}} aims to determine the slit losses induced by the fixed \ac{ADC} positions as function of the object position across the slit. The recipe aims to create a table with slitlosses (\hyperref[dataitem:lmadcslitloss]{\STATCALIB{LM_ADC_SLITLOSS}}), which is added to the static database and used in the recipes \hyperref[rec:lsslmflux]{\REC{metis_LM_lss_flux}}. This recipe is to be carried out only when an update of the database is needed. The algorithm and the workflow of the recipe to determine the slitlosses is given in Section~\ref{sssec:adc_slitlosses}, more information can be found in Section "Calibration of slit losses" in the Calibration Plan \cite{METIS-calibration_plan}. 


%------------------------------------------------------------------------------------------------------------------
\subsubsection{LM-LSS Flatfielding recipe \REC{metis\_LM\_lss\_rsrf}:}\label{rec:lsslmrsrf}
The recipe \hyperref[rec:lsslmrsrf]{\REC{metis_LM_lss_rsrf}} aims to create a spectroscopic master flatfield for determining the pixel-to-pixel sensitivity and to enable the order location algorithm (\hyperref[rec:lsslmtrace]{\REC{metis_LM_lss_trace}}).
\begin{figure}[ht]
  \centering
  \includegraphics[width=0.5\textheight]{figures/metis_lm_lss_rsrf_v0.74.pdf}
  \caption[Recipe: \REC{metis\_LM\_lss\_rsrf}]{\REC{metis\_LM\_lss\_rsrf} --
    Recipe workflow to create the spectroscopic flatfield by means of the \ac{RSRF}.}
  \label{Fig:rec_lm_lss_rsrf}
\end{figure}

\begin{recipedef}
Name:		& \hyperref[rec:lsslmrsrf]{\REC{metis_LM_lss_rsrf}}  \\
Purpose:	& Spectroscopic flatfielding with \ac{RSRF} \\
Type:		& Calibration\\
Requirements: & None \\
Templates:           & \TPL{METIS_spec_lm_cal_rsrf} \\
Input data:     & $N\times$ \hyperref[dataitem:lmlsswaveraw]{\RAW{LM_LSS_RSRF_RAW}} \\
                & \hyperref[dataitem:persistencemap]{\EXTCALIB{PERSISTENCE_MAP}}  \\
                & \hyperref[dataitem:gainmap2rg]{\STATCALIB{GAIN_MAP_2RG}}  \\
                & \hyperref[dataitem:badpixmap2rg]{\PROD{BADPIX_MAP_2RG}}  \\
                & \hyperref[dataitem:masterdark2rg]{\PROD{MASTER_DARK_2RG}}  \\
Parameters: 	& TBD\\
Algorithm:      & subtract master \ac{WCU} "OFF" frame from illumination frame (done on individual images)\\
                & median/mean filtering of subtracted images\\
                & division by blackbody spectrum\\
                & normalisation to achieve \ac{RSRF}\\
Output data:	& \hyperref[dataitem:lsslmrsrfmaster]{\PROD{MASTER\_LM\_LSS\_RSRF}} (\FITS{PRO.CATG=MASTER_LM_LSS_RSRF}): \\
                & \hyperref[dataitem:medianlmrsrfimg]{\PROD{MEDIAN_LM_LSS_RSRF_IMG}}\\
                & \hyperref[dataitem:meanlmrsrfimg]{\PROD{MEAN_LM_LSS_RSRF_IMG}}\\
Expected accuracies: & 3\% (cf. \cite{METIS-calibration_plan} and \cite{METIS_calerrbudget})\\
QC1 parameters: & \hyperref[qc:lmlssrsrfmeanlevel]{\QC{QC LM LSS RSRF MEAN LEVEL}}: Mean level of the \ac{RSRF}\\
                & \hyperref[qc:lmlssrsrfmedianlevel]{\QC{QC LM LSS RSRF MEDIAN LEVEL}}: Median level of the \ac{RSRF}\\
                & \hyperref[qc:lmlssrsrfintordrlevel]{\QC{QC LM LSS RSRF INTORDR LEVEL}}: Flux level of the interorder background\\
                & \hyperref[qc:lmlssrsrfnormstdev]{\QC{QC LM LSS RSRF NORM STDEV}}: Standard deviation of the normalised \ac{RSRF}\\
                & \hyperref[qc:lmlssrsrfnormsnr]{\QC{QC LM LSS RSRF NORM SNR}}: \ac{SNR} of the normalised \ac{RSRF}\\
                & more TBD\\
\end{recipedef}
\clearpage

%------------------------------------------------------------------------------------------------------------------
\subsubsection{LM-LSS Order detection \REC{metis\_LM\_lss\_trace}:}\label{rec:lsslmtrace}
The recipe \hyperref[rec:lsslmtrace]{\REC{metis_LM_lss_trace}} aims at detecting the orders and a polynomial fitting of the order locations (see \cite{pis02} and \cite{pis21} for details on the algorithms). The detection and polynomial fitting is based on flatfield frames taken through a pinhole mask, which leads to individual pinhole traces along the entire dispersion direction.

\begin{figure}[ht]
  \centering
  \includegraphics[width=0.5\textheight]{figures/metis_lm_lss_trace_v0.74.pdf}
  \caption[Recipe: \REC{metis_LM_lss_trace}]{\REC{metis_LM_lss_trace} --
    Detection and polynomial fitting of the order location.}
  \label{Fig:rec_lm_lss_wtrace}
\end{figure}

\begin{recipedef}
Name:		&  \hyperref[rec:lsslmtrace]{\REC{metis_LM_lss_trace}} \\
Purpose:	& Detection of order location \\
Type:		& Calibration\\
Requirements: & None \\
Templates:           & \TPL{METIS_spec_lm_cal_rsrfpinh}  \\
Input data:     & $N\times$ \hyperref[dataitem:lmlssrsrfpinhraw]{\RAW{LM_LSS_RSRF_PINH_RAW}} \\
                & \hyperref[dataitem:persistencemap]{\EXTCALIB{PERSISTENCE_MAP}}  \\
                & \hyperref[dataitem:gainmap2rg]{\STATCALIB{GAIN_MAP_2RG}}  \\
                & \hyperref[dataitem:badpixmap2rg]{\PROD{BADPIX_MAP_2RG}}  \\
                & \hyperref[dataitem:masterdark2rg]{\PROD{MASTER_DARK_2RG}}  \\
                & \hyperref[dataitem:lsslmrsrfmaster]{\PROD{MASTER\_LM\_LSS\_RSRF}} \\
Parameters: 	& polynomial degree\\
Algorithm:      & Detection of the order edges\\
                & Polynomial fitting\\
Output data:	& \hyperref[dataitem:lmlsstrace]{\PROD{LM_LSS_TRACE}} (\FITS{PRO.CATG=LM_LSS_TRACE}): Polynomial coefficients\\
Expected accuracies: & (TBD)\\
QC1 parameters: & \hyperref[qc:lmlsstracelpolydeg]{\QC{QC LM LSS TRACE LPOLYDEG}}: Degree of the polynomial fit of the left order edge\\
                & \hyperref[qc:lmlsstracelcoeffi]{\QC{QC LM LSS TRACE LCOEFF<i>}}: $i$-th coefficient of the polynomial of the left order edge\\
                & \hyperref[qc:lmlsstracerpolydeg]{\QC{QC LM LSS TRACE RPOLYDEG}}: Degree of the polynomial fit of the right order edge\\
                & \hyperref[qc:lmlsstracercoeffi]{\QC{QC LM LSS TRACE RCOEFF<i>}}: $i$-th coefficient of the polynomial of the right order edge\\
                & \hyperref[qc:lmlsstraceintrordrlevel]{\QC{QC LM LSS TRACE INTORDR LEVEL}}: Flux level of the interorder background\\
                & more TBD\\
\end{recipedef}

\clearpage
%------------------------------------------------------------------------------------------------------------------
\subsubsection{LM-LSS wavelength calibration recipe \REC{metis\_LM\_lss\_wave}:}\label{rec:lsslmwave}
This recipe aims at determining the first guess of the wavelength calibration on basis of the \ac{WCU} laser sources (c.f. \cite{METIS-calibration_plan}). Therefore the first steps are the removal of the detector signature of the \FITS{LM_WAVE_RAW} frames by applying the master calibration files derived in the previous steps, following by the background subtraction (if needed, TBD) and the application of the RSRF. The distortion of the lines (i.e. possible tilt, curvature,...) and the wavelength solution is determined by the algorithm developed by Piskunov et al. (\cite{pis02}, \cite{pis21}). The reference frame is defined by the laser line catalogue (\hyperref[dataitem:lasertab]{\STATCALIB{LASER_TAB}}).

\begin{figure}[ht]
  \centering
  \includegraphics[width=0.5\textheight]{figures/metis_lm_lss_wave_v0.74.pdf}
  \caption[Recipe: \REC{metis\_LM\_lss\_wave}]{\REC{metis\_LM\_lss\_wave} --
    Creation of the LM LSS master wavelength correction.}
  \label{Fig:rec_lm_lss_trace}
\end{figure}
\clearpage

\begin{recipedef}
Name:		& \hyperref[rec:lsslmwave]{\REC{metis_LM_lss_wave}} \\
Purpose:	& Wavelength calibration \\
Type:		& Calibration\\
Requirements: & METIS-6084, METIS-1371, METIS-6074 \\
Templates:           & \TPL{METIS_spec_lm_cal_internalwave}, \\
Input data: 	& \hyperref[dataitem:lmlsswaveraw]{\RAW{LM_LSS_WAVE_RAW}}\\
                & \hyperref[dataitem:persistencemap]{\EXTCALIB{PERSISTENCE_MAP}}  \\
                & \hyperref[dataitem:gainmap2rg]{\STATCALIB{GAIN_MAP_2RG}}  \\
                & \hyperref[dataitem:badpixmap2rg]{\PROD{BADPIX_MAP_2RG}}  \\
                & \hyperref[dataitem:masterdark2rg]{\PROD{MASTER_DARK_2RG}}  \\
                & \hyperref[dataitem:lsslmrsrfmaster]{\PROD{MASTER\_LM\_LSS\_RSRF}} \\
                & \hyperref[dataitem:lmlsstrace]{\PROD{LM_LSS_TRACE}} \\
                & \hyperref[dataitem:lasertab]{\STATCALIB{LASER_TAB}} \\
                % & \STATCALIB{REF_AIRG_CAT} \\
Parameters: 	& (TBD)\\
Algorithm:      & Application of detector master calibration files\\
                & Determination and application of the distortion correction\\
                & Determination of the first guess of the wavelength solution by polynomial fit of the detected laser source lines\\
Output data:	& \hyperref[dataitem:lmlsscurve]{\PROD{LM_LSS_CURVE}} (\FITS{PRO.CATG=LM_LSS_CURVE}): Curvature \\
                & \hyperref[dataitem:lmlssdistsol]{\PROD{LM_LSS_DIST_SOL}} (\FITS{PRO.CATG=LM_LSS_DIST_SOL}): Distortion solution\\
                & \hyperref[dataitem:lmlsswaveguess]{\PROD{LM_LSS_WAVE_GUESS}} (\FITS{PRO.CATG=LM_LSS_WAVE_GUESS}): Wavelength first guess\\
Expected accuracies: & 1/5th of a pixel after post-processing (cf. \cite{METIS-calibration_plan})\\
QC1 parameters: & \hyperref[qc:lmlsswavepolydeg]{\QC{QC LM LSS WAVE POLYDEG}}: Degree of the first guess polynomial\\
                & \hyperref[qc:lmlsswavecoeffi]{\QC{QC LM LSS WAVE COEFF<i>}}: $i$-th coefficient of the polynomial\\
                & \hyperref[qc:lmlsswavenlines]{\QC{QC LM LSS WAVE NLINES}}: Number of detected (laser) lines; should be constant\\
                & \hyperref[qc:lmlsswavelinefwhmavg]{\QC{QC LM LSS WAVE LINEFWHMAVG}}: Average of the \ac{FWHM} of the detected lines (should be widely constant)\\
                & \hyperref[qc:lmlsswaveinterordrlevel]{\QC{QC LM LSS WAVE INTORDR LEVEL}}: Flux level of the interorder background\\
                & more TBD: e.g. QC params for distortion determination and correction\\
\end{recipedef}

\clearpage
%------------------------------------------------------------------------------------------------------------------
\subsubsection{LM-LSS flux calibration recipe \REC{metis_LM_lss_flux}:}\label{rec:lsslmflux}
Flux calibration with spectrophotometric standard stars: As first step the detector master calibration files derived previously are applied followed by the background subtraction, if needed the distortion correction (\hyperref[dataitem:lmlssdistsol]{\PROD{LM_LSS_DIST_SOL}}), and
the wavelength calibration by means of the first guess solution (\hyperref[dataitem:lmlsswaveguess]{\PROD{LM_LSS_WAVE_GUESS}}) and the telluric sky lines (c.f. Sect.\,8.5 in \cite{DRLS}). Then the recipe extracts the standard star spectrum object, removes sky lines, collapses the 2D to 1D spectra and applies a telluric correction in an automated way to the standard star spectrum (in contrast to the science observations, which are telluric corrected in a dedicated recipe to achieve the best correction). The response curve is obtained by comparing the extracted spectrum with a model and/or another reference spectrum of the standard star. Currently it is foreseen to use the same standard stars as in \ac{CRIRES}/CRIRES+ and \ac{VISIR}. It is under investigation whether more stars are needed.
\begin{figure}[ht]
  \centering
  \includegraphics[width=0.4\textheight]{figures/metis_lm_lss_flux_v0.74.pdf}
  \caption[Recipe: \REC{metis_LM_lss_flux}]{\REC{metis_LM_lss_flux} --
    Flux calibration recipe.}
  \label{Fig:rec_lm_lss_flux}
\end{figure}
\clearpage
\begin{recipedef}
Name:		& \hyperref[rec:lsslmflux]{\REC{metis_LM_lss_flux}} \\
Purpose:	& Flux calibration \\
Type:		& Calibration\\
Requirements: & METIS-6084, METIS-6074 \\
Templates:           & \TPL{METIS_spec_lm_cal_standard}\\
Input data: 	& \hyperref[dataitem:lmlssfluxraw]{\RAW{LM_LSS_FLUX_RAW}}\\
                & \hyperref[dataitem:persistencemap]{\EXTCALIB{PERSISTENCE_MAP}}  \\
                & \hyperref[dataitem:gainmap2rg]{\STATCALIB{GAIN_MAP_2RG}}  \\
                & \hyperref[dataitem:badpixmap2rg]{\PROD{BADPIX_MAP_2RG}}  \\
                & \hyperref[dataitem:masterdark2rg]{\PROD{MASTER_DARK_2RG}}  \\
                & \hyperref[dataitem:lsslmrsrfmaster]{\PROD{MASTER\_LM\_LSS\_RSRF}} \\
                & \hyperref[dataitem:lmlssdistsol]{\PROD{LM_LSS_DIST_SOL}} \\
                & \hyperref[dataitem:lmlsswaveguess]{\PROD{LM_LSS_WAVE_GUESS}} \\
                & \hyperref[dataitem:aopsfmodel]{\EXTCALIB{AO_PSF_MODEL}} \\
                & \hyperref[dataitem:atmlinecat]{\EXTCALIB{ATM_LINE_CAT}} \\
                & \hyperref[dataitem:lmadcslitloss]{\STATCALIB{LM_ADC_SLITLOSS}}\\
                & \hyperref[dataitem:lmsynthtrans]{\STATCALIB{LM_SYNTH_TRANS}}\\
                & \hyperref[dataitem:reffluxcat]{\STATCALIB{REF_FLUX_CAT}} \\
Parameters: 	& (TBD)\\
Algorithm:      & Application of master calibration files\\
                & Background removal\\
                & Determination and application of the distortion correction\\
                & Determination and application of the wavelength solution\\
                & Identifying/separatiing sky/object pixels\\
                & Removing sky lines: Creation and Subtraction of 2D sky\\
                & Collapsing 2D to 1D spectrum, (see Fig.\,\ref{Fig:rec_lm_lss_sci})\\
                & Determination and application of response curve\\
Output data:	& \hyperref[dataitem:lmlssstdobjmap]{\PROD{LM_LSS_STD_OBJ_MAP}}: Pixel map of object pixels\\
            	& \hyperref[dataitem:lmlssstdskymap]{\PROD{LM_LSS_STD_SKY_MAP}}: Pixel map of sky pixels\\
              	& \hyperref[dataitem:lmlssstd1d]{\PROD{LM_LSS_STD_1D}}: coadded, wavelength calibrated, collapsed 1D spectrum\\
                & \hyperref[dataitem:lsslmresp]{\PROD{MASTER\_LM\_RESPONSE}}: response function (TBD)\\
Expected accuracies: & 10\% over an atmospheric band (ESO Req. R-MET-107)\\
            & $<30$\% absolute line flux accuracy (R-MET-107)\\
            & $<5$\% absolute flux calibration (R-MET-82)\\
QC1 parameters: & \hyperref[qc:lmlssstdbackgdmean]{\QC{QC LM LSS STD BACKGD MEAN}}: Mean value of background\\
                & \hyperref[qc:lmlssstdbackgdmedian]{\QC{QC LM LSS STD BACKGD MEDIAN}}: Median value of background\\
                & \hyperref[qc:lmlssstdbackgdstdev]{\QC{QC LM LSS STD BACKGD STDEV}}: Standard deviation value of background\\
                & \hyperref[qc:lmlssstdsnr]{\QC{QC LM LSS STD SNR}}: Signal-to-noise ration of flux standard star spectrum\\
                & \hyperref[qc:lmlssstdsnrnoise]{\QC{QC LM LSS STD SNRNOISE}}: Noise level of flux standard star spectrum\\
                & \hyperref[qc:lmlssstdfwhm]{\QC{QC LM LSS STD FWHM}}: FWHM of flux standard spectrum\\
                & \hyperref[qc:lmlssfluxintrordravglevel]{\QC{QC LM LSS FLUX INTORDR LEVEL}}: Flux level of the interorder background\\
                & \hyperref[qc:lmlssfluxlevel]{\QC{QC LM LSS FLUX AVGLEVEL}}: Average level of the standard star flux \\
                & \hyperref[qc:lmlssfluxwavecaldevmean]{\QC{QC LM LSS FLUX WAVECAL DEVMEAN}}: Mean deviation from the
                  wavelength reference frame (TBDef)\\
                & \hyperref[qc:lmlssfluxwavecalfwhm]{\QC{QC LM LSS FLUX WAVECAL FWHM}}: Measured FWHM of lines\\
                & \hyperref[qc:lmlssfluxwavecalnident]{\QC{QC LM LSS FLUX WAVECAL NIDENT}}: Number of identified lines\\
                & \hyperref[qc:lmlssfluxwavecalnmatch]{\QC{QC LM LSS FLUX WAVECAL NMATCH}}: Number of lines matched between
                    catalogue and spectrum\\
                & \hyperref[qc:lmlssfluxwavecalpolydeg]{\QC{QC LM LSS FLUX WAVECAL POLYDEG}}: Degree of the polynomial\\
                & \hyperref[qc:lmlssfluxwavecalpolycoeffn]{\QC{QC LM LSS FLUX WAVECAL POLYCOEFF\<n\>}}: $n$-th coefficient of the polynomial\\
                & \hyperref[qc:lmlssfluxstdsnr]{\QC{QC LM LSS FLUX STDSNR}}: Signal-to-noise ration of flux standard star spectrum\\
                & \hyperref[qc:lmlssfluxsnrnoise]{\QC{QC LM LSS FLUX SNRNOISE}}: Noise level of flux standard star spectrum\\
                & \hyperref[qc:lmlssfluxfwhm]{\QC{QC LM LSS FLUX FWHM}}: FWHM of flux standard spectrum\\
                & \hyperref[qc:lmlssfluxpsfloss]{\QC{QC LM LSS FLUX PSFLOSS}}: Fraction of AO induced slit losses (TBdef)\\
                & more TBD
\end{recipedef}

\subsubsection{LM-LSS science reduction recipe \REC{metis_LM_lss_sci}:}\label{rec:lsslmsci}
The science calibration recipe comprises the extraction of the object (i.e. separation of object/sky pixels), removing the sky lines, the application of the response curve previously defined, the 2D to 1D collapse and the coaddition. In contrast to the flux standard star reduction, the telluric correction on the science data is done in a dedicated recipe afterwards to achieve best quality for the correction.
\begin{figure}[ht]
  \centering
  \includegraphics[width=0.38\textheight]{figures/metis_lm_lss_sci_v0.74.pdf}
  \caption[Recipe: \REC{metis_LM_lss_sci}]{\REC{metis_LM_lss_sci} --
    Science reduction recipe.}
  \label{Fig:rec_lm_lss_sci}
\end{figure}
\clearpage

\begin{recipedef}
Name:		& \hyperref[rec:lsslmsci]{\REC{metis_LM_lss_sci}} \\
Purpose:    & Science data calibration\\
Type:		& Science reduction\\
Requirements: & METIS-6084 \\
Templates:           & \TPL{METIS_spec_lm_acq}, \\
                & \TPL{METIS_spec_lm_obs_AutoNodOnSlit}, \\
                & \TPL{METIS_spec_lm_obs_GenericOffset} \\
                & \TPL{METIS_spec_lm_cal_slit_adc}\\
Input data: 	& \hyperref[dataitem:lmlsssciraw]{\RAW{LM_LSS_SCI_RAW}}\\
                & \hyperref[dataitem:persistencemap]{\EXTCALIB{PERSISTENCE_MAP}}  \\
                & \hyperref[dataitem:gainmap2rg]{\STATCALIB{GAIN_MAP_2RG}}  \\
                & \hyperref[dataitem:badpixmap2rg]{\PROD{BADPIX_MAP_2RG}}  \\
                & \hyperref[dataitem:masterdark2rg]{\PROD{MASTER_DARK_2RG}}  \\
                & \hyperref[dataitem:lsslmrsrfmaster]{\PROD{MASTER\_LM\_LSS\_RSRF}} \\
                & \hyperref[dataitem:lmlssdistsol]{\PROD{LM_LSS_DIST_SOL}} \\
                & \hyperref[dataitem:lmlsswaveguess]{\PROD{LM_LSS_WAVE_GUESS}} \\
                & \hyperref[dataitem:atmlinecat]{\EXTCALIB{ATM_LINE_CAT}} \\
                & \hyperref[dataitem:lmadcslitloss]{\STATCALIB{LM_ADC_SLITLOSS}}\\
                %& \hyperref[dataitem:aopsfmodel]{\EXTCALIB{AO_PSF_MODEL}} \\
                %& \hyperref[dataitem:lsfkernel]{\STATCALIB{LSF_KERNEL}}\\
                & \hyperref[dataitem:lsslmresp]{\PROD{MASTER\_LM\_RESPONSE}} \\
Parameters: 	& (TBD)\\
Algorithm:      & Application of the detector master calib files\\
                & wavelength calibration \\
                & Identifying/separatiing sky/object pixels\\
                & Removing sky lines: Creation and Subtraction of 2D sky\\
                & Coaddition of individual object spectra of one OB\\
                & Collapsing 2D to 1D spectrum, (see Fig.\,\ref{Fig:rec_lm_lss_sci})\\
                & Application of the response function (flux calibration) \\
Output data:	& \hyperref[dataitem:lmlsssciobjmap]{\PROD{LM_LSS_SCI_OBJ_MAP}}: Pixel map of object pixels\\
            	& \hyperref[dataitem:lmlsssciskymap]{\PROD{LM_LSS_SCI_SKY_MAP}}: Pixel map of sky pixels\\
            	& \hyperref[dataitem:lmlsssci2d]{\PROD{LM_LSS_SCI_2D}}: coadded, wavelength calibrated 2D spectrum\\
                & (\FITS{PRO_CATG}: \FITS{LM_LSS_2d_coadd_wavecal}) \\
                & \hyperref[dataitem:lmlsssci1d]{\PROD{LM_LSS_SCI_1D}}: coadded, wavelength calibrated 1D spectrum\\
                & (\FITS{PRO_CATG}: \FITS{LM_LSS_1d_coadd_wavecal}) \\
                & \hyperref[dataitem:lmlsssciflux2d]{\PROD{LM_LSS_SCI_FLUX_2D}}: coadded, wavelength + flux calibrated 2D spectrum\\
                & (\FITS{PRO_CATG}: \FITS{LM_LSS_2d_coadd_wavecal}) \\
              	& \hyperref[dataitem:lmlsssciflux1d]{\PROD{LM_LSS_SCI_FLUX_1D}}: coadded, wavelength + flux 1D spectrum\\
                & (\FITS{PRO_CATG}: \FITS{LM_LSS_1d_coadd_wavecal}) \\
Expected accuracies: & (TBD)\\
QC1 parameters: & \hyperref[qc:lmlssscisnr]{\QC{QC LM LSS SCI SNR}}: Signal-to-noise ration of science spectrum\\
                & \hyperref[qc:lmlssscisnrnoise]{\QC{QC LM LSS SCI SNRNOISE}}: Noise level of science spectrum\\
                & \hyperref[qc:lmlssscifluxsnr]{\QC{QC LM LSS SCI FLUX SNR}}: Signal-to-noise ration of flux calibrated  science spectrum\\
                & \hyperref[qc:lmlssscifluxsnrnoise]{\QC{QC LM LSS SCI FLUX SNRNOISE}}: Noise level of flux calibrated science spectrum\\
                & \hyperref[qc:lmlsssciinterordrlevel]{\QC{QC LM LSS SCI INTORDR LEVEL}}: Flux level of the interorder background\\
                & \hyperref[qc:lmlsssciwavecaldevmean]{\QC{QC LM LSS SCI WAVECAL DEVMEAN}}: Mean deviation from the wavelength reference frame (TBDef)\\
                & \hyperref[qc:lmlsssciwavecalfwhm]{\QC{QC LM LSS SCI WAVECAL FWHM}}: Measured FWHM of lines\\
                & \hyperref[qc:lmlsssciwavecalnident]{\QC{QC LM LSS SCI WAVECAL NIDENT}}: Number of identified lines\\
                & \hyperref[qc:lmlsssciwavecalnmatch]{\QC{QC LM LSS SCI WAVECAL NMATCH}}: Number of lines matched between catalogue and spectrum\\
                & \hyperref[qc:lmlsssciwavecalpolydeg]{\QC{QC LM LSS SCI WAVECAL POLYDEG}}: Degree of the wavelength polynomial\\
                & \hyperref[qc:lmlsssciwavecalpolycoeffn]{\QC{QC LM LSS SCI WAVECAL POLYCOEFF\<n\>}}: $n$-th coefficient of the polynomial\\
                & more TBD\\
\end{recipedef}

\subsubsection{LM-LSS telluric correction recipe \REC{metis_LM_lss_mf_model}:}\label{rec:LMLSSmfmodel}
The telluric correction will be done with the package \texttt{molecfit}\footnote{\url{https://www.eso.org/sci/software/pipelines/molecfit/molecfit-pipe-recipes.html}}. It is realised in three individual recipes, \hyperref[rec:LMLSSmfmodel]{\REC{metis_LM_lss_mf_model}}, which calculates the best-fit model, \hyperref[rec:LMLSSmfcalctrans]{\REC{metis_LM_lss_mf_calctrans}}, which creates a synthetic transmission curve, and \hyperref[rec:LMLSSmfcorrect]{\REC{metis_LM_lss_mf_correct}}, which performs the actual telluric correction by means of the synthetic transmission.

\begin{figure}[ht]
  \centering
  \includegraphics[width=0.5\textheight]{figures/metis_lm_lss_mf_model_v0.74.pdf}
  \caption[Recipe: \REC{metis_LM_lss_mf_model}]{\REC{metis_LM_lss_mf_model} --
    Recipe to achieve the best-fit for the calculation of the synthetic transmission curve for the telluric correction.}
  \label{Fig:rec_lm_lss_mf_model}
\end{figure}
\clearpage

\begin{recipedef}
Name:		& \hyperref[rec:LMLSSmfmodel]{\REC{metis_LM_lss_mf_model}} \\
Purpose:	& Achieve the best fit for modelling the transmission curve to be applied as telluric correction \\
Type:		& Post-calibration\\
Requirements: & METIS-4051, METIS-6091 \\
Templates:           & None\\
Input data: 	& \hyperref[dataitem:lmlsssciflux1d]{\PROD{LM_LSS_SCI_FLUX_1D}}\\
                & \hyperref[dataitem:lsfkernel]{\STATCALIB{LSF_KERNEL}} \\
                & \hyperref[dataitem:atmprofile]{\EXTCALIB{ATM_PROFILE}} \\
                & \hyperref[dataitem:atmlinecat]{\EXTCALIB{ATM_LINE_CAT}} \\
Parameters: 	& \texttt{molecfit} parameters (c.f. \cite{molecfit})\\
Algorithm:      & Fit of telluric features visible in the science input spectrum\\
                & Determination of best-fit parameter set\\
Output data:	& \hyperref[dataitem:mfbestfittab]{\PROD{MF\_BEST\_FIT\_TAB}}: Table with best-fit parameters\\
Expected accuracies: & (TBD)\\
QC1 parameters: & cf. \cite{molecfit}\\
\end{recipedef}

\subsubsection{LM-LSS telluric correction recipe \REC{metis_LM_lss_mf_calctrans}:}\label{rec:LMLSSmfcalctrans}

\begin{figure}[ht]
  \centering
  \includegraphics[width=0.5\textheight]{figures/metis_lm_lss_mf_calctrans_v0.74.pdf}
  \caption[Recipe: \REC{metis_LM_lss_mf_calctrans}]{\REC{metis_LM_lss_mf_calctrans} --
    Recipe to calculate the synthetic transmission to be applied as telluric correction.}
  \label{Fig:rec_lm_lss_mf_calctrans}
\end{figure}
\clearpage

\begin{recipedef}
Name:		& \hyperref[rec:LMLSSmfcalctrans]{\REC{metis_LM_lss_mf_calctrans}} \\
Purpose:	& Calculation of the synthetic transmission \\
Type:		& Post-calibration\\
Requirements: & METIS-4051, METIS-6091 \\
Templates:           & None\\
Input data: 	& \hyperref[dataitem:mfbestfittab]{\PROD{MF\_BEST\_FIT\_TAB}}: Table with best-fit parameters\\
                & \hyperref[dataitem:lsfkernel]{\STATCALIB{LSF_KERNEL}} \\
                & \hyperref[dataitem:atmprofile]{\EXTCALIB{ATM_PROFILE}} \\
                & \hyperref[dataitem:atmlinecat]{\EXTCALIB{ATM_LINE_CAT}} \\
Parameters: 	& \texttt{molecfit} parameters (c.f.  \cite{molecfit})\\
Algorithm:      & Calculate the entire transmission curve by means of the best-fit parameters\\
Output data:	& \hyperref[dataitem:lmlsssynthttrans]{\PROD{LM_LSS_SYNTH_TRANS}}: synth. transmission\\
Expected accuracies: & (TBD)\\
QC1 parameters: & cf. \cite{molecfit}\\
\end{recipedef}

\subsubsection{LM-LSS telluric correction recipe \REC{metis_LM_lss_mf_correct}:}\label{rec:LMLSSmfcorrect}

\begin{figure}[ht]
  \centering
  \includegraphics[width=0.5\textheight]{figures/metis_lm_lss_mf_correct_v0.74.pdf}
  \caption[Recipe: \REC{metis_LM_lss_mf_correct}]{\REC{metis_LM_lss_mf_correct} --
    Recipe to apply the telluric correction.}
  \label{Fig:rec_lm_lss_mf_correct}
\end{figure}
\clearpage

\begin{recipedef}
Name:		& \hyperref[rec:LMLSSmfcorrect]{\REC{metis_LM_lss_mf_correct}} \\
Purpose:	& Apply the synthetic transmission to the science spectra \\
Type:		& Post-calibration\\
Requirements: & METIS-4051, METIS-6091 \\
Templates:           & None\\
Input data: 	& \hyperref[dataitem:lmlsssciflux1d]{\PROD{LM_LSS_SCI_FLUX_1D}}\\
                & \hyperref[dataitem:lmlsssynthttrans]{\PROD{LM_LSS_SYNTH_TRANS}}\\
Parameters: 	& None\\
Algorithm:      & Apply telluric correction, i.e. divide the input science spectrum\\
                & by the synthetic transmission\\
Output data:	& \hyperref[dataitem:lmlssscifluxtellcorr1d]{\PROD{LM_LSS_SCI_FLUX_TELLCORR_1D}}\\
Expected accuracies: & (TBD)\\
QC1 parameters: & cf. \cite{molecfit}\\
\end{recipedef}




\clearpage
\subsection{Long-slit spectroscopy, N band}
\label{ssec:recipes_lss_n}
A draft of the reduction cascade is shown in
Figs.~\ref{Fig:NLssAssomap1} and \ref{Fig:NLssAssomap2} together with the data processing table
(Tables~\ref{Tab:NLssDatProc1} and ~\ref{Tab:NLssDatProc2}). The first part aims to update the static calibration database, in particular the creation of the gain map (\hyperref[Sec:detector_calibration]{\REC{metis_det_lingain}}) and the determination of the \ac{ADC} slitlosses (\hyperref[rec:metisnadcmslitloss]{\REC{metis_n_adc_slitloss}}). Both are executed only when an update is required, e.g. after a major instrument interention or on yearly basis. The second part comprises the basic
calibrations, e.g. the 
dark correction and the spectroscopic flatfielding via \ac{RSRF}, followed by the third part, the main calibration steps, incorporating the wavelength calibration (by means of atmospheric lines visible in the respective spectra and the first guess wavelength solution created during \ac{AIT}) and the determination of the response curve for the flux calibration. Therefore the main step of the wavelength calibration is carried out in the recipes \hyperref[rec:lssnstd]{\REC{metis_N_lss_std}} and \hyperref[rec:lssnsci]{\REC{metis_LM_lss_sci}}. Finally, the telluric absorption correction is applied using the modelling approach with \texttt{molecfit}.
%------------------------------------------------------------------------------------------------------------------
\subsubsection{Recipes \REC{metis_det_lingain} and \REC{metis_det_dark}}
These recipes are described in Section~\ref{Sec:detector_calibration}.
%------------------------------------------------------------------------------------------------------------------
\subsubsection{Recipe \REC{metis_N_adc_slitloss}}\label{rec:metis_n_adc_slitloss}
The recipe \hyperref[sssec:adc_slitlosses]{\REC{metis_n_adc_slitloss}} aims to determine the slit losses induced by the fixed \ac{ADC} positions as function of the object position across the slit. The recipe aims to create a table with slitlosses (\hyperref[dataitem:n_adc_slitloss]{\STATCALIB{N_ADC_SLITLOSS}}), which is added to the static database and used in the recipes \hyperref[rec:lssnstd]{\REC{metis_N_lss_std}}. This recipe is to be carried out only when an update of the database is needed. The algorithm and the workflow of the recipe to determine the slitlosses is given in Section~\ref{sssec:adc_slitlosses}, more information can be found in Section "Calibration of slit losses" in the Calibration Plan \cite{METIS-calibration_plan}. 

%------------------------------------------------------------------------------------------------------------------
\subsubsection{N-LSS Flatfielding recipe \REC{metis_N_lss_rsrf}:}\label{rec:lssnrsrf}
The recipe \REC{metis_N_lss_rsrf} aims to create a spectroscopic master flatfield for determining the pixel-to-pixel sensitivity and to enable the order location algorithm (\REC{metis_N_ss_trace}).
\begin{figure}[ht]
  \centering
  \includegraphics[width=0.5\textheight]{figures/metis_n_lss_rsrf_v0.82.pdf}
  \caption[Recipe: \REC{metis_N_lss_rsrf}]{\REC{metis_N_lss_rsrf} --
    Spectroscopic faltfielding with \ac{RSRF}.}
  \label{Fig:rec_n_lss_rsrf}
\end{figure}

\begin{recipedef}
Name:		& \hyperref[rec:lssnrsrf]{\REC{metis_N_lss_rsrf}} \\
Purpose:	& Spectroscopic flatfielding with \ac{RSRF} \\
Type:		& Calibration\\
Requirements: & None \\
Templates:           & \TPL{METIS_spec_n_cal_rsrf} \\
Input data:     & $N\times$ \hyperref[dataitem:n_lss_rsrf_raw]{\RAW{N_LSS_RSRF_RAW}} \\
                & \hyperref[dataitem:persistence_map]{\EXTCALIB{PERSISTENCE_MAP}}  \\
                & \hyperref[dataitem:gain_map_n]{\STATCALIB{GAIN_MAP_N}}  \\
                & \hyperref[dataitem:badpix_map_n]{\PROD{BADPIX_MAP_N}}  \\
                & \hyperref[dataitem:master_dark_n]{\PROD{MASTER_DARK_N}}  \\
Parameters: 	& TBD\\
Algorithm:      & subtract master \ac{WCU} "OFF" frame from illumination frame (done on individual images)\\
                & median/mean filtering of subtracted images\\
                & division by blackbody spectrum\\
                & normalisation to achieve \ac{RSRF}\\
Output data:	&  \hyperref[dataitem:master_n_lss_rsrf]{\PROD{MASTER_N_LSS_RSRF}} (\FITS{PRO.CATG=MASTER_N_LSS_TRSRF}): \\
                & \hyperref[dataitem:median_n_lss_rsrf_img]{\PROD{MEDIAN_N_LSS_RSRF_IMG}}\\
                & \hyperref[dataitem:mean_n_lss_rsrf_img]{\PROD{MEAN_N_LSS_RSRF_IMG}}\\

Expected accuracies: & 3\% (cf. \cite{METIS-calibration_plan} and \cite{METIS_calerrbudget})\\
QC1 parameters: & \hyperref[qc:nlssrsrfmeanlevel]{\QC{QC N LSS RSRF MEAN LEVEL}}: Mean level of the \ac{RSRF}\\
                & \hyperref[qc:nlssrsrfmedianlevel]{\QC{QC N LSS RSRF MEDIAN LEVEL}}: Median level of the \ac{RSRF}\\
                & \hyperref[qc:nlssrsrfintordrlevel]{\QC{QC N LSS RSRF INTORDR LEVEL}}: Flux level of the interorder background\\
                & \hyperref[qc:nlssrsrfnormstdev]{\QC{QC N LSS RSRF NORM STDEV}}: Standard deviation of the normalised \ac{RSRF}\\
                & \hyperref[qc:nlssrsrfnormsnr]{\QC{QC N LSS RSRF NORM SNR}}: \ac{SNR} of the normalised \ac{RSRF}\\
                & more TBD\\
\end{recipedef}

\clearpage
\subsubsection{N-LSS Order detection \REC{metis_N_lss_trace}:}\label{rec:lssntrace}
The recipe \REC{metis_N_lss_trace} aims at detecting the orders and a polynomial fitting of the order locations (see \cite{pis02} and \cite{pis21} for details on the algorithms). The detection and polynomial fitting is based on flatfield frames taken through a pinhole mask, which leads to individual pinhole traces along the entire dispersion direction.

\begin{figure}[ht]
  \centering
  \includegraphics[width=0.5\textheight]{figures/metis_N_lss_trace_v0.82.pdf}
  \caption[Recipe: \REC{metis_N_lss_trace}]{\REC{metis_N_lss_trace} --
    Detection and polynomial fitting of the order location.}
  \label{Fig:rec_N_lss_wave}
\end{figure}

\begin{recipedef}
Name:		& \REC{metis_N_lss_trace} \\
Purpose:	& Detection of order location \\
Type:		& Calibration\\
Requirements: & None \\
Templates:           & \TPL{METIS_spec_n_cal_InternalWave} \\
Input data:     & $N\times$ \hyperref[dataitem:n_lss_rsrf_pinh_raw]{\RAW{N_LSS_RSRF_PINH_RAW}} \\
                & \hyperref[dataitem:persistence_map]{\EXTCALIB{PERSISTENCE_MAP}}  \\
                & \hyperref[dataitem:gain_map_n]{\STATCALIB{GAIN_MAP_N}}  \\
                & \hyperref[dataitem:badpix_map_n]{\PROD{BADPIX_MAP_N}}  \\
                & \hyperref[dataitem:master_dark_n]{\PROD{MASTER_DARK_N}}  \\
                &  \hyperref[dataitem:master_n_lss_rsrf]{\PROD{MASTER_N_LSS_RSRF}} \\
Parameters: 	& polynomial degree\\
Algorithm:      & Detection of the order edges\\
                & Polynomial fitting\\
Output data:	& \hyperref[dataitem:n_lss_trace]{\PROD{N_LSS_TRACE}} (\FITS{PRO.CATG=N_LSS_TRACE}): Polynomial coefficients\\
Expected accuracies: & (TBD)\\
QC1 parameters: & \hyperref[q:nlsstracelpolydeg]{\QC{QC N LSS TRACE LPOLYDEG}}: Degree of the polynomial fit of the left order edge\\
                & \hyperref[q:nlsstracelcoeffi]{\QC{QC N LSS TRACE LCOEFF<i>}}: $i$-th coefficient of the polynomial of the left order edge\\
                & \hyperref[q:nlsstracerpolydeg]{\QC{QC N LSS TRACE RPOLYDEG}}: Degree of the polynomial fit of the right order edge\\
                & \hyperref[q:nlsstracercoeffi]{\QC{QC N LSS TRACE RCOEFF<i>}}: $i$-th coefficient of the polynomial of the right order edge\\
                & \hyperref[q:nlsstraceintrordrlevel]{\QC{QC N LSS TRACE INTORDR LEVEL}}: Flux level of the interorder background\\
                & \QC{TBD}: TBD\\
\end{recipedef}

\clearpage

%------------------------------------------------------------------------------------------------------------------
\subsubsection{N-LSS standard star recipe \REC{metis_N_lss_std}:}\label{rec:lssnstd}
Flux calibration with spectrophotometric standard stars: As first step the detector master calibration files derived previously are applied followed by the background subtraction, if needed the distortion correction (\PROD{N_DIST_SOL}), and
the wavelength calibration by means of the first guess solution (\PROD{N_WAVE_GUESS}) and the telluric sky lines (c.f. Sect.\,8.5 in \cite{DRLS}). Then the recipe extracts the standard star spectrum object, removes sky lines, collapses the 2D to 1D spectra and applies a telluric correction in an automated way to the standard star spectrum (in contrast to the science observations, which are telluric corrected in a dedicated recipe to achieve the best correction). The response curve is obtained by comparing the extracted spectrum with a model and/or another reference spectrum of the standard star. Currently it is foreseen to use the same standard stars as in \ac{CRIRES}/CRIRES+ and \ac{VISIR}. It is under investigation whether more stars are needed.
\begin{figure}[ht]
  \centering
  \includegraphics[width=0.4\textheight]{figures/metis_n_lss_std_v0.82.pdf}
  \caption[Recipe: \REC{metis_N_lss_std}]{\REC{metis_N_lss_std} --
    Standard star calibration recipe.}
  \label{Fig:rec_N_lss_flux}
\end{figure}
\clearpage
\begin{recipedef}
Name:		& \REC{metis_N_lss_std} \\
Purpose:	& Flux calibration \\
Type:		& Calibration\\
Requirements: & METIS-6084, METIS-6074 \\
Templates:           & \TPL{METIS_spec_N_cal_standard}\\
Input data: 	& \hyperref[dataitem:n_lss_flux_raw]{\RAW{N_LSS_FLUX_RAW}}\\
                & \hyperref[dataitem:persistence_map]{\EXTCALIB{PERSISTENCE_MAP}}  \\
                & \hyperref[dataitem:gain_map_n]{\STATCALIB{GAIN_MAP_N}}  \\
                & \hyperref[dataitem:badpix_map_n]{\PROD{BADPIX_MAP_N}}  \\
                & \hyperref[dataitem:master_dark_n]{\PROD{MASTER_DARK_N}}  \\
                & \hyperref[dataitem:master_n_lss_rsrf]{\PROD{MASTER_N_LSS_RSRF}} \\
                & \hyperref[dataitem:n_lss_trace]{\PROD{N_LSS_TRACE}}\\
                & \hyperref[dataitem:n_lss_dist_sol]{\STATCALIB{N_LSS_DIST_SOL}} \\
                & \hyperref[dataitem:n_lss_wave_guess]{\STATCALIB{N_LSS_WAVE_GUESS}} \\
                & \hyperref[dataitem:n_synth_trans]{\STATCALIB{N_SYNTH_TRANS}}\\
                & \hyperref[dataitem:ao_psf_model]{\EXTCALIB{AO_PSF_MODEL}} \\
                & \hyperref[dataitem:atm_line_cat]{\EXTCALIB{ATM_LINE_CAT}} \\
                & \hyperref[dataitem:n_adc_slitloss]{\STATCALIB{N_ADC_SLITLOSS}}\\
                & \hyperref[dataitem:ref_std_cat]{\STATCALIB{REF_STD_CAT}} \\                
Parameters: 	& (TBD)\\
Algorithm:      & Application of master calibration files\\
                & Background removal\\
                & Determination and application of the distortion correction\\
                & Determination and application of the wavelength solution\\
                & Identifying/separatiing sky/object pixels\\
                & Removing sky lines: Creation and Subtraction of 2D sky\\
                & Collapsing 2D to 1D spectrum, (see Fig.\,\ref{Fig:rec_N_lss_sci})\\
                & Determination and application of response curve\\
Output data:	& \hyperref[dataitem:n_lss_std_obj_map]{\PROD{N_LSS_STD_OBJ_MAP}}: Pixel map of object pixels\\
            	& \hyperref[dataitem:n_lss_std_sky_map]{\PROD{N_LSS_STD_SKY_MAP}}: Pixel map of sky pixels\\
              	& \hyperref[dataitem:n_lss_std_1d]{\PROD{N_LSS_STD_1D}}  : coadded, wavelength calibrated, collapsed 1D spectrum\\
                & \hyperref[dataitem:master_n_response]{\PROD{MASTER_N_RESPONSE}}: response function (TBD)\\
Expected accuracies: & 10\% over an atmospheric band (ESO Req. R-MET-107)\\
            & $<30$\% absolute line flux accuracy (R-MET-107)\\
            & $<5$\% absolute flux calibration (R-MET-82)\\
QC1 parameters: & \hyperref[qc:nlssstdbackgdmean]{\QC{QC N LSS STD BACKGD MEAN}}: Mean value of background\\
                & \hyperref[qc:nlssstdbackgdmedian]{\QC{QC N LSS STD BACKGD MEDIAN}}: Median value of background\\
                & \hyperref[qc:nlssstdbackgdstdev]{\QC{QC N LSS STD BACKGD STDEV}}: Standard deviation value of background\\
                & \hyperref[qc:nlssstdsnr]{\QC{QC N LSS STD SNR}}: Signal-to-noise ration of flux standard star spectrum\\
                & \hyperref[qc:nlssstdsnrnoise]{\QC{QC N LSS STD SNRNOISE}}: Noise level of flux standard star spectrum\\
                & \hyperref[qc:nlssstdfwhm]{\QC{QC N LSS STD FWHM}}: FWHM of flux standard spectrum\\
                & \hyperref[qc:nlssfluxintrordravglevel]{\QC{QC N LSS FLUX INTORDR LEVEL}}: Flux level of the interorder background\\
                & \hyperref[qc:nlssfluxlevel]{\QC{QC N LSS FLUX AVGLEVEL}}: Average level of the standard star flux \\
                & \hyperref[qc:nlssfluxwavecaldevmean]{\QC{QC N LSS FLUX WAVECAL DEVMEAN}}: Mean deviation from the
                  wavelength reference frame (TBDef)\\
                & \hyperref[qc:nlssfluxwavecalfwhm]{\QC{QC N LSS FLUX WAVECAL FWHM}}: Measured FWHM of lines\\
                & \hyperref[qc:nlssfluxwavecalnident]{\QC{QC N LSS FLUX WAVECAL NIDENT}}: Number of identified lines\\
                & \hyperref[qc:nlssfluxwavecalnmatch]{\QC{QC N LSS FLUX WAVECAL NMATCH}}: Number of lines matched between
                    catalogue and spectrum\\
                & \hyperref[qc:nlssfluxwavecalpolydeg]{\QC{QC N LSS FLUX WAVECAL POLYDEG}}: Degree of the polynomial\\
                & \hyperref[qc:nlssfluxwavecalpolycoeffn]{\QC{QC N LSS FLUX WAVECAL POLYCOEFF\<n\>}}: $n$-th coefficient of the polynomial\\
                & \hyperref[qc:nlssfluxstdsnr]{\QC{QC N LSS FLUX STDSNR}}: Signal-to-noise ration of flux standard star spectrum\\
                & \hyperref[qc:nlssfluxsnrnoise]{\QC{QC N LSS FLUX SNRNOISE}}: Noise level of flux standard star spectrum\\
                & \hyperref[qc:nlssfluxfwhm]{\QC{QC N LSS FLUX FWHM}}: FWHM of flux standard spectrum\\
                & \hyperref[qc:nlssfluxpsfloss]{\QC{QC N LSS FLUX PSFLOSS}}: Fraction of AO induced slit losses (TBdef)\\
                & more TBD\\
\end{recipedef}

\subsubsection{N-LSS science reduction recipe \REC{metis_N_lss_sci}:}\label{rec:lssnsci}
The science calibration recipe comprises the extraction of the object (i.e. separation of object/sky pixels), removing the sky lines, the application of the response curve previously defined, the 2D to 1D collapse and the coaddition. In contrast to the flux standard star reduction, the telluric correction on the science data is done in a dedicated recipe afterwards to achieve best quality for the correction.
\begin{figure}[ht]
  \centering
  \includegraphics[width=0.38\textheight]{figures/metis_N_lss_sci_v0.82.pdf}
  \caption[Recipe: \REC{metis_N_lss_sci}]{\REC{metis_N_lss_sci} --
    Science reduction recipe.}
  \label{Fig:rec_N_lss_sci}
\end{figure}
\clearpage

\begin{recipedef}
Name:		& \REC{metis_N_lss_sci} \\
Purpose:  & Science data calibration\\
Type:		& Science reduction\\
Requirements: & METIS-6084 \\
Templates:           & \TPL{METIS_spec_N_acq}, \\
                & \TPL{METIS_spec_n_obs_AutoChopNodOnSlit}, \\
                & \TPL{METIS_spec_N_obs_GenericOffset} \\
                & \TPL{METIS_spec_n_cal_slit}\\
Input data: 	& \hyperref[dataitem:n_lss_sci_raw]{\RAW{N_LSS_SCI_RAW}}\\
                & \hyperref[dataitem:persistence_map]{\EXTCALIB{PERSISTENCE_MAP}}  \\
                & \hyperref[dataitem:gain_map_n]{\STATCALIB{GAIN_MAP_N}}  \\
                & \hyperref[dataitem:badpix_map_n]{\PROD{BADPIX_MAP_N}}  \\
                & \hyperref[dataitem:master_dark_n]{\PROD{MASTER_DARK_N}}  \\
                & \hyperref[dataitem:master_n_lss_rsrf]{\PROD{MASTER_N_LSS_RSRF}} \\
                & \hyperref[dataitem:n_lss_trace]{\PROD{N_LSS_TRACE}}\\
                & \hyperref[dataitem:n_lss_dist_sol]{\STATCALIB{N_LSS_DIST_SOL}}\\
                & \hyperref[dataitem:n_lss_wave_guess]{\STATCALIB{N_LSS_WAVE_GUESS}}\\
                & \hyperref[dataitem:atm_line_cat]{\EXTCALIB{ATM_LINE_CAT}} \\
                & \hyperref[dataitem:ao_psf_model]{\EXTCALIB{AO_PSF_MODEL}} \\
                & \hyperref[dataitem:n_adc_slitloss]{\STATCALIB{N_ADC_SLITLOSS}}\\
                & \hyperref[dataitem:master_n_response]{\PROD{MASTER_N_RESPONSE}} \\
Parameters: 	& (TBD)\\
Algorithm:      & Application of the detector master calib files\\
                & wavelength calibration \\
                & Identifying/separatiing sky/object pixels\\
                & Removing sky lines: Creation and Subtraction of 2D sky\\
                & Coaddition of individual object spectra of one OB\\
                & Collapsing 2D to 1D spectrum, (see Fig.\,\ref{Fig:rec_N_lss_sci})\\
                & Application of the response function (flux calibration) \\
Output data:	& \hyperref[dataitem:n_lss_sci_obj_map]{\PROD{N_LSS_SCI_OBJ_MAP}}: Pixel map of object pixels\\
            	& \hyperref[dataitem:n_lss_sci_sky_map]{\PROD{N_LSS_SCI_SKY_MAP}}: Pixel map of sky pixels\\
            	& \hyperref[dataitem:n_lss_sci_2d]{\PROD{N_LSS_SCI_2D}}: coadded, wavelength calibrated 2D spectrum\\
                & (\FITS{PRO_CATG}: \FITS{N_LSS_2d_coadd_wavecal}) \\
                & \hyperref[dataitem:n_lss_sci_1d]{\PROD{N_LSS_SCI_1D}}: coadded, wavelength calibrated 1D spectrum\\
                & (\FITS{PRO_CATG}: \FITS{N_LSS_1d_coadd_wavecal}) \\
                & \hyperref[dataitem:n_lss_sci_flux_2d]{\PROD{N_LSS_SCI_FLUX_2D}}: coadded, wavelength calibrated 2D spectrum\\
                & (\FITS{PRO_CATG}: \FITS{N_LSS_2d_coadd_wavecal}) \\
              	& \hyperref[dataitem:n_lss_sci_flux_1d]{\PROD{N_LSS_SCI_FLUX_1D}}: coadded, wavelength 1D spectrum\\
                & (\FITS{PRO_CATG}: \FITS{N_LSS_1d_coadd_wavecal}) \\
Expected accuracies: & (TBD)\\
QC1 parameters: & \hyperref[qc:nlssscisnr]{\QC{QC N LSS SCI SNR}}: Signal-to-noise ration of science spectrum\\
                & \hyperref[qc:nlssscisnrnoise]{\QC{QC N LSS SCI SNRNOISE}}: Noise level of science spectrum\\
                & \hyperref[qc:nlssscifluxsnr]{\QC{QC N LSS SCI FLUX SNR}}: Signal-to-noise ration of flux calibrated  science spectrum\\
                & \hyperref[qc:lmlssscifluxsnrnoise]{\QC{QC N LSS SCI FLUX SNRNOISE}}: Noise level of flux calibrated science spectrum\\
                & \hyperref[qc:nlsssciinterordrlevel]{\QC{QC N LSS SCI INTORDR LEVEL}}: Flux level of the interorder background\\
                & \hyperref[qc:nlsssciwavecaldevmean]{\QC{QC N LSS SCI WAVECAL DEVMEAN}}: Mean deviation from the wavelength reference frame (TBDef)\\
                & \hyperref[qc:nlsssciwavecalfwhm]{\QC{QC N LSS SCI WAVECAL FWHM}}: Measured FWHM of lines\\
                & \hyperref[qc:nlsssciwavecalnident]{\QC{QC N LSS SCI WAVECAL NIDENT}}: Number of identified lines\\
                & \hyperref[qc:nlsssciwavecalnmatch]{\QC{QC N LSS SCI WAVECAL NMATCH}}: Number of lines matched between catalogue and spectrum\\
                & \hyperref[qc:nlsssciwavecalpolydeg]{\QC{QC N LSS SCI WAVECAL POLYDEG}}: Degree of the wavelength polynomial\\
                & \hyperref[qc:nlsssciwavecalpolycoeffn]{\QC{QC N LSS SCI WAVECAL POLYCOEFF\<n\>}}: $n$-th coefficient of the polynomial\\
                & more TBD\\
\end{recipedef}

\subsubsection{N-LSS telluric correction recipe \REC{metis_N_lss_mf_model}:}\label{rec:NLSSmfmodel}
The telluric correction will be done with the package \texttt{molecfit}\footnote{\url{https://www.eso.org/sci/software/pipelines/molecfit/molecfit-pipe-recipes.html}}. It is realised in three individual recipes, \hyperref[rec:NLSSmfmodel]{\REC{metis_N_lss_mf_model}}, which calculates the best-fit model, \hyperref[rec:NLSSmfcalctrans]{\REC{metis_N_lss_mf_calctrans}}, which creates a synthetic transmission curve, and \hyperref[rec:NLSSmfcorrect]{\REC{metis_N_lss_mf_correct}}, which performs the actual telluric correction by means of the synthetic transmission.

\begin{figure}[ht]
  \centering
  \includegraphics[width=0.5\textheight]{figures/metis_N_lss_mf_model_v0.82.pdf}
  \caption[Recipe: \REC{metis_N_lss_mf_model}]{\REC{metis_N_lss_mf_model} --
    Recipe to achieve the best-fit for the calculation of the synthetic transmission curve for the telluric correction.}
  \label{Fig:rec_N_lss_mf_model}
\end{figure}
\clearpage

\begin{recipedef}
Name:		& \hyperref[rec:NLSSmfmodel]{\REC{metis_N_lss_mf_model}}\\
Purpose:	& Achieve the best fit for modelling the transmission curve to be applied as telluric correction \\
Type:		& Post-calibration\\
Requirements: & METIS-4051, METIS-6091 \\
Templates:           & None\\
Input data: 	& \hyperref[dataitem:n_lss_sci_flux_1d]{\PROD{N_LSS_SCI_FLUX_1D}}\\
                & \hyperref[dataitem:lsf_kernel]{\STATCALIB{LSF_KERNEL}} \\
                & \hyperref[dataitem:atm_profile]{\EXTCALIB{ATM_PROFILE}} \\
                & \hyperref[dataitem:atm_line_cat]{\EXTCALIB{ATM_LINE_CAT}} \\
Parameters: 	& \texttt{molecfit} parameters (c.f. \cite{molecfit})\\
Algorithm:      & Fit of telluric features visible in the science input spectrum\\
                & Determination of best-fit parameter set\\
Output data:	& \hyperref[dataitem:mf_best_fit_tab]{\PROD{MF_BEST_FIT_TAB}}\\
Expected accuracies: & (TBD)\\
QC1 parameters: & cf. \cite{molecfit}\\
\end{recipedef}

\subsubsection{N-LSS telluric correction recipe \REC{metis_N_lss_mf_calctrans}:}\label{rec:NLSSmfcalctrans}
\begin{figure}[ht]
  \centering
  \includegraphics[width=0.5\textheight]{figures/metis_N_lss_mf_calctrans_v0.82.pdf}
  \caption[Recipe: \REC{metis_N_lss_mf_calctrans}]{\REC{metis_N_lss_mf_calctrans} --
    Recipe to calculate the synthetic transmission to be applied as telluric correction.}
  \label{Fig:rec_N_lss_mf_calctrans}
\end{figure}
\clearpage

\begin{recipedef}
Name:		& \hyperref[rec:NLSSmfcalctrans]{\REC{metis_N_lss_mf_calctrans}}\\
Purpose:	& Calculation of the synthetic transmission \\
Type:		& Post-calibration\\
Requirements: & METIS-4051, METIS-6091 \\
Templates:           & None\\
Input data: 	& \hyperref[dataitem:mf_best_fit_tab]{\PROD{MF_BEST_FIT_TAB}}\\
                & \hyperref[dataitem:lsf_kernel]{\STATCALIB{LSF_KERNEL}} \\
                & \hyperref[dataitem:atm_profile]{\EXTCALIB{ATM_PROFILE}} \\
                & \hyperref[dataitem:atm_line_cat]{\EXTCALIB{ATM_LINE_CAT}} \\
Parameters: 	& \texttt{molecfit} parameters (c.f.  \cite{molecfit})\\
Algorithm:      & Calculate the entire transmission curve by means of the best-fit parameters\\
Output data:	& \hyperref[dataitem:n_lss_synth_trans]{\PROD{N_LSS_SYNTH_TRANS}}\\
\\
Expected accuracies: & (TBD)\\
QC1 parameters: & cf. \cite{molecfit}\\
\end{recipedef}

\subsubsection{N-LSS telluric correction recipe \REC{metis_N_lss_mf_correct}:}\label{rec:NLSSmfcorrect}
\begin{figure}[ht]
  \centering
  \includegraphics[width=0.5\textheight]{figures/metis_N_lss_mf_correct_v0.82.pdf}
  \caption[Recipe: \REC{metis_N_lss_mf_correct}]{\REC{metis_N_lss_mf_correct} --
    Recipe to apply the telluric correction.}
  \label{Fig:rec_N_lss_mf_correct}
\end{figure}
\clearpage

\begin{recipedef}
Name:		& \hyperref[rec:NLSSmfcorrect]{\REC{metis_N_lss_mf_correct}}\\
Purpose:	& Apply the synthetic transmission to the science spectra \\
Type:		& Post-calibration\\
Requirements: & METIS-4051, METIS-6091 \\
Templates:           & None\\
Input data: 	& \hyperref[dataitem:n_lss_sci_flux_1d]{\PROD{N_LSS_SCI_FLUX_1D}}\\
                & \hyperref[dataitem:n_lss_synth_trans]{\PROD{N_LSS_SYNTH_TRANS}}\\
Parameters: 	& None\\
Algorithm:      & Apply telluric correction, i.e. divide the input science spectrum\\
                & by the synthetic transmission\\
Output data:	& \hyperref[dataitem:n_lss_sci_flux_tellcorr_1d]{\PROD{N_LSS_SCI_FLUX_TELLCORR_1D}}\\
Expected accuracies: & (TBD)\\
QC1 parameters: & cf. \cite{molecfit}\\
\end{recipedef}








%% 06_5-Recipes_IFU.tex
%% Created:     Fri Aug 25 13:14:02 2017 by Koehler@I-Mac
%%
%% subsection for LMS recipes
%%
%%%%%%%%%%%%%%%%%%%%%%%%%%%%%%%%%%%%%%%%%%%%%%%%%%%%%%%%%%%%%%%%%%%%%%%%%%%%%%%

\clearpage
\subsection{LM integral-field spectroscopy (LMS)}
\label{ssec:LMS_recipes}

\TODO{This section is identical to the PDR document \cite{DRLS}. We
will consider rearranging the recipes to be in line with the imaging
pipelines. This would entail handling basic reduction and background
subtraction for of both science and standard exposures in common
recipes (\REC{metis_lms_basic}, \REC{metis_lms_background}), then
having a recipe to analyse the standard observations
(\REC{metis_lms_photstd}). The science exposures are then fully
calibrated (\REC{metis_lms_calibrate}. A full set of exposures would
then be assembled and restored with a fully sampled PSF in a
post-processing recipe (\REC{metis_lms_combine}).}

\subsubsection{LMS wavelength calibration}
\label{sssec:lms_wavecal}

This recipe processes daytime wavelength calibration images to derive
the pixel-to-wavelength relation for the LM integral-field
spectrograph. The calibration template will use the quantum cascade
laser (QCL) in the warm calibration unit to finely sample the desired
wavelength range. The image will consist of lines for each wavelength
and slice. The solution will have to provide for each detector pixel
$(x,y)$ the slice number $i$, the spatial position $\xi$ along the
slice and the wavelength in the dispersion correction. As the slices
and wavelength lines may be tilted with respect to the detector
columns and rows, a combined solution is required
\begin{eqnarray}
  \label{eq:wavelength_solution}
  \xi &= f_{i}(x, y) \\
  \lambda &= g_{i}(x, y)
\end{eqnarray}
The boundaries of the slice image on the detector are obtained by
measuring the left and right edges of the wavelength lines and
interpolating. The slice number is then obtained by counting the
slices according to the optical design of the spectrograph. The
wavelength of each line is known from the settings of the QCL, the $x$
coordinate is obtained by linear interpolation along the line (or
perhaps using the distortion table from \REC{metis_lms_distortion} if
necessary). The functions $f_{i}$ and $g_{i}$ are expected to be
sufficiently accurately described by low-order polynomials.

The recipe produces a multi-extension FITS file with an image
extension mapping wavelength across each detector in the array. A
table extension holds the polynomial coefficients.

\begin{recipedef}
  Name:                & \REC{metis_lms_wavecal}                                                \\
  Purpose:             & Determine pixel-to-wavelength transformation.                          \\
  Requirements:        & \REQ{METIS-6074}                                                       \\
  Type:                & Calibration                                                            \\
  Templates:           & \TPL{METIS_ifu_cal_LampWave}                                           \\
  Input data:          & Images taken with WCU QCL source                                       \\
                       & Master dark                                                            \\
                       & Bad-pixel map                                                          \\
                       & Distortion table (TBD)                                                 \\
  Parameters:          & TBD                                                                    \\
  Algorithm:           & Measure line locations (left and right edges, centroid by Gaussian fit)\\
                       & Compute wavelength solution $\xi(x,y, i)$, $\lambda(x, y, i)$          \\
  Output data:         & \PROD{LMS_WAVECAL}                                                     \\
  Expected accuracies: & TBD                                                                    \\
  QC1 parameters:      & \QC{QC LMS WAVECAL RMS}                                                \\
                       & \QC{QC LMS WAVECAL NLINES}                                             \\
                       & \QC{QC LMS WAVECAL PEAK CNTS}                                          \\
                       & \QC{QC LMS WAVECAL LINE WIDTH}                                         \\
  \end{recipedef}

\begin{figure}[hb]
  \centering
  \includegraphics[width=0.7\textwidth]{metis_lms_wavecal}
  \caption[Recipe: \REC{metis_lms_wavecal}]{\REC{metis_lms_wavecal} --
    daytime wavelength calibration for the LMS.}
  \label{fig:metis_lms_wavecal}
\end{figure}


\clearpage
\subsubsection{LMS relative spectral response function}
\label{sssec:lms_rsrf}

This recipe creates a spectroscopic master flat and determines the
relative spectral response function (RSRF) for the four HAWAII2RG
detectors of the LM spectrograph. The input data are obtained by
illuminating the field of view with the black-body calibration lamp at
two different temperatures. The RSRF is then determined by dividing
the image by the known lamp continuum shape for the respective
temperature. We refer to the two-dimensional image obtained by this
division as \PROD{MASTER_FLAT} and the one-dimensional reponse
function obtained by averaging at constant wavelength as
\PROD{RSRF}. The bad pixel mask can be updated by identifying pixels
that deviate strongly from their neighbours.

\begin{recipedef}
Name:                & \REC{metis_lms_rsrf}                                                     \\
Purpose:             & Create relative spectral response function for the LMS detector.         \\
Requirements:        & \REQ{METIS-6131}, \REQ{METIS-6698}                                       \\
Type:                & Calibration                                                              \\
Templates:           & \TPL{METIS_ifu_cal_rsrf}                                                 \\
Input data:          & Raw flats taken with black-body calibration lamp.                        \\
                     & \CODE{MASTER_DARK_LMS}                                                   \\
                     & \CODE{BADPIX_MAP_LMS}                                                    \\
                     & \CODE{LMS_WAVECAL}: image with wavelength at each pixel.                 \\
Parameters:          & TBD                                                                      \\
Algorithm:           & Create continuum image by mapping Planck spectrum at $T_{\mathrm{lamp}}$ to
                       wavelength image.                                                        \\
                     & Divide exposures by continuum image.                                     \\
                     & Average exposures to yield master flat (2D RSRF).                        \\
                     & Average in spatial direction to obtain relative response function        \\
Output data:         & \PROD{MASTER_FLAT_LMS}                                                   \\
                     & \PROD{RSRF_LMS}                                                          \\
                     & \PROD{BADPIX_MAP_LMS}                                                    \\
Expected accuracies: & TBD                                                                      \\
QC1 parameters:      & \QC{QC LMS RSRF NBADPIX}                                                    \\
                     & (more TBD)                                                               \\
\end{recipedef}

\begin{figure}[hb]
  \centering
  \includegraphics[width=0.6\textwidth]{metis_lms_rsrf}
  \caption[Recipe: \REC{metis_lms_rsrf}]{\REC{metis_lms_rsrf} --
    creation of LMS relative spectral response function.}
  \label{fig:metis_lms_rsrf}
\end{figure}


\clearpage
\subsubsection{LMS flux standard reduction}
\label{sssec:lms_std_process}

This recipe reduces and analyses a series of LMS observations of a
spectroscopic flux standard star. The comparison of the measured
detector counts (ADU) with the tabulated spectrum of the star gives
the wavelength-dependent conversion from ADU to physical units
(photons per second per wavelength bin per spatial bin).

The level of stray light is estimated in the dark areas between the
spectra and subtracted from the entire frame. The distribution of
stray light across the field can only be characterised once the
instrument is built. It is to be hoped that subtraction of a constant
or a low-level 2D polynomial fit will be sufficient.

The sky and thermal background is estimated from blank sky
observations (if obtained during the observing sequence) or by
combining the (dithered) science frames.

The wavelength calibration is taken from the daylight calibration. It
may be refined by measuring telluric emission and/or absorption lines
(by fitting with \lstinline{molecfit}).

\begin{recipedef}
  Name:                & \REC{metis_lms_std_process}                                            \\
  Purpose:             & Determine conversion between detector counts and physical source flux. \\
  Requirements:        & \REQ{METIS-6131}                                                       \\
  Type:                & Calibration                                                            \\
  Templates:           & \TPL{METIS_ifu_cal_standard}                                           \\
  Input data:          & Raw spectra of flux standard star                                      \\
                       & Master dark                                                            \\
                       & Master flat (2D relative spectral response function)                   \\
                       & Bad pixel mask                                                         \\
                       & Wavelength calibration image                                           \\
                       & Distortion table                                                       \\
  Parameters:          & TBD                                                                    \\
  Algorithm:           & Subtract dark, divide by master flat                                   \\
                       & Estimate stray light and subtract                                      \\
                       & Estimate background and subtract                                       \\
                       & Rectify spectra and assemble cube                                      \\
                       & Extract 1D spectrum of star                                            \\
                       & Compute and apply telluric correction                                  \\
                       & Compute conversion to physical units as function of wavelength.        \\
  Output data:         & \PROD{LMS_STD_REDUCED_CUBE}                                            \\
                       & \PROD{LMS_STD_BACKGROUND_CUBE}                                         \\
                       & \PROD{LMS_STD_REDUCED_1D}                                              \\
                       & \PROD{LMS_STD_TELLURIC_1D}                                             \\
                       & \PROD{FLUXCAL_TAB}                                                     \\
  Expected accuracies: & TBD                                                                    \\
  QC1 parameters:      & \QC{QC LMS STD STRAYLIGHT MEAN}                                        \\
\end{recipedef}

\begin{figure}[hb]
  \centering
  \includegraphics[width=0.7\textwidth]{metis_lms_std_process}
  \caption[Recipe: \REC{metis_lms_std_process}]{%
    \REC{metis_lms_std_process} -- reduction of LMS flux standard
    frames and flux calibration (not all data products are shown).}
  \label{fig:metis_lms_std_process}
\end{figure}

\clearpage
\subsubsection{LMS science reduction}
\label{sssec:lms_sci_process}

This recipe performs basic reduction of raw science exposures applying
dark and RSRF correction and flux calibration (i.e.~conversion of
pixel values to physical units) on each exposure individually. The
recipe shall be able to process data from either the nominal or the
extended wavelength mode. For the nominal mode, all slices belong to
the same echelle order. For the extended mode, slices belonging to the
same echelle order are grouped and processing is iterated over the
echelle orders.

The level of stray light is estimated in the dark areas between the
spectra and subtracted from the entire frame. The distribution of
stray light across the field can only be characterised once the
instrument is built. It is to be hoped that subtraction of a constant
or a low-level 2D polynomial fit will be sufficient.

The sky and thermal background, as well as residual straylight, is
estimated from blank sky observations if these are available in the
sequence of input frames or by combining (dithered) science
frames. The initial wavelength solution is taken from the daylight
calibration. It may be checked and corrected by measuring atmospheric
lines if a sufficient number is available in the limited wavelength
range.

A telluric correction is determined by this recipe by automatically
extracting a 1D spectrum from ``object'' pixels identified by a
thresholding algorithm. \lstinline{molecfit} is applied to this
spectrum and the correction is mapped back to the reduced 2D images or
3D cubes using the wavelength images. In an interactive environment
(Reflex workflow) the telluric correction may be improved by asking
the user to define an extraction aperture adapted to the target
structure.

Various levels of output data can be envisaged:
\begin{itemize}
\item Reduced 2D detector images. These are accompanied by additional
  information describing the geometry of the slice layout, target
  position and wavelength calibration to the extent that the exposure can be
  combined with other exposures into a single rectified spectral cube.
  This information can be stored in the FITS header or a table
  extension.
\item A rectified spectral cube for each exposure with a linear
  wavelength grid, constructed by resampling each spectral slice onto
  a spatial-wavelength grid common to all slices. The spatial pixels
  are rectangular with along-slit pixel scale given by the detector
  pixel scale and the across-slit pixel scale given by the slice
  width.
\item A spectral cube obtained by combining all exposures taken within
  a template. This step involves the image reconstruction discussed in
  Sect.~\ref{ssec:image_reconstruction}. Whether this step is included
  in the present recipe \REC{metis_lms_sci_process} or is postponed to
  the more general recipe \REC{metis_lms_sci_postprocess} is TBD. It
  may be formally required to do the image reconstruction here if
  templates are set up to obtain a fixed set of spatially dithered and
  rotated exposures aimed at reconstructing a fully sampled PSF in
  both spatial dimensions.
\end{itemize}

For the nominal mode, each output is a single-extension FITS file
corresponding to one echelle order. For the extended mode, each of the
echelle orders results in an extension in a multi-extension FITS
file.

The recipe as descibed here is run in the science pipelines. For the
observatory pipeline, a variant of the recipe may be implemented with
reduced functionality and output. The observatory recipe may also have
to include features to determine QC parameters for the LM-band images
that are taken in parallel with the LMS exposures, similar to
\REC{metis_lm_img_sci_process} (Sect.~\ref{sssec:lm_img_sci}; \REC{METIS-6072}).

\begin{recipedef}
Name:                & \REC{metis_lms_sci_process}                                                              \\
Purpose:             & Reduction of individual science exposures.                                               \\
Requirements:        & \REQ{METIS-6131}                                                                         \\
Type:                & Science                                                                                  \\
Templates:           & \TPL{METIS_ifu_obs_FixedSkyOffset}                                                       \\
                     & \TPL{METIS_ifu_obs_GenericOffset}                                                        \\
                     & \TPL{METIS_ifu_ext_obs_FixedSkyOffset}                                                   \\
                     & \TPL{METIS_ifu_ext_obs_GenericOffset}                                                    \\
                     & \TPL{METIS_ifu_app_obs_GenericOffset}                                                    \\
                     & \TPL{METIS_ifu_clc_obs_FixedSkyOffset}                                                   \\
                     & \TPL{METIS_ifu_vc_obs_FixedSkyOffset}                                                    \\
                     & \TPL{METIS_ifu_ext_app_obs_GenericOffset}                                                \\
                     & \TPL{METIS_ifu_ext_clc_obs_FixedSkyOffset}                                               \\
                     & \TPL{METIS_ifu_ext_vc_obs_FixedSkyOffset}                                                \\
                     & \TPL{METIS_ifu_cal_psf}                                                                  \\
Input data:          & Dithered science exposures.                                                              \\
                     & Blank sky images (if available)                                                          \\
                     & Master dark                                                                              \\
                     & Master flat (2D relative spectral response function)                                     \\
                     & Bad pixel mask                                                                           \\
                     & Wavelength calibration image                                                             \\
                     & Flux calibration table                                                                   \\
                     & Distortion table                                                                         \\
                     & Line spread kernel to be used with \CODE{molecfit}                                       \\
Parameters:          & telluric correction (yes/no)                                                             \\
                     & more TBD                                                                                 \\
Algorithm:           & Subtract dark, divide by master flat                                                     \\
                     & Estimate stray light and subtract                                                        \\
                     & Estimate background from dithered science exposures or blank-sky exposures and subtract. \\
                     & Apply flux calibration .                                                                 \\
                     & Rectify spectra and assemble cube                                                        \\
                     & Extract 1D object spectrum                                                               \\
                     & Compute telluric correction and apply to reduced images and cube                         \\
Output data:         & \PROD{LMS_SCI_REDUCED} (2D, per exposure)                                                \\
                     & \PROD{LMS_SCI_REDUCED_TAC} (2D, per exposure)                                            \\
                     & \PROD{LMS_SCI_BACKGROUND} (2D, per exposure)                                             \\
                     & \PROD{LMS_SCI_REDUCED_CUBE} (3D, per exposure)                                           \\
                     & \PROD{LMS_SCI_REDUCED_CUBE_TAC} (3D, per exposure)                                       \\
                     & \PROD{LMS_SCI_COMBINED} (3D)                                                             \\
                     & \PROD{LMS_SCI_COMBINED_TAC} (3D)                                                         \\
                     & \PROD{LMS_SCI_OBJECT_1D}  (1D)                                                           \\
                     & \PROD{LMS_SCI_TELLURIC_1D}                                                               \\
Expected accuracies: & TBD                                                                                      \\
QC1 parameters:      & TBD                                                                                      \\
\end{recipedef}

\begin{figure}[hb]
  \centering
  \includegraphics[width=0.7\textwidth]{metis_lms_sci_process}
  \caption[Recipe: \REC{metis_lms_sci_process}]{%
    \REC{metis_lms_sci_process} -- reduction of LMS science frames.}
  \label{fig:metis_lms_sci_process}
\end{figure}

\clearpage
\subsubsection{LMS telluric absorption correction}
\label{sssec:lms_tellcorr}

This recipe corrects for telluric absorption in a reduced LMS data
cube. The correction is done via a model atmospheric spectrum derived
with \CODE{molecfit}.

An automatic telluric correction can be performed as part of
\REC{metis_lms_sci_process}. In an interactive environment it may be
better to do the telluric correction as a separate post-processing
step with a user-defined aperture for the extraction of a 1D object
spectrum. The spectrum is extracted from a combined cube
(\CODE{LMS_SCI_COMBINED}) but may be applied to other products of
\REC{metis_lms_sci_process} specified in the input set of frames.

\begin{recipedef}
  Name:                & \REC{metis_lms_tellcorr}                                                        \\
  Purpose:             & Remove telluric absorption features                                             \\
  Requirements:        & \REQ{METIS-6091}                                                                \\
  Type:                & Calibration / post processing                                                   \\
  Templates:           & ---                                                                             \\
  Input data:          & \CODE{LMS_SCI_COMBINED} -- reduced combined LMS cube                            \\
                       & \CODE{LSF_KERNEL} -- Line spread kernel to be used with \CODE{molecfit}         \\
                       & \CODE{ATM_PROFILE} -- Atmospheric input profile to be used with \CODE{molecfit} \\
  Parameters:          & extraction aperture parameters                                                  \\
                       & \CODE{molecfit} parameters                                                      \\
                       & atmospheric profile incl.\ radiometer data                                      \\
                       & line spread kernel                                                              \\
  Algorithm:           & extract 1D spectrum                                                             \\
                       & Application of molecfit                                                         \\
  Output data:         & \PROD{LMS_SCI_REDUCED_TAC}                                                      \\
  Expected accuracies: & TBD                                                                             \\
  QC1 parameters:      & TBD                                                                             \\
\end{recipedef}

\begin{figure}[hb]
  \centering
  \includegraphics[width=0.7\textwidth]{metis_lms_tellcorr}
  \caption[Recipe: \REC{metis_lms_tellcorr}]{\REC{metis_lms_tellcorr}
    -- telluric correction of reduced LMS science cubes.}
  \label{fig:metis_lms_tellcorr}
\end{figure}


\clearpage
\subsubsection{LMS science postprocessing}
\label{sssec:lms_sci_postprocess}

This recipe combines a number of reduced LMS exposures covering a
different spatial and wavelength ranges into a single data cube. The
positions and orientations of the exposures may differ as follows:
\begin{description}
\item[Spatial dithering:] The target is placed at different positions
  along and across the slice. Along-slice dithering aids in background
  subtraction, across-slice dithering is necessary image
  reconstruction given that the slice width undersamples the PSF.
\item[Field rotation:] The field is rotated by 90 degrees between
  exposures. The cube of a single exposure has different pixel scales
  along and across the slice. The goal of combining exposures at
  different rotation angles is to reconstruct images on a square grid
  with pixel scale given by the detector scale (8.2\,mas). The exact
  procedure remains to be investigated; one of the major challenges is
  to find the exact centre of rotation
  (Sect.~\ref{ssec:image_reconstruction}).
\item[Spectral dithering:] Sequences of exposures are taken at various
  echelle angles in order to cover an increased contiguous wavelength
  range. In the extended mode, such a sequence may cover the
  wavelength gaps between echelle order coverage.
\end{description}

In order to allow coaddition of data from separate OBs, possibly taken
months apart, the wavelengths will be corrected to the heliocentric
reference system before coaddition.

The recipe is only used in the science-grade pipelines, not at the
observatory.

\begin{recipedef}
  Name:           & \REC{metis_lms_sci_postprocess}                                            \\
  Purpose:        & Coaddition and mosaicing of reduced science cubes.                         \\
  Requirements:   & \REQ{METIS-6131}                                                           \\
  Type:           & Science                                                                    \\
  Templates:      & ---                                                                        \\
  Input data:     & Reduced science cubes (\PROD{LMS_SCI_REDUCED}, \PROD{LMS_SCI_REDUCED_TAC}) \\
  Parameters:     & TBD                                                                        \\
  Algorithm:      & Determine cubic output grid encompassing all input cubes                   \\
                  & Resample input cubes to output grid                                        \\
                  & Coadd                                                                      \\
  Output data:    & \PROD{LMS_SCI_COADD}                                                       \\
                  & \PROD{LMS_SCI_COADD_ERROR}                                                 \\
  QC1 parameters: & ---                                                                        \\
\end{recipedef}

\begin{figure}[hb]
  \centering
  \includegraphics[width=0.7\textwidth]{metis_lms_sci_postprocess}
  \caption[Recipe: \REC{metis_lms_sci_postprocess}]{%
    \REC{metis_lms_sci_postprocess} -- post-processing (coaddition) of
    reduced LMS science frames.}
  \label{fig:metis_lms_sci_postprocess}
\end{figure}


\clearpage
\subsubsection{LMS distortion calibration}
\label{sssec:lms_distortion}

Calibration of the geometric distortion of the LMS is done by
observing a pin hole mask located in a focal plane within the
instrument. The distortion is described in terms of a polynomial model
whose coefficients can be used to map positions in in the detector
array to sky positions. Measurement of the FWHM of the spots gives an
indication of the variation of spectral resolution across the field of view.

\begin{recipedef}
  Name:                & \REC{metis_lms_distortion}                                                  \\
  Purpose:             & Determine geometric distortion coefficients for the LMS.                    \\
  Requirements:        & \REQ{METIS-6087}, \REQ{METIS-6073}                                          \\
  Type:                & Calibration                                                                 \\
  Templates:           & \TPL{METIS_ifu_cal_distortion}                                              \\
  Input data:          & Images of multi-pinhole mask.                                               \\
  Parameters:          & TBD                                                                         \\
  Algorithm:           & Calculate table mapping pixel position to position on sky.                  \\
  Output data:         & \PROD{LMS_DISTORTION_TABLE}                                                 \\
                       & \PROD{LMS_DIST_REDUCED}                                                     \\
  Expected accuracies: & TBD                                                                         \\
  QC1 parameters:      & \QC{QC LMS DISTORT RMS}: RMS deviation between measured position and model \\
                       & \QC{QC LMS DISTORT FWHM}:   Measured FWHM of spots                            \\
                       & \QC{QC LMS DISTORT NSPOTS}: Number of identified spots                        \\
\end{recipedef}

\begin{figure}[hb]
  \centering
  \includegraphics[width=0.6\textwidth]{metis_lms_distortion}
  \caption[Recipe: \REC{metis_lms_distortion}]{%
    \REC{metis_lms_distortion} -- LMS distortion calibration}
  \label{fig:metis_lms_distortion}
\end{figure}


\clearpage



%%%%%%%%%%%%%%%%%%%%%%%%%%%%%%%%%%%%%%%%%%%%%%%%%%%%%%%%%%%%%%%%%%%%%%%%%%%%%%%%

%%% Local Variables:
%%% TeX-master: "METIS_DRLD"
%%% End:


\subsection{ADI Post Processing}
\label{ssec:ADI_postprocessing}



The following recipes can be used by astronomers in an offline way
perform basic ADI processing on data which have already undergone
basic calibration, via the standard LM/N processing methods.  As it
relies on reduced data they will not be executed in a scheduled way at
the telescope. For more detailed HCI reductions the observers will
have to rely on their own more specialized code, but intermediate data
products will be optionally provided in these recipes to facilitate
their dedicated HCI reductions. Potentially these three recipes can be combined into one with a logical decision tree. To minimize interpolation artefacts any interpolation steps are to be combined as much as possible. Bad pixel maps and image stacks without background subtraction are available as output from the earlier science processing recipes.

\subsubsection{IMG\_LM/N RAVC/CVC(/CLC) ADI Post Processing}
\label{sssec:adi_img_vc}


The following recipe is applicable for ADI post processing for the LM
and N band, and CVC/RAVC(/CLC) coronagraphs. An input set of
observations consists of a time sequence of ADI images in LM or N band, which have already undergone basic calibration.

For each image, the centroid of the central source is determined, distortion corrections are performed, and the images are aligned on a subpixel scale. The median PSF is then estimated and subtracted from all images following the first step of the standard ADI technique of Marois et al (2006).
Each image is then derotated using the known position angle and coadded to produce the final science image. In addition, the images prior to PSF subtraction are derotated and combined to produce a second final image.

In addition to the final images, the cube of derotated, PSF subtracted images are used to calculate the raw and post-ADI contrast curves as well as the ADI throughput curve. The intrinsic radial throughput of the coronagraph  is taken from static calibrations while the post-processing losses are estimated from injection and retrieval of artificial companions with a known brightness and separation.

Off-axis unsaturated PSFs which are needed for the ADI process are either collected as part of the observations, static calibrations or available from the QACITS control loop.  If collected as part of the OB they are processed by the regular science recipes (with compensation of any neutral density transmission).

While not part of the PIP specs, the current generation of VIP\_HCI ADI reduction algorithms supports the more detailed Marois et al. 2006 ADI (including an annular optimization step) as well as PCA-based routines.

\begin{recipedef}
  Name:                & \REC{metis_lm_adi_ravc}                                        \\
  Purpose:             & Classical ADI post processing for CVC/RAVC(/CLC) coronagraphs      \\
  Requirements:        & \REQ{METIS-5989}                                               \\
  Type:                & Science                                                    \\
  Input data:          & \PROD{LM_SCI_BASIC_REDUCED}                            \\
                       & Time series of LM\_SCI\_REDUCED images                      \\
                       & LM distortion table                               \\
                       & Coronagraphic throughput map and profile                                                  \\
                       & Off-axis PSF references                                                  \\
                       &                                                  \\
   Matched keywords:   &              \\
                       &               \\
                       &               \\
                       &               \\
                       &               \\
  Parameters:          &  combination approach (median,mean,sigclip) \\
                       &   combination parameters (e.g., N-sigma)          \\
                       &  start and end limit to contrast curve (in $\lambda/D$) \\
  & frame exclusion thresholds dependent on AO parameters and centroid offset                \\

  Algorithm:           & Determine centroid of central source \\
                       & Distortion correction and sub-pixel alignment   \\
                       & Estimate median PSF   \\
                       & Subtract median PSF   \\
                       & De-rotate images   \\
                       & Coadd images   \\
  & Calculate contrast curve   \\
  Output data:       & \PROD{LM_RAVC_SCI_CALIBRATED} (Calibrated Image)                                    \\
                     & \PROD{LM_RAVC_SCI_CENTRED} (Cube of individually calibrated and recentered images)                                 \\
                     & \PROD{LM_RAVC_CENTROID_TAB} (Table of star centre estimages)                                 \\

                     & \PROD{LM_RAVC_SCI_SPECKLE} (PSF/speckle image)                                 \\
                     & \PROD{LM_RAVC_SCI_DEROTATED_PSFSUB_COADD} (Combined derotated image with PSF subtraction)                                 \\
                     & \PROD{LM_RAVC_SCI_DEROTATED_COADD} (Combined derotated image without PSF subtraction)                                  \\
                     & \PROD{LM_RAVC_SCI_CONTRAST_RAW} (Raw Contrast Curve)                                 \\
                     & \PROD{LM_RAVC_SCI_CONTRAST_ADI} (Post ADI Contrast Curve)                                 \\
                     & \PROD{LM_RAVC_SCI_THROUGHPUT} (ADI Throughput Curve)                               \\
                     & \PROD{LM_RAVC_SCI_COVERAGEMAP} (ADI Coverage Map: number of frames
                     \\
                     & \PROD{LM_RAVC_SCI_SNR} (ADI SNR Map)  \\

  Expected accuracies: & \TBD                                                           \\
  QC1 parameters:      & \QC{FWHM of PSF by frame}                                      \\
                       & \QC{Raw and post ADI contrast at seps (1,2,5,10,20,40 lam/D)}                                        \\
                       & \QC{Mean SNR in ADI map}                                        \\
                       & \QC{Peak SNR in ADI map}                                         \\
  hdrl functions:      & \CODE{}                                    \\
                       & \CODE{}                                 \\
                       & \CODE{}                                \\
\end{recipedef}

\begin{figure}[hb]
  \centering
  \includegraphics[width=0.6\textwidth]{./figures/metis_lm_adi_ravc}
  \caption[Recipe: \REC{metis_lm_adi_ravc}]{\REC{metis_lm_adi_ravc} -- LM ADI post processing for RAVC/CVC(/CLC) coronagraph.
    }
  \label{fig:metis_lm_adi_ravc}
\end{figure}




\subsubsection{IMG\_LM/N APP ADI Post Processing}
\label{sssec:adi_img_app}


The following recipe is applicable for ADI post processing for the LM and N band, in combination with the APP coronagraph. It is very similar to the recipe for RAVC/CVC(/CLC) coronagraphs, with the addition of steps for merging together the two half PSFs, and of applying an angular wedge mask before the derotation and stacking, and contrast curve calculations steps. An input set of observations consists of a time sequence of ADI images in LM/N band, which have already undergone basic calibration.

For each image, the centroid of the central source is determined for all three PSFs and distortion corrections are performed. The PSFs are aligned at a sub-pixel scale and extracted; the extracted coronagraphic PSFs are merged to produce a complete PSF and the third PSF is used to form a cube of calibrated leakage PSFs.
The mean/median/sigmaclipped PSF is estimated and subtracted from each frame of the merged coronagraphic PSF. Each image is
then derotated to place the off-axis source at the same on-sky angle and coadded to produce the final stacked science image. In addition, the images prior to PSF subtraction are derotated and combined to produce a second final stacked image.

In addition to the final images, the stack of derotated, PSF subtracted images are used to calculate the raw and post-ADI contrast curves as well as the ADI throughput curve and coverage map containing the effective number of included frames.



\begin{recipedef}
  Name:                & \REC{metis_lm_adi_app}                                        \\
  Purpose:             & Classical ADI post processing for APP coronagraph      \\
  Requirements:        & \REQ{METIS-5989}                                               \\
  Type:                & Science                                                    \\
  Input data:          & \PROD{LM_SCI_BASIC_REDUCED}                            \\
                       & Time series of LM\_SCI\_REDUCED images                      \\
                       & LM distortion table                               \\
                       & Off-axis PSF reference                                                  \\
                       &                                                  \\
   Matched keywords:   &              \\
                       &               \\
                       &               \\
                       &               \\
                       &               \\
  Parameters:          &  combination approach (median,mean,sigclip) \\
                       &   combination parameters (e.g., N-sigma)          \\
                       &  start and end limit to contrast curve (in $\lambda/D$) \\
  & frame exclusion thresholds dependent on AO parameters and centroid offset \\

  Algorithm:           & Determine centroid of central source \\
                       & Distortion correction and sub-pixel alignment   \\
                       & sub-pixel PSF extraction and alignment   \\
                       & Merge coronagraphic PSFs   \\
                       & Estimate median PSF   \\
                       & Subtract median PSF   \\
                       & De-rotate images   \\
                       & Coadd images   \\
                       & Calculate contrast curves   \\
  & Calculate contrast curve   \\
  Output data:       & \PROD{LM_APP_SCI_CALIBRATED} (Calibrated Image)                                    \\
                     & \PROD{LM_APP_SCI_CENTRED} (Cube of individually calibrated and recentered images)                                 \\
                     & \PROD{LM_APP_CENTROID_TAB} (Table of star centre estimates)                                 \\

                     & \PROD{LM_APP_SCI_SPECKLE} (PSF/speckle image)                                 \\
                     & \PROD{LM_APP_SCI_DEROTATED_PSFSUB} (Combined derotated image with PSF subtraction)                                 \\
                     & \PROD{LM_APP_SCI_DEROTATED_COADD} (Combined derotated image without PSF subtraction)                                  \\
                     & \PROD{LM_APP_SCI_CONTRAST_RAW} (Raw Contrast Curve)                                 \\
                     & \PROD{LM_APP_SCI_CONTRAST_ADI} (Post ADI Contrast Curve)                                 \\
                     & \PROD{LM_APP_SCI_THROUGHPUT} (ADI Throughput Curve)                               \\

                     & \PROD{LM_APP_SCI_COVERAGEMAP} (ADI Coverage Map: number of frames
                     \\
                     & \PROD{LM_APP_SCI_SNR} (ADI SNR Map)                            \\

  Expected accuracies: & \TBD                                                           \\
  QC1 parameters:      & \QC{FWHM of PSF by frame}                                      \\
                       & \QC{Raw and post ADI contrast at seps (1,2,5,10,20,40 lam/D)}                                        \\
                       & \QC{Mean SNR in ADI map}                                        \\
                       & \QC{Peak SNR in ADI map}                                         \\
  hdrl functions:      & \CODE{}                                    \\
                       & \CODE{}                                 \\
                       & \CODE{}                                \\
\end{recipedef}

\begin{figure}[hb]
  \centering
  \includegraphics[width=0.6\textwidth]{./figures/metis_lm_adi_app}
  \caption[Recipe: \REC{metis_lm_adi_app}]{\REC{metis_lm_adi_app} -- LM ADI post processing for APP coronagraph.
    }
  \label{fig:metis_lm_adi_app}
\end{figure}



\subsubsection{IFU ADI Post Processing}
\label{sssec:adi_ifu}


The following recipe is applicable for ADI post processing for the IFU data cubes and the RAVC/CVC and APP coronagraphs. As only a single target can be targeted in the limited field of view the methods overlap between coronagraphs.
For the IFU observations, the input is a set of reduced 3D (spectral and spatial) data cubes on a rectified grid.

For each wavelength slice in each cube the centroid is determined by a QACITS-like algorithm in the case of the focal plane coronagraphs (RAVC/CVC) or through 2D cross-correlation with a template PSF for the APP coronagraph. This centroid information is stored in a table together with timestamps, parallactic angle and bad frame flags (based on AO loop status, AO performance, atmospheric parameters and centroid offset).
As the ADI step requires square pixels following previous work on combining ADI techniques with IFUs (such as SPHERE/IFS, SINFONI), the rectangular spatial grid is interpolated or nearest neighbor filled to produce a square pixel image.
It is acknowledged that one spatial dimension is undersampled which may lead to reduced performance compared to a Nyquist-sampled PSF.
The mean/median/sigmaclipped PSF (in time) is estimated for each wavelength and subtracted from each image in the cube.
After derotation the cubes are combined in time to give a coadded cube. For the APP a wedge shape is used. The limited field of view of the IFU means that only PSF can be centered on the IFU. The derotated cubes are also used to generate post ADI contrast curves and contrast curves with input from the radial coronagraph throughput profile and off-axis PSFs. In addition coverage maps are produced.

While not part of the PIP specs, the current generation of VIP\_HCI ADI reduction algorithms supports
the more detailed Marois et al. 2006 ADI routine, PCA-based routines, as well as ADI+mSDI processing to improve the speckle PSF estimation with the additional wavelength information.



\begin{recipedef}
  Name:                & \REC{metis_ifu_adi_ravc}                                        \\
  Purpose:             & Classical ADI post processing for APP/CVC/RAVC(/CLC) coronagraphs with LMS IFU      \\
  Requirements:        & \REQ{METIS-5989}                                               \\
  Type:                & Science                                                    \\
  Input data:          & \PROD{IFU_SCI_REDUCED}                            \\
                       & Time series of IFU\_SCI\_REDUCED images                      \\
                       & IFU distortion table                               \\
                       & Coronagraphic throughput map and profile                                                  \\
                       & Off-axis PSF references                                                  \\
                       &                                                  \\
   Matched keywords:   &              \\
                       &               \\
                       &               \\
                       &               \\
                       &               \\
  Parameters:          &  combination approach (median,mean,sigclip) \\
                       &   combination parameters (e.g., N-sigma)          \\
                       &  start and end limit to contrast curve (in $\lambda/D$) \\
  & frame exclusion thresholds dependent on AO parameters and centroid offset \\

  Algorithm:           & Determine centroid of central source \\
                       & Distortion correction, square pixel reconstruction and sub-pixel alignment   \\
                       & Estimate median PSF   \\
                       & Subtract median PSF   \\
                       & Derotate images   \\
                       & Coadd images   \\
                       & Calculate contrast curves   \\

  Output data:       & \PROD{IFU_RAVC_SCI_CALIBRATED} (Calibrated Cube)                                    \\
                     & \PROD{IFU_RAVC_SCI_CENTRED} (Cube of individually calibrated and recentered cubes)                                 \\
                     & \PROD{IFU_RAVC_CENTROID_TAB} (Table of star centre estimages)                                 \\

                     & \PROD{IFU_RAVC_SCI_SPECKLE} (PSF/speckle cube)                                 \\
                     & \PROD{IFU_RAVC_SCI_DEROTATED_PSFSUB} (Combined derotated cube with PSF subtraction)                                 \\
                     & \PROD{IFU_RAVC_SCI_DEROTATED_COADD} (Combined derotated cube without PSF subtraction)                                  \\
                     & \PROD{IFU_RAVC_SCI_CONTRAST_RAW} (Raw Contrast Curves)                                 \\
                     & \PROD{IFU_RAVC_SCI_CONTRAST_ADI} (Post ADI Contrast Curves)                                 \\
                     & \PROD{IFU_RAVC_SCI_THROUGHPUT} (ADI Throughput Curves)                               \\
                     & \PROD{IFU_RAVC_SCI_SNR} (ADI SNR Map)                            \\
                     
                                      & \PROD{IFU_RAVC_SCI_COVERAGEMAP} (ADI Coverage Map: number of frames)                            \\

  Expected accuracies: & \TBD                                                           \\
  QC1 parameters:      & \QC{FWHM of PSF by frame}                                      \\
                       & \QC{Raw and post ADI contrast at seps (1,2,5,10,20,40 lam/D)}                                        \\
                       & \QC{Mean SNR in ADI map}                                        \\
                       & \QC{Peak SNR in ADI map}                                         \\
  hdrl functions:      & \CODE{}                                    \\
                       & \CODE{}                                 \\
                       & \CODE{}                                \\
\end{recipedef}

\begin{figure}[hb]
  \centering
  \includegraphics[width=0.6\textwidth]{./figures/metis_ifu_adi_ravc}
  \caption[Recipe: \REC{metis_ifu_adi_ravc}]{\REC{metis_ifu_adi_ravc} -- IFU ADI post processing for RAVC coronagraph.
    }
  \label{fig:metis_ifu_adi_ravc}
\end{figure}

%%% Local Variables:
%%% TeX-master: "METIS_DRLD"
%%% End:

 
\subsection{Recipes for engineering templates}
\label{ssec:recipes_technical}

\TBD It is not clear (2020-11-19) whether data from maintenance
templates need to be reduced by the pipeline. This section and the
recipes described therein serves as a placeholder and may be removed
later.

%------------------------------------------------------------------------------------------------------------------
\subsubsection{Pupil imaging}
\label{sssec:pupil_imaging}

\TBD This recipe refers to pupil imaging using the science
detectors. Presumably it only contains the most basic reduction
steps. There may have to be separate recipes for the various
subsystems.

\begin{recipedef}
  Name                 & \REC{metis_pupil_imaging}                     \\
  Purpose:             & Apply basic reduction to pupil imaging data.  \\
  Requirements:        & --                                            \\
  Type:                & Maintenance                                   \\
  Templates:           & \TBD (\cite{METIS-calibration_plan}, Sect.~6) \\
  Input data:          & \TBD                                          \\
  Matched keywords:    & Detector ID                                   \\
  Algorithm            & \TBD                                          \\
  Output data:         & \TBD                                          \\
  Expected accuracies: & \TBD                                          \\
  QC1 parameters:      & \TBD                                          \\
\end{recipedef}

\clearpage
%------------------------------------------------------------------------------------------------------------------
\subsubsection{Chopper Home Position}\label{ssec:metisimgchophome}
The recipe \REC{metis_img_chophome} aims to detect undefined chopper mirror zero positions after switching on the chopper (e.g. after instrument interventions) or induced by unforeseen events like earthquakes (cf. Section "Chopper Home Position" in  \cite{METIS-calibration_plan}).
\begin{figure}[ht]
  \centering
  \includegraphics[width=0.5\textheight]{figures/metis_img_chophome_v0.1.pdf}
  \caption[Recipe: \REC{metis_img_chophome}]{\REC{metis_img_chophome} --
    Recipe workflow to detect the zero position of the chopper mirror.}
  \label{Fig:rec_chop_home}
\end{figure}

\begin{recipedef}\label{rec:metisimgchophome}
Name:		& \REC{metis_img_chophome} \\
Purpose:	& Detection of the chopper mirror home position \\
Type:		& Calibration\\
Requirements: & TBD \\
Observing templates: & \TPL{METIS_img_lm_cal_ChopperHome} \\
Input data:     & \FITS{LM_IMG_RAW} \\
                & \hyperref[dataitem:persistencemap]{\EXTCALIB{PERSISTENCE_MAP}}  \\
                & \hyperref[dataitem:gainmap2rg]{\STATCALIB{GAIN_MAP_2RG}}  \\
                & \hyperref[dataitem:badpixmap2rg]{\PROD{BADPIX_MAP_2RG}}  \\
                & \hyperref[dataitem:masterdark2rg]{\PROD{MASTER_DARK_2RG}}  \\
                & \PROD{MASTER_FLAT_2RG}  \\
Parameters: 	& TBD\\
Algorithm:      & remove detector signature\\
                & remove median background\\
                & apply flatfield\\
                & detect reference source from \ac{WCU} via centroid peak detection\\
                & Calculate mirror offset\\
Output data:	& Offset of the chopper mirror to be piped either into the \ac{ICS} for correction \\
                & or to be used in thee pipeline for astrometric correction\\
Expected accuracies: & 0.1mas accuracy of the centroid position (cf. \cite{METIS-calibration_plan})\\
QC1 parameters: & \QC{TBD}: TBD\\
\end{recipedef}
\clearpage

%------------------------------------------------------------------------------------------------------------------
\subsubsection{Plate-scale calibration}


%------------------------------------------------------------------------------------------------------------------
\subsubsection{Fringing calibration}


%------------------------------------------------------------------------------------------------------------------
\subsubsection{Calibration of slit losses}\label{sssec:adc_slitlosses}
The recipes \hyperref[rec:metislmadcmslitloss]{\REC{metis_lm_adc_slitloss}} (Fig.~\ref{Fig:rec_lm_adc_slitloss}) and \hyperref[rec:metisnadcmslitloss]{\REC{metis_n_adc_slitloss}} (Fig.~\ref{Fig:rec_n_adc_slitloss}) aims to determine the slitlosses induced by the fixed positions of the \ac{ADC} (cf. Section "Calibration of slit losses" in  \cite{METIS-calibration_plan})). This recipe is to be carried out once in a while and creates the slitlosses to be included in the static calibration database.
\begin{figure}[ht]
  \centering
  \includegraphics[width=0.5\textheight]{figures/metis_lm_lss_adc_slitloss_v0.74.pdf}
  \caption[Recipe: \REC{metis_lm_adc_slitloss}]{\REC{metis_lm_adc_slitloss} --
    Recipe workflow to determine the \ac{ADC} induced slit losses.}
  \label{Fig:rec_lm_adc_slitloss}
\end{figure}

\begin{recipedef}\label{rec:metislmadcmslitloss}
Name:		& \REC{metis_lm_adc_slitloss} \\
Purpose:	& Determination of the \ac{ADC} induced slit losses \\
Type:		& Calibration\\
Requirements: & TBD \\
Observing templates: & \TPL{METIS_spec_lm_cal_SlitAdc} \\
Input data:     & \FITS{LM_SLITLOSSES_RAW} \\
                & \FITS{LM_WCU_OFF_RAW} \\
                & \hyperref[dataitem:persistencemap]{\EXTCALIB{PERSISTENCE_MAP}}  \\
                & \hyperref[dataitem:gainmap2rg]{\STATCALIB{GAIN_MAP_2RG}}  \\
                & \hyperref[dataitem:badpixmap2rg]{\PROD{BADPIX_MAP_2RG}}  \\
                & \hyperref[dataitem:masterdark2rg]{\PROD{MASTER_DARK_2RG}}  \\
                & \PROD{MASTER_FLAT_2RG}  \\
Parameters: 	& TBD\\
Algorithm:      & remove detector signature\\
                & remove median background\\
                & apply flatfield\\
                & detect reference source from \ac{WCU} via centroid peak detection\\
                & apply aperture photometry\\
                & calculate (simple) slitloss model (details to be defined)\\
Output data:	& Slit loss model as function of the wavelength and object position across the slit (\EXTCALIB{LM_ADC_SLITLOSS}) \\
Expected accuracies: & 3\% (cf. \cite{METIS_calerrbudget})\\
QC1 parameters: & \QC{TBD}: TBD\\
\end{recipedef}


\begin{figure}[ht]
  \centering
  \includegraphics[width=0.5\textheight]{figures/metis_n_lss_adc_slitloss_v0.74.pdf}
  \caption[Recipe: \REC{metis_n_adc_slitloss}]{\REC{metis_n_adc_slitloss} --
    Recipe workflow to determine the \ac{ADC} induced slit losses.}
  \label{Fig:rec_n_adc_slitloss}
\end{figure}

\begin{recipedef}\label{rec:metisnadcmslitloss}
Name:		& \REC{metis_n_adc_slitloss} \\
Purpose:	& Determination of the \ac{ADC} induced slit losses \\
Type:		& Calibration\\
Requirements: & TBD \\
Observing templates: & \TPL{METIS_spec_n_cal_SlitAdc} \\
Input data:     & \FITS{N_SLITLOSSES_RAW} \\
                & \FITS{N_WCU_OFF_RAW} \\
                & \hyperref[dataitem:persistencemap]{\EXTCALIB{PERSISTENCE_MAP}}  \\
                & \hyperref[dataitem:gainmapgeo]{\STATCALIB{GAIN_MAP_GEO}}  \\
                & \hyperref[dataitem:badpixmapgeo]{\PROD{BADPIX_MAP_GEO}}  \\
                & \hyperref[dataitem:masterdarkgeo]{\PROD{MASTER_DARK_GEO}}  \\
                & \PROD{MASTER_FLAT_GEO}  \\
Parameters: 	& TBD\\
Algorithm:      & remove detector signature\\
                & remove median background\\
                & apply flatfield\\
                & detect reference source from \ac{WCU} via centroid peak detection\\
                & apply aperture photometry\\
                & calculate (simple) slitloss model (details to be defined)\\
Output data:	& Slit loss model as function of the wavelength and object position across the slit (\EXTCALIB{N_ADC_SLITLOSS}) \\
Expected accuracies: & 3\% (cf. \cite{METIS_calerrbudget})\\
QC1 parameters: & \QC{TBD}: TBD\\
\end{recipedef}


%%% Local Variables:
%%% mode: latex
%%% TeX-master: "METIS_DRLD"
%%% End:


%%% Local Variables:
%%% TeX-master: "METIS_DRLD"
%%% End:

% \input{06_1a-Recipes_Detector}
% \newpage
% \input{06_1b-Recipes_Imaging_LM}
% \input{06_2-Recipes_Imaging_NQ}
% \clearpage
\subsection{Long-slit spectroscopy, LM band}
\label{ssec:recipes_lss_lm}

A draft of the reduction cascade is shown in
Fig.~\ref{Fig:LMLssAssomap} together with the data processing table
(Table~\ref{Tab:LMLssDatProc}). The first part aims to update the static calibration database, in particular the creation of the gain map (\hyperref[Sec:detector_calibration]{\REC{metis_det_lingain}}) and the determination of the \ac{ADC} slitlosses (\hyperref[rec:metislmadcmslitloss]{\REC{metis_lm_adc_slitloss}}). These are executed only when an update is required, e.g. after a major instrument interention or on yearly basis. The second part comprises the basic calibrations, e.g. the dark correction and the spectroscopic flatfielding via \ac{RSRF}, followed by the third part, the main calibration steps, incorporating the determination of the first guess wavelength solution by means of the laser sources in the \ac{WCU} and the determination of the response curve for the flux calibration. Subsequently, the main reduction is conducted, which applies the previously created master calibration files to the science frames. Both, the flux standard and the science observations are wavelength calibrated with the help of the atmospheric lines visible in the respective spectra. Therefore the main step of the wavelength calibration is carried out in the recipes \hyperref[rec:lsslmflux]{\REC{metis_LM_lss_flux}} and \hyperref[rec:lsslmsci]{\REC{metis_LM_lss_sci}}. Finally, the telluric absorption correction is applied using the modelling approach with \texttt{molecfit}.


%------------------------------------------------------------------------------------------------------------------
\subsubsection{Recipes \REC{metis\_det\_lingain} and \REC{metis\_det\_dark}}
These recipes are described in Section~\ref{Sec:detector_calibration}.

%------------------------------------------------------------------------------------------------------------------
\subsubsection{Recipe \REC{metis\_LM\_adc\_slitloss}}
The recipe \hyperref[sssec:adc_slitlosses]{\REC{metis_lm_adc_slitloss}} aims to determine the slit losses induced by the fixed \ac{ADC} positions as function of the object position across the slit. The recipe aims to create a table with slitlosses (\hyperref[dataitem:lmadcslitloss]{\STATCALIB{LM_ADC_SLITLOSS}}), which is added to the static database and used in the recipes \hyperref[rec:lsslmflux]{\REC{metis_LM_lss_flux}}. This recipe is to be carried out only when an update of the database is needed. The algorithm and the workflow of the recipe to determine the slitlosses is given in Section~\ref{sssec:adc_slitlosses}, more information can be found in Section "Calibration of slit losses" in the Calibration Plan \cite{METIS-calibration_plan}. 


%------------------------------------------------------------------------------------------------------------------
\subsubsection{LM-LSS Flatfielding recipe \REC{metis\_LM\_lss\_rsrf}:}\label{rec:lsslmrsrf}
The recipe \hyperref[rec:lsslmrsrf]{\REC{metis_LM_lss_rsrf}} aims to create a spectroscopic master flatfield for determining the pixel-to-pixel sensitivity and to enable the order location algorithm (\hyperref[rec:lsslmtrace]{\REC{metis_LM_lss_trace}}).
\begin{figure}[ht]
  \centering
  \includegraphics[width=0.5\textheight]{figures/metis_lm_lss_rsrf_v0.74.pdf}
  \caption[Recipe: \REC{metis\_LM\_lss\_rsrf}]{\REC{metis\_LM\_lss\_rsrf} --
    Recipe workflow to create the spectroscopic flatfield by means of the \ac{RSRF}.}
  \label{Fig:rec_lm_lss_rsrf}
\end{figure}

\begin{recipedef}
Name:		& \hyperref[rec:lsslmrsrf]{\REC{metis_LM_lss_rsrf}}  \\
Purpose:	& Spectroscopic flatfielding with \ac{RSRF} \\
Type:		& Calibration\\
Requirements: & None \\
Templates:           & \TPL{METIS_spec_lm_cal_rsrf} \\
Input data:     & $N\times$ \hyperref[dataitem:lmlsswaveraw]{\RAW{LM_LSS_RSRF_RAW}} \\
                & \hyperref[dataitem:persistencemap]{\EXTCALIB{PERSISTENCE_MAP}}  \\
                & \hyperref[dataitem:gainmap2rg]{\STATCALIB{GAIN_MAP_2RG}}  \\
                & \hyperref[dataitem:badpixmap2rg]{\PROD{BADPIX_MAP_2RG}}  \\
                & \hyperref[dataitem:masterdark2rg]{\PROD{MASTER_DARK_2RG}}  \\
Parameters: 	& TBD\\
Algorithm:      & subtract master \ac{WCU} "OFF" frame from illumination frame (done on individual images)\\
                & median/mean filtering of subtracted images\\
                & division by blackbody spectrum\\
                & normalisation to achieve \ac{RSRF}\\
Output data:	& \hyperref[dataitem:lsslmrsrfmaster]{\PROD{MASTER\_LM\_LSS\_RSRF}} (\FITS{PRO.CATG=MASTER_LM_LSS_RSRF}): \\
                & \hyperref[dataitem:medianlmrsrfimg]{\PROD{MEDIAN_LM_LSS_RSRF_IMG}}\\
                & \hyperref[dataitem:meanlmrsrfimg]{\PROD{MEAN_LM_LSS_RSRF_IMG}}\\
Expected accuracies: & 3\% (cf. \cite{METIS-calibration_plan} and \cite{METIS_calerrbudget})\\
QC1 parameters: & \hyperref[qc:lmlssrsrfmeanlevel]{\QC{QC LM LSS RSRF MEAN LEVEL}}: Mean level of the \ac{RSRF}\\
                & \hyperref[qc:lmlssrsrfmedianlevel]{\QC{QC LM LSS RSRF MEDIAN LEVEL}}: Median level of the \ac{RSRF}\\
                & \hyperref[qc:lmlssrsrfintordrlevel]{\QC{QC LM LSS RSRF INTORDR LEVEL}}: Flux level of the interorder background\\
                & \hyperref[qc:lmlssrsrfnormstdev]{\QC{QC LM LSS RSRF NORM STDEV}}: Standard deviation of the normalised \ac{RSRF}\\
                & \hyperref[qc:lmlssrsrfnormsnr]{\QC{QC LM LSS RSRF NORM SNR}}: \ac{SNR} of the normalised \ac{RSRF}\\
                & more TBD\\
\end{recipedef}
\clearpage

%------------------------------------------------------------------------------------------------------------------
\subsubsection{LM-LSS Order detection \REC{metis\_LM\_lss\_trace}:}\label{rec:lsslmtrace}
The recipe \hyperref[rec:lsslmtrace]{\REC{metis_LM_lss_trace}} aims at detecting the orders and a polynomial fitting of the order locations (see \cite{pis02} and \cite{pis21} for details on the algorithms). The detection and polynomial fitting is based on flatfield frames taken through a pinhole mask, which leads to individual pinhole traces along the entire dispersion direction.

\begin{figure}[ht]
  \centering
  \includegraphics[width=0.5\textheight]{figures/metis_lm_lss_trace_v0.74.pdf}
  \caption[Recipe: \REC{metis_LM_lss_trace}]{\REC{metis_LM_lss_trace} --
    Detection and polynomial fitting of the order location.}
  \label{Fig:rec_lm_lss_wtrace}
\end{figure}

\begin{recipedef}
Name:		&  \hyperref[rec:lsslmtrace]{\REC{metis_LM_lss_trace}} \\
Purpose:	& Detection of order location \\
Type:		& Calibration\\
Requirements: & None \\
Templates:           & \TPL{METIS_spec_lm_cal_rsrfpinh}  \\
Input data:     & $N\times$ \hyperref[dataitem:lmlssrsrfpinhraw]{\RAW{LM_LSS_RSRF_PINH_RAW}} \\
                & \hyperref[dataitem:persistencemap]{\EXTCALIB{PERSISTENCE_MAP}}  \\
                & \hyperref[dataitem:gainmap2rg]{\STATCALIB{GAIN_MAP_2RG}}  \\
                & \hyperref[dataitem:badpixmap2rg]{\PROD{BADPIX_MAP_2RG}}  \\
                & \hyperref[dataitem:masterdark2rg]{\PROD{MASTER_DARK_2RG}}  \\
                & \hyperref[dataitem:lsslmrsrfmaster]{\PROD{MASTER\_LM\_LSS\_RSRF}} \\
Parameters: 	& polynomial degree\\
Algorithm:      & Detection of the order edges\\
                & Polynomial fitting\\
Output data:	& \hyperref[dataitem:lmlsstrace]{\PROD{LM_LSS_TRACE}} (\FITS{PRO.CATG=LM_LSS_TRACE}): Polynomial coefficients\\
Expected accuracies: & (TBD)\\
QC1 parameters: & \hyperref[qc:lmlsstracelpolydeg]{\QC{QC LM LSS TRACE LPOLYDEG}}: Degree of the polynomial fit of the left order edge\\
                & \hyperref[qc:lmlsstracelcoeffi]{\QC{QC LM LSS TRACE LCOEFF<i>}}: $i$-th coefficient of the polynomial of the left order edge\\
                & \hyperref[qc:lmlsstracerpolydeg]{\QC{QC LM LSS TRACE RPOLYDEG}}: Degree of the polynomial fit of the right order edge\\
                & \hyperref[qc:lmlsstracercoeffi]{\QC{QC LM LSS TRACE RCOEFF<i>}}: $i$-th coefficient of the polynomial of the right order edge\\
                & \hyperref[qc:lmlsstraceintrordrlevel]{\QC{QC LM LSS TRACE INTORDR LEVEL}}: Flux level of the interorder background\\
                & more TBD\\
\end{recipedef}

\clearpage
%------------------------------------------------------------------------------------------------------------------
\subsubsection{LM-LSS wavelength calibration recipe \REC{metis\_LM\_lss\_wave}:}\label{rec:lsslmwave}
This recipe aims at determining the first guess of the wavelength calibration on basis of the \ac{WCU} laser sources (c.f. \cite{METIS-calibration_plan}). Therefore the first steps are the removal of the detector signature of the \FITS{LM_WAVE_RAW} frames by applying the master calibration files derived in the previous steps, following by the background subtraction (if needed, TBD) and the application of the RSRF. The distortion of the lines (i.e. possible tilt, curvature,...) and the wavelength solution is determined by the algorithm developed by Piskunov et al. (\cite{pis02}, \cite{pis21}). The reference frame is defined by the laser line catalogue (\hyperref[dataitem:lasertab]{\STATCALIB{LASER_TAB}}).

\begin{figure}[ht]
  \centering
  \includegraphics[width=0.5\textheight]{figures/metis_lm_lss_wave_v0.74.pdf}
  \caption[Recipe: \REC{metis\_LM\_lss\_wave}]{\REC{metis\_LM\_lss\_wave} --
    Creation of the LM LSS master wavelength correction.}
  \label{Fig:rec_lm_lss_trace}
\end{figure}
\clearpage

\begin{recipedef}
Name:		& \hyperref[rec:lsslmwave]{\REC{metis_LM_lss_wave}} \\
Purpose:	& Wavelength calibration \\
Type:		& Calibration\\
Requirements: & METIS-6084, METIS-1371, METIS-6074 \\
Templates:           & \TPL{METIS_spec_lm_cal_internalwave}, \\
Input data: 	& \hyperref[dataitem:lmlsswaveraw]{\RAW{LM_LSS_WAVE_RAW}}\\
                & \hyperref[dataitem:persistencemap]{\EXTCALIB{PERSISTENCE_MAP}}  \\
                & \hyperref[dataitem:gainmap2rg]{\STATCALIB{GAIN_MAP_2RG}}  \\
                & \hyperref[dataitem:badpixmap2rg]{\PROD{BADPIX_MAP_2RG}}  \\
                & \hyperref[dataitem:masterdark2rg]{\PROD{MASTER_DARK_2RG}}  \\
                & \hyperref[dataitem:lsslmrsrfmaster]{\PROD{MASTER\_LM\_LSS\_RSRF}} \\
                & \hyperref[dataitem:lmlsstrace]{\PROD{LM_LSS_TRACE}} \\
                & \hyperref[dataitem:lasertab]{\STATCALIB{LASER_TAB}} \\
                % & \STATCALIB{REF_AIRG_CAT} \\
Parameters: 	& (TBD)\\
Algorithm:      & Application of detector master calibration files\\
                & Determination and application of the distortion correction\\
                & Determination of the first guess of the wavelength solution by polynomial fit of the detected laser source lines\\
Output data:	& \hyperref[dataitem:lmlsscurve]{\PROD{LM_LSS_CURVE}} (\FITS{PRO.CATG=LM_LSS_CURVE}): Curvature \\
                & \hyperref[dataitem:lmlssdistsol]{\PROD{LM_LSS_DIST_SOL}} (\FITS{PRO.CATG=LM_LSS_DIST_SOL}): Distortion solution\\
                & \hyperref[dataitem:lmlsswaveguess]{\PROD{LM_LSS_WAVE_GUESS}} (\FITS{PRO.CATG=LM_LSS_WAVE_GUESS}): Wavelength first guess\\
Expected accuracies: & 1/5th of a pixel after post-processing (cf. \cite{METIS-calibration_plan})\\
QC1 parameters: & \hyperref[qc:lmlsswavepolydeg]{\QC{QC LM LSS WAVE POLYDEG}}: Degree of the first guess polynomial\\
                & \hyperref[qc:lmlsswavecoeffi]{\QC{QC LM LSS WAVE COEFF<i>}}: $i$-th coefficient of the polynomial\\
                & \hyperref[qc:lmlsswavenlines]{\QC{QC LM LSS WAVE NLINES}}: Number of detected (laser) lines; should be constant\\
                & \hyperref[qc:lmlsswavelinefwhmavg]{\QC{QC LM LSS WAVE LINEFWHMAVG}}: Average of the \ac{FWHM} of the detected lines (should be widely constant)\\
                & \hyperref[qc:lmlsswaveinterordrlevel]{\QC{QC LM LSS WAVE INTORDR LEVEL}}: Flux level of the interorder background\\
                & more TBD: e.g. QC params for distortion determination and correction\\
\end{recipedef}

\clearpage
%------------------------------------------------------------------------------------------------------------------
\subsubsection{LM-LSS flux calibration recipe \REC{metis_LM_lss_flux}:}\label{rec:lsslmflux}
Flux calibration with spectrophotometric standard stars: As first step the detector master calibration files derived previously are applied followed by the background subtraction, if needed the distortion correction (\hyperref[dataitem:lmlssdistsol]{\PROD{LM_LSS_DIST_SOL}}), and
the wavelength calibration by means of the first guess solution (\hyperref[dataitem:lmlsswaveguess]{\PROD{LM_LSS_WAVE_GUESS}}) and the telluric sky lines (c.f. Sect.\,8.5 in \cite{DRLS}). Then the recipe extracts the standard star spectrum object, removes sky lines, collapses the 2D to 1D spectra and applies a telluric correction in an automated way to the standard star spectrum (in contrast to the science observations, which are telluric corrected in a dedicated recipe to achieve the best correction). The response curve is obtained by comparing the extracted spectrum with a model and/or another reference spectrum of the standard star. Currently it is foreseen to use the same standard stars as in \ac{CRIRES}/CRIRES+ and \ac{VISIR}. It is under investigation whether more stars are needed.
\begin{figure}[ht]
  \centering
  \includegraphics[width=0.4\textheight]{figures/metis_lm_lss_flux_v0.74.pdf}
  \caption[Recipe: \REC{metis_LM_lss_flux}]{\REC{metis_LM_lss_flux} --
    Flux calibration recipe.}
  \label{Fig:rec_lm_lss_flux}
\end{figure}
\clearpage
\begin{recipedef}
Name:		& \hyperref[rec:lsslmflux]{\REC{metis_LM_lss_flux}} \\
Purpose:	& Flux calibration \\
Type:		& Calibration\\
Requirements: & METIS-6084, METIS-6074 \\
Templates:           & \TPL{METIS_spec_lm_cal_standard}\\
Input data: 	& \hyperref[dataitem:lmlssfluxraw]{\RAW{LM_LSS_FLUX_RAW}}\\
                & \hyperref[dataitem:persistencemap]{\EXTCALIB{PERSISTENCE_MAP}}  \\
                & \hyperref[dataitem:gainmap2rg]{\STATCALIB{GAIN_MAP_2RG}}  \\
                & \hyperref[dataitem:badpixmap2rg]{\PROD{BADPIX_MAP_2RG}}  \\
                & \hyperref[dataitem:masterdark2rg]{\PROD{MASTER_DARK_2RG}}  \\
                & \hyperref[dataitem:lsslmrsrfmaster]{\PROD{MASTER\_LM\_LSS\_RSRF}} \\
                & \hyperref[dataitem:lmlssdistsol]{\PROD{LM_LSS_DIST_SOL}} \\
                & \hyperref[dataitem:lmlsswaveguess]{\PROD{LM_LSS_WAVE_GUESS}} \\
                & \hyperref[dataitem:aopsfmodel]{\EXTCALIB{AO_PSF_MODEL}} \\
                & \hyperref[dataitem:atmlinecat]{\EXTCALIB{ATM_LINE_CAT}} \\
                & \hyperref[dataitem:lmadcslitloss]{\STATCALIB{LM_ADC_SLITLOSS}}\\
                & \hyperref[dataitem:lmsynthtrans]{\STATCALIB{LM_SYNTH_TRANS}}\\
                & \hyperref[dataitem:reffluxcat]{\STATCALIB{REF_FLUX_CAT}} \\
Parameters: 	& (TBD)\\
Algorithm:      & Application of master calibration files\\
                & Background removal\\
                & Determination and application of the distortion correction\\
                & Determination and application of the wavelength solution\\
                & Identifying/separatiing sky/object pixels\\
                & Removing sky lines: Creation and Subtraction of 2D sky\\
                & Collapsing 2D to 1D spectrum, (see Fig.\,\ref{Fig:rec_lm_lss_sci})\\
                & Determination and application of response curve\\
Output data:	& \hyperref[dataitem:lmlssstdobjmap]{\PROD{LM_LSS_STD_OBJ_MAP}}: Pixel map of object pixels\\
            	& \hyperref[dataitem:lmlssstdskymap]{\PROD{LM_LSS_STD_SKY_MAP}}: Pixel map of sky pixels\\
              	& \hyperref[dataitem:lmlssstd1d]{\PROD{LM_LSS_STD_1D}}: coadded, wavelength calibrated, collapsed 1D spectrum\\
                & \hyperref[dataitem:lsslmresp]{\PROD{MASTER\_LM\_RESPONSE}}: response function (TBD)\\
Expected accuracies: & 10\% over an atmospheric band (ESO Req. R-MET-107)\\
            & $<30$\% absolute line flux accuracy (R-MET-107)\\
            & $<5$\% absolute flux calibration (R-MET-82)\\
QC1 parameters: & \hyperref[qc:lmlssstdbackgdmean]{\QC{QC LM LSS STD BACKGD MEAN}}: Mean value of background\\
                & \hyperref[qc:lmlssstdbackgdmedian]{\QC{QC LM LSS STD BACKGD MEDIAN}}: Median value of background\\
                & \hyperref[qc:lmlssstdbackgdstdev]{\QC{QC LM LSS STD BACKGD STDEV}}: Standard deviation value of background\\
                & \hyperref[qc:lmlssstdsnr]{\QC{QC LM LSS STD SNR}}: Signal-to-noise ration of flux standard star spectrum\\
                & \hyperref[qc:lmlssstdsnrnoise]{\QC{QC LM LSS STD SNRNOISE}}: Noise level of flux standard star spectrum\\
                & \hyperref[qc:lmlssstdfwhm]{\QC{QC LM LSS STD FWHM}}: FWHM of flux standard spectrum\\
                & \hyperref[qc:lmlssfluxintrordravglevel]{\QC{QC LM LSS FLUX INTORDR LEVEL}}: Flux level of the interorder background\\
                & \hyperref[qc:lmlssfluxlevel]{\QC{QC LM LSS FLUX AVGLEVEL}}: Average level of the standard star flux \\
                & \hyperref[qc:lmlssfluxwavecaldevmean]{\QC{QC LM LSS FLUX WAVECAL DEVMEAN}}: Mean deviation from the
                  wavelength reference frame (TBDef)\\
                & \hyperref[qc:lmlssfluxwavecalfwhm]{\QC{QC LM LSS FLUX WAVECAL FWHM}}: Measured FWHM of lines\\
                & \hyperref[qc:lmlssfluxwavecalnident]{\QC{QC LM LSS FLUX WAVECAL NIDENT}}: Number of identified lines\\
                & \hyperref[qc:lmlssfluxwavecalnmatch]{\QC{QC LM LSS FLUX WAVECAL NMATCH}}: Number of lines matched between
                    catalogue and spectrum\\
                & \hyperref[qc:lmlssfluxwavecalpolydeg]{\QC{QC LM LSS FLUX WAVECAL POLYDEG}}: Degree of the polynomial\\
                & \hyperref[qc:lmlssfluxwavecalpolycoeffn]{\QC{QC LM LSS FLUX WAVECAL POLYCOEFF\<n\>}}: $n$-th coefficient of the polynomial\\
                & \hyperref[qc:lmlssfluxstdsnr]{\QC{QC LM LSS FLUX STDSNR}}: Signal-to-noise ration of flux standard star spectrum\\
                & \hyperref[qc:lmlssfluxsnrnoise]{\QC{QC LM LSS FLUX SNRNOISE}}: Noise level of flux standard star spectrum\\
                & \hyperref[qc:lmlssfluxfwhm]{\QC{QC LM LSS FLUX FWHM}}: FWHM of flux standard spectrum\\
                & \hyperref[qc:lmlssfluxpsfloss]{\QC{QC LM LSS FLUX PSFLOSS}}: Fraction of AO induced slit losses (TBdef)\\
                & more TBD
\end{recipedef}

\subsubsection{LM-LSS science reduction recipe \REC{metis_LM_lss_sci}:}\label{rec:lsslmsci}
The science calibration recipe comprises the extraction of the object (i.e. separation of object/sky pixels), removing the sky lines, the application of the response curve previously defined, the 2D to 1D collapse and the coaddition. In contrast to the flux standard star reduction, the telluric correction on the science data is done in a dedicated recipe afterwards to achieve best quality for the correction.
\begin{figure}[ht]
  \centering
  \includegraphics[width=0.38\textheight]{figures/metis_lm_lss_sci_v0.74.pdf}
  \caption[Recipe: \REC{metis_LM_lss_sci}]{\REC{metis_LM_lss_sci} --
    Science reduction recipe.}
  \label{Fig:rec_lm_lss_sci}
\end{figure}
\clearpage

\begin{recipedef}
Name:		& \hyperref[rec:lsslmsci]{\REC{metis_LM_lss_sci}} \\
Purpose:    & Science data calibration\\
Type:		& Science reduction\\
Requirements: & METIS-6084 \\
Templates:           & \TPL{METIS_spec_lm_acq}, \\
                & \TPL{METIS_spec_lm_obs_AutoNodOnSlit}, \\
                & \TPL{METIS_spec_lm_obs_GenericOffset} \\
                & \TPL{METIS_spec_lm_cal_slit_adc}\\
Input data: 	& \hyperref[dataitem:lmlsssciraw]{\RAW{LM_LSS_SCI_RAW}}\\
                & \hyperref[dataitem:persistencemap]{\EXTCALIB{PERSISTENCE_MAP}}  \\
                & \hyperref[dataitem:gainmap2rg]{\STATCALIB{GAIN_MAP_2RG}}  \\
                & \hyperref[dataitem:badpixmap2rg]{\PROD{BADPIX_MAP_2RG}}  \\
                & \hyperref[dataitem:masterdark2rg]{\PROD{MASTER_DARK_2RG}}  \\
                & \hyperref[dataitem:lsslmrsrfmaster]{\PROD{MASTER\_LM\_LSS\_RSRF}} \\
                & \hyperref[dataitem:lmlssdistsol]{\PROD{LM_LSS_DIST_SOL}} \\
                & \hyperref[dataitem:lmlsswaveguess]{\PROD{LM_LSS_WAVE_GUESS}} \\
                & \hyperref[dataitem:atmlinecat]{\EXTCALIB{ATM_LINE_CAT}} \\
                & \hyperref[dataitem:lmadcslitloss]{\STATCALIB{LM_ADC_SLITLOSS}}\\
                %& \hyperref[dataitem:aopsfmodel]{\EXTCALIB{AO_PSF_MODEL}} \\
                %& \hyperref[dataitem:lsfkernel]{\STATCALIB{LSF_KERNEL}}\\
                & \hyperref[dataitem:lsslmresp]{\PROD{MASTER\_LM\_RESPONSE}} \\
Parameters: 	& (TBD)\\
Algorithm:      & Application of the detector master calib files\\
                & wavelength calibration \\
                & Identifying/separatiing sky/object pixels\\
                & Removing sky lines: Creation and Subtraction of 2D sky\\
                & Coaddition of individual object spectra of one OB\\
                & Collapsing 2D to 1D spectrum, (see Fig.\,\ref{Fig:rec_lm_lss_sci})\\
                & Application of the response function (flux calibration) \\
Output data:	& \hyperref[dataitem:lmlsssciobjmap]{\PROD{LM_LSS_SCI_OBJ_MAP}}: Pixel map of object pixels\\
            	& \hyperref[dataitem:lmlsssciskymap]{\PROD{LM_LSS_SCI_SKY_MAP}}: Pixel map of sky pixels\\
            	& \hyperref[dataitem:lmlsssci2d]{\PROD{LM_LSS_SCI_2D}}: coadded, wavelength calibrated 2D spectrum\\
                & (\FITS{PRO_CATG}: \FITS{LM_LSS_2d_coadd_wavecal}) \\
                & \hyperref[dataitem:lmlsssci1d]{\PROD{LM_LSS_SCI_1D}}: coadded, wavelength calibrated 1D spectrum\\
                & (\FITS{PRO_CATG}: \FITS{LM_LSS_1d_coadd_wavecal}) \\
                & \hyperref[dataitem:lmlsssciflux2d]{\PROD{LM_LSS_SCI_FLUX_2D}}: coadded, wavelength + flux calibrated 2D spectrum\\
                & (\FITS{PRO_CATG}: \FITS{LM_LSS_2d_coadd_wavecal}) \\
              	& \hyperref[dataitem:lmlsssciflux1d]{\PROD{LM_LSS_SCI_FLUX_1D}}: coadded, wavelength + flux 1D spectrum\\
                & (\FITS{PRO_CATG}: \FITS{LM_LSS_1d_coadd_wavecal}) \\
Expected accuracies: & (TBD)\\
QC1 parameters: & \hyperref[qc:lmlssscisnr]{\QC{QC LM LSS SCI SNR}}: Signal-to-noise ration of science spectrum\\
                & \hyperref[qc:lmlssscisnrnoise]{\QC{QC LM LSS SCI SNRNOISE}}: Noise level of science spectrum\\
                & \hyperref[qc:lmlssscifluxsnr]{\QC{QC LM LSS SCI FLUX SNR}}: Signal-to-noise ration of flux calibrated  science spectrum\\
                & \hyperref[qc:lmlssscifluxsnrnoise]{\QC{QC LM LSS SCI FLUX SNRNOISE}}: Noise level of flux calibrated science spectrum\\
                & \hyperref[qc:lmlsssciinterordrlevel]{\QC{QC LM LSS SCI INTORDR LEVEL}}: Flux level of the interorder background\\
                & \hyperref[qc:lmlsssciwavecaldevmean]{\QC{QC LM LSS SCI WAVECAL DEVMEAN}}: Mean deviation from the wavelength reference frame (TBDef)\\
                & \hyperref[qc:lmlsssciwavecalfwhm]{\QC{QC LM LSS SCI WAVECAL FWHM}}: Measured FWHM of lines\\
                & \hyperref[qc:lmlsssciwavecalnident]{\QC{QC LM LSS SCI WAVECAL NIDENT}}: Number of identified lines\\
                & \hyperref[qc:lmlsssciwavecalnmatch]{\QC{QC LM LSS SCI WAVECAL NMATCH}}: Number of lines matched between catalogue and spectrum\\
                & \hyperref[qc:lmlsssciwavecalpolydeg]{\QC{QC LM LSS SCI WAVECAL POLYDEG}}: Degree of the wavelength polynomial\\
                & \hyperref[qc:lmlsssciwavecalpolycoeffn]{\QC{QC LM LSS SCI WAVECAL POLYCOEFF\<n\>}}: $n$-th coefficient of the polynomial\\
                & more TBD\\
\end{recipedef}

\subsubsection{LM-LSS telluric correction recipe \REC{metis_LM_lss_mf_model}:}\label{rec:LMLSSmfmodel}
The telluric correction will be done with the package \texttt{molecfit}\footnote{\url{https://www.eso.org/sci/software/pipelines/molecfit/molecfit-pipe-recipes.html}}. It is realised in three individual recipes, \hyperref[rec:LMLSSmfmodel]{\REC{metis_LM_lss_mf_model}}, which calculates the best-fit model, \hyperref[rec:LMLSSmfcalctrans]{\REC{metis_LM_lss_mf_calctrans}}, which creates a synthetic transmission curve, and \hyperref[rec:LMLSSmfcorrect]{\REC{metis_LM_lss_mf_correct}}, which performs the actual telluric correction by means of the synthetic transmission.

\begin{figure}[ht]
  \centering
  \includegraphics[width=0.5\textheight]{figures/metis_lm_lss_mf_model_v0.74.pdf}
  \caption[Recipe: \REC{metis_LM_lss_mf_model}]{\REC{metis_LM_lss_mf_model} --
    Recipe to achieve the best-fit for the calculation of the synthetic transmission curve for the telluric correction.}
  \label{Fig:rec_lm_lss_mf_model}
\end{figure}
\clearpage

\begin{recipedef}
Name:		& \hyperref[rec:LMLSSmfmodel]{\REC{metis_LM_lss_mf_model}} \\
Purpose:	& Achieve the best fit for modelling the transmission curve to be applied as telluric correction \\
Type:		& Post-calibration\\
Requirements: & METIS-4051, METIS-6091 \\
Templates:           & None\\
Input data: 	& \hyperref[dataitem:lmlsssciflux1d]{\PROD{LM_LSS_SCI_FLUX_1D}}\\
                & \hyperref[dataitem:lsfkernel]{\STATCALIB{LSF_KERNEL}} \\
                & \hyperref[dataitem:atmprofile]{\EXTCALIB{ATM_PROFILE}} \\
                & \hyperref[dataitem:atmlinecat]{\EXTCALIB{ATM_LINE_CAT}} \\
Parameters: 	& \texttt{molecfit} parameters (c.f. \cite{molecfit})\\
Algorithm:      & Fit of telluric features visible in the science input spectrum\\
                & Determination of best-fit parameter set\\
Output data:	& \hyperref[dataitem:mfbestfittab]{\PROD{MF\_BEST\_FIT\_TAB}}: Table with best-fit parameters\\
Expected accuracies: & (TBD)\\
QC1 parameters: & cf. \cite{molecfit}\\
\end{recipedef}

\subsubsection{LM-LSS telluric correction recipe \REC{metis_LM_lss_mf_calctrans}:}\label{rec:LMLSSmfcalctrans}

\begin{figure}[ht]
  \centering
  \includegraphics[width=0.5\textheight]{figures/metis_lm_lss_mf_calctrans_v0.74.pdf}
  \caption[Recipe: \REC{metis_LM_lss_mf_calctrans}]{\REC{metis_LM_lss_mf_calctrans} --
    Recipe to calculate the synthetic transmission to be applied as telluric correction.}
  \label{Fig:rec_lm_lss_mf_calctrans}
\end{figure}
\clearpage

\begin{recipedef}
Name:		& \hyperref[rec:LMLSSmfcalctrans]{\REC{metis_LM_lss_mf_calctrans}} \\
Purpose:	& Calculation of the synthetic transmission \\
Type:		& Post-calibration\\
Requirements: & METIS-4051, METIS-6091 \\
Templates:           & None\\
Input data: 	& \hyperref[dataitem:mfbestfittab]{\PROD{MF\_BEST\_FIT\_TAB}}: Table with best-fit parameters\\
                & \hyperref[dataitem:lsfkernel]{\STATCALIB{LSF_KERNEL}} \\
                & \hyperref[dataitem:atmprofile]{\EXTCALIB{ATM_PROFILE}} \\
                & \hyperref[dataitem:atmlinecat]{\EXTCALIB{ATM_LINE_CAT}} \\
Parameters: 	& \texttt{molecfit} parameters (c.f.  \cite{molecfit})\\
Algorithm:      & Calculate the entire transmission curve by means of the best-fit parameters\\
Output data:	& \hyperref[dataitem:lmlsssynthttrans]{\PROD{LM_LSS_SYNTH_TRANS}}: synth. transmission\\
Expected accuracies: & (TBD)\\
QC1 parameters: & cf. \cite{molecfit}\\
\end{recipedef}

\subsubsection{LM-LSS telluric correction recipe \REC{metis_LM_lss_mf_correct}:}\label{rec:LMLSSmfcorrect}

\begin{figure}[ht]
  \centering
  \includegraphics[width=0.5\textheight]{figures/metis_lm_lss_mf_correct_v0.74.pdf}
  \caption[Recipe: \REC{metis_LM_lss_mf_correct}]{\REC{metis_LM_lss_mf_correct} --
    Recipe to apply the telluric correction.}
  \label{Fig:rec_lm_lss_mf_correct}
\end{figure}
\clearpage

\begin{recipedef}
Name:		& \hyperref[rec:LMLSSmfcorrect]{\REC{metis_LM_lss_mf_correct}} \\
Purpose:	& Apply the synthetic transmission to the science spectra \\
Type:		& Post-calibration\\
Requirements: & METIS-4051, METIS-6091 \\
Templates:           & None\\
Input data: 	& \hyperref[dataitem:lmlsssciflux1d]{\PROD{LM_LSS_SCI_FLUX_1D}}\\
                & \hyperref[dataitem:lmlsssynthttrans]{\PROD{LM_LSS_SYNTH_TRANS}}\\
Parameters: 	& None\\
Algorithm:      & Apply telluric correction, i.e. divide the input science spectrum\\
                & by the synthetic transmission\\
Output data:	& \hyperref[dataitem:lmlssscifluxtellcorr1d]{\PROD{LM_LSS_SCI_FLUX_TELLCORR_1D}}\\
Expected accuracies: & (TBD)\\
QC1 parameters: & cf. \cite{molecfit}\\
\end{recipedef}




% \input{06_4-Recipes_LSS_N}
% \input{06_5-Recipes_IFU}
% \input{06_6-Recipes_HCI}


% \subsection{Specifications and requirements}
% \label{Subsec:polarion}
% \input{01-Requirements}

\clearpage
\section{DRL Functions}\label{sec:drl_functions}

% Include Detecor Functions
\subsection{Detector Functions}\label{sec:drl_functions_det}

%---------------------------------------------------------------------
\subsubsection{Determine Dark Frames}\label{drl:det_dark}
\begin{recipedef}
Name: & \hyperref[drl:det_dark]{\DRL{metis_determine_dark}} \\
Purpose: & Determine the dark current of the detectors and create dark images\\
Used in recipes: & \hyperref[rec:metis_det_dark]{\REC{metis_det_dark}}, \hyperref[rec:metis_det_lingain]{\REC{metis_det_lingain}}\\
%Working remarks: & None \\
%Function Parameters: & None \\
Input: & $n\times$ \texttt{const hdrl\_image * input} \\
Other inputs: &  combination method (\texttt{median}, \texttt{mean}, \texttt{sigclip},\dots)\\
& parameters for combination method\\
QC outputs: & QC DARK MEAN\\
& QC DARK MEDIAN\\
& QC DARK RMS\\
%Output FITS files: & None \\
Outputs: & Dark frame\\
               & \texttt{cpl\_error\_code} \\
General description: & Determination of dark frames for subtraction of dark current \\
Mathematical description: & Median or mean of input images, with sigma clipping \\
Quality assessment: & Through QC parameters \\
Error conditions: & See \cite{DRLVT}. \\
Unit tests: & See \cite{DRLVT}. \\
\end{recipedef}

\subsubsection{Flag Deviant Pixels in Dark Frame}\label{drl:update_dark_mask}
\begin{recipedef}
Name: & \hyperref[drl:update_dark_mask]{\DRL{metis_update_dark_mask}} \\
Purpose: & Flag deviant (hot, cold, bad) pixels in the master dark and update the image mask\\
Used in recipes: & \hyperref[drl:det_dark]{\REC{metis_det_dark}}\\
%Working remarks: & None \\
%Function Parameters: & None \\
Input: & \texttt{const hdrl\_image * input} \\ 
Other inputs: & Threshold(s) for deviant pixel detection \\
QC outputs: & QC DARK NHOTPIX\\
& QC DARK NCOLDPIX \\
& QC DARK NBADPIX \\
%Output FITS files: & None \\
Outputs: & Updated dark frame mask\\
               & \texttt{cpl\_error\_code} \\
General description: & Flag deviant (hot/cold/bad) pixels in master dark \\
Mathematical description: & flag pixels outside the provided thresholds \\
Quality assessment: & Through QC parameters \\
Error conditions: & See \cite{DRLVT}. \\
Unit tests: & See \cite{DRLVT}. \\
\end{recipedef}

\subsubsection{Non-linearity Correction}\label{drl:img_nonlinear_correction}
\begin{recipedef}
Name: & \hyperref[drl:img_nonlinear_correction]{\DRL{img_nonlinear_correction}} \\
Purpose: &Correction for detector non-linearity\\
Used in recipes: & \hyperref[rec:metis_lm_img_basic_reduce]{\REC{metis_lm_img_basic_reduce}}\\
%Working remarks: & None \\
%Function Parameters: & None \\
Input: & $n\times$ \texttt{const hdrl\_image * input} \\
%Other inputs: & None \\
%QC outputs: & None\\
%Output FITS files: & None \\
Outputs: & Non-linearity corrected images [px]\\
                & \texttt{cpl\_error\_code} \\
General description: & Correction of detector non-linearity \\
Mathematical description: & TBD \\
Quality assessment: & Through QC parameters \\
Error conditions: & See \cite{DRLVT}. \\
Unit tests: & See \cite{DRLVT}. \\
\end{recipedef}

\subsubsection{Crosstalk correction}\label{drl:img_crosstalk_correction}
\begin{recipedef}
Name: & \hyperref[drl:img_crosstalk_correction]{\DRL{img_crosstalk_correction}} \\
Purpose: &Correction for detector crosstalk\\
Used in recipes: & \hyperref[rec:metis_lm_img_basic_reduce]{\REC{metis_lm_img_basic_reduce}}\\
%Working remarks: & None \\
%Function Parameters: & None \\
Input: & $n\times$ \texttt{const hdrl\_image * input} \\
%Other inputs: & None \\
%QC outputs: & None\\
%Output FITS files: & None \\
Outputs: & Crosstalk corrected images [px]\\
                & \texttt{cpl\_error\_code} \\
General description: & Correction of detector crosstalk \\
Mathematical description: & TBD \\
Quality assessment: & Through QC parameters \\
Error conditions: & See \cite{DRLVT}. \\
Unit tests: & See \cite{DRLVT}. \\
\end{recipedef}


\subsubsection{Apply Persistance Correction}\label{drl:img_apply_persistence_correction}
\begin{recipedef}
Name: & \hyperref[drl:img_apply_persistence_correction]{\DRL{metis_img_apply_persistence_correction}} \\
Purpose: & Apply the persistence correction to raw images\\
Used in recipes: & \hyperref[sssec:lm_img_flatfield]{\REC{metis_lm_img_flat}}\\
%Working remarks: & None \\
%Function Parameters: & None \\
Input: & $n\times$ \texttt{const hdrl\_image * input} \\
Other inputs: \hyperref[dataitem:persistence_map]{\PROD{PERSISTENCE_MAP}} \\
QC outputs: \hyperref[qc:qc_persist_count]{\QC{QC PERSIST COUNT}} \\
%Output FITS files: & None \\
Outputs: & Corrected raw images\\
         & \texttt{cpl\_error\_code} \\
General description: & Persistence correction of raw images \\
Mathematical description: & see Section~\ref{sec_persistence_correction} \\
Quality assessment: & Through QC parameters \\
Error conditions: & See \cite{DRLVT}. \\
Unit tests: & See \cite{DRLVT}. \\
\end{recipedef}

% Include LMS mode
\subsection{LMS observing mode}\label{sec:drl_functions_lms}

%Include IMG mode
\subsection{IMG observing mode}\label{sec:drl_functions_img}

%---------------------------------------------------------------------
\subsubsection{metis\_derive\_gain}\label{drl:metis_derive_gain}
\begin{recipedef}
Name: & \hyperref[drl:metis_derive_gain]{\DRL{metis_derive_gain}} \\
Purpose: & Determine the gain. \\
Used in recipes: & \hyperref[sssec:metis_det_lingain]{\REC{metis_det_lingain}}\\
%Working remarks: & None \\
%Function Parameters: & None \\
Input: & $n\times$ \texttt{const double * means} \\
%Other inputs: & None \\
%QC outputs: & None\\
%Output FITS files: & None \\
Outputs: & gain [\TODO{units}]\\
               & \texttt{cpl\_error\_code} \\
General description: & Determine the gain as the slope of variance against mean of detlin images. \\
Mathematical description: & TBD \\
Quality assessment: & Through QC parameters \\
Error conditions: & None. \\
%Unit tests: & See \cite{DRLVT} (TBD). \\
\end{recipedef}
%---------------------------------------------------------------------
\subsubsection{Detect peak centroid location}\label{drl:img_peakcentroid}
\begin{recipedef}
Name: & \hyperref[drl:img_peakcentroid]{\DRL{detect\_centroid\_peak}} \\
Purpose: &Detect the location of a source peak by a centroid\\
Used in recipes: & \hyperref[rec:metisimgchophome]{\REC{metis_img_chophome}}\newline
\hyperref[rec:metislmadcmslitloss]{\REC{metis_lm_adc_slitloss}} \newline
\hyperref[rec:metisnadcmslitloss]{\REC{metis_n_adc_slitloss}}\\
%Working remarks: & None \\
%Function Parameters: & None \\
Input: & $n\times$ \texttt{const hdrl\_image * input} \\
%Other inputs: & None \\
%QC outputs: & None\\
%Output FITS files: & None \\
Outputs: & Location of detected peak [px]\\
               & \texttt{cpl\_error\_code} \\
General description: & Detection of a source by centroid fit \\
Mathematical description: & see Section~\ref{sssec:centroid} TBD \\
Quality assessment: & Through QC parameters \\
Error conditions: & See \cite{DRLVT} (TBD). \\
Unit tests: & See \cite{DRLVT} (TBD). \\
\end{recipedef}

%Include LSS mode
\subsection{LSS observing mode}\label{sec:drl_functions_lss}

%---------------------------------------------------------------------
\subsubsection{Order background correction}\label{drl:correctorder}
\begin{recipedef}
Name: & \hyperref[drl:correctorder]{\DRL{correct\_order\_bg}} \\
Purpose: & Removal of order background contamination (e.g. stray light)\\
Used in recipes: & \hyperref[rec:lsslmrsrf]{\REC{metis\_LM\_lss\_rsrf}} \newline
                  \hyperref[rec:lssnrsrf]{\REC{metis\_N\_lss\_rsrf}} \newline
                  \hyperref[rec:lsslmwave]{\REC{metis\_LM\_lss\_wave}} \newline
                  \hyperref[rec:lsslmflux]{\REC{metis\_LM\_lss\_flux}} \newline
                  \hyperref[rec:lssnflux]{\REC{metis\_N\_lss\_flux}} \newline
                  \hyperref[rec:lsslmsci]{\REC{metis\_LM\_lss\_sci}}\newline
                  \hyperref[rec:lssnsci]{\REC{metis\_N\_lss\_sci}}\\
%Working remarks: & None \\
%Function Parameters: & None \\
Input: & $n\times$ \texttt{const hdrl\_image * input} \\
%Other inputs: & None \\
%QC outputs: & None\\
%Output FITS files: & None \\
Other outputs: & \texttt{cpl\_error\_code} \\
General description: & Removal of stray light by a 2D-polynomial fits of the order background \\
Mathematical description: & see Section~\ref{ssec:orderbg} \\
Quality assessment: & Through QC parameters \\
Error conditions: & See \cite{DRLVT}. \\
Unit tests: & See \cite{DRLVT}. \\
\end{recipedef}

%---------------------------------------------------------------------
\subsubsection{Slit curvature detection}\label{drl:slitcurvature}
\begin{recipedef}\label{rec:slitcurvature}
Name: & \hyperref[drl:slitcurvature]{\DRL{slit\_curvature}} \\
Purpose: & Determines the slit curvature along the orders \\
Used in recipes: & \hyperref[rec:lsslmwave]{\REC{metis\_LM\_lss\_wave}} \\
%Working remarks: & None \\
%Function Parameters: & None \\
Input: & \texttt{const hdrl\_image * input} \\
Other inputs: & None\\
%QC outputs: & None \\
%Output FITS files: & None \\
Other outputs: & \texttt{cpl\_error\_code} \\
General description: & Determines the tilt and shear of the orders using the sky emission lines \\
Mathematical description: &  see \cite{pis02} and \cite{pis21}\\
Quality assessment: & Through QC parameters \\
Error conditions: & See \cite{DRLVT}. \\
Unit tests: & See \cite{DRLVT}. \\
\end{recipedef}

%---------------------------------------------------------------------
\subsubsection{Subtract WCU OFF frame from frame}\label{drl:subtrwcuoffillum}
Used in recipes:\\ 
\hyperref[rec:lsslmrsrf]{\REC{metis\_LM\_lss\_rsrf}} \newline
\hyperref[rec:lssnrsrf]{\REC{metis\_N\_lss\_rsrf}} \newline
\hyperref[rec:metislmadcmslitloss]{\REC{metis_lm_adc_slitloss}} \newline
\hyperref[rec:metisnadcmslitloss]{\REC{metis_n_adc_slitloss}} \newline
%\hyperref[rec:]{\REC{}} \newline
TBD
%---------------------------------------------------------------------
\subsubsection{Calculate blackbody spectrum}\label{drl:calcbb}
Used in recipes:\\ 
\hyperref[rec:lsslmrsrf]{\REC{metis\_LM\_lss\_rsrf}} \newline
\hyperref[rec:lssnrsrf]{\REC{metis\_N\_lss\_rsrf}} \newline
TBD
%---------------------------------------------------------------------
\subsubsection{Normalise RSRF}\label{drl:normrsrf}
\begin{recipedef}\label{drl:normflat}
Name: & \hyperref[drl:normflat]{\DRL{norm\_flat}} \\
Purpose: & Creates normalised flats from \hyperref[dataitem:lmlssrsrfraw]{\PROD{LM\_LSS_RSRF\_RAW}} / \hyperref[dataitem:nlssrsrfraw]{\PROD{N\_LSS_RSRF\_RAW}}\\
Used in recipes: & \hyperref[rec:lsslmrsrf]{\REC{metis\_LM\_lss\_rsrf}} \\
& \hyperref[rec:lssnrsrf]{\REC{metis\_N\_lss\_rsrf}} \\
%Working remarks: & None \\
%Function Parameters: & None \\
Input: & \texttt{const hdrl\_image * input} \\
%Other inputs: & None\\
%QC outputs: & None \\
%Output FITS files: & None \\
Other outputs: & \texttt{cpl\_error\_code} \\
General description: & Creates normalised \ac{RSRF} \\
Mathematical description: &  see \cite{pis02} and \cite{pis21}\\
Quality assessment: & Through QC parameters \\
Error conditions: & See \cite{DRLVT}. \\
Unit tests: & See \cite{DRLVT}. \\
\end{recipedef}

%---------------------------------------------------------------------
\subsubsection{Detect order traces}\label{drl:tracedetect}
Used in recipes:\\ 
\hyperref[rec:lsslmtrace]{\REC{metis\_LM\_lss\_trace}} \newline
\hyperref[rec:lssntrace]{\REC{metis\_N\_lss\_trace}} \newline
TBD



%---------------------------------------------------------------------
\subsubsection{Apply RSRF}\label{drl:applyrsrf}
\begin{recipedef}
Name: & \hyperref[drl:applyrsrf]{\DRL{apply\_rsrf}}\\
Purpose: & applies \hyperref[dataitem:lsslmrsrfmaster]{\PROD{MASTER\_LM\_LSS\_RSRF}} to LM-frames\newline
           applies \hyperref[dataitem:lssnrsrfmaster]{\PROD{MASTER\_N\_LSS\_RSRF}} to N-frames\\
Used in recipes: & \hyperref[rec:lsslmtrace]{\REC{metis\_LM\_lss\_trace}} \newline
                 \hyperref[rec:lssntrace]{\REC{metis\_N\_lss\_trace}} \newline
                 \hyperref[rec:lsslmwave]{\REC{metis\_LM\_lss\_wave}} \newline
                 \hyperref[rec:lsslmflux]{\REC{metis\_LM\_lss\_flux}} \newline
                 \hyperref[rec:lssnflux]{\REC{metis\_N\_lss\_flux}} \newline
                 \hyperref[rec:lsslmsci]{\REC{metis\_LM\_lss\_sci}} \newline
                 \hyperref[rec:lssnsci]{\REC{metis\_N\_lss\_sci}} \newline 
                 \hyperref[rec:metislmadcmslitloss]{\REC{metis_lm_adc_slitloss}} \newline
                \hyperref[rec:metisnadcmslitloss]{\REC{metis_n_adc_slitloss}}\\
%Working remarks: & None \\
%Function Parameters: & None \\
%Input FITS files: & None \\
Inputs: & \texttt{const hdrl\_image * input}\\
%QC outputs: & None \\
%Output FITS files: & None \\
Other outputs: & \texttt{cpl\_error\_code} \\
General description: & Normal flatfield process for correcting pixel-to-pixel sensitivities \\
Mathematical description: & Division by \hyperref[dataitem:lsslmrsrfmaster]{\PROD{MASTER\_LM\_LSS\_RSRF}} or  \hyperref[dataitem:lssnrsrfmaster]{\PROD{MASTER\_N\_LSS\_RSRF}}, respectively \\
Quality assessment: & Through QC parameters \\
Error conditions: & See \cite{DRLVT}. \\
Unit tests: & See \cite{DRLVT}. \\
\end{recipedef}

%%---------------------------------------------------------------------
%\subsubsection{Wavelength calibration}\label{drl:wavecal}
%\begin{recipedef}\label{drl:wavecal}
%Name: & \hyperref[drl:wavecal]{\DRL{wave\_cal}} \\
%Purpose: & Computes wavelength calibration \\
%Used in recipes: & \hyperref[rec:lsslmwave]{\REC{metis\_LM\_lss\_wave}} \\
%%Working remarks: & None \\
%%Function Parameters: & None \\
%Input: & \texttt{const hdrl\_image * input} \\https://www.overleaf.com/project/5f1abb4137d7690001f8aeb1
%%Other inputs: & None\\
%%QC outputs: & None \\
%%Output FITS files: & None \\
%Other outputs: & \texttt{cpl\_error\_code} \\
%General description: & Determines wavelength calibration by means of line lamp spectra \\
%Mathematical description: &  see \cite{pis02} and \cite{pis21}\\
%Quality assessment: & Through QC parameters \\
%Error conditions: & See \cite{DRLVT}. \\
%Unit tests: & See \cite{DRLVT}. \\
%\end{recipedef}

%---------------------------------------------------------------------
\subsubsection{Detect line peak}\label{drl:linedetect}
Used in recipes:\\ 
\hyperref[rec:lsslmwave]{\REC{metis\_LM\_lss\_wave}} \newline
\hyperref[rec:lsslmflux]{\REC{metis\_LM\_lss\_flux}} \newline
\hyperref[rec:lssnflux]{\REC{metis\_N\_lss\_flux}} \newline
\hyperref[rec:lsslmsci]{\REC{metis\_LM\_lss\_sci}} \newline
\hyperref[rec:lssnsci]{\REC{metis\_N\_lss\_sci}} \newline
TBD



%---------------------------------------------------------------------
\subsubsection{Determine response}\label{drl:determineresponse}
\begin{recipedef}
Name: & \hyperref[drl:determineresponse]{\DRL{determine\_response}}\\
Purpose: & Determination of the spectral response function\\
Used in recipes: &  \hyperref[rec:lsslmflux]{\REC{metis\_LM\_lss\_flux}} \newline
                 \hyperref[rec:lssnflux]{\REC{metis\_N\_lss\_flux}} \\
%Working remarks: & None \\
%Function Parameters: & None \\
Input: & \texttt{const hdrl\_LM\_lsstrum1d * input} \\
Other inputs: & \texttt{const hdrl\_LM\_lsstrum1d * input}\\
%QC outputs: & None \\
%Output FITS files: & None \\
Other outputs: & \texttt{cpl\_error\_code} \\
General description: & Determination of the instrumental response function to be used for the absolute flux calibration \\
Mathematical description: & see HDRL manual \\
Quality assessment: & Through QC parameters \\
Error conditions: & See \cite{DRLVT}. \\
Unit tests: & See \cite{DRLVT}. \\
\end{recipedef}

%---------------------------------------------------------------------
\subsubsection{Object extraction}\label{drl:extractobject}
\begin{recipedef}
Name: & \hyperref[drl:extractobject]{\DRL{extract\_object}}\\
Purpose: & Extract object spectrum from 2D-spectrum\\
Used in recipes: & \hyperref[rec:lsslmflux]{\REC{metis\_LM\_lss\_flux}} \newline
                 \hyperref[rec:lssnflux]{\REC{metis\_N\_lss\_flux}} \newline
                 \hyperref[rec:lsslmsci]{\REC{metis\_LM\_lss\_sci}} \newline
                 \hyperref[rec:lssnsci]{\REC{metis\_N\_lss\_sci}} \\
%Working remarks: & None \\
%Function Parameters: & None \\
Input: & \texttt{const hdrl\_image * input\newline const hdrl\_image * background\newline hdrl\_image * output}  \\
%Other inputs: & None\\
%QC outputs: & None \\
%Output FITS files: & None \\
Other outputs: & \texttt{cpl\_error\_code} \\
General description: & Routine to extract a 1D spectrum of the target from the 2D spectrum\\
Mathematical description: & See optimal extraction algorithm \cite{pis02} and \cite{pis21} \\
Quality assessment: & Through QC parameters \\
Error conditions: & See \cite{DRLVT}. \\
Unit tests: & See \cite{DRLVT}. \\
\end{recipedef}


%---------------------------------------------------------------------
\subsubsection{Apply flux calibration}\label{drl:applyfluxcal}
\begin{recipedef}
Name: & \hyperref[drl:applyfluxcal]{\DRL{apply\_fluxcal}}\\
Purpose: & applies \hyperref[dataitem:lsslmresp]{\PROD{MASTER\_LM\_RESPONSE}} or \hyperref[dataitem:lssnresp]{\PROD{MASTER\_N\_RESPONSE}} to science frames\\
Used in recipes: & \hyperref[rec:lsslmsci]{\REC{metis\_LM\_lss\_sci}} \newline
                 \hyperref[rec:lssnsci]{\REC{metis\_N\_lss\_sci}} \\
%Working remarks: & None \\
%Function Parameters: & None \\
%Input FITS files: & None \\
Inputs: & \texttt{const hdrl\_image * input}\\
%QC outputs: & None \\
%Output FITS files: & None \\
Other outputs: & \texttt{cpl\_error\_code} \\
General description: & Applies the absolute flux calibration to science frames \\
%Mathematical description: & Applies projection from detector space to $(y, \lambda)$-space\\
Quality assessment: & Through QC parameters \\
Error conditions: & See \cite{DRLVT}. \\
Unit tests: & See \cite{DRLVT}. \\
\end{recipedef}




%Include ADI mode

\subsection{ADI observing mode}\label{sec:drl_functions_adi}

%---------------------------------------------------------------------

%lm_adi_cgrph_centroid

\subsubsection{\DRL{lm\_adi\_cgrph\_centroid}}\label{drl:lm_adi_cgrph_centroid}
\begin{recipedef}
Name: & \hyperref[drl:lm_adi_cgrph_centroid]{\DRL{lm\_adi\_cgrph\_centroid}} \\
Purpose: & Detect the centroid in a sequence of LM band CVC/RAVC/CLC coronographic images\\
Used in recipes: & \hyperref[rec:metis_det_adi_cgrph]{\REC{metis\_det\_adi\_cgrph}}\\
%Working remarks: & None \\
%Function Parameters: & None \\
Input: & $n\times$ \texttt{const hdrl\_image * input} \\
%Other inputs: & None \\
QC outputs: & QC LM CGRPH FWHM NN TBD\\
%Output FITS files: & None \\
Outputs: & \PROD{lm\_cgrph\_CENTROID\_TAB}\\
                & \texttt{cpl\_error\_code} \\
General description: & Centroiding of source in LM band coronagraphic images \\
Mathematical description: & see Section~\ref{ssec:algo_app_imaging}  \\
Quality assessment: & Through QC parameters \\
Error conditions: & See \cite{DRLVT} (TBD). \\
Unit tests: & See \cite{DRLVT} (TBD). \\
\end{recipedef}

%n_adi_cgrph_centroid

\subsubsection{\DRL{n\_adi\_cgrph\_centroid}}\label{drl:n_adi_cgrph_centroid}
\begin{recipedef}
Name: & \hyperref[drl:n_adi_cgrph_centroid]{\DRL{n\_adi\_cgrph\_centroid}} \\
Purpose: & Detect the centroids in a sequence of N band CVC/RAVC/CLC coronographic images\\
Used in recipes: & \hyperref[rec:metis_det_adi_cgrph]{\REC{metis\_det\_adi\_cgrph}}\\
%Working remarks: & None \\
%Function Parameters: & None \\
Input: & $n\times$ \texttt{const hdrl\_image * input} \\
%Other inputs: & None \\
QC outputs: & QC N CGRPH FWHM NN TBD\\
%Output FITS files: & None \\
Outputs: & \PROD{n\_cgrph\_CENTROID\_TAB}\\
                & \texttt{cpl\_error\_code} \\
General description: & Centroiding of source in N band coronagraphic images \\
Mathematical description: & see Section~\ref{ssec:algo_app_imaging}  \\
Quality assessment: & Through QC parameters \\
Error conditions: & See \cite{DRLVT} (TBD). \\
Unit tests: & See \cite{DRLVT} (TBD). \\
\end{recipedef}


%lm_adi_app_centroid

\subsubsection{\DRL{lm\_adi\_app\_centroid}}\label{drl:lm_adi_app_centroid}
\begin{recipedef}
Name: & \hyperref[drl:lm_adi_app_centroid]{\DRL{lm\_adi\_app\_centroid}} \\
Purpose: & Detect the centroid in a sequence of LM band APP coronographic images\\
Used in recipes: & \hyperref[rec:metis_det_adi_app]{\REC{metis\_det\_adi\_app}}\\
%Working remarks: & None \\
%Function Parameters: & None \\
Input: & $n\times$ \texttt{const hdrl\_image * input} \\
%Other inputs: & None \\
QC outputs: & QC LM APP FWHM NN TBD\\
%Output FITS files: & None \\
Outputs: & \PROD{lm\_app\_CENTROID\_TAB}\\
                & \texttt{cpl\_error\_code} \\
General description: & Centroiding of source in LM band APP coronagraphic images \\
Mathematical description: & see Section~\ref{ssec:algo_app_imaging}  \\
Quality assessment: & Through QC parameters \\
Error conditions: & See \cite{DRLVT} (TBD). \\
Unit tests: & See \cite{DRLVT} (TBD). \\
\end{recipedef}

%n_adi_app_centroid

\subsubsection{\DRL{n\_adi\_app\_centroid}}\label{drl:n_adi_app_centroid}
\begin{recipedef}
Name: & \hyperref[drl:n_adi_app_centroid]{\DRL{n\_adi\_app\_centroid}} \\
Purpose: & Detect the centroids in a sequence of N band APP coronographic images\\
Used in recipes: & \hyperref[rec:metis_det_adi_app]{\REC{metis\_det\_adi\_app}}\\
%Working remarks: & None \\
%Function Parameters: & None \\
Input: & $n\times$ \texttt{const hdrl\_image * input} \\
%Other inputs: & None \\
QC outputs: & QC N APP FWHM NN TBD\\
%Output FITS files: & None \\
Outputs: & \PROD{n\_app\_CENTROID\_TAB}\\
                & \texttt{cpl\_error\_code} \\
General description: & Centroiding of source in N band APP coronagraphic images \\
Mathematical description: & see Section~\ref{ssec:algo_app_imaging}  \\
Quality assessment: & Through QC parameters \\
Error conditions: & See \cite{DRLVT} (TBD). \\
Unit tests: & See \cite{DRLVT} (TBD). \\
\end{recipedef}


%lm_adi_cgrph_psf

\subsubsection{\DRL{lm\_adi\_cgrph\_psf}}\label{drl:lm_adi_cgrph_psf}
\begin{recipedef}
Name: & \hyperref[drl:lm_adi_cgrph_psf]{\DRL{lm_adi\_cgrph\_psf}} \\
Purpose: & Calculate median PSF for sequence of ADI images\\
Used in recipes: & \hyperref[rec:metis_det_adi_cgrph]{\REC{metis\_det\_adi\_cgrph}}\\
%Working remarks: & None \\
Function Parameters: & combination method (median, mean, sigma clipped...)\\
                     & parameters for combination method\\
Input: & $n\times$ \texttt{const hdrl\_image * input} \\
       &  \PROD{lm\_cgrph\_CENTROID\_TAB}\\
%Other inputs: & None \\
QC outputs: & QC DET CGRPH SCI FWHM ??\\
%Output FITS files: & None \\
Outputs: & \PROD{lm\_cgrph\_PSF\_MEDIAN}\\
                & \texttt{cpl\_error\_code} \\
General description: & Calculation of median PSF for sequence of ADI images\ \\
Mathematical description: & see Section~\ref{ssec:algo_app_imaging} \TBD \\
Quality assessment: & Through QC parameters \\
Error conditions: & See \cite{DRLVT} (TBD). \\
Unit tests: & See \cite{DRLVT} (TBD). \\
\end{recipedef}


%n_adi_cgrph_psf

\subsubsection{\DRL{n\_adi\_cgrph\_psf}}\label{drl:n_adi_cgrph_psf}
\begin{recipedef}
Name: & \hyperref[drl:n_adi_cgrph_psf]{\DRL{n_adi\_cgrph\_psf}} \\
Purpose: & Calculate median PSF for sequence of ADI images\\
Used in recipes: & \hyperref[rec:metis_det_adi_cgrph]{\REC{metis\_det\_adi\_cgrph}}\\
%Working remarks: & None \\
Function Parameters: & combination method (median, mean, sigma clipped...)\\
                     & parameters for combination method\\
Input: & $n\times$ \texttt{const hdrl\_image * input} \\
       &  \PROD{n\_cgrph\_CENTROID\_TAB}\\
%Other inputs: & None \\
QC outputs: & QC DET CGRPH SCI FWHM ??\\
%Output FITS files: & None \\
Outputs: & \PROD{n\_cgrph\_PSF\_MEDIAN}\\
                & \texttt{cpl\_error\_code} \\
General description: & Calculation of median PSF for sequence of ADI images\ \\
Mathematical description: & see Section~\ref{ssec:algo_app_imaging} \TBD \\
Quality assessment: & Through QC parameters \\
Error conditions: & See \cite{DRLVT} (TBD). \\
Unit tests: & See \cite{DRLVT} (TBD). \\
\end{recipedef}


%lm_adi_cgrph_psf

\subsubsection{\DRL{lm\_adi\_app\_psf}}\label{drl:lm_adi_app_psf}
\begin{recipedef}
Name: & \hyperref[drl:lm_adi_app_psf]{\DRL{lm_adi\_app\_psf}} \\
Purpose: & Calculate median PSF for sequence of ADI images\\
Used in recipes: & \hyperref[rec:metis_det_adi_app]{\REC{metis\_det\_adi\_app}}\\
%Working remarks: & None \\
Function Parameters: & combination method (median, mean, sigma clipped...)\\
                     & parameters for combination method\\
Input: & $n\times$ \texttt{const hdrl\_image * input} \\
       &  \PROD{lm\_app\_CENTROID\_TAB}\\
%Other inputs: & None \\
QC outputs: & QC DET APP SCI FWHM ??\\
%Output FITS files: & None \\
Outputs: & \PROD{lm\_app\_PSF\_MEDIAN}\\
                & \texttt{cpl\_error\_code} \\
General description: & Calculation of median PSF for sequence of ADI images\ \\
Mathematical description: & see Section~\ref{ssec:algo_app_imaging} \TBD \\
Quality assessment: & Through QC parameters \\
Error conditions: & See \cite{DRLVT} (TBD). \\
Unit tests: & See \cite{DRLVT} (TBD). \\
\end{recipedef}


%n_adi_app_psf

\subsubsection{\DRL{n\_adi\_app\_psf}}\label{drl:n_adi_app_psf}
\begin{recipedef}
Name: & \hyperref[drl:n_adi_app_psf]{\DRL{n_adi\_app\_psf}} \\
Purpose: & Calculate median PSF for sequence of ADI images for the APP coronograph\\
Used in recipes: & \hyperref[rec:metis_det_adi_app]{\REC{metis\_det\_adi\_app}}\\
%Working remarks: & None \\
Function Parameters: & combination method (median, mean, sigma clipped...)\\
                     & parameters for combination method\\
Input: & $n\times$ \texttt{const hdrl\_image * input} \\
       &  \PROD{n\_app\_CENTROID\_TAB}\\
%Other inputs: & None \\
QC outputs: & QC DET APP SCI FWHM ??\\
%Output FITS files: & None \\
Outputs: & \PROD{n\_app\_PSF\_MEDIAN}\\
                & \texttt{cpl\_error\_code} \\
General description: & Calculation of median PSF for sequence of ADI images  for the APP coronograph \\
Mathematical description: & see Section~\ref{ssec:algo_app_imaging} \TBD \\
Quality assessment: & Through QC parameters \\
Error conditions: & See \cite{DRLVT} (TBD). \\
Unit tests: & See \cite{DRLVT} (TBD). \\
\end{recipedef}

%n_adi_app_psf

\subsubsection{\DRL{lm\_merge\_app\_psf}}\label{drl:lm_merge_app_adi_psf}
\begin{recipedef}
Name: & \hyperref[drl:lm_merge_app_adi_psf]{\DRL{lm\_merge\_app\_psf}} \\
Purpose: & Merge the components of the PSF for an LM band APP coronagraph image\\
Used in recipes: & \hyperref[rec:metis_det_adi_app]{\REC{metis\_det\_adi\_app}}\\
%Working remarks: & None \\
Function Parameters: & combination method (median, mean, sigma clipped...)\\
                     & parameters for combination method\\
Input: & $n\times$ \texttt{const hdrl\_image * input} \\
%Other inputs: & None \\
QC outputs: & \TBD\\
%Output FITS files: & None \\
Outputs: & \texttt{cpl\_error\_code} \\
General description: & Merge the components of the PSF for an LM band APP coronagraph image \\
Mathematical description: & see Section~\ref{ssec:algo_app_imaging} \TBD \\
Quality assessment: & Through QC parameters \\
Error conditions: & See \cite{DRLVT} (TBD). \\
Unit tests: & See \cite{DRLVT} (TBD). \\
\end{recipedef}


\subsubsection{\DRL{n\_merge\_app\_psf}}\label{drl:n_merge_app_adi_psf}
\begin{recipedef}
Name: & \hyperref[drl:n_merge_app_adi_psf]{\DRL{n\_merge\_app\_psf}} \\
Purpose: & Merge the components of the PSF for an N band APP coronagraph image\\
Used in recipes: & \hyperref[rec:metis_det_adi_app]{\REC{metis\_det\_adi\_app}}\\
%Working remarks: & None \\
Function Parameters: & combination method (median, mean, sigma clipped...)\\
                     & parameters for combination method\\
Input: & $n\times$ \texttt{const hdrl\_image * input} \\
%Other inputs: & None \\
QC outputs: & \TBD\\
%Output FITS files: & None \\
Outputs: & \texttt{cpl\_error\_code} \\
General description: & Merge the components of the PSF for an N band APP coronagraph image \\
Mathematical description: & see Section~\ref{ssec:algo_app_imaging} \TBD \\
Quality assessment: & Through QC parameters \\
Error conditions: & See \cite{DRLVT} (TBD). \\
Unit tests: & See \cite{DRLVT} (TBD). \\
\end{recipedef}

%apply distortion correcton and regrid

\subsubsection{\DRL{adi\_regrid}}\label{drl:adi_regrid}
\begin{recipedef}
Name: & \hyperref[drl:adi_regrid]{\DRL{adi\_regrid}} \\
Purpose: & Apply distortion correction and regrid to align images\\
Used in recipes: & \hyperref[rec:metis_det_adi_cgrph]{\REC{metis\_det\_adi\_cgrph}}\\
%Working remarks: & None \\
Function Parameters: & Resampling method\\
                     & parameters for resampling method\\
Input: & $n\times$ \texttt{const hdrl\_image * input} \\
       & \PROD{det\_cgrph\_CENTROID\_TAB}\\
       & \PROD{det\_DISTORTION\_TABLE}\\
%Other inputs: & Parameters for regridding \\
QC outputs: & None\\
Output FITS files: & \PROD{det\_cgrph\_SCI\_CENTRED} \\
Outputs: &   \texttt{cpl\_error\_code} \\
General description: & Apply the distortion correction and align images on a subpixel scale \\
Mathematical description: & see Section~\ref{ssec:algo_app_imaging} \TBD \\
Quality assessment: & Through QC parameters \\
Error conditions: & See \cite{DRLVT} (TBD). \\
Unit tests: & See \cite{DRLVT} (TBD). \\
\end{recipedef}


%apply distortion correcton and regrid

\subsubsection{\DRL{ifu_adi\_regrid}}\label{drl:ifu_adi_regrid}
\begin{recipedef}
Name: & \hyperref[drl:ifu_adi_regrid]{\DRL{ifu_adi\_regrid}} \\
Purpose: & Apply distortion correction, perform square pixel reconstruction, and regrid to align images\\
Used in recipes: & \hyperref[rec:metis_det_adi_cgrph]{\REC{metis\_det\_adi\_cgrph}}\\
%Working remarks: & None \\
Function Parameters: & Resampling method\\
                     & parameters for resampling method\\
Input: & $n\times$ \texttt{const hdrl\_image * input} \\
       & \PROD{det\_cgrph\_CENTROID\_TAB}\\
       & \PROD{det\_DISTORTION\_TABLE}]]\\
%Other inputs: & Parameters for regridding \\
QC outputs: & None\\
Output FITS files: & \PROD{det\_cgrph\_SCI\_CENTRED} \\
Outputs: &   \texttt{cpl\_error\_code} \\
General description: & Apply the distortion correction, perform square pixel reconstruction, and align images on a subpixel scale \\
Mathematical description: & see Section~\ref{ssec:algo_app_imaging} \TBD \\
Quality assessment: & Through QC parameters \\
Error conditions: & See \cite{DRLVT} (TBD). \\
Unit tests: & See \cite{DRLVT} (TBD). \\
\end{recipedef}


%derotate images

\subsubsection{\DRL{adi\_derotate}}\label{drl:adi_derotate}
\begin{recipedef}
Name: & \hyperref[drl:adi_derotate]{\DRL{adi\_derotate}} \\
Purpose: & Apply disotrtion correction and Derotate a squence of coronagraphic images\\
Used in recipes: & \hyperref[rec:metis_det_adi_cgrph]{\REC{metis\_det\_adi\_cgrph}}\\
%Working remarks: & None \\
Function Parameters: method for resampling\\
Input: & $n\times$ \texttt{const hdrl\_image * input} \\
%Other inputs: & None \\
QC outputs: &  QC DET CGRPH SCI SNR MEAN \\
            &  QC DET CGRPH SCI SNR PEAK \\
Output FITS files: & \PROD{det\_cgrph\_SCI\_DEROTATED\_PFSUB} \\
                   & \PROD{det\_cgrph\_SCI\_DEROTATED} \\
%Outputs: & \PROD{det\_cgrph\_PSF\_MEDIAN}\\
%                & \texttt{cpl\_error\_code} \\
General description: & derotate a sequence of ADI images \\
Mathematical description: & see Section~\ref{ssec:algo_app_imaging} \TBD \\
Quality assessment: & Through QC parameters \\
Error conditions: & See \cite{DRLVT} (TBD). \\
Unit tests: & See \cite{DRLVT} (TBD). \\
\end{recipedef}

%calculate lm contrast curve 

\subsubsection{\DRL{lm\_adi\_cgrph\_contrast}}\label{drl:lm_adi_contrast}
\begin{recipedef}
Name: & \hyperref[drl:lm_adi_contrast]{\DRL{lm\_adi\_contrast}} \\
Purpose: & Calculate contrast curves for a sequence of LM band coronagraphic images, for the RAVC/CDC/CLC coronagraph\\
Used in recipes: & \hyperref[rec:metis_det_adi_cgrph]{\REC{metis\_det\_adi\_cgrph}}\\
%Working remarks: & None \\
Function Parameters: \TBD \\
Input: & $n\times$ \texttt{const hdrl\_image * input} \\
Other inputs: & Off-axis PSF reference \\
QC outputs: & QC DET CGRPH SCI CONTRAST RAW LAMD\\
            & QC DET CGRPH SCI CONTRAST ADI LAMD\\
  Output FITS files: & \PROD{det\_cgrph\_SCI\_CONTRAST\_RAW} \\
                     & \PROD{det\_cgrph\_SCI\_CONTRAST\_ADI} \\
                     & \PROD{det\_cgrph\_SCI\_THROUGHPUT} \\
Outputs: & \texttt{cpl\_error\_code} \\
General description: &  Calculate contrast curves for a sequence of LM band coronagraphic images, for the  RAVC/CDC/CLC coronagraph\\
Mathematical description: & see Section~\ref{ssec:algo_app_imaging} \TBD \\
Quality assessment: & Through QC parameters \\
Error conditions: & See \cite{DRLVT} (TBD). \\
Unit tests: & See \cite{DRLVT} (TBD). \\
\end{recipedef}


%calculate n contrast curve 

\subsubsection{\DRL{n\_adi\_cgrph\_contrast}}\label{drl:n_adi_contrast}
\begin{recipedef}
Name: & \hyperref[drl:n_adi_contrast]{\DRL{n\_adi\_contrast}} \\
Purpose: & Calculate contrast curves for a sequence of N band coronagraphic images, for the RAVC/CDC/CLC coronagraph\\
Used in recipes: & \hyperref[rec:metis_det_adi_cgrph]{\REC{metis\_det\_adi\_cgrph}}\\
%Working remarks: & None \\
Function Parameters: \TBD \\
Input: & $n\times$ \texttt{const hdrl\_image * input} \\
Other inputs: & Off-axis PSF reference \\
QC outputs: & QC DET CGRPH SCI CONTRAST RAW LAMD\\
            & QC DET CGRPH SCI CONTRAST ADI LAMD\\
  Output FITS files: & \PROD{det\_cgrph\_SCI\_CONTRAST\_RAW} \\
                     & \PROD{det\_cgrph\_SCI\_CONTRAST\_ADI} \\
                     & \PROD{det\_cgrph\_SCI\_THROUGHPUT} \\
Outputs: & \texttt{cpl\_error\_code} \\
General description: &  Calculate contrast curves for a sequence of N band coronagraphic images, for the  RAVC/CDC/CLC coronagraph\\
Mathematical description: & see Section~\ref{ssec:algo_app_imaging} \TBD \\
Quality assessment: & Through QC parameters \\
Error conditions: & See \cite{DRLVT} (TBD). \\
Unit tests: & See \cite{DRLVT} (TBD). \\
\end{recipedef}


%calculate lm contrast curve 

\subsubsection{\DRL{lm\_adi\_app\_contrast}}\label{drl:lm_adi_contrast}
\begin{recipedef}
Name: & \hyperref[drl:lm_adi_contrast]{\DRL{lm\_adi\_contrast}} \\
Purpose: & Calculate contrast curves for a sequence of LM band coronagraphic images, for the APP coronagraph\\
Used in recipes: & \hyperref[rec:metis_det_adi_app]{\REC{metis\_det\_adi\_app}}\\
%Working remarks: & None \\
Function Parameters: \TBD \\
Input: & $n\times$ \texttt{const hdrl\_image * input} \\
Other inputs: & Off-axis PSF reference \\
QC outputs: & QC DET APP SCI CONTRAST RAW LAMD\\
            & QC DET APP SCI CONTRAST ADI LAMD\\
  Output FITS files: & \PROD{det\_app\_SCI\_CONTRAST\_RAW} \\
                     & \PROD{det\_app\_SCI\_CONTRAST\_ADI} \\
                     & \PROD{det\_app\_SCI\_THROUGHPUT} \\
Outputs: & \texttt{cpl\_error\_code} \\
General description: &  Calculate contrast curves for a sequence of LM band coronagraphic images, for the  APP coronagraph\\
Mathematical description: & see Section~\ref{ssec:algo_app_imaging} \TBD \\
Quality assessment: & Through QC parameters \\
Error conditions: & See \cite{DRLVT} (TBD). \\
Unit tests: & See \cite{DRLVT} (TBD). \\
\end{recipedef}


%calculate n contrast curve 

\subsubsection{\DRL{n\_adi\_app\_contrast}}\label{drl:n_adi_contrast}
\begin{recipedef}
Name: & \hyperref[drl:n_adi_contrast]{\DRL{n\_adi\_contrast}} \\
Purpose: & Calculate contrast curves for a sequence of N band coronagraphic images, for the APP coronagraph\\
Used in recipes: & \hyperref[rec:metis_det_adi_app]{\REC{metis\_det\_adi\_app}}\\
%Working remarks: & None \\
Function Parameters: \TBD \\
Input: & $n\times$ \texttt{const hdrl\_image * input} \\
Other inputs: & Off-axis PSF reference \\
QC outputs: & QC DET APP SCI CONTRAST RAW LAMD\\
            & QC DET APP SCI CONTRAST ADI LAMD\\
  Output FITS files: & \PROD{det\_APP\_SCI\_CONTRAST\_RAW} \\
                     & \PROD{det\_APP\_SCI\_CONTRAST\_ADI} \\
                     & \PROD{det\_APP\_SCI\_THROUGHPUT} \\
Outputs: & \texttt{cpl\_error\_code} \\
General description: &  Calculate contrast curves for a sequence of N band coronagraphic images, for the  APP coronagraph\\
Mathematical description: & see Section~\ref{ssec:algo_app_imaging} \TBD \\
Quality assessment: & Through QC parameters \\
Error conditions: & See \cite{DRLVT} (TBD). \\
Unit tests: & See \cite{DRLVT} (TBD). \\
\end{recipedef}




\clearpage
\section{DRL Data Structures}\label{sec:drl_data_structures}

\subsection{DRL Data Structures of the LMS mode}\label{ssec:lms_drl_data_structures}

\subsection{DRL Data Structures of the IMG mode}\label{ssec:img_drl_data_structures}

\subsection{DRL Data Structures of the LSS mode}\label{ssec:lss_drl_data_structures}
\clearpage
\section{HDRL algorithms}\label{sec:hdrl_algorithms}

The METIS \ac{DRS} will make good use of \ac{HDRL} functions and data structures, not the least by relying on its built-in error-propagation.
The recipe descriptions (Ch.~\ref{sec:pipeline_recipes}) also list the functions that will be used.

Please note: It is expected that the following lists of HDRL functions will be expanded during the implementation phase to reflect the list of functions actually used by the pipeline.
%-----------------------------------------------------------------------------------------------------------
\subsection{HDRL functions common to all modes}\label{ssec:hdrlcommon}
The use of following HDRL functions is common to all the reduction modes of METIS:

\begin{itemize}
    \item \CODE{hdrl_bpm_3d_compute}
    \item \CODE{hdrl_bpm_fit_compute}
    \item \CODE{hdrl_imagelist_collapse}
    \item \CODE{hdrl_imagelist_mult_scalar}
    \item \CODE{hdrl_imagelist_sub_image}
    \item \CODE{hdrl_persistence_compute}
\end{itemize}


%-----------------------------------------------------------------------------------------------------------
\subsection{HDRL for IMG}\label{ssec:hdrlimg}
We will use the following \ac{HDRL} functions for the imaging pipeline:

\begin{itemize}
    \item \CODE{hdrl_catalogue_create}
    \item \CODE{hdrl_strehl_compute}
\end{itemize}

We do not foresee a need for new \ac{HDRL} algorithms for imaging parts of the \ac{DRS}.

% Moved to https://github.com/AstarVienna/METIS_DRLD/issues/102 for discussion
% \TODO{Is this true? Should we argue for ADI or anything else?}


%-----------------------------------------------------------------------------------------------------------
\subsection{HDRL for LSS}\label{ssec:hdrllss}
We are currently intending the usage of the following \ac{HDRL} algorithms for spectroscopy whenever applicable:

\begin{itemize}
    \item \texttt{hdrl\_spectrum1D\_resample}       # : Resampling of 1D spectra
    \item \texttt{hdrl\_efficiency\_compute}        # : For monitoring the system's health and performance we intend to use this built-in function to compute the efficiency of the \ac{LSS} spectroscopic mode.
    \item \texttt{hdrl\_response\_compute()}        # : This function provides a 1D response as function of the wavelength.
    \item \texttt{hdrl\_utils\_airmass()}
\end{itemize}

Nowadays there are well developed algorithms available for spectroscopy-specific tasks, which are already in use in ESO pipelines (e.g. CRIRES+) and which can be generalized to be used in the \ac{HDRL}.
This especially applies to algorithms developed by Piskunov and collaborators (\cite{pis21}, \cite{pis02}).
We therefore recommend including their following algorithms into the \ac{HDRL}:

\begin{itemize}
    \item flatfield normalisation
    \item curvature determination
    \item slit decomposition
    \item wavelength calibration
    \item continuum normalisation
\end{itemize}

As some of these algorithms are already implemented in the \ac{CRIRES}+ pipeline
and are therefore already based on \ac{CPL} an inclusion into \ac{HDRL} should
be possible.

%-----------------------------------------------------------------------------------------------------------
\subsection{HDRL for IFU}
\label{ssec:hdrllms}

In addition to the HDRL functions listed in section \ref{\ssec:hdrlcommon}, we will also investigate how applicable the following functions are for the METIS IFU reduction algorithms:
\begin{itemize}
    \item \CODE{hdrl_resample_imagelist_to_table}
    \item \CODE{hdrl_resample_compute}
\end{itemize}

Similar to the LSS above, the reduction of the IFU will rely on the algorithms
from the \ac{CRIRES} pipeline, as shown for the distortion correction algorithm
in Ch.~\ref{ssec:criticalwavelengthanddistortionifu} where this pipeline
directly gets applied to METIS simulations.

The METIS \ac{DRS} will benefit from the availability of such routines in the ESO-HDRL.


\clearpage
\section{Critical algorithms}\label{sec:critical_algorithms}

This section discusses the progress made on the critical algorithms identified in the Data Reduction Library Specifications document [AD1].

\begin{enumerate}
    \item Persistence correction
    \item Detector Masks
    \item Bad pixel determination
    \item Background subtraction
    \item Wavelength calibration and distortion correction
    \item Telluric correction
    \item Error propagation
    \item N-band image restoration
    \item LMS image and cube reconstruction
    \item LMS data rate
\end{enumerate}

The following additional critical algorithms were identified in the period between PDR and FDR:

\begin{enumerate}
    \item ADI algorithm
\end{enumerate}

% Export these to individual files if they become too wordy
\subsection{Persistence correction}

Will have access to a HDRL function from ESO. 
The interface is TBD.
Subtraction of persistence image as determined by ESO.
Need to reference ESO docs for this.
How does the scaling happen?
Is there an API endpoint for this?

\subsection{Detector Masks}

Look into the MATISSE docs as to how this was done
Delivers a bias/dark/ron value?

\subsection{Bad pixel determination}

Surely every other instrument on planet earth has some sort of algo for this.
Need to determine the format of the bad pixel mask (bool?)
Take XSHOOTER as example

\subsection{Background subtraction}

N-band: Standard chop and nod using telescope chopper and internal nodder
See VISIR pipeline

\subsection{Wavelength calibration and distortion correction}

Pyreduce, will be described by Nadeen
IMG modes are not critical, see recipes

\subsection{Telluric correction}
Telluric correction, i.e. the removal of absorption features arising in the Earth's atmosphere, is a critical issue as the imprint of molecular species present in our air may vary on different timescales down to minutes due to changes in their composition and their amount. In particular the \ac{MIR} range is affected (see App~\ref{app:atmo_trans}).\\
There are two well established ways to correct for these absorptions:
\begin{itemize}
    \item \textit{Classical approach}: A telluric standard star (\ac{TSS}) spectrum is taken ideally directly before/after the science observations at the same airmass. This \ac{TSS}-spectrum is processed in the same way as the science spectrum (except the absolute flux calibration) and finally its continuum is normalised to unity. In addition, a model of this specific \ac{TSS} spectrum is used to remove intrinsic spectral features. The remaining normalised spectrum (ideally) only contains the fingerprint of the Earth's atmospheric absorptions and can be used for the telluric correction. In case the model spectrum also contains absolute flux values, this star could also be used for the absolute flux calibration.
    \item \textit{Modelling approach}: In the last years a new method has evolved which is based on radiative transfer modelling of the Earth's atmosphere (\cite{mf1, mf2, molecfit}\footnote{\url{https://www.eso.org/sci/software/pipelines/skytools/molecfit}}). A model of the Earth's atmosphere in combination with a radiative transfer model and a molecular line list containing lines of various species is used to determine the transmission of the Earth's atmosphere at the time of observations by fitting specific molecular absorption features in the science spectra. The best-fit transmission function is finally used for the telluric correction.
\end{itemize}
These two methods can also be combined in the way that the modelling approach is applied to a \ac{TSS} spectrum, and the resulting transmission function is then applied to the science spectrum. This combination is specifically useful if the science target continuum is too weak to use the absorptions for a fit.\\
For the \ac{METIS} pipelines we intend to incorporate both ways (including their combination). A detailed description is given in Section~\ref{ssec:tellcorr}). As both methods are well established, we deem a dedicated prototyping not necessary.

\subsection{Error propagation}

HDRL functions do error propogation. Look into docs for these. 
Understand how HDRL does this and then copy it.

\subsection{N-band image restoration}

Chop-nod inversion and stacking
Look at how VISIR does this
Need to look at sub-pixel chop shifts, and whether the images need to be resampled
Plate scale distortions are not really a issue, because we only care about compact objects

\subsection{LMS image and cube reconstruction}

Completely open and a bit of a beast. 
Help from Linz?
Drizzling from HST?

\subsection{LMS data rate}

See data rate document. 
Hand-wave-y arguments to make this go away.

% Additional critical algos
\subsection{ADI algorithm}
\subsubsection{Generation of representative set of IMG\_LM data: planetary system}
Using a combination of the HEEPS code and the ScopeSim code with the METIS package we have performed simulations of a star and a point source companion in order to test critical algorithms for high contrast imaging on simulated data of METIS.
Using the HEEPS code we simulate a sequence of up to 12000 PSFs of an unresolved star behind the RAVC coronagraph and of an unresolved companion sufficiently far away from the coronagraph to be considered off-axis, with each PSF corresponding with an integration time of 300 milliseconds for a total of 3600 seconds of observing time. Each PSF is generated with a (SCAO-only) wavefront sampled every 300 ms (comparable to the coherence time t0 at L-band), thereby effectively time sampling the atmospheric wavefront aberrations. Both PSFs are scaled to the magnitude of the parent star ($L=3.5$). For this demonstration we use a time sequence of 12000 PSFs but will apply some windowing and temporal binning to increase execution speed on a laptop. 
The two PSFs are combined together with an offset of 100 mas, an additional delta magnitude ($\Delta L=7.7$) for the companion and field rotation over 1 hour (corresponding with 30 degrees change in position angle). The system could be described at Beta Pic b orbiting at 100 mas around its host star. A second source is added at opposite position angle to see the symmetry of the reduction (of use for the APP coronagraph which only has one dark hole). This combined source is fed into ScopeSim as a fits file with the WCS information conveying the spatial extent. The coronagraphic transmissions are transferred to ScopeSim as well during this process.
In the ScopeSim step the instrumental noise and background expected from both METIS and ELT is injected and the source flux reduced according to system transmission. 
After these processing steps we are left with a large 3D cube of a star (x, y, time) with two planets undergoing field rotation which we use for the ADI processing step using the VIP\_HCI package. For this example, the output images have been truncated to 201x201 pixels and the images have been binned to 30 seconds exposure time ($N_\mathrm{bin}=100$) to further reduce execution time.
\subsubsection{Demonstration of ADI algorithm on simulated IMG\_LM data}
The most basic PSF subtraction of VIP\_HCI is the standard ADI algorithm from Marois et al 2006, where the median of the cube is first subtracted and in a second step optimally-scaled time-localized medians are subtracted in annular rings. Afterwards each image in the cube is derotated with the known position angle. Subsequently the derotated and PSF subtracted images are averaged to give a final stacked version. When the second annular optimized subtraction step is skipped this procedure is equal to the one described in the PIP Spec.
For the LMS IFU, this ADI procedure can similarly be followed for each wavelength in the reduced cubes. In VIP\_HCI it is possible to do a two-step (first ADI then SDI) or single-step reduction, in which rescaled images from other wavelength slices are used to provide additional PSF reference material.
Example output images for ADI reduction for LM RAVC data.


\begin{figure}[!ht]
  \centering
  \includegraphics[width=0.6\textwidth]{./figures/onaxis_psf.png}
  \caption{On-axis coronagraphic PSF with instrumental noise injected and with corrected atmospheric turbulence. The circular control region from the AO system is clearly visible.}
  %\label{fig:metis_l}
\end{figure}
\begin{figure}[!ht]
  \centering
  \includegraphics[width=0.6\textwidth]{./figures/offaxis_psf.png}
  \caption{Off-axis PSF without instrumental noise.}
  %\label{fig:metis_l}
\end{figure}

\begin{figure}[!ht]
  \centering
  \includegraphics[width=0.6\textwidth]{./figures/adi_meansub.png}
  \caption{Single frame of image stack with mean PSF subtracted.}
  %\label{fig:metis_l}
\end{figure}

\begin{figure}[!ht]
  \centering
  \includegraphics[width=0.6\textwidth]{./figures/adi_derotstack.png}
  \caption{Derotated stack of ADI reduced images.}
  %\label{fig:metis_l}
\end{figure}

Additionally, the contrast curves can be extracted (raw contrast over time-averaged cube, 5 sigma contrast after ADI processing). Please note that the plotted contrast is dependent on the assumptions made for this demonstration (SCAO-only effects, no additional jitter / pupil drifts / no residual atmospheric dispersion, 120 binned frames with each 100 frames of 0.3s integration time, 30 degrees image rotation). 
Injection and retrieval of point sources of known intensity is performed to calculate the efficiency or throughput of the ADI post-processing technique. Closer to the star the same amount of position angle movement will be seen as an increasingly smaller physical movement. Therefore, self-subtraction of a point source signal will be more pronounced and increased losses are seen close to the star. The following plot shows the post-processing only component of the ADI reduction losses. The radial coronagraphic throughput losses due to partial suppression of off-axis sources will be provided separately (and is not included in these sensitivity plots for this example).



Following the SNR definitions of Mawet et al 2014 the noise is calculated in annular rings and corrected for the relatively low number of effective samples. A 5-sigma sensitivity curve is produced by injection and retrieval of sources. In addition, a post-ADI SNR map is generated. Note that the two injection planets are so bright that an unmasked SNR map gives an underestimated SNR as their presence strongly alters the local noise estimate. Nonetheless they are both recognized as significant detections.

\begin{figure}[!ht]
  \centering
  \includegraphics[width=0.6\textwidth]{./figures/adi_throughput.png}
  \caption{Efficiency of ADI algorithm in retrieving inserted point sources as a function of angular separation.}
  %\label{fig:metis_l}
\end{figure}

\begin{figure}[!ht]
  \centering
  \includegraphics[width=0.6\textwidth]{./figures/adi_contrast.png}
  \caption{Achievable contrast for the $L=3.5$ target in 1 hour integration time as a function of angular separation.}
  %\label{fig:metis_l}
\end{figure}

\begin{figure}[!ht]
  \centering
  \includegraphics[width=0.6\textwidth]{./figures/adi_snrmap.png}
  \caption{Two dimensional ADI map of post ADI signal to noise of point sources. Two point sources that were inserted at 100 mas are seen.}
  %\label{fig:metis_l}
\end{figure}

\subsubsection{Generation of representative set of IFU data}

To demonstrate the ADI/SDI reduction of an IFU dataset which contains spatial, temporal and spectral information we first generate a METIS IFU HCI dataset of 12000 frames by X by Y pixels and Z spectral bins. Each frame corresponds with an exposure time of 0.3 seconds, with a total integration time of 1 hour. At a central wavelength of 3.8 micron this means approximately X by X lambda/D. 
We generate PSFs with the HEEPS python code in L band with the RAVC coronagraph at a pixel scale of X by X mas. This data is FFT interpolated with VIP\_HCI with a scaling factor to generate 1600 spectral bins in a wavelength range of X micron at a wavelength of X micron.
As the METIS IFU projected on-sky has rectangular pixels we reinterpolate this data for each temporal and spectral slice to a grid of X by X mas and then average the data in one dimension to get a pixel scale of X by X. The sum of all flux is set to be equal to the total photons of the star in L-band.
To go back to square pixels, we duplicate the undersampled axis. We inject noise according to the shot noise of the source, sky background and detector read noise with a single resolution element spanning about x by x pixels, while also using a binning factor of X spectrally and Y temporally to reduce memory consumption.



 


\subsubsection{Demonstration of ADI+SDI algorithm on IFU data}


This dataset if used together with an array containing a set of position angles given 1 hour of sky rotation and an array containing the wavelengths of the spectral axis to perform a median PSF subtraction technique using VIP\_HCI on the generated data. This is a two-step process where first for each timestep all spectral slices are rescaled to be on the same lambda/D grid. Then the median of the cube in the spectral dimension is subtracted. This is repeated for every timestep to give a time varying array of residuals. For this array the median in time is also calculated and subtracted. Afterwards the residuals are derotated and stacked. 

\clearpage
\section{QC Parameters}\label{sec:qc_parameters}
\clearpage
\section{Development plan}\label{sec:development_plan}

This section describes the development plan for the METIS data reduction software to comply with the the specifications given in the in the ESO dataflow deliverables standards document (\cite{1618}). The time lines are subject to regular updates and review with
\begin{itemize}
    \item AIT team: coordination of the pipeline implementation schedule
    \item ICS team: coordination with ICS implementation schedule to test the data flow between ICS and pipeline and check consistency of FITS keywords
    \item Detector group: coordination when detector characteristics become available
    \item ESO: check consistency of FITS keywords, QC parameters
\end{itemize}



\appendix
% This section contains a description of the format of all file types
% that are produced by our recipes, intermediate or final. It is
% modelled after the Muse pipeline manual, Appendix A.3.
\section{Recipe product files}
\label{app:recipe_products}

\subsection{METIS\_IMAGE}
\label{sapp:METIS_IMAGE}

\subsection{METIS\_CUBE}
\label{sapp:METIS_CUBE}


%%% Local Variables:
%%% mode: latex
%%% TeX-master: "METIS_DRLD"
%%% End:

% This section contains a description of the format of all FITS keywords
\section{FITS keywords}
\label{app:fits_keywords}
% This section contains a transmission fro the Earth's atmopshere
% 
\section{Atmopsheric transmission}
\label{app:atmo_trans}

\begin{figure}[ht]
  \centering
  \includegraphics[width=0.7\textheight]{figures/tra}
  \caption[Recipe: \REC{metis\_LM\_lss\_rsrf}]{\REC{metis\_LM\_lss\_rsrf} --
    Recipe workflow to create the spectroscopic flatfield by means of the \ac{RSRF}.}
  \label{Fig:rec_lm_lss_rsrf}
\end{figure}

\end{document}

%%
%% THE END
%%
%%%%%%%%%%%%%%%%%%%%%%%%%%%%%%%%%%%%%%%%%%%%%%%%%%%%%%%%%%%%%%%%%%%%%%%%%%%%%
