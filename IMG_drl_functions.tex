\subsection{IMG observing mode}\label{sec:drl_functions_img}

%---------------------------------------------------------------------
\subsubsection{metis\_derive\_gain}\label{drl:metis_derive_gain}
\begin{recipedef}
Name: & \hyperref[drl:metis_derive_gain]{\DRL{metis_derive_gain}} \\
Purpose: & Determine the gain. \\
Used in recipes: & \hyperref[sssec:metis_det_lingain]{\REC{metis_det_lingain}}\\
%Working remarks: & None \\
%Function Parameters: & None \\
Input: & $n\times$ \texttt{const double * means} \\
%Other inputs: & None \\
%QC outputs: & None\\
%Output FITS files: & None \\
Outputs: & gain [\TODO{units}]\\
               & \texttt{cpl\_error\_code} \\
General description: & Determine the gain as the slope of variance against mean of detlin images. \\
Mathematical description: & TBD \\
Quality assessment: & Through QC parameters \\
Error conditions: & None. \\
%Unit tests: & See \cite{DRLVT} (TBD). \\
\end{recipedef}
%---------------------------------------------------------------------
\subsubsection{Detect peak centroid location}\label{drl:img_peakcentroid}
\begin{recipedef}
Name: & \hyperref[drl:img_peakcentroid]{\DRL{detect\_centroid\_peak}} \\
Purpose: &Detect the location of a source peak by a centroid\\
Used in recipes: & \hyperref[rec:metisimgchophome]{\REC{metis_img_chophome}}\newline
\hyperref[rec:metislmadcmslitloss]{\REC{metis_lm_adc_slitloss}} \newline
\hyperref[rec:metisnadcmslitloss]{\REC{metis_n_adc_slitloss}}\\
%Working remarks: & None \\
%Function Parameters: & None \\
Input: & $n\times$ \texttt{const hdrl\_image * input} \\
%Other inputs: & None \\
%QC outputs: & None\\
%Output FITS files: & None \\
Outputs: & Location of detected peak [px]\\
               & \texttt{cpl\_error\_code} \\
General description: & Detection of a source by centroid fit \\
Mathematical description: & see Section~\ref{sssec:centroid} TBD \\
Quality assessment: & Through QC parameters \\
Error conditions: & See \cite{DRLVT} (TBD). \\
Unit tests: & See \cite{DRLVT} (TBD). \\
\end{recipedef}