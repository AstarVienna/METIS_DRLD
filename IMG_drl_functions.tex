
\subsection{IMG observing mode}\label{sec:drl_functions_img}

%---------------------------------------------------------------------

%\subsubsection{Function for basic processing}




\subsubsection{Detrending}\label{drl:img_detrend}
\begin{recipedef}
Name: & \DRL{img_detrend} \\
Purpose: &Remove detector signals\\
Used in recipes: & \REC{metis_lm_img_basic_reduce}\\
%Working remarks: & None \\
%Function Parameters: & None \\
Input: &  \texttt{metis\_image *input} \\
       &  \texttt{const metis\_image *dark} (\PROD{MASTER_DARK_det})\\
       &  \texttt{const metis\_image *flat}  (\PROD{MASTER_IMG_FLAT_LAMP_LM}, \PROD{MASTER_IMG_FLAT_LAMP_N}, \PROD{MASTER_IMG_FLAT_TWILIGHT_LM}, \PROD{MASTER_IMG_FLAT_TWILIGHT_N})\\
%Other inputs: & None \\
%QC outputs: & None\\
%Output FITS files: & None \\https://www.overleaf.com/project/5f1abb4137d7690001f8aeb1
Outputs: & \texttt{metis\_image *output} (Detrended Images) \\
                & \texttt{cpl\_error\_code} \\
General description: & De-trending the raw image \\
Mathematical description: & See Section~\ref{rec:metis_lm_img_basic_reduce} \\
Quality assessment: & Through QC parameters \\
Error conditions: & See~\cite{DRLVT}. \\
Unit tests: & See~\cite{DRLVT}. \\
\end{recipedef}


\subsubsection{Building sky background}\label{drl:img_skybackground_build}
\begin{recipedef}
Name: & \DRL{img_skybackground_build} \\
Purpose: & Building thermal background signals\\
Used in recipes: & \REC{metis_lm_img_background}\\
%Working remarks: & None \\
%Function Parameters: & None \\
Input: &  \texttt{metis\_imagelist *input} \\
        %& \texttt{const ls\_skybackground\_image} \\
        %\texttt{const ls\_master_dark_image *input} \\
    %\texttt{const ls\_master_flat_image *input} \\
%Other inputs: & None \\
%QC outputs: & None\\
%Output FITS files: & None \\
Outputs: & \texttt{const metis\_image *sky} (\PROD{LM_SCI_BKG}, \PROD{LM_STD_BKG}) \\
                & \texttt{cpl\_error\_code} \\
General description: & Removing sky backround from images \\
Mathematical description: & See Section~\ref{rec:metis_lm_img_background} \\
Quality assessment: & Through QC parameters \\
Error conditions: & See~\cite{DRLVT}. \\
Unit tests: & See~\cite{DRLVT}. \\
\end{recipedef}
        
    
\subsubsection{Remove sky background}\label{drl:img_skybackground_removal}
\begin{recipedef}
Name: & \DRL{img_skybackground_removal} \\
Purpose: & Remove sky background signals\\
Used in recipes: & \REC{metis_lm_img_background}\\
%Working remarks: & None \\
%Function Parameters: & None \\
Input: & \texttt{metis\_image *input} \\
       & \texttt{const metis\_image *sky} (\PROD{LM_SCI_BKG}, \PROD{LM_STD_BKG})\\
     %\texttt{const ls\_master_dark_image *input} \\
    %\texttt{const ls\_master_flat_image *input} \\
%Other inputs: & None \\
%QC outputs: & None\\
%Output FITS files: & None \\
Outputs: & \texttt{metis\_image *output} (sky subtracted image) \\
                & \texttt{cpl\_error\_code} \\
General description: & Removing sky backround from images \\
Mathematical description: & See Section~\ref{rec:metis_lm_img_background} \\
Quality assessment: & Through QC parameters \\
Error conditions: & See~\cite{DRLVT}. \\
Unit tests: & See~\cite{DRLVT}. \\
\end{recipedef}
    
%\subsubsection{Source extraction}\label{drl:img_source_extraction}
%\begin{recipedef}
%Name: & \DRL{img_source_extraction} \\
%Purpose: &Measure and extract source information for position calibration\\
%% HB 20230628: The metis_lm_img_astrometry_process seems unnecessary, so it
%%    is commented out. See https://github.com/AstarVienna/METIS_DRLD/pull/65
%%    where we can discuss this further.
%%Used in recipes: & \REC*{metis_lm_img_astrometry_process}\\
%Used in recipes: & None\\
%%Working remarks: & None \\
%%Function Parameters: & None \\
%Input: & \texttt{const hdrl\_image *input} \\
%     %\texttt{const ls\_master_dark_image *input} \\
%    %\texttt{const ls\_master_flat_image *input} \\
%%Other inputs: & None \\
%%QC outputs: & None\\
%%Output FITS files: & None \\
%Outputs: & star flux [counts] and location [px]\\
%                & \texttt{cpl\_error\_code} \\
%General description: & Measure and extract sources on a image \\
%%Mathematical description: & see Section~\ref{sssec:centroid} TBD \\
%Mathematical description: & TBD \\
%Quality assessment: & Through QC parameters \\
%Error conditions: & See~\cite{DRLVT}. \\
%Unit tests: & See~\cite{DRLVT}. \\
%\end{recipedef}

% HB 20230726: Apparently we should do all this without cross-correlation but instead use the WFS-FS mirror information. Nevertheless maybe having this function is useful.
%\subsubsection{Astrometry calibration}\label{drl:img_astrometry_calib}
%\begin{recipedef}
%Name: & \DRL{img_astrometry_calib} \\
%Purpose: &Check and refine WCS of input images by cross-correlation and dtermine the grid\\
%Used in recipes: & \REC{metis_lm_img_sci_postprocess}\\
%%Working remarks: & None \\
%%Function Parameters: & None \\
%Input: & \texttt{const hdrl\_image *input} \\
%     %\texttt{const ls\_master_dark_image *input} \\
%    %\texttt{const ls\_master_flat_image *input} \\
%%Other inputs: & None \\
%%QC outputs: & None\\
%%Output FITS files: & None \\
%Outputs: & \texttt{cpl\_error\_code} \\
%General description: & Astrometry calibration \\
%%Mathematical description: & see Section~\ref{sssec:centroid} TBD \\
%Mathematical description: & TBD \\
%Quality assessment: & Through QC parameters \\
%Error conditions: & See~\cite{DRLVT}. \\
%Unit tests: & See~\cite{DRLVT}. \\
%\end{recipedef}

% HB 20230726: We can use hdrl functionality for this.
%   For stacking we can use hdrl_imagelist_combine or hdrl_imagelist_collapse.
%   For resampling we can use hdrl_resample_imagelist_to_table or hdrl_resample_compute.
%\subsubsection{Image Resample and Coadding}\label{drl:img_resample_coadding}
%\begin{recipedef}
%Name: & \DRL{img_resample_coadding} \\
%Purpose: &Calculate the conversion bewtween pixels and sky coordinates\\
%Used in recipes: & \REC{metis_lm_img_sci_postprocess}\\
%%Working remarks: & None \\
%%Function Parameters: & None \\
%Input: & \texttt{const hdrl\_image *input} \\
%     %\texttt{const ls\_master_dark_image *input} \\
%    %\texttt{const ls\_master_flat_image *input} \\
%%Other inputs: & None \\
%%QC outputs: & None\\
%%Output FITS files: & None \\
%Outputs: & \texttt{cpl\_error\_code} \\
%General description: & Image resample and coadding \\
%%Mathematical description: & see Section~\ref{sssec:centroid} TBD \\
%Mathematical description: & TBD \\
%Quality assessment: & Through QC parameters \\
%Error conditions: & See~\cite{DRLVT}. \\
%Unit tests: & See~\cite{DRLVT}. \\
%\end{recipedef}


\subsubsection{Create master LM flat}\label{drl:lm_img_flat}\label{drl:metis_lm_img_flat}
\begin{recipedef}
Name: & \DRL{metis_lm_img_flat} \\
Purpose: & Create master flat field for the LM-band imaging detector\\
Used in recipes: & \REC{metis_lm_img_flat}\\
%Working remarks: & None \\
%Function Parameters: & None \\
Input: & \texttt{metis\_imagelist *input} \\
Other inputs: &  combination method (\texttt{median}, \texttt{mean}, \texttt{sigclip},\dots)\\
& parameters for combination method\\
%&  \PROD{BADPIX_MAP_2RG}   \\
QC outputs: & QC LM MFLAT RMS\\
%Output FITS files: & None \\
Outputs: &  \texttt{metis\_image *flat} (\PROD{MASTER_IMG_FLAT_LAMP_LM}, \PROD{MASTER_IMG_FLAT_TWILIGHT_LM}) \\
         & \texttt{cpl\_error\_code} \\
General description: & Determination of master flat for flat fielding \\
Mathematical description: & Combine images with the same DIT, fit slope of pixels against illumination level \\
Quality assessment: & Through QC parameters \\
Error conditions: & See~\cite{DRLVT}. \\
Unit tests: & See~\cite{DRLVT}. \\
\end{recipedef}

\subsubsection{Create master N flat}\label{drl:n_img_flat}\label{drl:metis_n_img_flat}
\begin{recipedef}
Name: & \DRL{metis_n_img_flat} \\
Purpose: & Create master flat field for the N-band imaging detector\\
Used in recipes: & \REC{metis_n_img_flat}\\
%Working remarks: & None \\
%Function Parameters: & None \\
Input: & \texttt{metis\_imagelist *input} \\
Other inputs: &  combination method (\texttt{median}, \texttt{mean}, \texttt{sigclip},\dots)\\
& parameters for combination method\\
%&  \PROD{BADPIX_MAP_GEO}   \\
QC outputs: & QC N MFLAT RMS\\
%Output FITS files: & None \\
Outputs: &  \texttt{metis\_image *flat} (\PROD{MASTER_IMG_FLAT_LAMP_N}, \PROD{MASTER_IMG_FLAT_TWILIGHT_N}) \\
         & \texttt{cpl\_error\_code} \\
General description: & Determination of master flat for flat fielding \\
Mathematical description: & Combine images with the same DIT, fit slope of pixels against illumination level \\
Quality assessment: & Through QC parameters \\
Error conditions: & See~\cite{DRLVT}. \\
Unit tests: & See~\cite{DRLVT}. \\
\end{recipedef}


\subsubsection{Flag deviant pixels in LM flat field}\label{drl:update_lm_flat_mask}\label{drl:metis_update_lm_flat_mask}
\begin{recipedef}
Name: & \DRL{metis_update_lm_flat_mask} \\
Purpose: & Flag deviant pixels in the LM master flat and update the image mask\\
Used in recipes: & \REC{metis_lm_img_flat}\\
%Working remarks: & None \\
%Function Parameters: & None \\
Input: & \texttt{const metis\_image *input} \\
& Threshold(s) for deviant pixel detection\\
QC outputs: & QC LM FLAT NBADPIX\\
            & QC LM FLAT MEAN\\
            & QC LM FLAT RMS\\
%Output FITS files: & None \\
Outputs:         &  \texttt{metis\_image *mask} (updated mask for flat field image) \\
                 & \texttt{cpl\_error\_code} \\
General description: &  Flag deviant pixels in master flat \\
Mathematical description: & flag pixels outside the provided threshold(s) \\
Quality assessment: & Through QC parameters \\
Error conditions: & See~\cite{DRLVT}. \\
Unit tests: & See~\cite{DRLVT}. \\
\end{recipedef}

\subsubsection{Flag deviant pixels in N flat field}\label{drl:update_n_flat_mask}\label{drl:metis_update_n_flat_mask}
\begin{recipedef}
Name: & \DRL{metis_update_n_flat_mask} \\
Purpose: & Flag deviant pixels in the N master flat and update the image mask\\
Used in recipes: & \REC{metis_n_img_flat}\\
%Working remarks: & None \\
%Function Parameters: & None \\
Input: & flat \texttt{const metis\_image *input} \\
Other Inputs: & Threshold(s) for deviant pixel detection\\
QC outputs: & QC N FLAT NBADPIX\\
            & QC N FLAT MEAN\\
            & QC N FLAT RMS\\
%Output FITS files: & None \\
Outputs:         &  \texttt{metis\_image *mask} (updated mask for flat field image) \\
                 & \texttt{cpl\_error\_code} \\
General description: &  Flag deviant pixels in master flat \\
Mathematical description: & flag pixels outside the provided threshold(s) \\
Quality assessment: & Through QC parameters \\
Error conditions: & See~\cite{DRLVT}. \\
Unit tests: & See~\cite{DRLVT}. \\
\end{recipedef}




\subsubsection{metis\_derive\_gain}\label{drl:metis_derive_gain}
\begin{recipedef}
Name: & \DRL{metis_derive_gain} \\
Purpose: & Determine the gain. \\
Used in recipes: & \REC{metis_det_lingain}\\
%Working remarks: & None \\
%Function Parameters: & None \\
Input: & $n\times$ \texttt{const double *means} \\
%Other inputs: & None \\
%QC outputs: & None\\
%Output FITS files: & None \\
Outputs: & double *gain (gain [e/adu]) \\
               & \texttt{cpl\_error\_code} \\
General description: & Determine the gain as the slope of variance against mean of detlin images. \\
Mathematical description: & See Section~\ref{sssec:gain} \\
Quality assessment: & Through QC parameters \\
Error conditions: & None. \\
%Unit tests: & See~\cite{DRLVT}. \\
\end{recipedef}
%---------------------------------------------------------------------
\subsubsection{metis\_derive\_nonlinearity}\label{drl:metis_derive_nonlinearity}
\begin{recipedef}
Name: & \DRL{metis_derive_nonlinearity} \\
Purpose: & Determine the non-linearity coefficients. \\
Used in recipes: & \REC{metis_det_lingain}\\
%Working remarks: & None \\
%Function Parameters: & None \\
Input: &  \texttt{metis\_imagelist *input} (Dark corrected non-linearity raws) \\
%Other inputs: & None \\
%QC outputs: & None\\
%Output FITS files: & None \\
Outputs: &  \texttt{metis\_image *linearity} (\PROD{LINEARITY_det}) \\
               & \texttt{cpl\_error\_code} \\
General description: & Determine the non-linearity coefficients. \\
Mathematical description: & low-order polynomial fit of the individual pixel values as function of the illumination intensity\\
Quality assessment: & Through QC parameters \\
Error conditions: & None. \\
%Unit tests: & See~\cite{DRLVT}. \\
\end{recipedef}

\subsubsection{Measure LM band flux of standard star}\label{drl:lm_calculate_std_flux}
\begin{recipedef}
Name: & \DRL{lm_calculate_std_flux} \\
Purpose: & Calculate LM band flux and position of standard star in detector units \\
Used in recipes: & \REC{metis_lm_img_std_process}\\
%Working remarks: & None \\
%Function Parameters: & None \\
Input: &  \texttt{const metis\_image *input} \\
%Other inputs: & photometric standard catalogue \\
QC outputs: & \QC{QC LM IMG STD BACKGD RMS}\\
            & \QC{QC LM STD PEAK CNTS}\\
            & \QC{QC LM STD APERTURE CNTS}\\
            & \QC{QC LM STD STREHL}\\
            & \QC{QC LM STD AIRMASS}                                                       \\
%Output FITS files: & \PROD{LM_STD_COMBINED} \\
Outputs: & \texttt{hdrl\_catalogue\_result *star} (position and flux of stars in detector units) \\
               & \texttt{cpl\_error\_code} \\
General description: & Photometry of standard star \\
Mathematical description: & See Section~\ref{sssec:lm_img_photstd} \\
Quality assessment: & Through QC parameters \\
Error conditions: & See~\cite{DRLVT}. \\
Unit tests: & See~\cite{DRLVT}. \\
\end{recipedef}


\subsubsection{Measure N band flux of standard star}\label{drl:n_std_flux}\label{drl:n_calculate_std_flux}
\begin{recipedef}
Name: & \DRL{n_calculate_std_flux} \\
Purpose: & Calculate N band flux and position of standard star in detector units \\
Used in recipes: & \REC{metis_n_img_std_process}\\
%Working remarks: & None \\
%Function Parameters: & None \\
Input: & \texttt{const metis\_image *input} \\
%Other inputs: & photometric standard catalogue \\
QC outputs: & \QC{QC N IMG STD BACKGD RMS}\\
            & \QC{QC N STD PEAK CNTS}\\
            & \QC{QC N STD APERTURE CNTS}\\
            & \QC{QC N STD STREHL}\\
            & \QC{QC N STD AIRMASS}                                                       \\
%Output FITS files: & \PROD{LM_STD_COMBINED} \\
Outputs: & \texttt{hdrl\_catalogue\_result *star} (position and flux of stars in detector units) \\
               & \texttt{cpl\_error\_code} \\
General description: & Photometry of standard star \\
Mathematical description: & See Section~\ref{n_img_std_process} \\
Quality assessment: & Through QC parameters \\
Error conditions: & See~\cite{DRLVT}. \\
Unit tests: & See~\cite{DRLVT}. \\
\end{recipedef}


% HB 20230723: We should use hdrl functinos for resampling / shifting / stacking of images
%   For stacking we can use hdrl_imagelist_combine or hdrl_imagelist_collapse.
%   For resampling we can use hdrl_resample_imagelist_to_table or hdrl_resample_compute.
%\subsubsection{Recentre and Stack Images}\label{drl:recentre_img}
%\begin{recipedef}
%Name: & \DRL{recentre_img} \\
%Purpose: & Recentre images based on position of standard star, and stack \\
%Used in recipes: & \REC{metis_lm_img_std_process}\\
%%Working remarks: & None \\
%%Function Parameters: & None \\
%Input: & $n\times$ \texttt{const hdrl\_image *input} \\
%Other inputs: & position and flux of stars in detector units (hdrl\_catalogue\_result structure)\\
%%QC outputs: & QC det IMG STD BACKGD RMS\\
%%            & QC det STD PEAK CNTS\\
%%            & QC det STD APERTURE CNTS\\
%%            & QC det STD STREHL\\
%Output FITS files: & \PROD{LM_STD_COMBINED} \\
%Outputs: & \texttt{cpl\_error\_code} \\
%General description: & shifting and stacking of a sequence of images \\
%Mathematical description: & See Section~\ref{sssec:lm_img_photstd} \\
%Quality assessment: & Through QC parameters \\
%Error conditions: & See~\cite{DRLVT}. \\
%Unit tests: & See~\cite{DRLVT}. \\
%\end{recipedef}

\subsubsection{Calculate flux calibration}\label{drl:calculate_std_fluxcal}
\begin{recipedef}
Name: & \DRL{calculate_std_fluxcal} \\
Purpose: & Calculate the conversion between instrumental and physical flux units \\
Used in recipes: & \REC{metis_lm_img_std_process}\\
%Working remarks: & None \\
%Function Parameters: & None \\
Input: &  \texttt{hdrl\_catalogue\_result *star} (Flux of standard star in instrumental units) \\
Other inputs: & \texttt{hdrl\_catalogue\_result *standard}  (photometric standard catalogue) \\
%QC outputs: & QC det IMG STD BACKGD RMS\\
%            & QC det STD PEAK CNTS\\
%            & QC det STD APERTURE CNTS\\
%            & QC det STD STREHL\\
%Output FITS files: & \PROD{LM_STD_COMBINED} \\
Outputs: & \texttt{cpl\_table *fluxcal} (\PROD{FLUXCAL_TAB}) \\
               & \texttt{cpl\_error\_code} \\
General description: & Calculate flux conversion to physical units \\
Mathematical description: & See Section~\ref{sssec:lm_img_photstd} \\
Quality assessment: & Through QC parameters \\
Error conditions: & See~\cite{DRLVT}. \\
Unit tests: & See~\cite{DRLVT}. \\
\end{recipedef}

\subsubsection{Calculate detection limits}\label{drl:calculate_detection_limits}
\begin{recipedef}
Name: & \DRL{calculate_detection_limits} \\
Purpose: & Calculate Detection Limits \\
Used in recipes: & \REC{metis_lm_img_std_process}\\
%Working remarks: & None \\
%Function Parameters: & None \\
Input: &  \texttt{const metis\_image *input} \\
Other Inputs: & \texttt{const cpl\_table *fluxcal} (\PROD{FLUXCAL_TAB}) \\
QC outputs: & QC det LM SENSITIVITY\\
            & QC det LM AREA SENSITIVITY\\
%Output FITS files: & \PROD{LM_STD_COMBINED} \\
Outputs: & \texttt{double *detlim} (detection limits for standard star)  \\
               & \texttt{cpl\_error\_code} \\
General description: & Calculate detection limits of standard star \\
Mathematical description: & See Section~\ref{sssec:lm_img_photstd} \\
Quality assessment: & Through QC parameters \\
Error conditions: & See~\cite{DRLVT}. \\
Unit tests: & See~\cite{DRLVT}. \\
\end{recipedef}



\subsubsection{Detect peak centroid location}\label{drl:img_peakcentroid}\label{drl:detect_centroid_peak}
\begin{recipedef}
Name: & \DRL{detect_centroid_peak} \\
Purpose: & Detect the location of a source peak by a centroid\\
Used in recipes: & \REC{metis_img_chophome}\newline
\REC{metis_lm_adc_slitloss} \newline
\REC{metis_n_adc_slitloss}\\
%Working remarks: & None \\
%Function Parameters: & None \\
Input: &  \texttt{metis\_imagelist *input} \\
%Other inputs: & None \\
%QC outputs: & None\\
%Output FITS files: & None \\
Outputs: &  \texttt{cpl\_table *centroid} (Location of detected peak in pixels)\\
               & \texttt{cpl\_error\_code} \\
General description: & Detection of a source by centroid fit \\
Mathematical description: & standard method \\
Quality assessment: & Through QC parameters \\
Error conditions: & See~\cite{DRLVT}. \\
Unit tests: & See~\cite{DRLVT}. \\
\end{recipedef}


\subsubsection{Convert LM image to physical flux}\label{drl:lm_scale_image_flux}
\begin{recipedef}
Name: & \DRL{lm_scale_image_flux} \\
Purpose: & Calculate the conversion between instrumental and physical flux units \\
Used in recipes: & \REC{metis_lm_img_calibrate}\\
%Working remarks: & None \\
%Function Parameters: & None \\
Input: &  \texttt{metis\_image *input} \\
       & \texttt{const cpl\_table *fluxcal} (\PROD{FLUXCAL_TAB}) \\
Other inputs: & None \\
QC outputs: & None\\
%Output FITS files: & \PROD{LM_STD_COMBINED} \\
Outputs: &  \texttt{metis\_image *output} (image in physical flux units) \\
General description: & Convert instrumental flux to ph/s \\
Mathematical description: & See Section~\ref{sssec:lm_img_calibrate} \\
Quality assessment: & Through QC parameters \\
Error conditions: & See~\cite{DRLVT}. \\
Unit tests: & See~\cite{DRLVT}. \\
\end{recipedef}



\subsubsection{Add LM calibration information to header}\label{drl:lm_update_header_distortion}
\begin{recipedef}
Name: & \DRL{lm_update_header_distortion} \\
Purpose: & Update the FITS header with calibration data (WCS, distortion, units)  \\
Used in recipes: & \REC{metis_lm_img_calibrate}\\
%Working remarks: & None \\
%Function Parameters: & None \\
Input: &  \texttt{metis\_image *input} \\
       &  \texttt{const cpl\_table *disttab} (\PROD{LM_DISTORTION_TABLE})\\
QC outputs: & None \\
Outputs: & \texttt{metis\_image *output} (updated header for image) \\
         & \texttt{cpl\_error\_code} \\
General description: & Add disortion, WCS and BUNIT information to FITS header \\
Mathematical description: & See Section~\ref{sssec:lm_img_calibrate} \\
Quality assessment: & None \\
Error conditions: & See~\cite{DRLVT}. \\
Unit tests: & See~\cite{DRLVT}. \\
\end{recipedef}


\subsubsection{Convert N image to physical flux}\label{drl:n_scale_image_flux}
\begin{recipedef}
Name: & \DRL{n_scale_image_flux} \\
Purpose: & Calculate the conversion between instrumental and physical flux units \\
Used in recipes: & \REC{metis_n_img_calibrate}\\
%Working remarks: & None \\
%Function Parameters: & None \\
Input: &  \texttt{metis\_image *input} \\
       & \texttt{cpl\_table *fluxcal} (\PROD{FLUXCAL_TAB}) \\
Other inputs: & None \\
QC outputs: & None\\
%Output FITS files: & \PROD{N_STD_COMBINED} \\
Outputs: &  \texttt{metis\_image *output} (image in physical flux units)\\
         &\texttt{cpl\_error\_code} \\
General description: & Convert instrumental flux to ph/s \\
Mathematical description: & See Section~\ref{sssec:n_img_calibrate} \\
Quality assessment: & Through QC parameters \\
Error conditions: & See~\cite{DRLVT}. \\
Unit tests: & See~\cite{DRLVT}. \\
\end{recipedef}



\subsubsection{Add N calibration information to header}\label{drl:n_update_header_distortion}
\begin{recipedef}
Name: & \DRL{n_update_header_distortion} \\
Purpose: & Update the FITS header with calibration data (WCS, distortion, units)  \\
Used in recipes: & \REC{metis_n_img_calibrate}\\
%Working remarks: & None \\
%Function Parameters: & None \\
Input: &    \texttt{metis\_image *input} \\
       &  \texttt{const cpl\_table *disttab} (\PROD{N_DISTORTION_TABLE})\\
QC outputs: & None \\
Output FITS files: & \texttt{metis\_image *output} (updated header for image)\\
Outputs: & \texttt{cpl\_error\_code} \\
General description: & Add disortion and BUNIT information to FITS header \\
Mathematical description: & See Section~\ref{sssec:n_img_calibrate} \\
Quality assessment: & None \\
Error conditions: & See~\cite{DRLVT}. \\
Unit tests: & See~\cite{DRLVT}. \\
\end{recipedef}



\subsubsection{Fit distortion parameters}\label{drl:fit_distortion}
\begin{recipedef}
Name: & \DRL{fit_distortion} \\
Purpose: & Fit optical distortion coefficients based on pinhole mask image  \\
Used in recipes: & \REC{metis_lm_img_distortion}\\
                 & \REC{metis_n_img_distortion}\\
%Working remarks: & None \\
Function Parameters: & parameters for fitting routine \\
Input: &    \texttt{const hdrl\_catalogue\_result *input} (catalogue of input sources)\\
       &    \texttt{const cpl\_table *pinhole} (\EXTCALIB{PINHOLE_TABLE})\\
QC outputs: & \QC{QC LM DISTORT RMS} or \QC{QC N DISTORT RMS}  \\
            & \QC{QC LM DISTORT NSOURCE} or \QC{QC N DISTORT NSOURCE} \\
Outputs:    & \texttt{cpl\_table *dist} (\PROD{det_DIST_REDUCED})\\
            &  \texttt{metis\_image *distmap}  (\PROD{det_DISTORTION_MAP})\\
            &   \texttt{cpl\_table *disttab} \PROD{det_DISTORTION_TABLE} \\
            & \texttt{cpl\_error\_code} \\
General description: &  Fit optical distortion coefficients based on pinhole mask image \\
Mathematical description: & See Section~\ref{sssec:lm_img_distortion} \\
Quality assessment: & None \\
Error conditions: & See~\cite{DRLVT}. \\
Unit tests: & See~\cite{DRLVT}. \\
\end{recipedef}


\subsubsection{Cutout region around beam}\label{drl:cutout_region}
\begin{recipedef}
Name: & \DRL{cutout_region} \\
Purpose: & Cutout a region around a beam in N band image  \\
Used in recipes: & \REC{metis_n_img_restore}\\
%Working remarks: & None \\
Function Parameters: & Size of cutout region \\
Input: &   \texttt{metis\_image *input} \\
QC outputs: None \\
%Output FITS files:\PROD{N_SCI_RESTORED} \\
Outputs:  & \texttt{metis\_image *output} (image cutout) \\
          & \texttt{cpl\_error\_code} \\
%Outputs:  &  \PROD{det_DISTORTION_TABLE} \\
%          & \texttt{cpl\_error\_code} \\
General description: & Cutout a region around the positive and negative beams in an N-band image \\
Mathematical description: & See Section~\ref{sssec:n_img_restoration} \\
Quality assessment: & None \\
Error conditions: & See~\cite{DRLVT}. \\
Unit tests: & See~\cite{DRLVT}. \\
\end{recipedef}
