\subsection{Imaging in LM and N band}
\label{ssec:overview_imaging}

The purpose of the pipeline is to correct or remove contributions from
the instrument, telescope, and atmosphere and produce science-grade
data products.  In the case of the METIS imaging modes the main
contributions to correct or remove are dark current, flatfield, bad
pixels, and, most importantly, thermal background emission from the
sky and the telescope. Further effects include persistence,
cross-talk, geometric distortions, etc. The final product of the
imaging pipeline is one or more flux-calibrated image(s) in units of
photons/s/pixel. Several images can be stacked into a single possibly
mosaiced image.

Due to the differences in characteristics between the HAWAII2RG
detector used for imaging in the L and M bands and the GeoSnap
detector used for the N band, the operational concept for the two
imager subsystems are quite different. This induces differences in the
way the data have to be reduced.

The GeoSnap detector has more stable gain than AQUARIUS detector,
which was still in the baseline at PDR.  Chopping is still necessary,
albeit at a lower frequency of a few Hz, and the standard chop/nod
technique will be employed for background subtraction.  As the dark
signal is automatically removed when the exposures from the different
chop and nod positions are combined no master dark is required for the
reduction of science data. Flat fielding may be possible, pending
further investigation of the detector stability

Observations and reduction of LM band data with the HAWAII2RG detecotr
can proceed as in the near infrared. After dark subtraction and
flat-fielding, the background is estimated from a series of dithered
science exposures or from exposures on a nearby blank patch of sky.

\TODO{Include association maps after possible revision. For HCI data,
  ADI may need to be part of reduction recipe if individual background
subtracted images are the goal?}

%%%%%%%%%%%%%%%%%%%%%%%%%%%%%%%%%%%%%%%%%%%%%%%%%%%%%%
%%% Local Variables:
%%% TeX-master: "METIS_DRLD"
%%% End:
