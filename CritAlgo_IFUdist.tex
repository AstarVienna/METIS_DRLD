The following is a summary of the steps that were taken to apply the
CRIRES+ data reduction to simulation the METIS LMS IFU.
\begin{itemize}
    \item ScopeSim runs
    \item Insert data into CRIRES frames
    \item Order tracing
    \item Tilt determination
    \item Extracting 1D spectra
    \item Assigning Wavelength and spatial scale to pixels in 2D spectra
\end{itemize}

\paragraph{ScopeSim}
As of April 2022, when these simulations were carried out, the
release-version of the simulator did not support the IFU mode of METIS.
Therefore a development version of the LMS-branch was istalled, after
instructions from the team. The mode was set such that the The scipt
that creates the frame with the sky spectrum is called
\texttt{lms\_sky.py} and does very little besides calling the simulator
with default arguments, and istructing it to set the central wavelength
of the spectrograph to arbitrary $3.55 \micron$. In addition, it retrieves
the wavelength scale from one of the slices, in order to later ingest it
into the CRIRES-pipeline.

\paragraph{CRIRES DRS}
The upgraded CRIRES is a cross-dispersed echelle slit-spectrograph for
the near-infrared bands YJHKLM. It uses three of the same HAWAII2RG
detectors that METIS will use four of. The fact that CRIRES images
several spectral orders while METIS does slices of spatial regions, is
no hindrance to applying the same methods for finding the location and
shape of each order/slice in the detector grid, and to determine the
orientation of the ``slit''.

The CRIRES DRS is a standard ESO-pipeline that was developped from
scratch for the upgraded instrument and is in operation since late 2021.
It can be installed from the usual sources for ESO pipelines on their
website, no particular version is needed for this exercise since the
basic algorithms have not changed.

ScopeSim produces frames for all four of the LMS detectors. Simply
ignoring the fourth, the pixel values were copied into a raw frame
CRIRES, thus preserving its headers and replacing the data.

\paragraph{Order tracing}

By ``trace'' we mean the polynomial in detector coordinates (x,y) that
describes the mid-line of the spectral order. This is done by smoothing
and thresholding the sky frame (which has enough continuum) in order to
distinguish in-order from inter-order pixels. Continuous clusters of
in-order pixels are then fitted with a second-degree polynomial. The
command executed was

\begin{verbatim}
# esorex cr2res_util_trace trace_sky.sof
\end{verbatim}

where \texttt{trace\_sky.sof} contains only one line, pretending the
sky frame is a FLAT (which is what is usually used for tracing)

\begin{verbatim}
CIRES_sky.fits FLAT
\end{verbatim}

The output file is \texttt{CIRES\_sky\_tw.fits} which is a table that
now has the traces for each slice. In addition to the mid-line, it also
has the polynomials for the upper and lower edges of each slice.

\paragraph{Slit tilt}

Next up is measuring the orientation of the ``slit'' in each slice,
i.e.~the tilt of spectral lines w.r.t. the detector columns. Using the
trace from the previous step, each slice gets rectified by shifting
columns by integer values according to the trace.

Then a peak-finder is run on each row to find spectral lines, which are
then fitted with Gaussians to determine their line centers. Comparing
how the line-centers change with detector rows, allows us to fit the
slit tilt as a polynomyal P(y). In order to reject outliers and ensure a
smooth change in the tilt, the \emph{coefficients} of the tilt
polynomials are then each fit as P(x).

These coefficients are saved in an updated FITS-table, called
\texttt{CIRES\_sky\_tw\_tw.fits}. Using the included script that
evaluates the polynomials and plots them on top of the sky frame, we can
have a look at the result.

    \begin{tcolorbox}[breakable, size=fbox, boxrule=1pt, pad at break*=1mm,colback=cellbackground, colframe=cellborder]
\prompt{In}{incolor}{25}{\boxspacing}
\begin{Verbatim}[commandchars=\\\{\}]
\PY{o}{\PYZpc{}}\PY{k}{run} show\PYZus{}trace\PYZus{}curv.py CIRES\PYZus{}sky\PYZus{}tw\PYZus{}tw.fits CIRES\PYZus{}sky.fits
\PY{c+c1}{\PYZsh{} (This plots only the top row of two detectors)}
\end{Verbatim}
\end{tcolorbox}

\begin{center}
\adjustimage{max size={0.9\linewidth}{0.9\paperheight}}{LMS distortion_files/LMS distortion_7_0.png}
\end{center}
{ \hspace*{\fill} \\}

\begin{tcolorbox}[breakable, size=fbox, boxrule=1pt, pad at break*=1mm,colback=cellbackground, colframe=cellborder]
\prompt{In}{incolor}{26}{\boxspacing}
\begin{Verbatim}[commandchars=\\\{\}]
\PY{c+c1}{\PYZsh{} Zooming in allows to perceive the tilt in the regularly plotted vertical lines that align with the data.}
\PY{k+kn}{from} \PY{n+nn}{IPython}\PY{n+nn}{.}\PY{n+nn}{display} \PY{k}{import} \PY{n}{Image}\PY{p}{,} \PY{n}{display}
\PY{n}{display}\PY{p}{(}\PY{n}{Image}\PY{p}{(}\PY{n}{filename}\PY{o}{=}\PY{l+s+s1}{\PYZsq{}}\PY{l+s+s1}{lms\PYZus{}sky\PYZus{}tracetilt\PYZus{}zoom.png}\PY{l+s+s1}{\PYZsq{}}\PY{p}{)}\PY{p}{)}
\end{Verbatim}
\end{tcolorbox}

\begin{center}
\adjustimage{max size={0.9\linewidth}{0.9\paperheight}}{LMS distortion_files/LMS distortion_8_0.png}
\end{center}
{ \hspace*{\fill} \\}

\begin{tcolorbox}[breakable, size=fbox, boxrule=1pt, pad at break*=1mm,colback=cellbackground, colframe=cellborder]
\prompt{In}{incolor}{27}{\boxspacing}
\begin{Verbatim}[commandchars=\\\{\}]
\PY{n}{tilts} \PY{o}{=} \PY{n}{fo}\PY{p}{(}\PY{l+s+s1}{\PYZsq{}}\PY{l+s+s1}{CIRES\PYZus{}sky\PYZus{}tw\PYZus{}tw.fits}\PY{l+s+s1}{\PYZsq{}}\PY{p}{)}\PY{p}{[}\PY{l+m+mi}{1}\PY{p}{]}\PY{o}{.}\PY{n}{data}\PY{p}{[}\PY{l+s+s1}{\PYZsq{}}\PY{l+s+s1}{SlitPolyB}\PY{l+s+s1}{\PYZsq{}}\PY{p}{]}
\PY{n}{tilts}
\end{Verbatim}
\end{tcolorbox}

\begin{tcolorbox}[breakable, size=fbox, boxrule=.5pt, pad at break*=1mm, opacityfill=0]
\prompt{Out}{outcolor}{27}{\boxspacing}
\begin{Verbatim}[commandchars=\\\{\}]
array([[-1.45030918e-02,  4.97644175e-06, -8.81093015e-10],
       [-1.71465266e-02,  1.19526702e-05, -4.11230539e-09],
       [-1.60470680e-02,  9.62834066e-06, -2.46179182e-09],
       [-1.56843290e-02,  1.04739179e-05, -3.39557250e-09],
       [-9.89049691e-03, -7.80064422e-07,  1.55907143e-09],
       [-1.37041189e-02,  7.36109573e-06, -1.76092183e-09],
       [-1.21335419e-02,  3.73431498e-06,  3.62338925e-10],
       [-9.88651568e-03,  1.90595104e-06,  9.70922842e-10],
       [-1.25626738e-02,  9.42000943e-06, -3.06478176e-09],
       [-1.23701034e-02,  4.81279417e-06,  6.75878571e-10],
       [-1.30267035e-02,  9.46268944e-06, -2.61068479e-09],
       [-9.47728454e-03,  3.20871517e-06,  1.27630431e-09],
       [-1.29999200e-02,  1.02179873e-05, -1.89150864e-09],
       [-1.04358752e-02,  2.52729614e-06,  1.30881039e-09]])
\end{Verbatim}
\end{tcolorbox}

When we evaluate one of the polynomials at the detector edges and
center, we find that the slit angle varies between 0.5 and 0.9 degrees.

\begin{tcolorbox}[breakable, size=fbox, boxrule=1pt, pad at break*=1mm,colback=cellbackground, colframe=cellborder]
\prompt{In}{incolor}{28}{\boxspacing}
\begin{Verbatim}[commandchars=\\\{\}]
\PY{n}{dx}\PY{o}{=}\PY{n}{np}\PY{o}{.}\PY{n}{polyval}\PY{p}{(}\PY{n}{tilts}\PY{p}{[}\PY{l+m+mi}{3}\PY{p}{]}\PY{p}{[}\PY{p}{:}\PY{p}{:}\PY{o}{\PYZhy{}}\PY{l+m+mi}{1}\PY{p}{]}\PY{p}{,}\PY{p}{[}\PY{l+m+mi}{1}\PY{p}{,}\PY{l+m+mi}{1024}\PY{p}{,}\PY{l+m+mi}{2048}\PY{p}{]}\PY{p}{)} \PY{c+c1}{\PYZsh{} reverse order of coeffs in python}
\PY{n}{dx}
\end{Verbatim}
\end{tcolorbox}

\begin{tcolorbox}[breakable, size=fbox, boxrule=.5pt, pad at break*=1mm, opacityfill=0]
\prompt{Out}{outcolor}{28}{\boxspacing}
\begin{Verbatim}[commandchars=\\\{\}]
array([-0.01567386, -0.00851955, -0.00847581])
\end{Verbatim}
\end{tcolorbox}

\begin{tcolorbox}[breakable, size=fbox, boxrule=1pt, pad at break*=1mm,colback=cellbackground, colframe=cellborder]
\prompt{In}{incolor}{29}{\boxspacing}
\begin{Verbatim}[commandchars=\\\{\}]
\PY{n}{np}\PY{o}{.}\PY{n}{degrees}\PY{p}{(}\PY{n}{np}\PY{o}{.}\PY{n}{arctan}\PY{p}{(}\PY{n}{dx}\PY{p}{)}\PY{p}{)} \PY{c+c1}{\PYZsh{} Tilt angles}
\end{Verbatim}
\end{tcolorbox}

\begin{tcolorbox}[breakable, size=fbox, boxrule=.5pt, pad at break*=1mm, opacityfill=0]
\prompt{Out}{outcolor}{29}{\boxspacing}
\begin{Verbatim}[commandchars=\\\{\}]
array([-0.89797241, -0.48812261, -0.48561642])
\end{Verbatim}
\end{tcolorbox}

\begin{tcolorbox}[breakable, size=fbox, boxrule=1pt, pad at break*=1mm,colback=cellbackground, colframe=cellborder]
\prompt{In}{incolor}{30}{\boxspacing}
\begin{Verbatim}[commandchars=\\\{\}]
\PY{n}{dx}\PY{o}{*}\PY{l+m+mi}{80} \PY{c+c1}{\PYZsh{} pixel difference between top and bottom of a slice, \PYZti{}80pix high}
\end{Verbatim}
\end{tcolorbox}

\begin{tcolorbox}[breakable, size=fbox, boxrule=.5pt, pad at break*=1mm, opacityfill=0]
\prompt{Out}{outcolor}{30}{\boxspacing}
\begin{Verbatim}[commandchars=\\\{\}]
array([-1.25390868, -0.68156423, -0.67806467])
\end{Verbatim}
\end{tcolorbox}

\paragraph{Extracting into 1D spectra}
Before we use the ``TW''-table from above to extract 1D-spectra, we need
to fix the wavelength scale that is also part of the table (TW stands
for TraceWave-table in CRIRES-lingo), because otherwise the spectra
would have a totally wrong wavelengh scale. For simplicity, the
numbering of the METIS slices was not matched with the CRIRES numbering
of orders, instead we set the wavelength of one slice (saved into
\texttt{lam\_1.npy} by the script above) for all spectra of the same
detector.

    \begin{tcolorbox}[breakable, size=fbox, boxrule=1pt, pad at break*=1mm,colback=cellbackground, colframe=cellborder]
\prompt{In}{incolor}{31}{\boxspacing}
\begin{Verbatim}[commandchars=\\\{\}]
\PY{n}{tw}\PY{o}{=}\PY{n}{fo}\PY{p}{(}\PY{l+s+s1}{\PYZsq{}}\PY{l+s+s1}{CIRES\PYZus{}sky\PYZus{}tw\PYZus{}tw.fits}\PY{l+s+s1}{\PYZsq{}}\PY{p}{)}
\PY{k}{for} \PY{n}{detec} \PY{o+ow}{in} \PY{p}{[}\PY{l+m+mi}{1}\PY{p}{,}\PY{l+m+mi}{2}\PY{p}{]}\PY{p}{:}
    \PY{n}{lam}\PY{o}{=}\PY{n}{np}\PY{o}{.}\PY{n}{load}\PY{p}{(}\PY{l+s+s1}{\PYZsq{}}\PY{l+s+s1}{lam\PYZus{}}\PY{l+s+si}{\PYZpc{}d}\PY{l+s+s1}{.npy}\PY{l+s+s1}{\PYZsq{}}\PY{o}{\PYZpc{}}\PY{k}{detec})
    \PY{n}{x}\PY{o}{=}\PY{n}{np}\PY{o}{.}\PY{n}{arange}\PY{p}{(}\PY{n+nb}{len}\PY{p}{(}\PY{n}{lam}\PY{p}{)}\PY{p}{)}\PY{o}{+}\PY{l+m+mi}{1} 
    \PY{n}{p}\PY{o}{=}\PY{n}{np}\PY{o}{.}\PY{n}{polyfit}\PY{p}{(}\PY{n}{x}\PY{p}{,}\PY{n}{lam}\PY{p}{,}\PY{l+m+mi}{2}\PY{p}{)}\PY{p}{[}\PY{p}{:}\PY{p}{:}\PY{o}{\PYZhy{}}\PY{l+m+mi}{1}\PY{p}{]}
    \PY{n+nb}{print}\PY{p}{(}\PY{n}{p}\PY{p}{)}
    \PY{k}{for} \PY{n}{row} \PY{o+ow}{in} \PY{n}{tw}\PY{p}{[}\PY{l+s+s1}{\PYZsq{}}\PY{l+s+s1}{CHIP}\PY{l+s+si}{\PYZpc{}d}\PY{l+s+s1}{.INT1}\PY{l+s+s1}{\PYZsq{}}\PY{o}{\PYZpc{}}\PY{k}{detec}].data:
        \PY{n}{row}\PY{p}{[}\PY{l+s+s1}{\PYZsq{}}\PY{l+s+s1}{Wavelength}\PY{l+s+s1}{\PYZsq{}}\PY{p}{]}\PY{o}{=}\PY{n}{p}
\PY{n}{tw}\PY{o}{.}\PY{n}{writeto}\PY{p}{(}\PY{n}{tw}\PY{o}{.}\PY{n}{filename}\PY{p}{(}\PY{p}{)}\PY{p}{,}\PY{n}{overwrite}\PY{o}{=}\PY{k+kc}{True}\PY{p}{)}
\end{Verbatim}
\end{tcolorbox}

    \begin{Verbatim}[commandchars=\\\{\}]
[ 3.55110788e+00  1.28761289e-05 -5.17343629e-11]
[ 3.52304458e+00  1.29677244e-05 -2.01943107e-11]
    \end{Verbatim}

The following command runs the optimal extraction which collapses the
full height of each slice into a 1D-spectrum, taking the trace and slit
tilt into account and iterating to minimize the residual error. (We
arbitrarily select ``order \#3''; again this is CRIRES-numbering and has
no meaning for the purpose of this exercise.)

\begin{verbatim}
# esorex cr2res_util_extract --detector=1 --order=3 --trace=1 extract_sky.sof
\end{verbatim}

Now we can plot the spectrum from SkyCalc, that was used by ScopeSim,
against the our extracted spectrum. For size considerations, the input
sky spectrum is not included in this package and we display the pre-made
figure here.

    \begin{tcolorbox}[breakable, size=fbox, boxrule=1pt, pad at break*=1mm,colback=cellbackground, colframe=cellborder]
\prompt{In}{incolor}{32}{\boxspacing}
\begin{Verbatim}[commandchars=\\\{\}]
\PY{n}{display}\PY{p}{(}\PY{n}{Image}\PY{p}{(}\PY{n}{filename}\PY{o}{=}\PY{l+s+s1}{\PYZsq{}}\PY{l+s+s1}{lms\PYZus{}skyrecover.png}\PY{l+s+s1}{\PYZsq{}}\PY{p}{)}\PY{p}{)}
\end{Verbatim}
\end{tcolorbox}

\begin{center}
    \adjustimage{max size={0.9\linewidth}{0.9\paperheight}}{LMS distortion_files/LMS distortion_17_0.png}
\end{center}
{ \hspace*{\fill} \\}

In blue the input spectrum, in orange and green the spectra from the two
detectors that cover the slice. Vertical offset and scaling are
arbitrary. The slight offset in wavelength is due to the mismatch in
slice number between the assigned wavelength and the one that got
extracted. This nicely illustrates the task of the wavelength
calibration, namely to determine this shift by cross-correlation, but
this in not part of the current exercise.

\paragraph{2D spectra}
Collapsing slices (or parts of one) into 1D spectra is certainly useful,
not the least for calibrations. But METIS is all about spatial
resolution, of course, so this whole exercise would be incomplete
without assigning to each pixel a wavelength and a speatial coordinate
along the slice. Luckily, this is quite straight-forward from the
information in the TW-table. The polynomials for the slice edges and
mid-line, together with the known on-sky length of the slice, define the
position of each pixel on the sky. To arrive at the correct wavelenth,
one evaluates the slit-tilt polynomial at the current pixel's distance
from the mid-line, to get the displacement in dispersion direction. Then
this delta-x gets converted to delta-lambda using the wavelength
solution.

\begin{tcolorbox}[breakable, size=fbox, boxrule=1pt, pad at break*=1mm,colback=cellbackground, colframe=cellborder]
    \prompt{In}{incolor}{33}{\boxspacing}
\begin{Verbatim}[commandchars=\\\{\}]
\PY{c+c1}{\PYZsh{} plot missing, due to technical difficulties, this is a stand\PYZhy{}in from CRIRES\PYZhy{}data.}
\PY{n}{display}\PY{p}{(}\PY{n}{Image}\PY{p}{(}\PY{n}{filename}\PY{o}{=}\PY{l+s+s1}{\PYZsq{}}\PY{l+s+s1}{crires\PYZus{}2d\PYZus{}rect.png}\PY{l+s+s1}{\PYZsq{}}\PY{p}{)}\PY{p}{)}
\end{Verbatim}
\end{tcolorbox}

\begin{center}
    \adjustimage{max size={0.9\linewidth}{0.9\paperheight}}{LMS distortion_files/LMS distortion_20_0.png}
\end{center}
{ \hspace*{\fill} \\}

Any non-linear spatial distortion along the slice, as measured from the
full field-of-view of the LM-Imager, or by illuminating the IFU slices
with the grid of the pin-hole mask in the WCU, can be added to the
spatial scale as a last step.

