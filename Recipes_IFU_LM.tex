%% 06_5-Recipes_IFU.tex
%% Created:     Fri Aug 25 13:14:02 2017 by Koehler@I-Mac
%%
%% subsection for IFU recipes
%%
%%%%%%%%%%%%%%%%%%%%%%%%%%%%%%%%%%%%%%%%%%%%%%%%%%%%%%%%%%%%%%%%%%%%%%%%%%%%%%%

\clearpage
\subsection{LM integral-field spectroscopy (IFU) recipes}
\label{ssec:IFU_recipes}

% Moved to https://github.com/AstarVienna/METIS_DRLD/issues/100
% \TODO{This section is identical to the PDR document~\cite{DRLS}. We
% will consider rearranging the recipes to be in line with the imaging
% pipelines. This would entail handling basic reduction and background
% subtraction for both science and standard exposures in common
% recipes (\CODE{metis_ifu_basic}, \CODE{metis_ifu_background}), then
% having a recipe to analyse the standard observations
% (\CODE{metis_ifu_photstd}). The science exposures are then fully
% calibrated (\CODE{metis_ifu_calibrate}). A full set of exposures would
% then be assembled and restored with a fully sampled PSF in a
% post-processing recipe (\CODE{metis_ifu_combine}).}

%------------------------------------------------------------------------------------------------------------------
\subsubsection{\REC*{metis_ifu_wavecal}: IFU wavelength calibration}
\label{sssec:ifu_wavecal}
\label{rec:metis_ifu_wavecal}

This recipe processes daytime wavelength calibration images to derive
the pixel-to-wavelength relation for the LM integral-field
spectrograph.

The calibration template will use the lasers in the warm calibration
unit to finely sample the desired wavelength range.
The \ac{WCU} has three lasers \cite{METIS-calibration_plan}:
(1) a fixed laser at \SI{3.39}{\micro\metre},
(2) the \ac{QCL}, a laser tuneable from at least \SI{4.68}{\micro\metre} to \SI{4.78}{\micro\metre}), and
(3) a fixed laser at \SI{5.26}{\micro\metre}.
The tuneable laser can be tuned so quickly during a long exposure with the
long-slit spectrograph or the LMS such that a single exposure with multiple
lines is created.


The image will consist of lines for each wavelength and slice.
The solution will have to provide for each detector pixel
$(x,y)$ the slice number $i$, the spatial position $\xi$ along the
slice and the wavelength in the dispersion correction. As the slices
and wavelength lines may be tilted with respect to the detector
columns and rows, a combined solution is required
\begin{align}
  \label{eq:wavelength_solution}
  \xi &= f_{i}(x, y) \\
  \lambda &= g_{i}(x, y)
\end{align}
The functions $f_{i}$ and $g_{i}$ are expected to be
sufficiently accurately described by low-order polynomials.

In principle we know this relation from our optical models, but we expect small
deviations ($\lesssim$ 1 px) because e.g.\ the main dispersion grating mechanism
in the \ac{LMS} did not come back to exactly to the nominal position.
We expect the offset from the optical model to be well represented either by a
simple linear function or a low-order polynomial.
The exact shape of the wavelength-to-pixel relation will be determined during
\ac{AIT}.

The wavelength range of the LM band is only partially covered by the lasers.
The model will therefore be calibrated only on the part of the range covered by
the lasers and subsequently interpolated to the whole image.
This calibrated optical model is then used by the other recipes as a first
guess for the wavelength solution.
%, even though it was directly verified only in a subset of the settings.

The boundaries of the slice image on the detector are obtained by
measuring the left and right edges of the wavelength lines.
The slice number is then obtained by counting the
slices according to the optical design of the spectrograph.
The wavelength of each line is known from the settings of the \ac{QCL}, the $x$
coordinate is obtained by
%linear interpolation along the line (or perhaps
using the distortion table from \REC{metis_ifu_distortion}
%if necessary).

The recipe produces a multi-extension FITS file with an image
extension mapping wavelength across each detector in the array.
A table extension holds the polynomial coefficients.

\begin{recipedef}
  Name:                & \REC{metis_ifu_wavecal}               \\
  Purpose:             & Determine pixel-to-wavelength transformation.                           \\
  Requirements:        & \REQ{METIS-6074}, \REQ{METIS-10300}                                     \\
  Type:                & Calibration                                                             \\
  Templates:           & \TPL{METIS_ifu_cal_InternalWave}                                        \\
  Input data:          & \RAW{IFU_WAVE_RAW}                    \\
                       & \PROD{MASTER_DARK_IFU}             \\
                       & \PROD{BADPIX_MAP_IFU}               \\
                       & \PROD{IFU_DISTORTION_TABLE}   \\
  Parameters:          & None                                                                    \\
  Algorithm:           & Measure line locations (left and right edges, centroid by Gaussian fit).\\
                       & Compute deviation from optical models.                                  \\
                       & Compute wavelength solution $\xi(x, y, i)$, $\lambda(x, y, i)$.         \\
                       & Compute wavelength map.                                                 \\
  Output data:         & \PROD{IFU_WAVECAL}                     \\
Expected accuracies:   & 1/5th of a pixel after post-processing (cf.~\cite{METIS-calibration_plan}, \REQ{METIS-6074}) \\
  QC1 parameters:      & \QC*{QC IFU WAVECAL RMS}                                                 \\
                       & \QC*{QC IFU WAVECAL NLINES}                                              \\
                       & \QC*{QC IFU WAVECAL PEAK CNTS}                                           \\
                       & \QC*{QC IFU WAVECAL LINE WIDTH}                                          \\
  \end{recipedef}

\begin{figure}[hb]
    \centering
    \def \globalscale {0.700000}
    \fontsize{10}{12}\selectfont
    %% Document preamble. Comment out for final figure! Footer too!
%\documentclass[tikz, margin=5mm, dvipsnames]{standalone}
%\usepackage{listings}
%\usepackage{hyperref}
%
% ADDING NEW DEFINITIONS -------------------------------------------- start
\definecolor{listingbg}{gray}{0.95}
\definecolor{darkgreen}{rgb}{0.0, 0.7, 0.0}
\definecolor{darkblue} {rgb}{0.0, 0.0, 0.7}
\definecolor{cyan} {rgb}{0.0, 0.4, 0.4}
\definecolor{darkred}  {rgb}{0.7, 0.0, 0.0}
\definecolor{darkorange}{rgb}{1.0, 0.49, 0.0}
\definecolor{violett}{rgb}{255, 0, 255}
\definecolor{turq}{rgb}{0.0, 0.7, 0.8}
\definecolor{fits}{rgb}{0.4, 0.1, 1}


\makeatletter
\lstdefinestyle{RAWstyle}{%
  basicstyle=\ttfamily\color{black}%
  \lst@ifdisplaystyle\scriptsize\fi}

\lstdefinestyle{PARstyle}{%
  basicstyle=\ttfamily\color{black}%
  \lst@ifdisplaystyle\scriptsize\fi}

\lstdefinestyle{DRLstyle}{%
  basicstyle=\ttfamily\color{black}%
  \lst@ifdisplaystyle\scriptsize\fi}

\lstdefinestyle{RECstyle}{%
  basicstyle=\ttfamily\color{black}%
  \lst@ifdisplaystyle\scriptsize\fi}

\lstdefinestyle{QCstyle}{%
  basicstyle=\ttfamily\color{black}%
  \lst@ifdisplaystyle\scriptsize\fi}

\lstdefinestyle{TPLstyle}{%
  basicstyle=\ttfamily\color{black}%
  \lst@ifdisplaystyle\scriptsize\fi}

\lstdefinestyle{PRODstyle}{%
  basicstyle=\ttfamily\color{black}%
  \lst@ifdisplaystyle\scriptsize\fi}

\lstdefinestyle{EXTCALIBstyle}{%
  basicstyle=\ttfamily\color{black}%
  \lst@ifdisplaystyle\scriptsize\fi}

\lstdefinestyle{STATCALIBstyle}{%
  basicstyle=\ttfamily\color{black}%
  \lst@ifdisplaystyle\scriptsize\fi}
\makeatother

%\makeatletter
\newcommand{\replaceunderscores}[1]{\expandafter\replace@underscores#1_\relax}

\def\replace@underscores#1_#2\relax{%
    \ifx \relax #2\relax
        #1%
    \else
        #1%
        \textunderscore
        \replace@underscores#2\relax
    \fi
}

\ExplSyntaxOn
% Generic \Smart@Item macro:
%   use \NEWRAW*{WHATEVER_THIS_IS} where hyperlinks are not needed (TOC, sections...)
%   and \RAW{WHATEVER_THIS_IS} for a full hyperlink-enabled version in regular text and tikz figures
% #1 boolean: use hyperlink
% #2 string: uppercase type of the item
% #3 string: id of the item
% #4 string: hyperref prefix
%
% NewExpandableDocumentCommand is used instead of NewDocumentCommand because
% this ensures the table of contents is correct if the macros are used in
% section headers. The xparse manual warns that
%
%    There are very rare occasion when it may be useful to create functions
%    using a fully-expandable argument grabber. [...] This facility should
%    only be used when absolutely necessary; if you do not understand when
%    this might be, do not use these functions!
%
% Nevertheless, NewExpandableDocumentCommand seems to work well. However,
% the last argument has to be m, r, R, l, or u, and not O.
\NewExpandableDocumentCommand{\Smart@Item}{m m m m}{%
    \IfBooleanTF{#1}{%
        \texorpdfstring{\lstinline[style=#2style]!#3!}{\replaceunderscores{#3}}%
    }{%
        % Replace spaces (e.g. in "QC ABC DEF") with underscores in order to
        % create a hyperref. The result of xstring macros (StrSubstitute) is not
        % expandable, and thus the result has to be stored, in MyHyper.
        % see section 3.2 of xstring manual.
        \StrSubstitute{#3}{\space}{_}[\MyHyper]
        \texorpdfstring{\hyperref[#4:\text_lowercase:n{\MyHyper}]{\lstinline[style=#2style]!#3!}}{\replaceunderscores{#3}}%
    }%
}
\ExplSyntaxOff

% Raw FITS file: \RAW{LM_SCI_RAW}
\NewDocumentCommand{\RAW}      {s m}{\Smart@Item{#1}{RAW}      {#2}{dataitem}}
\NewDocumentCommand{\PAR}      {s m}{\Smart@Item{#1}{PAR}      {#2}{dataitem}}
\NewDocumentCommand{\DRL}      {s m}{\Smart@Item{#1}{DRL}      {#2}{drl}}
\NewExpandableDocumentCommand{\REC}      {s m}{\Smart@Item{#1}{REC}      {#2}{rec}}
\NewDocumentCommand{\QC}       {s m}{\Smart@Item{#1}{QC}       {#2}{qc}}
\NewDocumentCommand{\PROD}     {s m}{\Smart@Item{#1}{PROD}     {#2}{dataitem}}
\NewDocumentCommand{\EXTCALIB} {s m}{\Smart@Item{#1}{EXTCALIB} {#2}{dataitem}}
\NewDocumentCommand{\STATCALIB}{s m}{\Smart@Item{#1}{STATCALIB}{#2}{dataitem}}
\NewDocumentCommand{\FITS}     {s m}{\Smart@Item{#1}{FITS}     {#2}{fits}}

% Templates: they do not have the starred version, no hyperrefs are created
\NewDocumentCommand{\TPL}      {m}{\Smart@Item{\BooleanTrue}{TPL}{#1}{}}
% Requirements: this only points to Polarion, no internal hyperrefs are created
\NewDocumentCommand{\REQ}      {m}{\href{https://polarion.astron.nl/polarion/\#/project/METIS/workitem?id=#1}{\textcolor{brown}{#1}}}

\makeatother

%\begin{document}


% ADDING NEW DEFINITIONS -------------------------------------------- start
\definecolor{listingbg}{gray}{0.95}
\definecolor{darkgreen}{rgb}{0.0, 0.7, 0.0}
\definecolor{darkblue} {rgb}{0.0, 0.0, 0.7}
\definecolor{cyan} {rgb}{0.0, 0.4, 0.4}
\definecolor{darkred}  {rgb}{0.7, 0.0, 0.0}
\definecolor{darkorange}{rgb}{1.0, 0.49, 0.0}
\definecolor{violett}{rgb}{255, 0, 255}
\definecolor{turq}{rgb}{0.0, 0.7, 0.8}
\definecolor{fits}{rgb}{0.4, 0.1, 1}


\makeatletter
\lstdefinestyle{RAWstyle}{%
  basicstyle=\ttfamily\color{black}%
  \lst@ifdisplaystyle\scriptsize\fi}

\lstdefinestyle{PARstyle}{%
  basicstyle=\ttfamily\color{black}%
  \lst@ifdisplaystyle\scriptsize\fi}

\lstdefinestyle{DRLstyle}{%
  basicstyle=\ttfamily\color{black}%
  \lst@ifdisplaystyle\scriptsize\fi}

\lstdefinestyle{RECstyle}{%
  basicstyle=\ttfamily\color{black}%
  \lst@ifdisplaystyle\scriptsize\fi}

\lstdefinestyle{QCstyle}{%
  basicstyle=\ttfamily\color{black}%
  \lst@ifdisplaystyle\scriptsize\fi}

\lstdefinestyle{TPLstyle}{%
  basicstyle=\ttfamily\color{black}%
  \lst@ifdisplaystyle\scriptsize\fi}

\lstdefinestyle{PRODstyle}{%
  basicstyle=\ttfamily\color{black}%
  \lst@ifdisplaystyle\scriptsize\fi}

\lstdefinestyle{EXTCALIBstyle}{%
  basicstyle=\ttfamily\color{black}%
  \lst@ifdisplaystyle\scriptsize\fi}

\lstdefinestyle{STATCALIBstyle}{%
  basicstyle=\ttfamily\color{black}%
  \lst@ifdisplaystyle\scriptsize\fi}
\makeatother

%%% This file contains definitions of shapes and nodes used
%%% for a recipe workflow
%%% Author       : Oliver Czoske
%%% Created      : 2021-03-03
%%% Last Changed : 2021-03-03
%%% Changes:
%%%

\usetikzlibrary{
  shapes.misc,
  positioning,
  calc,
  arrows.meta}

%% All connecting lines have an arrow
\tikzset{
  every path/.style={->, >=Latex[open], thick}
}

%% Start and stop buttons (black disks, stop with ring)
%% These are pics, use as
%%         \pic (name) [above of=..] {picname};
\tikzset{
  start/.pic = {
    \node (-m) at (0, 0){};
    \filldraw [fill=black] (0, 0) circle (0.2);
  }
}

\tikzset{
  stop/.pic = {
    \node (-m) at (0, 0){};
    \node (-t) at (0, -0.3){};
    \filldraw [fill=black] (0, 0) circle(0.2);
    \draw[black] (0, 0) circle (0.3);
  }
}


%%%% Various boxes and their colours
%%%% These are nodes, use as
%%%% \node (name) [type, location]  {text};

\definecolor{stepcolor}{RGB}{210,169,188}
\definecolor{rawcolor}{RGB}{235,235,235}
\definecolor{externalcolor}{RGB}{183,255,255}
\definecolor{calibcolor}{RGB}{255,250,216}
\definecolor{calproductcolor}{RGB}{185,184,237}
\definecolor{qcproductcolor}{RGB}{255,201,165}
\definecolor{sciproductcolor}{RGB}{197,219,183}
\definecolor{framecolor}{RGB}{127,13,65}

\tikzset{
  %% template : the template(s) that trigger(s) the recipe
  template/.style={
    rectangle,
    draw=black,
    minimum width=4.0cm,
    minimum height=0.5cm,
    align=center
  },
  %% input : the input files
  input/.style={
    rectangle,
    fill=rawcolor,
    minimum width=4.0cm,
    minimum height=0.75cm,
    text width=3cm,
    align=center
  },
  %% calib : calibration input
  calib/.style={
    rectangle,
    fill=calibcolor,
    minimum width=4.0cm,
    minimum height=0.75cm,
    text width=3cm,
    align=center
  },
  %% external : external input
  external/.style={
    rectangle,
    fill=externalcolor,
    minimum width=4.0cm,
    minimum height=0.75cm,
    text width=3.5cm,
    align=center
  },
  %% params : parameters
  params/.style={
    rectangle,
    draw=red,
    thick,
    minimum width=4.0cm,
    minimum height=0.75cm,
    text width=3cm,
    align=center
  },
  %% redstep : a reduction step
  %%      ("step" is predefined and can't be used)
  redstep/.style={
    rectangle,
    rounded corners=0.2cm,
    fill=stepcolor,   %%% define colour!
    minimum width=4.0cm,
    minimum height=1cm,
    text width=3cm,
    align=center
  },
  %% connection : connection to input or output
  connection/.style={
    circle,
    fill=black,
    minimum size=0.15cm,
    inner sep=0pt
  },
  %% sciproduct : a science product
  sciproduct/.style={
    rectangle,
    fill=sciproductcolor,
    minimum width=4.0cm,
    minimum height=0.75cm,
    text width=3.5cm,
    align=center
  },
  %% calproduct : a calibration product
  calproduct/.style={
    rectangle,
    fill=calproductcolor,
    minimum width=4.0cm,
    minimum height=0.75cm,
    text width=3.5cm,
    align=center
  },
  %% frame : frame around the recipe
  %% This is a path, use as
  %%    \draw [frame] (upper left) rectangle (lower right);
  frame/.style={framecolor, very thick, dashed}
}



\begin{tikzpicture}
  [x=1cm,
  y=-1cm,
  align=center,
  node distance=2cm and 3cm]
  \sffamily

  %% Grid for orientation. Comment out for final figure!
  % \draw[help lines, green](-5, 0) grid (8, 11);

  %%% Put workflow commands here:
  %% Main reduction workflow

  \node (template)  [template]{%
    \TPL{METIS_ifu_cal_InternalWave}
  };

  \pic (start) [below=0.75cm of template] {start};

  \node (input) [below=0.75cm of start-m, input, text width=3.8cm]{%
    \RAW{IFU_WAVE_RAW}
  };

  \node (step_detrend) [below=2.5cm of input, redstep]{%
    remove detector\\
    signature
  };

  \node (step_locatelines) [below=1.cm of step_detrend, redstep]{%
    locate lines
  };

  \node (step_fit) [below=1.cm of step_locatelines, redstep]{%
    fit polynomial
  };

  \pic (stop) [below=2.0cm of step_fit]{stop};

  %% Connections
  \draw [connection_arrow] (template) -- (input);
  \draw [connection_arrow] (input) -- (step_detrend);
  \draw [connection_arrow] (step_detrend) -- (step_locatelines);
  \draw [connection_arrow] (step_locatelines) -- (step_fit);
  \draw [connection_arrow] (step_fit) -- (stop-t);

  %% Input
  \node (connect_dark) [connection] at
    ($(input)!0.35!(step_detrend)$) {};
  \node (darkin) [left=of connect_dark, calproduct]{%
    \STATCALIB{MASTER_DARK_IFU}
  };
  \draw [connection_arrow] (darkin.east) -- (connect_dark);

%  \node (connect_bp) [connection] at($(input)!0.65!(step_detrend)$) {};
%  \node (bpmin) [left=of connect_bp, calproduct]{\STATCALIB{BADPIX_MAP_IFU}};
%  \draw [connection_arrow] (bpmin.east) -- (connect_bp);

  \node (connect_distortion) [connection] at
    ($(step_detrend)!0.5!(step_locatelines)$) {};
  \node (distortion) [left=of connect_distortion, calproduct]{%
    \STATCALIB{IFU_DISTORTION_TABLE}
  };
  \draw [connection_arrow] (distortion.east) -- (connect_distortion);

  %% Output
  \node (connectwavecal) [connection] at
  ($(step_fit.south)!0.35!(stop-t)$) {};
  \node (wavecal) [right=of connectwavecal, calproduct]{%
    \STATCALIB{IFU_WAVECAL}
  };
  \draw [connection_arrow] (connectwavecal) -- (wavecal);

  %% Frame around recipe
  \draw [frame] ($(input)!0.2!(step_detrend) - (3,0)$)
  rectangle ($(step_fit)!0.7!(stop-t) + (2.5,0)$);
  \node [framecolor, anchor=north west] at
  ($(input)!0.2!(step_detrend) - (3, 0)$) {%
    \REC{metis_ifu_wavecal}};

\end{tikzpicture}

% ADDING NEW DEFINITIONS -------------------------------------------- start
\definecolor{listingbg}{gray}{0.95}
\definecolor{darkgreen}{rgb}{0.0, 0.7, 0.0}
\definecolor{darkblue} {rgb}{0.0, 0.0, 0.7}
\definecolor{cyan} {rgb}{0.0, 0.4, 0.4}
\definecolor{darkred}  {rgb}{0.7, 0.0, 0.0}
\definecolor{darkorange}{rgb}{1.0, 0.49, 0.0}
\definecolor{violet}{rgb}{255, 0, 255}
\definecolor{turq}{rgb}{0.0, 0.7, 0.8}
\definecolor{fits}{rgb}{0.4, 0.1, 1}


\makeatletter
\lstdefinestyle{RAWstyle}{%
  basicstyle=\ttfamily\color{fits}%
  \lst@ifdisplaystyle\scriptsize\fi}

\lstdefinestyle{PARstyle}{%
  basicstyle=\ttfamily\color{cyan}%
  \lst@ifdisplaystyle\scriptsize\fi}

\lstdefinestyle{DRLstyle}{%
  basicstyle=\ttfamily\color{violet}%
  \lst@ifdisplaystyle\scriptsize\fi}

\lstdefinestyle{RECstyle}{%
  basicstyle=\ttfamily\color{darkgreen}%
  \lst@ifdisplaystyle\scriptsize\fi}

%% Write QC parameters like this: \QC*{QC_SOMETHING_OR_OTHER}
\lstdefinestyle{QCstyle}{%
  basicstyle=\ttfamily\color{darkblue}%
  \lst@ifdisplaystyle\scriptsize\fi}

%% Write templates like this: \TPL{DARK_LM}
\lstdefinestyle{TPLstyle}{%
  basicstyle=\ttfamily\color{darkred}%
  \lst@ifdisplaystyle\scriptsize\fi}

%% Write products like this: \PROD{SOME_THING}
\lstdefinestyle{PRODstyle}{%
  basicstyle=\ttfamily\color{darkorange}%
  \lst@ifdisplaystyle\scriptsize\fi}

%% external calib files
\lstdefinestyle{EXTCALIBstyle}{%
  basicstyle=\ttfamily\color{Turquoise}%
  \lst@ifdisplaystyle\scriptsize\fi}

% static calib files
\lstdefinestyle{STATCALIBstyle}{%
  basicstyle=\ttfamily\color{teal}%
  \lst@ifdisplaystyle\scriptsize\fi}

% static calib files
\lstdefinestyle{FITSstyle}{%
  basicstyle=\ttfamily\color{black}%
  \lst@ifdisplaystyle\scriptsize\fi}
\makeatother



%% Document footer. Comment out for final figure! Header too!
%\end{document}

  \caption[Recipe: \REC*{metis_ifu_wavecal}]{\REC*{metis_ifu_wavecal} --
    daytime wavelength calibration for the IFU.}
  \label{fig:metis_ifu_wavecal}
\end{figure}


%------------------------------------------------------------------------------------------------------------------
\clearpage
\subsubsection{\REC*{metis_ifu_rsrf}: IFU relative spectral response function}
\label{sssec:ifu_rsrf}
\label{rec:metis_ifu_rsrf}

This recipe creates a spectroscopic master flat and determines the
relative spectral response function (RSRF) for the four HAWAII2RG
detectors of the LM spectrograph. The input data are obtained by
illuminating the field of view with the black-body calibration lamp at
two different temperatures. The RSRF is then determined by dividing
the image by the known lamp continuum shape for the respective
temperature. We refer to the two-dimensional image obtained by this
division as \PROD{MASTER_FLAT_IFU} and the one-dimensional reponse
function obtained by averaging at constant wavelength as
\PROD{RSRF_IFU}. The bad pixel mask can be updated by identifying pixels
that deviate strongly from their neighbours.

\begin{recipedef}
Name:                & \REC{metis_ifu_rsrf}                                                     \\
Purpose:             & Create relative spectral response function for the IFU detector.         \\
Requirements:        & \REQ{METIS-6131}, \REQ{METIS-6698}                                       \\
Type:                & Calibration                                                              \\
Templates:           & \TPL{METIS_ifu_cal_rsrf}                                                 \\
Input data:          & \RAW{IFU_RSRF_RAW} (Raw flats taken with black-body calibration lamp.)   \\
                     & \PROD{MASTER_DARK_IFU}              \\
                     & \PROD{BADPIX_MAP_IFU}                                                    \\
                     & \PROD{IFU_WAVECAL}: image with wavelength at each pixel.                 \\
Parameters:          & None                                                                     \\
Algorithm:           & Create continuum image by mapping Planck spectrum at $T_{\mathrm{lamp}}$ to
                       wavelength image.                                                        \\
                     & Divide exposures by continuum image.                                     \\
                     & Average exposures to yield master flat (2D RSRF).                        \\
                     & Average in spatial direction to obtain relative response function        \\
Output data:         & \PROD{MASTER_FLAT_IFU}                                                   \\
                     & \PROD{RSRF_IFU}                                                          \\
                     & \PROD{BADPIX_MAP_IFU}                                                    \\
Expected accuracies: & 3\% (\REQ{METIS-6698})                                                   \\
QC1 parameters:      & \QC*{QC IFU RSRF NBADPIX}                                                    \\
%                     & (more TBD)                                                               \\
\end{recipedef}

\begin{figure}[hb]
    \centering
    \def \globalscale {0.700000}
    \fontsize{10}{12}\selectfont
    %% Document preamble. Comment out for final figure! Footer too!
%\documentclass[tikz, margin=5mm, dvipsnames]{standalone}
%\usepackage{listings}
%\usepackage{hyperref}
%
% ADDING NEW DEFINITIONS -------------------------------------------- start
\definecolor{listingbg}{gray}{0.95}
\definecolor{darkgreen}{rgb}{0.0, 0.7, 0.0}
\definecolor{darkblue} {rgb}{0.0, 0.0, 0.7}
\definecolor{cyan} {rgb}{0.0, 0.4, 0.4}
\definecolor{darkred}  {rgb}{0.7, 0.0, 0.0}
\definecolor{darkorange}{rgb}{1.0, 0.49, 0.0}
\definecolor{violett}{rgb}{255, 0, 255}
\definecolor{turq}{rgb}{0.0, 0.7, 0.8}
\definecolor{fits}{rgb}{0.4, 0.1, 1}


\makeatletter
\lstdefinestyle{RAWstyle}{%
  basicstyle=\ttfamily\color{black}%
  \lst@ifdisplaystyle\scriptsize\fi}

\lstdefinestyle{PARstyle}{%
  basicstyle=\ttfamily\color{black}%
  \lst@ifdisplaystyle\scriptsize\fi}

\lstdefinestyle{DRLstyle}{%
  basicstyle=\ttfamily\color{black}%
  \lst@ifdisplaystyle\scriptsize\fi}

\lstdefinestyle{RECstyle}{%
  basicstyle=\ttfamily\color{black}%
  \lst@ifdisplaystyle\scriptsize\fi}

\lstdefinestyle{QCstyle}{%
  basicstyle=\ttfamily\color{black}%
  \lst@ifdisplaystyle\scriptsize\fi}

\lstdefinestyle{TPLstyle}{%
  basicstyle=\ttfamily\color{black}%
  \lst@ifdisplaystyle\scriptsize\fi}

\lstdefinestyle{PRODstyle}{%
  basicstyle=\ttfamily\color{black}%
  \lst@ifdisplaystyle\scriptsize\fi}

\lstdefinestyle{EXTCALIBstyle}{%
  basicstyle=\ttfamily\color{black}%
  \lst@ifdisplaystyle\scriptsize\fi}

\lstdefinestyle{STATCALIBstyle}{%
  basicstyle=\ttfamily\color{black}%
  \lst@ifdisplaystyle\scriptsize\fi}
\makeatother

%\makeatletter
\newcommand{\replaceunderscores}[1]{\expandafter\replace@underscores#1_\relax}

\def\replace@underscores#1_#2\relax{%
    \ifx \relax #2\relax
        #1%
    \else
        #1%
        \textunderscore
        \replace@underscores#2\relax
    \fi
}

\ExplSyntaxOn
% Generic \Smart@Item macro:
%   use \NEWRAW*{WHATEVER_THIS_IS} where hyperlinks are not needed (TOC, sections...)
%   and \RAW{WHATEVER_THIS_IS} for a full hyperlink-enabled version in regular text and tikz figures
% #1 boolean: use hyperlink
% #2 string: uppercase type of the item
% #3 string: id of the item
% #4 string: hyperref prefix
%
% NewExpandableDocumentCommand is used instead of NewDocumentCommand because
% this ensures the table of contents is correct if the macros are used in
% section headers. The xparse manual warns that
%
%    There are very rare occasion when it may be useful to create functions
%    using a fully-expandable argument grabber. [...] This facility should
%    only be used when absolutely necessary; if you do not understand when
%    this might be, do not use these functions!
%
% Nevertheless, NewExpandableDocumentCommand seems to work well. However,
% the last argument has to be m, r, R, l, or u, and not O.
\NewExpandableDocumentCommand{\Smart@Item}{m m m m}{%
    \IfBooleanTF{#1}{%
        \texorpdfstring{\lstinline[style=#2style]!#3!}{\replaceunderscores{#3}}%
    }{%
        % Replace spaces (e.g. in "QC ABC DEF") with underscores in order to
        % create a hyperref. The result of xstring macros (StrSubstitute) is not
        % expandable, and thus the result has to be stored, in MyHyper.
        % see section 3.2 of xstring manual.
        \StrSubstitute{#3}{\space}{_}[\MyHyper]
        \texorpdfstring{\hyperref[#4:\text_lowercase:n{\MyHyper}]{\lstinline[style=#2style]!#3!}}{\replaceunderscores{#3}}%
    }%
}
\ExplSyntaxOff

% Raw FITS file: \RAW{LM_SCI_RAW}
\NewDocumentCommand{\RAW}      {s m}{\Smart@Item{#1}{RAW}      {#2}{dataitem}}
\NewDocumentCommand{\PAR}      {s m}{\Smart@Item{#1}{PAR}      {#2}{dataitem}}
\NewDocumentCommand{\DRL}      {s m}{\Smart@Item{#1}{DRL}      {#2}{drl}}
\NewExpandableDocumentCommand{\REC}      {s m}{\Smart@Item{#1}{REC}      {#2}{rec}}
\NewDocumentCommand{\QC}       {s m}{\Smart@Item{#1}{QC}       {#2}{qc}}
\NewDocumentCommand{\PROD}     {s m}{\Smart@Item{#1}{PROD}     {#2}{dataitem}}
\NewDocumentCommand{\EXTCALIB} {s m}{\Smart@Item{#1}{EXTCALIB} {#2}{dataitem}}
\NewDocumentCommand{\STATCALIB}{s m}{\Smart@Item{#1}{STATCALIB}{#2}{dataitem}}
\NewDocumentCommand{\FITS}     {s m}{\Smart@Item{#1}{FITS}     {#2}{fits}}

% Templates: they do not have the starred version, no hyperrefs are created
\NewDocumentCommand{\TPL}      {m}{\Smart@Item{\BooleanTrue}{TPL}{#1}{}}
% Requirements: this only points to Polarion, no internal hyperrefs are created
\NewDocumentCommand{\REQ}      {m}{\href{https://polarion.astron.nl/polarion/\#/project/METIS/workitem?id=#1}{\textcolor{brown}{#1}}}

\makeatother

%\begin{document}


% ADDING NEW DEFINITIONS -------------------------------------------- start
\definecolor{listingbg}{gray}{0.95}
\definecolor{darkgreen}{rgb}{0.0, 0.7, 0.0}
\definecolor{darkblue} {rgb}{0.0, 0.0, 0.7}
\definecolor{cyan} {rgb}{0.0, 0.4, 0.4}
\definecolor{darkred}  {rgb}{0.7, 0.0, 0.0}
\definecolor{darkorange}{rgb}{1.0, 0.49, 0.0}
\definecolor{violett}{rgb}{255, 0, 255}
\definecolor{turq}{rgb}{0.0, 0.7, 0.8}
\definecolor{fits}{rgb}{0.4, 0.1, 1}


\makeatletter
\lstdefinestyle{RAWstyle}{%
  basicstyle=\ttfamily\color{black}%
  \lst@ifdisplaystyle\scriptsize\fi}

\lstdefinestyle{PARstyle}{%
  basicstyle=\ttfamily\color{black}%
  \lst@ifdisplaystyle\scriptsize\fi}

\lstdefinestyle{DRLstyle}{%
  basicstyle=\ttfamily\color{black}%
  \lst@ifdisplaystyle\scriptsize\fi}

\lstdefinestyle{RECstyle}{%
  basicstyle=\ttfamily\color{black}%
  \lst@ifdisplaystyle\scriptsize\fi}

\lstdefinestyle{QCstyle}{%
  basicstyle=\ttfamily\color{black}%
  \lst@ifdisplaystyle\scriptsize\fi}

\lstdefinestyle{TPLstyle}{%
  basicstyle=\ttfamily\color{black}%
  \lst@ifdisplaystyle\scriptsize\fi}

\lstdefinestyle{PRODstyle}{%
  basicstyle=\ttfamily\color{black}%
  \lst@ifdisplaystyle\scriptsize\fi}

\lstdefinestyle{EXTCALIBstyle}{%
  basicstyle=\ttfamily\color{black}%
  \lst@ifdisplaystyle\scriptsize\fi}

\lstdefinestyle{STATCALIBstyle}{%
  basicstyle=\ttfamily\color{black}%
  \lst@ifdisplaystyle\scriptsize\fi}
\makeatother

%%% This file contains definitions of shapes and nodes used
%%% for a recipe workflow
%%% Author       : Oliver Czoske
%%% Created      : 2021-03-03
%%% Last Changed : 2021-03-03
%%% Changes:
%%%

\usetikzlibrary{
  shapes.misc,
  positioning,
  calc,
  arrows.meta}

%% All connecting lines have an arrow
\tikzset{
  every path/.style={->, >=Latex[open], thick}
}

%% Start and stop buttons (black disks, stop with ring)
%% These are pics, use as
%%         \pic (name) [above of=..] {picname};
\tikzset{
  start/.pic = {
    \node (-m) at (0, 0){};
    \filldraw [fill=black] (0, 0) circle (0.2);
  }
}

\tikzset{
  stop/.pic = {
    \node (-m) at (0, 0){};
    \node (-t) at (0, -0.3){};
    \filldraw [fill=black] (0, 0) circle(0.2);
    \draw[black] (0, 0) circle (0.3);
  }
}


%%%% Various boxes and their colours
%%%% These are nodes, use as
%%%% \node (name) [type, location]  {text};

\definecolor{stepcolor}{RGB}{210,169,188}
\definecolor{rawcolor}{RGB}{235,235,235}
\definecolor{externalcolor}{RGB}{183,255,255}
\definecolor{calibcolor}{RGB}{255,250,216}
\definecolor{calproductcolor}{RGB}{185,184,237}
\definecolor{qcproductcolor}{RGB}{255,201,165}
\definecolor{sciproductcolor}{RGB}{197,219,183}
\definecolor{framecolor}{RGB}{127,13,65}

\tikzset{
  %% template : the template(s) that trigger(s) the recipe
  template/.style={
    rectangle,
    draw=black,
    minimum width=4.0cm,
    minimum height=0.5cm,
    align=center
  },
  %% input : the input files
  input/.style={
    rectangle,
    fill=rawcolor,
    minimum width=4.0cm,
    minimum height=0.75cm,
    text width=3cm,
    align=center
  },
  %% calib : calibration input
  calib/.style={
    rectangle,
    fill=calibcolor,
    minimum width=4.0cm,
    minimum height=0.75cm,
    text width=3cm,
    align=center
  },
  %% external : external input
  external/.style={
    rectangle,
    fill=externalcolor,
    minimum width=4.0cm,
    minimum height=0.75cm,
    text width=3.5cm,
    align=center
  },
  %% params : parameters
  params/.style={
    rectangle,
    draw=red,
    thick,
    minimum width=4.0cm,
    minimum height=0.75cm,
    text width=3cm,
    align=center
  },
  %% redstep : a reduction step
  %%      ("step" is predefined and can't be used)
  redstep/.style={
    rectangle,
    rounded corners=0.2cm,
    fill=stepcolor,   %%% define colour!
    minimum width=4.0cm,
    minimum height=1cm,
    text width=3cm,
    align=center
  },
  %% connection : connection to input or output
  connection/.style={
    circle,
    fill=black,
    minimum size=0.15cm,
    inner sep=0pt
  },
  %% sciproduct : a science product
  sciproduct/.style={
    rectangle,
    fill=sciproductcolor,
    minimum width=4.0cm,
    minimum height=0.75cm,
    text width=3.5cm,
    align=center
  },
  %% calproduct : a calibration product
  calproduct/.style={
    rectangle,
    fill=calproductcolor,
    minimum width=4.0cm,
    minimum height=0.75cm,
    text width=3.5cm,
    align=center
  },
  %% frame : frame around the recipe
  %% This is a path, use as
  %%    \draw [frame] (upper left) rectangle (lower right);
  frame/.style={framecolor, very thick, dashed}
}



\begin{tikzpicture}
  [x=1cm,
  y=-1cm,
  align=center,
  node distance=2cm and 3cm]
  \sffamily

  %% Grid for orientation. Comment out for final figure!
  % \draw[help lines, green](-5, 0) grid (8, 11);

  %%% Put workflow commands here:
  %% Main reduction workflow

  \node (template) [template]{%
    \TPL{METIS_ifu_cal_rsrf}
  };

  \pic (start) [below=0.75cm of template] {start};

  \node (input) [below=0.75cm of start-m, input]{%
    \textsl{N} \RAW{IFU_RSRF_RAW}
  };

  \node (step_detrend) [below=3.5cm of input, redstep]{%
    detector signature\\
    removal
  };


  \node (step_normalize) [below=1.cm of step_detrend, redstep]{%
    continuum\\
    normalisation
  };

  \node (step_average) [below=1.cm of step_normalize, redstep]{%
    average/median
  };

  \pic (stop) [below=3.5cm of step_average]{stop};

  %% Connections
  \draw [connection_arrow] (template) -- (input);
  \draw [connection_arrow] (input) -- (step_detrend);
  \draw [connection_arrow] (step_detrend) -- (step_normalize);
  \draw [connection_arrow] (step_normalize) -- (step_average);
  \draw [connection_arrow] (step_average) -- (stop-t);

  %% Input
% if your rsrf raw has persistence, you are doin
%  \node (connectpersistence) [connection] at
%  ($(input)!0.35!(step_detrend)$) {};
%  \node (persistence) [left=3.95cm of connectpersistence, external]{%
%    PERSISTENCE\_MAP
%  };
%  \draw [connection_arrow] (persistence) -- (connectpersistence);

  \node (connect_persistence) [connection] at ($(input)!0.35!(step_detrend)$) {};
  \node (persistence) [left=of connect_persistence, calproduct]{\EXTCALIB{PERSISTENCE_MAP}};
  \draw [connection_arrow] (persistence.east) -- (connect_persistence);

  \node (connect_dark) [connection] at ($(input)!0.55!(step_detrend)$) {};
  \node (darkin) [left=of connect_dark, calproduct]{\STATCALIB{MASTER_DARK_IFU}};
  \draw [connection_arrow] (darkin.east) -- (connect_dark);

  \node (connect_distortion) [connection] at ($(input)!0.75!(step_detrend)$) {};
  \node (distortion) [left=of connect_distortion, calproduct]{\STATCALIB{IFU_DISTORTION_TABLE}};
  \draw [connection_arrow] (distortion.east) -- (connect_distortion);

%  \node (connect_bp) [connection] at ($(input)!0.65!(step_detrend)$) {};
%  \node (bpmin) [left=of connect_bp, calproduct]{\STATCALIB{BADPIX_MAP_IFU}};
%  \draw [connection_arrow] (bpmin.east) -- (connect_bp);

  \node (connect_wavecal) [connection] at ($(step_detrend)!0.5!(step_normalize)$) {};
  \node (wavecal) [left=of connect_wavecal, calproduct]{\STATCALIB{IFU_WAVECAL}};
  \draw [connection_arrow] (wavecal) -- (connect_wavecal);

  %% Output
  \node (connectrsrf) [connection] at ($(step_average)!0.25!(stop-t)$) {};
  \node (rsrf) [right=of connectrsrf, calproduct]{\STATCALIB{RSRF_IFU}};
  \draw [connection_arrow] (connectrsrf) -- (rsrf);

  \node (connectflat) [connection] at ($(step_average)!0.5!(stop-t)$) {};
  \node (flat) [right=of connectflat, calproduct]{\STATCALIB{MASTER_FLAT_IFU}};
  \draw [connection_arrow] (connectflat) -- (flat);

  \node (connectbpm) [connection] at ($(step_average)!0.75!(stop-t)$) {};
  \node (bpm) [right=of connectbpm, calproduct]{\STATCALIB{BADPIX_MAP_IFU}};
  \draw [connection_arrow] (connectbpm) -- (bpm);

  %% Frame around recipe
  \draw [frame] ($(input)!0.2!(step_detrend) - (2.5,0)$)
    rectangle ($(step_average)!0.85!(stop-t) + (2.5,0)$);
  \node [framecolor, anchor=north west] at ($(input)!0.2!(step_detrend) - (2.5, 0)$)
    {\REC{metis_ifu_rsrf}};

\end{tikzpicture}

% ADDING NEW DEFINITIONS -------------------------------------------- start
\definecolor{listingbg}{gray}{0.95}
\definecolor{darkgreen}{rgb}{0.0, 0.7, 0.0}
\definecolor{darkblue} {rgb}{0.0, 0.0, 0.7}
\definecolor{cyan} {rgb}{0.0, 0.4, 0.4}
\definecolor{darkred}  {rgb}{0.7, 0.0, 0.0}
\definecolor{darkorange}{rgb}{1.0, 0.49, 0.0}
\definecolor{violet}{rgb}{255, 0, 255}
\definecolor{turq}{rgb}{0.0, 0.7, 0.8}
\definecolor{fits}{rgb}{0.4, 0.1, 1}


\makeatletter
\lstdefinestyle{RAWstyle}{%
  basicstyle=\ttfamily\color{fits}%
  \lst@ifdisplaystyle\scriptsize\fi}

\lstdefinestyle{PARstyle}{%
  basicstyle=\ttfamily\color{cyan}%
  \lst@ifdisplaystyle\scriptsize\fi}

\lstdefinestyle{DRLstyle}{%
  basicstyle=\ttfamily\color{violet}%
  \lst@ifdisplaystyle\scriptsize\fi}

\lstdefinestyle{RECstyle}{%
  basicstyle=\ttfamily\color{darkgreen}%
  \lst@ifdisplaystyle\scriptsize\fi}

%% Write QC parameters like this: \QC*{QC_SOMETHING_OR_OTHER}
\lstdefinestyle{QCstyle}{%
  basicstyle=\ttfamily\color{darkblue}%
  \lst@ifdisplaystyle\scriptsize\fi}

%% Write templates like this: \TPL{DARK_LM}
\lstdefinestyle{TPLstyle}{%
  basicstyle=\ttfamily\color{darkred}%
  \lst@ifdisplaystyle\scriptsize\fi}

%% Write products like this: \PROD{SOME_THING}
\lstdefinestyle{PRODstyle}{%
  basicstyle=\ttfamily\color{darkorange}%
  \lst@ifdisplaystyle\scriptsize\fi}

%% external calib files
\lstdefinestyle{EXTCALIBstyle}{%
  basicstyle=\ttfamily\color{Turquoise}%
  \lst@ifdisplaystyle\scriptsize\fi}

% static calib files
\lstdefinestyle{STATCALIBstyle}{%
  basicstyle=\ttfamily\color{teal}%
  \lst@ifdisplaystyle\scriptsize\fi}

% static calib files
\lstdefinestyle{FITSstyle}{%
  basicstyle=\ttfamily\color{black}%
  \lst@ifdisplaystyle\scriptsize\fi}
\makeatother



%% Document footer. Comment out for final figure! Header too!
%\end{document}

  \caption[Recipe: \REC*{metis_ifu_rsrf}]{\REC*{metis_ifu_rsrf} --
    creation of IFU relative spectral response function.}
  \label{fig:metis_ifu_rsrf}
\end{figure}


%------------------------------------------------------------------------------------------------------------------
\clearpage
\subsubsection{\REC*{metis_ifu_std_process}: IFU flux standard reduction}
\label{sssec:ifu_std_process}
\label{rec:metis_ifu_std_process}

This recipe reduces and analyses a series of IFU observations of a
spectroscopic flux standard star. The comparison of the measured
detector counts (ADU) with the tabulated spectrum of the star gives
the wavelength-dependent conversion from ADU to physical units
(photons per second per centimetre square per micron per arcsec square).

The level of stray light is estimated in the dark areas between the
spectra and subtracted from the entire frame. The distribution of
stray light across the field can only be characterised once the
instrument is built. It is to be hoped that subtraction of a constant
or a low-level 2D polynomial fit will be sufficient.

The sky and thermal background is estimated from blank sky
observations (if obtained during the observing sequence) or by
combining the (dithered) science frames.

The wavelength calibration is taken from the daylight calibration. It
may be refined by measuring telluric emission and/or absorption lines
(by fitting with \lstinline{molecfit}).

\begin{recipedef}
  Name:                & \REC{metis_ifu_std_process}                                            \\
  Purpose:             & Determine conversion between detector counts and physical source flux. \\
  Requirements:        & \REQ{METIS-6131}                                                       \\
  Type:                & Calibration                                                            \\
  Templates:           & \TPL{METIS_ifu_cal_standard}                                           \\
  Input data:          & \RAW{IFU_STD_RAW} (Raw spectra of flux standard star)                  \\
                       & \PROD{MASTER_DARK_IFU}            \\
                       & \PROD{RSRF_IFU} (2D relative spectral response function)               \\
                       & \PROD{BADPIX_MAP_IFU}  \\
                       & \PROD{IFU_WAVECAL} \\
                       & \PROD{IFU_DISTORTION_TABLE} \\
  Parameters:          & None                                                                   \\
  Algorithm:           & Subtract dark, divide by master flat                                   \\
                       & Estimate stray light and subtract                                      \\
                       & Estimate background and subtract                                       \\
                       & Rectify spectra and assemble cube                                      \\
                       & Extract 1D spectrum of star                                            \\
                       & Compute and apply telluric correction                                  \\
                       & Compute conversion to physical units as function of wavelength.        \\
  Output data:         & \PROD{IFU_STD_REDUCED_CUBE}  \\
                       & \PROD{IFU_STD_BACKGROUND_CUBE}                                         \\
                       & \PROD{IFU_STD_REDUCED_1D}                                              \\
                       & \PROD{IFU_STD_TELLURIC_1D}                                             \\
                       & \PROD{FLUXCAL_TAB}                                                     \\
  Expected accuracies: & $<5$\% absolute flux calibration \\
  QC1 parameters:      & \QC*{QC IFU STD STRAYLIGHT MEAN}                                        \\
\end{recipedef}

\begin{figure}[hb]
  \centering
    \def \globalscale {0.700000}
    \fontsize{10}{12}\selectfont
    %%% This file contains definitions of shapes and nodes used
%%% for a recipe workflow
%%% Author       : Oliver Czoske
%%% Created      : 2021-03-03
%%% Last Changed : 2021-03-03
%%% Changes:
%%%

\usetikzlibrary{
  shapes.misc,
  positioning,
  calc,
  arrows.meta}

%% All connecting lines have an arrow
\tikzset{
  every path/.style={->, >=Latex[open], thick}
}

%% Start and stop buttons (black disks, stop with ring)
%% These are pics, use as
%%         \pic (name) [above of=..] {picname};
\tikzset{
  start/.pic = {
    \node (-m) at (0, 0){};
    \filldraw [fill=black] (0, 0) circle (0.2);
  }
}

\tikzset{
  stop/.pic = {
    \node (-m) at (0, 0){};
    \node (-t) at (0, -0.3){};
    \filldraw [fill=black] (0, 0) circle(0.2);
    \draw[black] (0, 0) circle (0.3);
  }
}


%%%% Various boxes and their colours
%%%% These are nodes, use as
%%%% \node (name) [type, location]  {text};

\definecolor{stepcolor}{RGB}{210,169,188}
\definecolor{rawcolor}{RGB}{235,235,235}
\definecolor{externalcolor}{RGB}{183,255,255}
\definecolor{calibcolor}{RGB}{255,250,216}
\definecolor{calproductcolor}{RGB}{185,184,237}
\definecolor{qcproductcolor}{RGB}{255,201,165}
\definecolor{sciproductcolor}{RGB}{197,219,183}
\definecolor{framecolor}{RGB}{127,13,65}

\tikzset{
  %% template : the template(s) that trigger(s) the recipe
  template/.style={
    rectangle,
    draw=black,
    minimum width=4.0cm,
    minimum height=0.5cm,
    align=center
  },
  %% input : the input files
  input/.style={
    rectangle,
    fill=rawcolor,
    minimum width=4.0cm,
    minimum height=0.75cm,
    text width=3cm,
    align=center
  },
  %% calib : calibration input
  calib/.style={
    rectangle,
    fill=calibcolor,
    minimum width=4.0cm,
    minimum height=0.75cm,
    text width=3cm,
    align=center
  },
  %% external : external input
  external/.style={
    rectangle,
    fill=externalcolor,
    minimum width=4.0cm,
    minimum height=0.75cm,
    text width=3.5cm,
    align=center
  },
  %% params : parameters
  params/.style={
    rectangle,
    draw=red,
    thick,
    minimum width=4.0cm,
    minimum height=0.75cm,
    text width=3cm,
    align=center
  },
  %% redstep : a reduction step
  %%      ("step" is predefined and can't be used)
  redstep/.style={
    rectangle,
    rounded corners=0.2cm,
    fill=stepcolor,   %%% define colour!
    minimum width=4.0cm,
    minimum height=1cm,
    text width=3cm,
    align=center
  },
  %% connection : connection to input or output
  connection/.style={
    circle,
    fill=black,
    minimum size=0.15cm,
    inner sep=0pt
  },
  %% sciproduct : a science product
  sciproduct/.style={
    rectangle,
    fill=sciproductcolor,
    minimum width=4.0cm,
    minimum height=0.75cm,
    text width=3.5cm,
    align=center
  },
  %% calproduct : a calibration product
  calproduct/.style={
    rectangle,
    fill=calproductcolor,
    minimum width=4.0cm,
    minimum height=0.75cm,
    text width=3.5cm,
    align=center
  },
  %% frame : frame around the recipe
  %% This is a path, use as
  %%    \draw [frame] (upper left) rectangle (lower right);
  frame/.style={framecolor, very thick, dashed}
}


\begin{tikzpicture}
  [x=1cm,
  y=-1cm,
  align=center,
  node distance=2cm and 3cm]
  \sffamily

  %% Grid for orientation. Comment out for final figure!
  %\draw[help lines, green](-5, 0) grid (8, 21);

  %%% Put workflow commands here:
  %% Main reduction workflow

  \node (template) [template]{%
    METIS\_ifu\_cal\_standard
  };

  \pic (start) [below=0.75cm of template] {start};

  \node (input) [below=0.75cm of start-m, input]{%
    \textsl{N} IFU\_STD\_RAW
  };

  \node (step1) [below=2cm of input, redstep]{%
    detector signature\\
    removal
  };

  \node (step2)[below=1.cm of step1, redstep]{%
    background\\
    subtraction
  };

  \node (step3) [below=1.cm of step2, redstep]{%
    rectification
  };

  \node (step4) [below=1.cm of step3, redstep]{%
    1D extraction
  };

  \node (step5) [below=1.cm of step4, redstep]{%
    telluric correction
  };

  \node (step6) [below=1.cm of step5, redstep]{%
    flux calibration
  };

  \pic (stop) [below=2.5cm of step6]{stop};

  %% Connections
  \draw (template) -- (input);
  \draw (input) -- (step1);
  \draw (step1) -- (step2);
  \draw (step2) -- (step3);
  \draw (step3) -- (step4);
  \draw (step4) -- (step5);
  \draw (step5) -- (step6);
  \draw (step6) -- (stop-t);

  %% Input
  \node (connectpers) [connection] at
  ($(input)!0.35!(step1)$) {};
  \node (persistence) [left=3.95cm of connectpers, external]{%
    PERSISTENCE\_MAP
  };
  \draw (persistence) -- (connectpers);

  % Input for detector signature removal (step1)
  \node (bpmin) [left=2cm of step1, yshift=0.8cm, calproduct]{%
    BADPIX\_MAP\_IFU
  };
  \draw (bpmin.east) -- ++(1., 0) -- ++(0., 0.6) -- ++(1., 0);

  \node (darkin) [left=2cm of step1, calproduct] {%
    MASTER\_DARK\_IFU
  };
  \draw (darkin) -- (step1);

  \node (flatin) [left=2cm of step1, yshift=-0.8cm, calproduct]{%
    MASTER\_FLAT\_IFU
  };
  \draw (flatin.east) -- ++(1., 0) -- ++(0., -0.6) -- ++(1., 0);

  % Input for rectification (step3)
  \node (wavecal) [left=2cm of step3, yshift=0.4cm, calproduct]{%
    IFU\_WAVECAL
  };
  \draw (wavecal.east) -- ++(1., 0) -- ++(0., 0.3) -- ++(1., 0);

  \node (distortion) [left=2cm of step3, yshift=-0.4cm, calproduct]{%
    IFU\_DISTORT\_TAB
  };
  \draw (distortion.east) -- ++(1., 0) -- ++(0., -0.3) -- ++(1., 0);

  % Further input
  \node (molecparams) [left=2cm of step5, params]{%
    molecfit parameters
  };
  \draw (molecparams) -- (step5);

  \node (stdcat) [left=2cm of step6, external] {%
    FLUXSTD\_CATALOG
  };
  \draw (stdcat.east) -- (step6);

  %% Output
  \node (connectreduced) [connection] at
  ($(step6)!0.25!(stop-t)$) {};
  \node (reduced) [right=of connectreduced, calproduct]{%
    IFU\_STD\_REDUCED
  };
  \draw (connectreduced) -- (reduced);

  \node (connectfluxcal) [connection] at
  ($(step6)!0.6!(stop-t)$) {};
  \node (fluxcal) [right=of connectfluxcal, calproduct]{%
    FLUXCAL\_TAB
  };
  \draw (connectfluxcal) -- (fluxcal);

  %% Frame around recipe
  \draw [frame] ($(input)!0.5!(step1) - (3.5,0)$)
  rectangle ($(step6)!0.75!(stop-t) + (2.5,0)$);
  \node [framecolor, anchor=north west] at
  ($(input)!0.5!(step1) - (3.5, 0)$) {%
    \textsl{metis\_ifu\_std\_process}};

\end{tikzpicture}

  \caption[Recipe: \REC*{metis_ifu_std_process}]{%
    \REC{metis_ifu_std_process} -- reduction of IFU flux standard
    frames and flux calibration (not all data products are shown).}
  \label{fig:metis_ifu_std_process}
\end{figure}

\clearpage
\subsubsection{\REC*{metis_ifu_sci_process}: IFU science reduction}
\label{sssec:ifu_sci_process}
\label{rec:metis_ifu_sci_process}

This recipe performs basic reduction of raw science exposures applying
dark and RSRF correction and flux calibration (i.e.~conversion of
pixel values to physical units) on each exposure individually. The
recipe will be able to process data from either the nominal or the
extended wavelength mode. For the nominal mode, all slices belong to
the same echelle order. For the extended mode, slices belonging to the
same echelle order are grouped and processing is iterated over the
echelle orders.

% from https://polarion.astron.nl/polarion/#/project/METIS/workitem?id=METIS-9141
For the LM band detectors in the IFU, the pixels on the edges of the detectors will be masked with the following widths~\cite{matisse_minutes} (\REQ{METIS-9141}) (Fig.~\ref{fig:ifu_detector_masking}):
\begin{itemize}
\item 64 columns on each ``outside'' of the 2x2 detector array in the dispersion direction
\item 0 columns on each ``inside'' of the 2x2 detector array in the dispersion direction
\item 32 rows at the top and bottom of each of the 4 detectors.
\end{itemize}
Note: We define rows and columns in the H2RG detector as follow:
\begin{itemize}
\item A row is readout by 32 outputs
\item A column is readout by 1 output
\end{itemize}

\begin{figure}[hb]
  \centering
  \includegraphics[width=0.7\textwidth]{LMS_detector_masking}
  \caption[The IFU masking scheme]{%
    The IFU masking scheme. Masked regions are in light blue.}
  \label{fig:ifu_detector_masking}
\end{figure}



The level of stray light is estimated in the dark areas between the
spectra and subtracted from the entire frame. The distribution of
stray light across the field can only be characterised once the
instrument is built. It is to be hoped that subtraction of a constant
or a low-level 2D polynomial fit will be sufficient.

The sky and thermal background, as well as residual straylight, is
estimated from blank sky observations if these are available in the
sequence of input frames or by combining (dithered) science
frames. The initial wavelength solution is taken from the daylight
calibration. It may be checked and corrected by measuring atmospheric
lines if a sufficient number is available in the limited wavelength
range.

A telluric correction is determined by this recipe by automatically
extracting a 1D spectrum from ``object'' pixels identified by a
thresholding algorithm. \lstinline{molecfit} is applied to this
spectrum and the correction is mapped back to the reduced 2D images or
3D cubes using the wavelength images. In an interactive environment
(Reflex workflow) the telluric correction may be improved by asking
the user to define an extraction aperture adapted to the target
structure.

The recipe produces the following (intermediate) data products:
\begin{itemize}\item Reduced 2D detector images. These are accompanied by additional
  information describing the geometry of the slice layout, target
  position and wavelength calibration to the extent that the exposure can be
  combined with other exposures into a single rectified spectral cube.
  This information can be stored in the FITS header or a table
  extension.
\item A rectified spectral cube for each exposure with a linear
  wavelength grid, constructed by resampling each spectral slice onto
  a spatial-wavelength grid common to all slices. The spatial pixels
  are rectangular with along-slit pixel scale given by the detector
  pixel scale and the across-slit pixel scale given by the slice
  width.
\item A spectral cube obtained by combining all exposures taken within
  a template. This step involves the image reconstruction discussed in
  Sect.~8.9 of~\cite{DRLS}.
% TODO: Decide on the below.
%  Whether this step is included
%  in the present recipe \REC{metis_ifu_sci_process} or is postponed to
%  the more general recipe \REC{metis_ifu_sci_postprocess} is TBD. It
%  may be formally required to do the image reconstruction here if
%  templates are set up to obtain a fixed set of spatially dithered and
%  rotated exposures aimed at reconstructing a fully sampled PSF in
%  both spatial dimensions.
\end{itemize}

For the nominal mode, each output is a single-extension FITS file
corresponding to one echelle order. For the extended mode, each of the
echelle orders results in an extension in a multi-extension FITS
file.

The recipe as described here is run in the science pipelines. For the
observatory pipeline, a variant of the recipe may be implemented with
reduced functionality and output. The observatory recipe may also have
to include features to determine QC parameters for the LM-band images
that are taken in parallel with the IFU exposures, similar to
 \REC{metis_lm_img_basic_reduce}.

\begin{recipedef}
Name:                & \REC{metis_ifu_sci_process}                                                              \\
Purpose:             & Reduction of individual science exposures.                                               \\
Requirements:        & \REQ{METIS-6131}, \REQ{METIS-6309}                                                       \\
Type:                & Science                                                                                  \\
Templates:           & \TPL{METIS_ifu_obs_FixedSkyOffset}                                                       \\
                     & \TPL{METIS_ifu_obs_GenericOffset}                                                        \\
                     & \TPL{METIS_ifu_ext_obs_FixedSkyOffset}                                                   \\
                     & \TPL{METIS_ifu_ext_obs_GenericOffset}                                                    \\
% TODO: Decide what to do about app
%                     & \TPL{METIS_ifu_app_obs_GenericOffset}                                                    \\
                     & \TPL{METIS_ifu_vc_obs_FixedSkyOffset}                                                    \\
%                     & \TPL{METIS_ifu_ext_app_obs_GenericOffset}                                                \\
                     & \TPL{METIS_ifu_ext_vc_obs_FixedSkyOffset}                                                \\
% HB 20230626: Not sure the *_obs_Stare templates should go here, but seems the most logical place.
                     & \TPL{METIS_ifu_app_obs_Stare}                                                            \\
                     & \TPL{METIS_ifu_ext_app_obs_Stare}                                                        \\
                     & \TPL{METIS_ifu_cal_psf}                                                                  \\
Input data:          & \RAW{IFU_SCI_RAW} (Dithered science exposures.) \\
                     & \RAW{IFU_SKY_RAW} (Blank sky images, if available.) \\
                     & \PROD{MASTER_DARK_IFU}                              \\
                       & \PROD{RSRF_IFU} (2D relative spectral response function)               \\
                       & \PROD{BADPIX_MAP_IFU}  \\
                       & \PROD{IFU_WAVECAL} \\
% TODO: I believe this should not be just FLUXCAL_TAB as it is not a table but an image
                     & \PROD{FLUXCAL_TAB} (Flux calibration table) \\
                       & \PROD{IFU_DISTORTION_TABLE} \\
                     & \STATCALIB{LSF_KERNEL} (Line spread kernel to be used with \CODE{molecfit}) \\
Parameters:          & telluric correction (yes/no)                                                             \\
%                     & more TBD                                                                                 \\
Algorithm:           & Subtract dark, divide by master flat                                                     \\
                     & Analyse and optionally remove masked regions and correct crosstalk and ghosts \\
                     & Estimate stray light and subtract                                                        \\
                     & Estimate background from dithered science exposures or blank-sky exposures and subtract. \\
                     & Apply flux calibration.                                                                  \\
                     & Rectify spectra and assemble cube                                                        \\
                     & Extract 1D object spectrum                                                               \\
                     & Compute telluric correction and apply to reduced images and cube                         \\
Output data:         & \PROD{IFU_SCI_REDUCED} (2D, per exposure)           \\
                     & \PROD{IFU_SCI_REDUCED_TAC} (2D, per exposure)   \\
                     & \PROD{IFU_SCI_BACKGROUND} (2D, per exposure)     \\
                     & \PROD{IFU_SCI_REDUCED_CUBE} (3D, per exposure) \\
                     & \PROD{IFU_SCI_REDUCED_CUBE_TAC} (3D, per exposure) \\
                     & \PROD{IFU_SCI_COMBINED} (3D)                       \\
                     & \PROD{IFU_SCI_COMBINED_TAC} (3D)               \\
                     & \PROD{IFU_SCI_OBJECT_1D}  (1D)                    \\
                     & \PROD{IFU_SCI_TELLURIC_1D}                      \\
Expected accuracies: & for wavelength: 1/5th of a pixel after post-processing\\
            & (cf.~\cite{METIS-calibration_plan}, \REQ{METIS-6074} \\
            & for flux: 10\% over an atmospheric band \\
            & $<30$\% absolute line flux accuracy\\
            & $<5$\% absolute flux calibration \\
            & (cf.~\cite{METIS-calibration_plan}, R-MET-107, R-MET-82)\\
QC1 parameters:      & None                                                                                     \\
\end{recipedef}

\begin{figure}[hb]
  \centering
    \def \globalscale {0.700000}
    \fontsize{10}{12}\selectfont
    %%% This file contains definitions of shapes and nodes used
%%% for a recipe workflow
%%% Author       : Oliver Czoske
%%% Created      : 2021-03-03
%%% Last Changed : 2021-03-03
%%% Changes:
%%%

\usetikzlibrary{
  shapes.misc,
  positioning,
  calc,
  arrows.meta}

%% All connecting lines have an arrow
\tikzset{
  every path/.style={->, >=Latex[open], thick}
}

%% Start and stop buttons (black disks, stop with ring)
%% These are pics, use as
%%         \pic (name) [above of=..] {picname};
\tikzset{
  start/.pic = {
    \node (-m) at (0, 0){};
    \filldraw [fill=black] (0, 0) circle (0.2);
  }
}

\tikzset{
  stop/.pic = {
    \node (-m) at (0, 0){};
    \node (-t) at (0, -0.3){};
    \filldraw [fill=black] (0, 0) circle(0.2);
    \draw[black] (0, 0) circle (0.3);
  }
}


%%%% Various boxes and their colours
%%%% These are nodes, use as
%%%% \node (name) [type, location]  {text};

\definecolor{stepcolor}{RGB}{210,169,188}
\definecolor{rawcolor}{RGB}{235,235,235}
\definecolor{externalcolor}{RGB}{183,255,255}
\definecolor{calibcolor}{RGB}{255,250,216}
\definecolor{calproductcolor}{RGB}{185,184,237}
\definecolor{qcproductcolor}{RGB}{255,201,165}
\definecolor{sciproductcolor}{RGB}{197,219,183}
\definecolor{framecolor}{RGB}{127,13,65}

\tikzset{
  %% template : the template(s) that trigger(s) the recipe
  template/.style={
    rectangle,
    draw=black,
    minimum width=4.0cm,
    minimum height=0.5cm,
    align=center
  },
  %% input : the input files
  input/.style={
    rectangle,
    fill=rawcolor,
    minimum width=4.0cm,
    minimum height=0.75cm,
    text width=3cm,
    align=center
  },
  %% calib : calibration input
  calib/.style={
    rectangle,
    fill=calibcolor,
    minimum width=4.0cm,
    minimum height=0.75cm,
    text width=3cm,
    align=center
  },
  %% external : external input
  external/.style={
    rectangle,
    fill=externalcolor,
    minimum width=4.0cm,
    minimum height=0.75cm,
    text width=3.5cm,
    align=center
  },
  %% params : parameters
  params/.style={
    rectangle,
    draw=red,
    thick,
    minimum width=4.0cm,
    minimum height=0.75cm,
    text width=3cm,
    align=center
  },
  %% redstep : a reduction step
  %%      ("step" is predefined and can't be used)
  redstep/.style={
    rectangle,
    rounded corners=0.2cm,
    fill=stepcolor,   %%% define colour!
    minimum width=4.0cm,
    minimum height=1cm,
    text width=3cm,
    align=center
  },
  %% connection : connection to input or output
  connection/.style={
    circle,
    fill=black,
    minimum size=0.15cm,
    inner sep=0pt
  },
  %% sciproduct : a science product
  sciproduct/.style={
    rectangle,
    fill=sciproductcolor,
    minimum width=4.0cm,
    minimum height=0.75cm,
    text width=3.5cm,
    align=center
  },
  %% calproduct : a calibration product
  calproduct/.style={
    rectangle,
    fill=calproductcolor,
    minimum width=4.0cm,
    minimum height=0.75cm,
    text width=3.5cm,
    align=center
  },
  %% frame : frame around the recipe
  %% This is a path, use as
  %%    \draw [frame] (upper left) rectangle (lower right);
  frame/.style={framecolor, very thick, dashed}
}


\begin{tikzpicture}
  [x=1cm,
  y=-1cm,
  align=center,
  node distance=2cm and 3cm]
  \sffamily

  %% Grid for orientation. Comment out for final figure!
  %\draw[help lines, green](-5, 0) grid (8, 21);

  %%% Put workflow commands here:
  %% Main reduction workflow

  \node (template) [template]{%
    METIS\_ifu\_obs\_$\ast$ \\
    METIS\_ifu\_ext\_obs\_$\ast$ \\
    METIS\_ifu\_$\langle$hci$\rangle$\_obs\_$\ast$ \\
    METIS\_ifu\_ext\_$\langle$hci$\rangle$\_obs\_$\ast$ \\
    METIS\_ifu\_cal\_psf
  };

  \pic (start) [below=0.75cm of template] {start};

  \node (input) [below=0.75cm of start-m, input]{%
    \textsl{N} IFU\_SCI\_RAW
  };

  \node (step1) [below=2cm of input, redstep]{%
    detector signature\\
    removal
  };

  \node (step2)[below=0.7cm of step1, redstep]{%
    background\\
    subtraction
  };

  \node (step3)[below=0.7cm of step2, redstep]{%
    flux calibration
  };

  \node (step4) [below=0.7cm of step3, redstep]{%
    rectification
  };

  \node (step5) [below=0.7cm of step4, redstep]{%
    Image\,reconstruction
  };

  \node (step6) [below=0.7cm of step5, redstep]{%
    1D extraction
  };

  \node (step7) [below=0.7cm of step6, redstep]{%
    telluric correction
  };

  \pic (stop) [below=1.5cm of step7]{stop};

  %% Connections
  \draw (template) -- (input);
  \draw (input) -- (step1);
  \draw (step1) -- (step2);
  \draw (step2) -- (step3);
  \draw (step3) -- (step4);
  \draw (step4) -- (step5);
  \draw (step5) -- (step6);
  \draw (step6) -- (step7);
  \draw (step7) -- (stop-t);

  %% Input
  \node (connectpers) [connection] at
  ($(input)!0.35!(step1)$) {};
  \node (persistence) [left=3.95cm of connectpers, external]{%
    PERSISTENCE\_MAP
  };
  \draw [-](persistence) -- (connectpers);

  % Input for detector signature removal (step1)
  \node (bpmin) [left=2cm of step1, yshift=0.8cm, calproduct]{%
    BADPIX\_MAP\_IFU
  };
  \draw (bpmin.east) -- ++(1., 0) -- ++(0., 0.6) -- ++(1., 0);

  \node (darkin) [left=2cm of step1, calproduct] {%
    MASTER\_DARK\_IFU
  };
  \draw (darkin) -- (step1);

  \node (flatin) [left=2cm of step1, yshift=-0.8cm, calproduct]{%
    MASTER\_FLAT\_IFU
  };
  \draw (flatin.east) -- ++(1., 0) -- ++(0., -0.6) -- ++(1., 0);

  % Input for flux calibration (step3)
  \node (fcalin) [left=2cm of step3, calproduct]{%
    FLUXCAL\_TAB
  };
  \draw (fcalin) -- (step3);

  % Input for rectification (step4)
  \node (wavecal) [left=2cm of step4, yshift=0.4cm, calproduct]{%
    IFU\_WAVECAL
  };
  \draw (wavecal.east) -- ++(1., 0) -- ++(0., 0.3) -- ++(1., 0);

  \node (distortion) [left=2cm of step4, yshift=-0.4cm, calproduct]{%
    IFU\_DISTORT\_TAB
  };
  \draw (distortion.east) -- ++(1., 0) -- ++(0., -0.3) -- ++(1., 0);

  % Further input
  \node (molecparams) [left=2cm of step7, params]{%
    molecfit parameters
  };
  \draw (molecparams) -- (step7);


  %% Output

  % Output of background subtraction (step2)
  \node (backgroundreduced) [right=1.cm of step2, calproduct]{%
    SCI\_BACKGROUND
  };
  \draw (step2) -- (backgroundreduced);

  % Output of flux calibration (step3)
  \node (sci_reduced) [right=1cm of step3, sciproduct]{%
    SCI\_REDUCED
  };
  \draw (step3) -- (sci_reduced);

  % Output of rectification (step4)
  \node (sci_reduced_cube) [right=1cm of step4, sciproduct]{%
    SCI\_REDUCED\_CUBE
  };
  \draw (step4) -- (sci_reduced_cube);

  % Output of image reconstruction (step5)
  \node (sci_combined) [right=1cm of step5, sciproduct]{%
    SCI\_COMBINED
  };
  \draw (step5) -- (sci_combined);

  % Output of 1d extraction (step6)
  \node (sci_object) [right=1cm of step6, sciproduct]{%
    SCI\_OBJECT
  };
  \draw (step6) -- (sci_object);

  % Output of telluric correction (step7)
  \node (sci_reduced_tac) [right=1cm of step7, sciproduct]{%
    SCI\_REDUCED\_TAC
  };
  \draw (step7) -- (sci_reduced_tac);

  %% Frame around recipe
  \draw [frame] ($(input)!0.5!(step1) - (3.5,0)$)
  rectangle ($(step7)!0.5!(stop-t) + (2.5,0)$);
  \node [framecolor, anchor=north west] at
  ($(input)!0.5!(step1) - (3.5, 0)$) {%
    \textsl{metis\_ifu\_sci\_process}};

\end{tikzpicture}

  \caption[Recipe: \REC*{metis_ifu_sci_process}]{%
    \REC{metis_ifu_sci_process} -- reduction of IFU science frames.}
  \label{fig:metis_ifu_sci_process}
\end{figure}


%------------------------------------------------------------------------------------------------------------------
\clearpage
\subsubsection{\REC*{metis_ifu_tellcorr}: IFU telluric absorption correction}
\label{sssec:ifu_tellcorr}
\label{rec:metis_ifu_tellcorr}

This recipe corrects for telluric absorption in a reduced IFU data
cube. The correction is done via a model atmospheric spectrum derived
with \CODE{molecfit}.

An automatic telluric correction can be performed as part of
\REC{metis_ifu_sci_process}. In an interactive environment it may be
better to do the telluric correction as a separate post-processing
step with a user-defined aperture for the extraction of a 1D object
spectrum. The spectrum is extracted from a combined cube
(\PROD{IFU_SCI_COMBINED}) but may be applied to other products of
\REC{metis_ifu_sci_process} specified in the input set of frames.

\begin{recipedef}
  Name:                & \REC{metis_ifu_tellcorr}                                                        \\
  Purpose:             & Remove telluric absorption features                                             \\
  Requirements:        & \REQ{METIS-6091}                                                                \\
  Type:                & Calibration / post processing                                                   \\
  Templates:           & ---                                                                             \\
  Input data:          & \PROD{IFU_SCI_COMBINED} -- reduced combined IFU cube                            \\
                       & \STATCALIB{LSF_KERNEL} -- Line spread kernel to be used with \CODE{molecfit}         \\
                       & \EXTCALIB{ATM_PROFILE} -- Atmospheric input profile to be used with \CODE{molecfit} \\
  Parameters:          & extraction aperture parameters                                                  \\
                       & \CODE{molecfit} parameters                                                      \\
                       & atmospheric profile incl.\ radiometer data                                      \\
                       & line spread kernel                                                              \\
  Algorithm:           & extract 1D spectrum                                                             \\
                       & Application of molecfit                                                         \\
  Output data:         & \PROD{IFU_SCI_REDUCED_TAC}                                                      \\
  Expected accuracies: & 2\%~\cite{METIS_calerrbudget}                                                   \\
  QC1 parameters:      & None                                                                            \\
\end{recipedef}

\begin{figure}[hb]
  \centering
    \def \globalscale {0.700000}
    \fontsize{10}{12}\selectfont
    
%%% This file contains definitions of shapes and nodes used
%%% for a recipe workflow
%%% Author       : Oliver Czoske
%%% Created      : 2021-03-03
%%% Last Changed : 2021-03-03
%%% Changes:
%%%

\usetikzlibrary{
  shapes.misc,
  positioning,
  calc,
  arrows.meta}

%% All connecting lines have an arrow
\tikzset{
  every path/.style={->, >=Latex[open], thick}
}

%% Start and stop buttons (black disks, stop with ring)
%% These are pics, use as
%%         \pic (name) [above of=..] {picname};
\tikzset{
  start/.pic = {
    \node (-m) at (0, 0){};
    \filldraw [fill=black] (0, 0) circle (0.2);
  }
}

\tikzset{
  stop/.pic = {
    \node (-m) at (0, 0){};
    \node (-t) at (0, -0.3){};
    \filldraw [fill=black] (0, 0) circle(0.2);
    \draw[black] (0, 0) circle (0.3);
  }
}


%%%% Various boxes and their colours
%%%% These are nodes, use as
%%%% \node (name) [type, location]  {text};

\definecolor{stepcolor}{RGB}{210,169,188}
\definecolor{rawcolor}{RGB}{235,235,235}
\definecolor{externalcolor}{RGB}{183,255,255}
\definecolor{calibcolor}{RGB}{255,250,216}
\definecolor{calproductcolor}{RGB}{185,184,237}
\definecolor{qcproductcolor}{RGB}{255,201,165}
\definecolor{sciproductcolor}{RGB}{197,219,183}
\definecolor{framecolor}{RGB}{127,13,65}

\tikzset{
  %% template : the template(s) that trigger(s) the recipe
  template/.style={
    rectangle,
    draw=black,
    minimum width=4.0cm,
    minimum height=0.5cm,
    align=center
  },
  %% input : the input files
  input/.style={
    rectangle,
    fill=rawcolor,
    minimum width=4.0cm,
    minimum height=0.75cm,
    text width=3cm,
    align=center
  },
  %% calib : calibration input
  calib/.style={
    rectangle,
    fill=calibcolor,
    minimum width=4.0cm,
    minimum height=0.75cm,
    text width=3cm,
    align=center
  },
  %% external : external input
  external/.style={
    rectangle,
    fill=externalcolor,
    minimum width=4.0cm,
    minimum height=0.75cm,
    text width=3.5cm,
    align=center
  },
  %% params : parameters
  params/.style={
    rectangle,
    draw=red,
    thick,
    minimum width=4.0cm,
    minimum height=0.75cm,
    text width=3cm,
    align=center
  },
  %% redstep : a reduction step
  %%      ("step" is predefined and can't be used)
  redstep/.style={
    rectangle,
    rounded corners=0.2cm,
    fill=stepcolor,   %%% define colour!
    minimum width=4.0cm,
    minimum height=1cm,
    text width=3cm,
    align=center
  },
  %% connection : connection to input or output
  connection/.style={
    circle,
    fill=black,
    minimum size=0.15cm,
    inner sep=0pt
  },
  %% sciproduct : a science product
  sciproduct/.style={
    rectangle,
    fill=sciproductcolor,
    minimum width=4.0cm,
    minimum height=0.75cm,
    text width=3.5cm,
    align=center
  },
  %% calproduct : a calibration product
  calproduct/.style={
    rectangle,
    fill=calproductcolor,
    minimum width=4.0cm,
    minimum height=0.75cm,
    text width=3.5cm,
    align=center
  },
  %% frame : frame around the recipe
  %% This is a path, use as
  %%    \draw [frame] (upper left) rectangle (lower right);
  frame/.style={framecolor, very thick, dashed}
}


\begin{tikzpicture}
  [x=1cm,
  y=-1cm,
  align=center,
  node distance=2cm and 3cm]
  \sffamily

  %% Grid for orientation. Comment out for final figure!
  % \draw[help lines, green](-5, 0) grid (8, 11);

  %%% Put workflow commands here:
  %% Main reduction workflow

  \pic (start) {start};

  \node (input) [below=0.75cm of start-m, input, fill=sciproductcolor]{%
    IFU\_SCI\_REDUCED
  };

  \node (step1) [below=2cm of input, redstep]{%
    Molecfit
  };

  \pic (stop) [below=2.cm of step1]{stop};

  %% Connections
  \draw (start-m) -- (input);
  \draw (input) -- (step1);
  \draw (step1) -- (stop-t);

  %% Input
  \node (params) [left=1.5cm of step1, yshift=1.cm, params]{%
    Molecfit params\\
    INS.IFU.SETUP
  };
  \draw (params.east) -- ++(1, 0) -- ++(0., 0.8) -- ++(0.5, 0);

  \node (lsfkernel) [left=1.5cm of step1, calib]{%
    IFU\_KERNEL
  };
  \draw (lsfkernel.east) -- (step1);

  \node (atmprofile) [left=1.5cm of step1, yshift=-1cm, external]{%
    ATM\_PROFILE
  };
  \draw (atmprofile.east) -- ++(1, 0) -- ++(0., -0.8) -- ++(0.5, 0);

  %% Output
  \node (connecttac) [connection] at
  ($(step1)!0.5!(stop-t)$) {};
  \node (reduced) [right=of connecttac, sciproduct, text width=4cm]{%
    IFU\_SCI\_REDUCED\_TAC
  };
  \draw (connecttac) -- (reduced);

  %% Frame around recipe
  \draw [frame] ($(input)!0.35!(step1) - (3,0)$)
  rectangle ($(step1)!0.75!(stop-t) + (2.5,0)$);
  \node [framecolor, anchor=north west] at
  ($(input)!0.35!(step1) - (3, 0)$) {%
    \textsl{metis\_ifu\_tellcorr}};

\end{tikzpicture}

  \caption[Recipe: \REC*{metis_ifu_tellcorr}]{\REC*{metis_ifu_tellcorr}
    -- telluric correction of reduced IFU science cubes.}
  \label{fig:metis_ifu_tellcorr}
\end{figure}


%------------------------------------------------------------------------------------------------------------------
\clearpage
\subsubsection{\REC*{metis_ifu_sci_postprocess}: IFU science postprocessing}
\label{sssec:ifu_sci_postprocess}
\label{rec:metis_ifu_sci_postprocess}

This recipe combines a number of reduced IFU exposures covering a
different spatial and wavelength ranges into a single data cube. The
positions and orientations of the exposures may differ as follows (cf.~\cite{METIS-operational_concept}): %\TODO{Reference to operational concept}
\begin{description}
\item[Spatial dithering:] The target is placed at different positions
  along and across the slice. Along-slice dithering aids in background
  subtraction, across-slice dithering is necessary image
  reconstruction given that the slice width undersamples the PSF\@.
\item[Field rotation:] The field is rotated by 90 degrees between
  exposures. The cube of a single exposure has different pixel scales
  along and across the slice. The goal of combining exposures at
  different rotation angles is to reconstruct images on a square grid
  with pixel scale given by the detector scale (8.2\,mas). The exact
  procedure remains to be investigated; one of the major challenges is
  to find the exact centre of rotation
  (Sect.~8.9 of~\cite{DRLS}).
\item[Spectral dithering:] Sequences of exposures are taken at various
  echelle angles in order to cover an increased contiguous wavelength
  range. In the extended mode, such a sequence may cover the
  wavelength gaps between echelle order coverage.
\end{description}

In order to allow co-addition of data from separate OBs, possibly taken
months apart, the wavelengths will be corrected to the heliocentric
reference system before co-addition.

The recipe is only used in the science-grade pipelines, not at the
observatory.

\begin{recipedef}
  Name:           & \REC{metis_ifu_sci_postprocess}  \\
  Purpose:        & Coaddition and mosaicing of reduced science cubes.                         \\
  Requirements:   & \REQ{METIS-6131}                                                           \\
  Type:           & Science                                                                    \\
  Templates:      & None                                                                       \\
  Input data:     & Reduced science cubes (\PROD{IFU_SCI_REDUCED}, \PROD{IFU_SCI_REDUCED_TAC}) \\
  Parameters:     & None                                                                       \\
  Algorithm:      & Call \DRL{ifu_grid_output} to find the output grid encompassing all input cubes \\
                  & Call \DRL{ifu_resampling} to resample input cubes to output grid   \\
                  & Call \DRL{ifu_coadd} to stack the images                    \\
  Output data:    & \PROD{IFU_SCI_COADD}                    \\
                  & \PROD{IFU_SCI_COADD_ERROR}        \\
  QC1 parameters: & ---                                                                        \\
\end{recipedef}

\begin{figure}[hb]
  \centering
    \def \globalscale {0.700000}
    \fontsize{10}{12}\selectfont
    \input{tikz/metis_ifu_sci_postprocess}
  \caption[Recipe: \REC*{metis_ifu_sci_postprocess}]{%
    \REC{metis_ifu_sci_postprocess} -- post-processing (coaddition) of
    reduced IFU science frames.}
  \label{fig:metis_ifu_sci_postprocess}
\end{figure}


%------------------------------------------------------------------------------------------------------------------
\clearpage
\subsubsection{\REC*{metis_ifu_distortion}: IFU distortion calibration}
\label{sssec:ifu_distortion}
\label{rec:metis_ifu_distortion}

Calibration of the geometric distortion of the IFU is done by
observing a pin hole mask located in a focal plane within the
instrument. The distortion is described in terms of a polynomial model
whose coefficients can be used to map positions in in the detector
array to sky positions. Measurement of the FWHM of the spots gives an
indication of the variation of spectral resolution across the field of view.

\begin{recipedef}
  Name:                & \REC{metis_ifu_distortion}                                                  \\
  Purpose:             & Determine geometric distortion coefficients for the IFU.                    \\
  Requirements:        & \REQ{METIS-6087}, \REQ{METIS-6073}                                          \\
  Type:                & Calibration                                                                 \\
  Templates:           & \TPL{METIS_ifu_cal_distortion}                                              \\
  Input data:          & \RAW{IFU_DISTORTION_RAW} (Images of multi-pinhole mask.) \\
  Parameters:          & None                                                                        \\
  Algorithm:           & Calculate table mapping pixel position to position on sky.                  \\
  Output data:         & \PROD{IFU_DISTORTION_TABLE}                                                 \\
                       & \PROD{IFU_DIST_REDUCED}                                                     \\
Expected accuracies: & 1/10th of a pixel after post-processing\\
               & (cf.~\cite{METIS-calibration_plan}, R-MET-106, \REQ{METIS-167}, \REQ{METIS-1371})\\
  QC1 parameters:      & \QC*{QC IFU DISTORT RMS}: RMS deviation between measured position and model \\
                       & \QC*{QC IFU DISTORT FWHM}:   Measured FWHM of spots                            \\
                       & \QC*{QC IFU DISTORT NSPOTS}: Number of identified spots                        \\
\end{recipedef}

\begin{figure}[hb]
  \centering
    \def \globalscale {0.700000}
    \fontsize{10}{12}\selectfont
    
% ADDING NEW DEFINITIONS -------------------------------------------- start
\definecolor{listingbg}{gray}{0.95}
\definecolor{darkgreen}{rgb}{0.0, 0.7, 0.0}
\definecolor{darkblue} {rgb}{0.0, 0.0, 0.7}
\definecolor{cyan} {rgb}{0.0, 0.4, 0.4}
\definecolor{darkred}  {rgb}{0.7, 0.0, 0.0}
\definecolor{darkorange}{rgb}{1.0, 0.49, 0.0}
\definecolor{violett}{rgb}{255, 0, 255}
\definecolor{turq}{rgb}{0.0, 0.7, 0.8}
\definecolor{fits}{rgb}{0.4, 0.1, 1}


\makeatletter
\lstdefinestyle{RAWstyle}{%
  basicstyle=\ttfamily\color{black}%
  \lst@ifdisplaystyle\scriptsize\fi}

\lstdefinestyle{PARstyle}{%
  basicstyle=\ttfamily\color{black}%
  \lst@ifdisplaystyle\scriptsize\fi}

\lstdefinestyle{DRLstyle}{%
  basicstyle=\ttfamily\color{black}%
  \lst@ifdisplaystyle\scriptsize\fi}

\lstdefinestyle{RECstyle}{%
  basicstyle=\ttfamily\color{black}%
  \lst@ifdisplaystyle\scriptsize\fi}

\lstdefinestyle{QCstyle}{%
  basicstyle=\ttfamily\color{black}%
  \lst@ifdisplaystyle\scriptsize\fi}

\lstdefinestyle{TPLstyle}{%
  basicstyle=\ttfamily\color{black}%
  \lst@ifdisplaystyle\scriptsize\fi}

\lstdefinestyle{PRODstyle}{%
  basicstyle=\ttfamily\color{black}%
  \lst@ifdisplaystyle\scriptsize\fi}

\lstdefinestyle{EXTCALIBstyle}{%
  basicstyle=\ttfamily\color{black}%
  \lst@ifdisplaystyle\scriptsize\fi}

\lstdefinestyle{STATCALIBstyle}{%
  basicstyle=\ttfamily\color{black}%
  \lst@ifdisplaystyle\scriptsize\fi}
\makeatother

%%% This file contains definitions of shapes and nodes used
%%% for a recipe workflow
%%% Author       : Oliver Czoske
%%% Created      : 2021-03-03
%%% Last Changed : 2021-03-03
%%% Changes:
%%%

\usetikzlibrary{
  shapes.misc,
  positioning,
  calc,
  arrows.meta}

%% All connecting lines have an arrow
\tikzset{
  every path/.style={->, >=Latex[open], thick}
}

%% Start and stop buttons (black disks, stop with ring)
%% These are pics, use as
%%         \pic (name) [above of=..] {picname};
\tikzset{
  start/.pic = {
    \node (-m) at (0, 0){};
    \filldraw [fill=black] (0, 0) circle (0.2);
  }
}

\tikzset{
  stop/.pic = {
    \node (-m) at (0, 0){};
    \node (-t) at (0, -0.3){};
    \filldraw [fill=black] (0, 0) circle(0.2);
    \draw[black] (0, 0) circle (0.3);
  }
}


%%%% Various boxes and their colours
%%%% These are nodes, use as
%%%% \node (name) [type, location]  {text};

\definecolor{stepcolor}{RGB}{210,169,188}
\definecolor{rawcolor}{RGB}{235,235,235}
\definecolor{externalcolor}{RGB}{183,255,255}
\definecolor{calibcolor}{RGB}{255,250,216}
\definecolor{calproductcolor}{RGB}{185,184,237}
\definecolor{qcproductcolor}{RGB}{255,201,165}
\definecolor{sciproductcolor}{RGB}{197,219,183}
\definecolor{framecolor}{RGB}{127,13,65}

\tikzset{
  %% template : the template(s) that trigger(s) the recipe
  template/.style={
    rectangle,
    draw=black,
    minimum width=4.0cm,
    minimum height=0.5cm,
    align=center
  },
  %% input : the input files
  input/.style={
    rectangle,
    fill=rawcolor,
    minimum width=4.0cm,
    minimum height=0.75cm,
    text width=3cm,
    align=center
  },
  %% calib : calibration input
  calib/.style={
    rectangle,
    fill=calibcolor,
    minimum width=4.0cm,
    minimum height=0.75cm,
    text width=3cm,
    align=center
  },
  %% external : external input
  external/.style={
    rectangle,
    fill=externalcolor,
    minimum width=4.0cm,
    minimum height=0.75cm,
    text width=3.5cm,
    align=center
  },
  %% params : parameters
  params/.style={
    rectangle,
    draw=red,
    thick,
    minimum width=4.0cm,
    minimum height=0.75cm,
    text width=3cm,
    align=center
  },
  %% redstep : a reduction step
  %%      ("step" is predefined and can't be used)
  redstep/.style={
    rectangle,
    rounded corners=0.2cm,
    fill=stepcolor,   %%% define colour!
    minimum width=4.0cm,
    minimum height=1cm,
    text width=3cm,
    align=center
  },
  %% connection : connection to input or output
  connection/.style={
    circle,
    fill=black,
    minimum size=0.15cm,
    inner sep=0pt
  },
  %% sciproduct : a science product
  sciproduct/.style={
    rectangle,
    fill=sciproductcolor,
    minimum width=4.0cm,
    minimum height=0.75cm,
    text width=3.5cm,
    align=center
  },
  %% calproduct : a calibration product
  calproduct/.style={
    rectangle,
    fill=calproductcolor,
    minimum width=4.0cm,
    minimum height=0.75cm,
    text width=3.5cm,
    align=center
  },
  %% frame : frame around the recipe
  %% This is a path, use as
  %%    \draw [frame] (upper left) rectangle (lower right);
  frame/.style={framecolor, very thick, dashed}
}


\begin{tikzpicture}
  [x=1cm,
  y=-1cm,
  align=center,
  node distance=2cm and 3cm]
  \sffamily

  %% Grid for orientation. Comment out for final figure!
  % \draw[help lines, green](-5, 0) grid (8, 11);

  %%% Put workflow commands here:
  %% Main reduction workflow

  \node (template) [template]{%
    \TPL{METIS_ifu_cal_distortion}
  };

  \pic (start) [below=0.75cm of template] {start};

  \node (input) [below=0.75cm of start-m, input]{\RAW{IFU_DISTORTION_RAW}};

%  \node (step1) [below=1.5cm of input, redstep]{%
%    subtract DIST\_OFF
%  };

  \node (step_signature) [below=6.cm of input, redstep]{detector signature\\ removal};
  \node (step2) [below=4.cm of step_signature, redstep]{locate images};
  \node (step3) [below=1.cm of step2, redstep]{fit polynomial};

  \pic (stop) [below=2.5cm of step3]{stop};

  %% Connections
  \draw [connection_arrow] (template) -- (input);
  \draw [connection_arrow] (input) -- (step_signature);
  \draw [connection_arrow] (step_signature) -- (step2);
  \draw [connection_arrow] (step2) -- (step3);
  \draw [connection_arrow] (step3) -- (stop-t);

  %% Input
%  \node (bpmin) [left=2cm of step2, yshift=0, calproduct]{%
%    BADPIX\_MAP\_IFU
%  };
%  \draw [connection_arrow] (bpmin) -- (step2);

  \node (connect_bpm) [connection] at ($(input)!0.20!(step_signature)$) {};
  \node (bpm) [left=of connect_bpm, external] {\EXTCALIB{BADPIX_MAP_IFU}};
  \draw [connection_arrow, dashed] (bpm) -- (connect_bpm);

  \node (connect_gain) [connection] at ($(input)!0.35!(step_signature)$) {};
  \node (gain) [left=of connect_gain, external] {\EXTCALIB{GAIN_MAP_IFU}};
  \draw [connection_arrow] (gain) -- (connect_gain);

  \node (connect_linearity) [connection] at ($(input)!0.50!(step_signature)$){};
  \node (linearity) [left=of connect_linearity, external]{\EXTCALIB{LINEARITY_IFU}};
  \draw [connection_arrow] (linearity) -- (connect_linearity);

  \node (connect_persistence) [connection] at ($(input)!0.65!(step_signature)$){};
  \node (persistence) [left=of connect_persistence, external]{\EXTCALIB{PERSISTENCE_MAP}};
  \draw [connection_arrow] (persistence) -- (connect_persistence);

  \node (connect_dark) [connection] at ($(input)!0.8!(step_signature)$) {};
  \node (dark) [left=of connect_dark, calproduct] {\PROD{MASTER_DARK_IFU}};
  \draw [connection_arrow] (dark) -- (connect_dark);

  \node (connect_pinhole) [connection] at ($(step_signature)!0.5!(step2)$) {};
  \node (pinhole) [left=of connect_pinhole, external] {\EXTCALIB{PINHOLE_TABLE}};
  \draw [connection_arrow] (pinhole) -- (connect_pinhole);


  %% Output
  \node (connect dist table) [connection] at
  ($(step3)!0.4!(stop-t)$) {};
  \node (dist table) [right=of connect dist table, calproduct]{\PROD{IFU_DISTORTION_TABLE}};
  \draw [connection_arrow] (connect dist table) -- (dist table);

  \node (connect dist reduced) [connection] at
  ($(step3)!0.75!(stop-t)$) {};
  \node (dist reduced) [right=of connect dist reduced, calproduct]{%
    \PROD{IFU_DIST_REDUCED}
  };
  \draw [connection_arrow] (connect dist reduced) -- (dist reduced);

  %% Frame around recipe
  \draw [frame] ($(input)!0.1!(step_signature) - (3.,0)$)
  rectangle ($(step3)!0.275!(stop-t) + (2.75,0)$);
  \node [framecolor, anchor=north west] at
  ($(input)!0.1!(step_signature) - (3., 0)$) {%
    \REC{metis_ifu_distortion}};

\end{tikzpicture}

% ADDING NEW DEFINITIONS -------------------------------------------- start
\definecolor{listingbg}{gray}{0.95}
\definecolor{darkgreen}{rgb}{0.0, 0.7, 0.0}
\definecolor{darkblue} {rgb}{0.0, 0.0, 0.7}
\definecolor{cyan} {rgb}{0.0, 0.4, 0.4}
\definecolor{darkred}  {rgb}{0.7, 0.0, 0.0}
\definecolor{darkorange}{rgb}{1.0, 0.49, 0.0}
\definecolor{violet}{rgb}{255, 0, 255}
\definecolor{turq}{rgb}{0.0, 0.7, 0.8}
\definecolor{fits}{rgb}{0.4, 0.1, 1}


\makeatletter
\lstdefinestyle{RAWstyle}{%
  basicstyle=\ttfamily\color{fits}%
  \lst@ifdisplaystyle\scriptsize\fi}

\lstdefinestyle{PARstyle}{%
  basicstyle=\ttfamily\color{cyan}%
  \lst@ifdisplaystyle\scriptsize\fi}

\lstdefinestyle{DRLstyle}{%
  basicstyle=\ttfamily\color{violet}%
  \lst@ifdisplaystyle\scriptsize\fi}

\lstdefinestyle{RECstyle}{%
  basicstyle=\ttfamily\color{darkgreen}%
  \lst@ifdisplaystyle\scriptsize\fi}

%% Write QC parameters like this: \QC*{QC_SOMETHING_OR_OTHER}
\lstdefinestyle{QCstyle}{%
  basicstyle=\ttfamily\color{darkblue}%
  \lst@ifdisplaystyle\scriptsize\fi}

%% Write templates like this: \TPL{DARK_LM}
\lstdefinestyle{TPLstyle}{%
  basicstyle=\ttfamily\color{darkred}%
  \lst@ifdisplaystyle\scriptsize\fi}

%% Write products like this: \PROD{SOME_THING}
\lstdefinestyle{PRODstyle}{%
  basicstyle=\ttfamily\color{darkorange}%
  \lst@ifdisplaystyle\scriptsize\fi}

%% external calib files
\lstdefinestyle{EXTCALIBstyle}{%
  basicstyle=\ttfamily\color{Turquoise}%
  \lst@ifdisplaystyle\scriptsize\fi}

% static calib files
\lstdefinestyle{STATCALIBstyle}{%
  basicstyle=\ttfamily\color{teal}%
  \lst@ifdisplaystyle\scriptsize\fi}

% static calib files
\lstdefinestyle{FITSstyle}{%
  basicstyle=\ttfamily\color{black}%
  \lst@ifdisplaystyle\scriptsize\fi}
\makeatother


  \caption[Recipe: \REC*{metis_ifu_distortion}]{%
    \REC{metis_ifu_distortion} -- IFU distortion calibration}
  \label{fig:metis_ifu_distortion}
\end{figure}


\clearpage



%%%%%%%%%%%%%%%%%%%%%%%%%%%%%%%%%%%%%%%%%%%%%%%%%%%%%%%%%%%%%%%%%%%%%%%%%%%%%%%%

%%% Local Variables:
%%% TeX-master: "METIS_DRLD"
%%% End:
