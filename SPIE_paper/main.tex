\documentclass[]{spie}  %>>> use for US letter paper
%\documentclass[a4paper]{spie}  %>>> use this instead for A4 paper
%\documentclass[nocompress]{spie}  %>>> to avoid compression of citations

\renewcommand{\baselinestretch}{1.0} % Change to 1.65 for double spacing
 
\usepackage{amsmath,amsfonts,amssymb}
\usepackage{graphicx}
\usepackage[colorlinks=true, allcolors=blue]{hyperref}

\title{Implementation plans for the data reduction pipeline for METIS at the ELT }

\author[a]{Kieran Leschinski}
\author[a]{Hugo Buddelmeijer}
\author[a]{Oliver Czoske}
\author[a]{Martin Balaz}
\author[a]{Fabian Haberhauer}
\author[a]{Jennifer Karr}
\author[a]{Gilles Otten}
\author[a]{Chi-Hung Yan}
\author[a]{Wolfgang Kausch}
\author[a]{Nadeen Sabha}
\author[a]{Thomas Marquart}
\author[a]{Werner Zeilinger}
\author[a]{Norbert Pryzbilla}
\author[a]{Shiang-Yu Wang}

\affil[a]{University of Vienna, T\"urkenschanztra\ss e 18, 1180 Vienna, Austria}

\authorinfo{Further author information contact Kieran Leschinski - E-mail: kieran.leschinski@univie.ac.at}

% Option to view page numbers
\pagestyle{empty} % change to \pagestyle{plain} for page numbers   
\setcounter{page}{301} % Set start page numbering at e.g. 301
 
\begin{document} 
\maketitle

\begin{abstract}

\end{abstract}

% Include a list of keywords after the abstract 
\keywords{Manuscript format, template, SPIE Proceedings, LaTeX}


\section{Introduction}
\label{sec:introduction}


\section{METIS and the pipeline}
\label{sec:environment}

\subsection{Description of METIS}
\label{ssec:env_metis}
	- MIR
	- SCAO
\subsection{Observation Modes}
\label{ssec:env_modes}
	- IFU
	- IMG
	- LSS
	- HCI
 
\subsection{What challenges does this pose for the pipeline}
\label{ssec:env_challenges}

- IFU undersampled pixels due to slit-width of IFU pseudo-slits
		- IFU reconstruction on rectangular pixels
	- MIR Background subtraction
	- Fixed ADCs
	- Multiple HCI modes
	- Geosnap cosmetics
	- Highly distriubuted team


\section{Implementation plans}
\label{sec:implementation}

\subsection{ESO pipeline environment}
\label{ssec:imp_eso}

	- Legacy C code: CPL, HDRL
	- ESO's shift to Python: EDPS, PyCPL
	- Integration with qcFlow

\subsection{Pipeline design}
\label{ssec:imp_pip}

	- DRL-D + DRL-VT, PIP-FDR success
	- Association matrices
	- Pipeline to be fully developed in python 
	- Integration into EDPS
 
\subsection{Connection to the METIS AIV archive}
\label{ssec:imp_archive}


\section{Development scheduling and tooling}
\label{sec:development}

\subsection{Scheduling matched to AIV plans}
\label{ssec:dev_aiv}

\begin{table}[]
    \centering
\caption{Caption}
\label{tab:dev_aiv_milestones}
    \begin{tabular}{c|cl}
         Date&   Internal AIV phase ID&Description METIS AIV phase\\
         02-2025&   AIV-010&PIP workstation setup in Leiden\\
         11-2025&   AIV-250&Dress-rehearsal of data flow chain\\
 02-2026& &IFU system testing\\
 &  &\\
 & &\\
 & &\\
 & &\\
 & &\\
 & &\\
 \end{tabular}
   
   
\end{table}


 - Major METIS AIV milestones for which recipes need to be ready
		-  :  : 
		-  :  : 
		-  : AIV-370 : 
		- 10-2026 : AIV-490 : IMG/LSS system testing
		- 12-2027 : AIV-640 : PAE
		- 12-2028 : AIV-920 : Commissioning starts
\subsection{Functionality levels}
\label{ssec:dev_levels}
	- Skel
	- Func
	- Perf
	- Sci
\subsection{Integration with data simulator}
\label{ssec:dev_scopesim}
	- Part of the new ESO strategy (1618 v4) to have coherent simulated data sets from the consortia before the pipeline 
	- ScopeSim with METIS
	- Wrapper for generate mock data sets
	- Efforts to 
\subsection{Official release schedule}
\label{ssec:dev_releases}
	- Table of release dates
\subsection{Github toolchain}
\label{ssec:dev_testing}
	- CI, testing on multiple systems
	- Issue tracking
	- Project management and defining release goals

\section{Conclusion}
\label{sec:conclusion}






\appendix    %>>>> this command starts appendixes

\acknowledgments % equivalent to \section*{ACKNOWLEDGMENTS}       

% References
\bibliography{report} % bibliography data in report.bib
\bibliographystyle{spiebib} % makes bibtex use spiebib.bst

\end{document} 
