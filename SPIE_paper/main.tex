\documentclass[a4paper]{spie}  %>>> use for US letter paper
%\documentclass[a4paper]{spie}  %>>> use this instead for A4 paper
%\documentclass[nocompress]{spie}  %>>> to avoid compression of citations

\renewcommand{\baselinestretch}{1.0} % Change to 1.65 for double spacing
 
\usepackage{amsmath,amsfonts,amssymb}
\usepackage{graphicx}
\usepackage{todonotes}
\usepackage{lipsum}
\usepackage[colorlinks=true, allcolors=blue]{hyperref}

\title{Implementation plans for the data reduction pipeline for METIS at the ELT as part of a greater ELT data analysis network}

\author[a]{Kieran Leschinski}
\author[a]{Hugo Buddelmeijer}
\author[a]{Oliver Czoske}
\author[a]{Martin Balaz}
\author[a]{Fabian Haberhauer}
\author[a]{Jennifer Karr}
\author[a]{Gilles Otten}
\author[a]{Chi-Hung Yan}
\author[a]{Wolfgang Kausch}
\author[a]{Nadeen Sabha}
\author[a]{Thomas Marquart}
\author[a]{Werner Zeilinger}
\author[a]{Norbert Pryzbilla}
\author[a]{Shiang-Yu Wang}

\affil[a]{University of Vienna, T\"urkenschanztra\ss e 18, 1180 Vienna, Austria}

\authorinfo{Further author information contact Kieran Leschinski - E-mail: kieran.leschinski@univie.ac.at}

% Option to view page numbers
\pagestyle{empty} % change to \pagestyle{plain} for page numbers   
\setcounter{page}{301} % Set start page numbering at e.g. 301
 
\begin{document} 
\maketitle

\begin{abstract}


\end{abstract}

% Include a list of keywords after the abstract 
\keywords{Manuscript format, template, SPIE Proceedings, LaTeX}


\section{Introduction}
% ELT and METIS overview
ELT will be the largest optical telescope ever built, and optimaised for infrared wavelength.
It will a collecting area of almost 1000m2 with its 39m mirror 
From the outset adaptive optics were built into the operational concept, with deformable and tip-tilt mirrors part of the telescope's 5 mirror design.
At mid-infrared wavelengths this will allow for high-strehl (>90\%) observations at the diffraction limit of the telescope. 
E.g. FWHM of 20mas at 3µm and 100mas at 15µm.
This will open up 



- TLRs of the pipeline
- Role of the pipeline

\section{Environmental constraints for the pipeline}





- METIS TLR for performance dictate the level to which the METIS pipeline must process data
- Pipeline is to also support AIV activities
- Pipeline will need to be able to have functionality available at different levels through out AIV to enable testing
- AIV team will use pipeline to monitor performance of METIS during the integration phase
- ESO is expecting METIS to be on-sky for decades, pipeline must work for this length too
- ESO is rolling out new pipeline tool chains in python: EDPS, PyCPL. This affects the implementation efforts
- PIP is a very distributed team

\section{Need for integrated development and testing architecture}
- Pipeline as a single element in an integrated software ecosystem
- AIV data archive for storing AIV data and for monitoring the system integration of the METIS instrument
- AIV system integration tests monitor both functionality and performance of METIS
- Heavy use of instrument data simulator (ScopeSim) for pre-testing use cases in advance of AIV needs
- Pipeline as a way of connecting predictions (simulations) to reality


\section{Pipeline implementation plans}

\subsection{Scheduling matched to AIV plans}
Early on the 



\subsection{Functionality levels}



\subsection{Integration with data simulator}





\subsection{Development tools}
- Pipeline to be fully developed in python, using EDPS and PyCPL frameworks
- Utilising Github toolchain

\subsection{Release schedule}


\section{Pipeline topology}
- Association matrices
- See DRL-D + DRL-VT, PIP-FDR success




\section{Outlook until PAE}
- Issues on the horizon
- Upcoming milestones


\appendix    %>>>> this command starts appendixes

\acknowledgments % equivalent to \section*{ACKNOWLEDGMENTS}       


% References
\bibliography{report} % bibliography data in report.bib
\bibliographystyle{spiebib} % makes bibtex use spiebib.bst

\end{document} 
