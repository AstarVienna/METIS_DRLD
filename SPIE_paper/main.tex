\documentclass[a4paper]{spie}  %>>> use for US letter paper
%\documentclass[a4paper]{spie}  %>>> use this instead for A4 paper
%\documentclass[nocompress]{spie}  %>>> to avoid compression of citations

\renewcommand{\baselinestretch}{1.0} % Change to 1.65 for double spacing
 
\usepackage{amsmath,amsfonts,amssymb}
\usepackage{graphicx}
\usepackage{todonotes}
\usepackage{lipsum}
\usepackage[colorlinks=true, allcolors=blue]{hyperref}

\title{Implementation plans for the data reduction pipeline for METIS at the ELT }

\author[a]{Kieran Leschinski}
\author[a]{Hugo Buddelmeijer}
\author[a]{Oliver Czoske}
\author[a]{Martin Balaz}
\author[a]{Fabian Haberhauer}
\author[a]{Jennifer Karr}
\author[a]{Gilles Otten}
\author[a]{Chi-Hung Yan}
\author[a]{Wolfgang Kausch}
\author[a]{Nadeen Sabha}
\author[a]{Thomas Marquart}
\author[a]{Werner Zeilinger}
\author[a]{Norbert Pryzbilla}
\author[a]{Shiang-Yu Wang}

\affil[a]{University of Vienna, T\"urkenschanztra\ss e 18, 1180 Vienna, Austria}

\authorinfo{Further author information contact Kieran Leschinski - E-mail: kieran.leschinski@univie.ac.at}

% Option to view page numbers
\pagestyle{empty} % change to \pagestyle{plain} for page numbers   
\setcounter{page}{301} % Set start page numbering at e.g. 301
 
\begin{document} 
\maketitle

\begin{abstract}
The METIS data reduction software is designed to transform raw data from the instrument's diverse observing modes into science-ready data, facilitating seamless analysis for astronomers. This process will be executed automatically on the ESO servers using the software, ensuring efficient and consistent data processing. Additionally, the pipeline software will be made available to the broader astronomical community, allowing for widespread use and collaboration. The development of this software is closely integrated with the Assembly, Integration, and Verification (AIV) activities, providing valuable feedback to both the instrument builders and the pipeline development team. This iterative process ensures that both the instrument and its software are finely tuned well before METIS begins observations with the ELT in 2028. 

\end{abstract}

% Include a list of keywords after the abstract 
\keywords{Manuscript format, template, SPIE Proceedings, LaTeX}


\section{Introduction}
\label{sec:introduction}

% - METIS will be the first generation mid infrared instrument on the ELT
% - It will utilise both the immense light collecting ability of the ELT as well as the  unparalleled diffraction limited resolution of the ELTs 39m mirror to look at both intra and extra galactic sources. 


% - it will contain a suite of observation modes including 
%    - medium resolution long-slit spectroscopy
%    - a high-resolution IFU
%    - imaging cameras in the LM and N bands
%    - various HCI modes using different types of coronagraphs
%  - METIS will 

The Mid-infrared ELT Imager and Spectrograph (METIS) is an advanced instrument designed for the Extremely Large Telescope (ELT), equipped with a variety of observing modes to explore the universe in unprecedented detail. METIS can perform direct imaging to capture clear pictures of celestial objects, high-resolution spectroscopy to analyze the composition and physical properties of those objects, and coronagraphy to block out starlight and reveal faint nearby features. This combination of capabilities allows METIS to offer a comprehensive toolkit for astronomers studying the mid-infrared spectrum.

METIS is being built to address several key scientific questions in modern astronomy. One of its primary science cases is the study of exoplanets and their atmospheres, which can provide insights into the potential for life beyond Earth. Additionally, METIS will investigate the formation and evolution of protoplanetary disks, the birthplaces of planets, and will observe the environments around supermassive black holes in the centers of galaxies. Its ability to capture detailed mid-infrared data will also enhance our understanding of the interstellar medium and the processes that lead to star formation.

The METIS instrument is expected to begin its observations in the late 2020s, soon after the ELT becomes operational. This timeline positions METIS to be at the forefront of astronomical discovery, leveraging the unprecedented capabilities of the ELT to push the boundaries of our knowledge about the universe. As the ELT and its suite of instruments come online, METIS will play a crucial role in unlocking new cosmic secrets and furthering our understanding of the cosmos.

The instrument's operational details and implementation plans are meticulously outlined in the accompanying paper. Section 2 introduces the data reduction pipeline, a critical component that processes raw data into scientifically usable information. Section 3 discusses the implementation plans, including the integration of METIS into the ESO pipeline environment and the design of the pipeline itself, ensuring seamless operation and data handling. Section 4 presents the development schedule, detailing how the timeline for METIS's completion aligns with Assembly, Integration, and Verification (AIV) activities occurring simultaneously. This coordinated approach ensures that METIS will be ready to begin its groundbreaking observations soon after the ELT becomes operational in the late 2020s. 



\section{METIS and the pipeline}
\label{sec:environment}

\subsection{Description of METIS}
\label{ssec:env_metis}
	- MIR
	- SCAO
\subsection{Observation Modes}
\label{ssec:env_modes}
 \begin{itemize}
     \item - IFU
     \item - IMG
     \item - LSS
     \item - HCI
 \end{itemize}
 
\subsection{What challenges does this pose for the pipeline}
\label{ssec:env_challenges}

\begin{itemize}
    \item - IFU undersampled pixels due to slit-width of IFU pseudo-slits
    \item - IFU reconstruction on rectangular pixels
    \item - MIR Background subtraction
    \item - Fixed ADCs
    \item - Multiple HCI modes
    \item - Geosnap cosmetics
    \item - Highly distriubuted team

\end{itemize}

\section{Implementation plans}
\label{sec:implementation}

\subsection{ESO pipeline environment}
\label{ssec:imp_eso}

 \begin{itemize}
     \item - Legacy C code: CPL, HDRL
     \item - ESO's shift to Python: EDPS, PyCPL
     \item - Integration with qcFlow
 \end{itemize}

\subsection{Pipeline design}
\label{ssec:imp_pip}

 \begin{itemize}
     \item - DRL-D + DRL-VT, PIP-FDR success
     \item - Association matrices
     \item - Pipeline to be fully developed in python
     \item - Integration into EDPS
 \end{itemize}
 
\subsection{Connection to the METIS AIV archive}
\label{ssec:imp_archive}

\missingfigure{Add figure from GIlles slide show}


\section{Development scheduling and tooling}
\label{sec:development}

\subsection{Scheduling matched to AIV plans}
\label{ssec:dev_aiv}

\begin{table}[]
    \centering
\caption{Major METIS AIV milestones for which pipeline recipes need to be ready}
\label{tab:dev_aiv_milestones}
    \begin{tabular}{c|c|l}
    \hline
    \hline
         Date &   METIS AIV phase ID & Description\\
         \hline
         02-2025 &   AIV-010&PIP workstation setup in Leiden\\
         11-2025 &   AIV-250&Dress-rehearsal of data flow chain\\
 02-2026 & AIV-370&IFU system testing\\
 10-2026 &  AIV-490&IMG/LSS system testing\\
 12-2027 & AIV-640&PAE\\
 12-2028 & AIV-920&Commissioning on-sky at ELT starts\\
 \hline
 \end{tabular}
      
\end{table}

\begin{itemize}
    \item AIV-010
    \item AIV-250
    \item AIV-370
    \item AIV-490
    \item AIV-640
    \item AIV-920
\end{itemize}



\subsection{Functionality levels}
\label{ssec:dev_levels}
  
 \begin{enumerate}
     \item Skeleton
     \item Functional
     \item Performance
     \item Science
 \end{enumerate}

\subsection{Integration with data simulator}
\label{ssec:dev_scopesim}

 \begin{itemize}
     \item - Part of the new ESO strategy (1618 v4) to have coherent simulated data sets from the consortia before the pipeline
     \item - ScopeSim with METIS
     \item - Wrapper for generate mock data sets
     \item - Efforts to integrate feasibility studies from the science team into the pixel in the test data set
 \end{itemize}

 
\subsection{Official release schedule}
\label{ssec:dev_releases}

\missingfigure{Add figure showing gantt chart of releases}


\subsection{Github toolchain}
\label{ssec:dev_testing}
	- CI, testing on multiple systems
	- Issue tracking
	- Project management and defining release goals

\section{Conclusion}
\label{sec:conclusion}



\appendix    %>>>> this command starts appendixes

\acknowledgments % equivalent to \section*{ACKNOWLEDGMENTS}       


% References
\bibliography{report} % bibliography data in report.bib
\bibliographystyle{spiebib} % makes bibtex use spiebib.bst

\end{document} 
