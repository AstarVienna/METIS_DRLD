\section{Functional and Workflows Description}
% ESO-037611 expects a section called "Functional and Workflows Description" in the DRLD
% "Data Processing Overview" is the term used for the DRLS.
%\section{Data Processing Overview}
\label{sec:data_processing_overview}

The METIS data reduction system runs in different environments and
serves various purposes.  According to the setting, the following
pipeline levels are distinguished~\cite{1618}:

\begin{description}
\item[Quality Control Level 0 (QC0):] The QC0 pipeline runs
  automatically in real time on a dedicated pipeline workstation in
  the instrument control room at the observatory. Its purpose is to
  analyse every FITS file created by the instrument and produce
  quality control parameters that allow assessment of whether the
  observation and instrument performance were within specifications.
  The appropriate reduction recipe is triggered either when a single
  FITS file is delivered to the workstation or when a template is
  finished. The files are classified based on header keywords, grouped
  and associated to the necessary standard calibration files.

\item[Quality Control Level 1 (QC1):] The goal of the QC1 pipeline is
  to produce certified calibration products from calibration
  observations as well as to produce QC parameters that are used to
  check the quality of observations and to monitor observing
  conditions and instrument health.
  The QC1 pipeline is run automatically by ESO in Garching.
  Calibration products and QC parameters are ingested into the ESO Science Archive.

\item[Quality Control Level 2 (QC2):] The QC2 pipeline produces
  Science Data Products compliant with~\cite{ESO-products_standard} as
  well as QC parameters derived from science exposures. It runs
  offline in an automatic way and uses the best calibration products
  for the night of observation (produced by the QC1 pipeline).
  The QC2 pipeline is run automatically by ESO in Garching.
  Science data products and QC parameters are ingested into the ESO Science Archive.

\item[Science-Grade Desktop Environment:] The pipeline recipes are
  delivered to the astronomical community to enable users to reduce
  data in an optimal and interactive way. Recipes can be run from the
  command line using the \lstinline{esorex} front-end or in the
  context of a \lstinline{Reflex}/\ac{EDPS} workflow. While the desktop recipes
  are identical to those used in the QC2 pipeline, the user can change
  recipe parameters to optimise the reduction. Within the
  \lstinline{Reflex}/\ac{EDPS} environment, interactive tools are provided that
  allow the user to assess the quality of individual reduction steps
  and to repeat them with different parameters. The products of this
  pipeline are compliant with~\cite{ESO-products_standard}.

\end{description}


The rest of this document describes the recipes primarily from the perspective of the desktop pipeline.
The QC0, QC1, QC2 pipelines use the subset of these recipes necessary for the goals described above.
Recipes used in the QC0 environment are written such that they can be run in real time, possibly requiring different defaults for processing parameters.


\subsection{Required calibrations}
\label{ssec:calibrations}

Table~\ref{tab:calibrations_per_mode} (taken from
\cite{METIS-calibration_plan}) lists the main calibration steps that
are required for each instrument mode.

%\TODO{Do we apply NCPA + PSF to HCI data? For ADI a simple recipe is foreseen.} The NCPA estimating algorithms are still under study by the HCI team.
%\begin{landscape}
\begin{table}
  \newcommand{\yes}{\tikz\fill[scale=0.35,color=green!50!black](0,.35) -- (.25,0) -- (0.9,.7) -- (.25,.15) -- cycle;}
  \newcommand{\no}{\textcolor{red!50!black}{---}}
    \caption[Overview of required calibrations per instrument mode]{Overview of required calibrations per instrument mode.
    The IFU modes refer to both the nominal configuration and to the extended wavelength configuration. From~\cite{METIS-calibration_plan}.}
  \label{tab:calibrations_per_mode}
  \centering\scriptsize
  \begin{tabularx}{\textwidth}{lXccXccccc}
    \hline
                           & Dark / Linearity & Flat & Wave & Offset type & Telluric & Flux & Distortion & NCPA + PSF & RSRF \\
    \hline\hline
    \CODE{IMG_LM}          & \yes & \yes & \no  & Dither         & \no      & \yes & \yes       & \no        & \no  \\
    \CODE{IMG_LM_(RA/C)VC} & \yes & \yes & \no  & ADI            & \no      & \yes & \yes       & \yes       & \no  \\
 %   \CODE{IMG_LM_CLC}      & \yes & \yes & \no  & ADI            & \no      & \yes & \yes       & \yes       & \no  \\
    \CODE{IMG_LM_APP}      & \yes & \yes & \no  & Dither + ADI   & \no      & \yes & \yes       & \yes       & \no  \\
    \CODE{SPEC_LM}         & \yes & \no  & \yes & Dither along slit & \yes  & \yes & \yes       & \no        & \yes \\
    \CODE{IFU}             & \yes & \no  & \yes & Dither         & \yes     & \yes & \yes       & \no        & \yes \\
    \CODE{IFU_APP}         & \yes & \no  & \yes & Dither\footnote{Dithering for background subtraction + IFU\_APP will not be practical due to AO halo.}  + ADI  & \yes     & \yes & \yes       & \yes       & \yes \\
    \CODE{IFU_(RA/C)VC}    & \yes  & \no  & \yes & ADI            & \yes     & \yes & \yes       & \yes       & \yes \\
   % \CODE{IFU_CLC}         & \yes & \no  & \yes & Dither + ADI   & \yes     & \yes & \yes       & \yes       & \yes \\
    \hline
    \CODE{IMG_N}           & \yes  & \yes & \no  & chop/nod       & \no      & \yes & \yes       & \no        & \no  \\
    \CODE{IMG_N_CVC}       & \yes  & \yes & \no  & three-point chopping & \no & \yes & \no       & \yes       & \no  \\
 %   \CODE{IMG_N_CLC}       & \no  & \yes & \no  & out-of-field chopping & \no & \yes & \no      & \yes       & \no  \\
    \CODE{SPEC_N_LOW}      & \yes  & \no  & \yes & chop/nod along slit & \yes & \yes & \yes      & \no        & \yes \\
    \hline
  \end{tabularx}
\end{table}
%\end{landscape}
\FloatBarrier

%%%
\subsection{Imaging in LM and N}
\label{ssec:overview_lm_imaging}

\textbf{Note: The pipeline layout has been modified compared to the
  PDR design in order to achieve better modularity. Basic reduction
  and background subtraction have been split into two recipes that now
  are applied to both standard calibration and science data. ADI recipes have been added since PDR however integration of \ac{HCI}
  into this workflow requires more work: \ac{HCI} images will be treated
  the same way at least through basic reduction, possibly through
  background subtraction. \ac{ADI} combination may require a separate
  recipe, at least for some \ac{HCI} configurations.}

The purpose of the pipeline is to correct or remove contributions from
the instrument, telescope, and atmosphere and produce science-grade
data products.  In the case of the METIS imaging modes the main
contributions to correct or remove are dark current, flatfield, bad
pixels, and, most importantly, thermal background emission from the
sky and the telescope. Further effects include persistence,
cross-talk, geometric distortions, etc. The final product of the
imaging pipeline is one or more image(s) that are flux-calibrated in units of
photons/s/pixel against a standard star.
Several images can be stacked into a single possibly mosaiced image.

Due to the differences in characteristics between the HAWAII2RG
detector used for imaging in the L and M bands and the GeoSnap
detector used for the N band, the operational concept for the two
imager subsystems are quite different. This induces differences in the
way the data have to be reduced.

The GeoSnap detector has more stable gain than AQUARIUS detector,
which was still in the baseline at PDR.  Chopping is still necessary,
albeit at a lower frequency of a few Hz, and the standard chop/nod
technique will be employed for background subtraction.  As the dark
signal is automatically removed when the exposures from the different
chop and nod positions are combined no master dark is required for the
reduction of science data. Flat fielding may be possible, pending
further investigation of the detector stability

Observations and reduction of LM band data with the HAWAII2RG detector
can proceed as in the near infrared. After dark subtraction and
flat-fielding, the background is estimated from a series of dithered
science exposures or from exposures on a nearby blank patch of sky.

The association maps for the current designs of the imaging pipelines
in~LM and~N are shown in Figs.~\ref{fig:IMG_LM_Assomap}
and~\ref{fig:IMG_N_Assomap}, respectively.

%\TODO{For \ac{HCI} data, \ac{ADI} may need to be part of reduction recipe if
%  individual background subtracted images are the goal?} We provide ADI recipes since PDR.

\begin{landscape}
  \begin{figure}
    \centering
    \includegraphics{IMG_LM_assomap_tikz}
    \caption[Reduction cascade and association map for imaing in L and
      M]{Association map for imaging in the LM band. The figure shows only
      the primary product created from each recipe; for a full list of
      products refer to the recipe descriptions in
      Sect.~\ref{ssec:recipes_img_lm}. The dashed line separates
      calibration tasks that are done at AIT or infrequently during
      operations (left) from daily tasks (right). The prefix ``\REC{metis_}'' has been
      omitted from the recipe names to improve clarity. The product
      names omit ``\FITS{LM_}''.}
    \label{fig:IMG_LM_Assomap}
  \end{figure}
\end{landscape}

\begin{landscape}
\begin{figure}
  \centering
    \includegraphics{IMG_N_assomap_tikz}
    \caption[Reduction cascade and association map for imaing in N]{%
      Association map for imaging in the N band. The figure shows
      only the primary product created from each recipe; for a full
      list of products refer to the recipe descriptions in
      Sect.~\ref{ssec:recipes_img_n}. The dashed line separates
      calibration tasks that are done at AIT or infrequently during
      operations (left) from daily tasks (right). The prefix ``\REC{metis_}'' has
      been omitted from the recipe names to improve clarity. The
      product names omit ``\PROD{N_}''.}
    \label{fig:IMG_N_Assomap}
  \end{figure}
\end{landscape}

%%%%%%%%%%%%%%%%%%%%%%%%%%%%%%%%%%%%%%%%%%%%%%%%%%%%%%
%%% Local Variables:
%%% TeX-master: "METIS_DRLD"
%%% End:


%%%
\subsection{Long-Slit Spectroscopy in L/M- and N-bands}

The purpose of the pipeline is to correct or remove contributions from
the instrument, telescope, and atmosphere and produce science-grade
data products for the L/M- and N-band long-slit spectroscopy
mode. Since the type of the detectors are not yet defined, we assume
the same reduction cascade for both spectral ranges LM and
N. However, to keep flexibility and independence of both branches, we
define different recipes for the time being, although they will be
mostly based on the same algorithms.

Figures~\ref{Fig:LMLssAssomap} and \ref{Fig:NQLssAssomap} show the reduction cascade and the
association map for the recipes handling L/M- and N-band long-slit
spectroscopy data.  Table~\ref{Tab:LssDatProc} contains the data processing table for these
modes. For the time being it is not clear whether a geometric
distortion correction will be needed. We therefore consider to investigate
the geometry of atmospheric lines for this purpose. These lines also will
serve as reference frame for the wavelength calibration.

\begin{sidewaysfigure}[ht]
  \centering
  \includegraphics[width=0.9\textheight]{figures/LM_LSS_pipeline_wf_draft_latest_v0.62.pdf}
  \caption[Reduction cascade and association map for LM long-slit
  spectroscopy]{Reduction cascade and association map for long-slit
    spectroscopy in the LM bands.  }
  \label{Fig:LMLssAssomap}
\end{sidewaysfigure}

% \begin{sidewaysfigure}[ht]
%   \centering
%   \includegraphics[width=0.9\textheight]{figures/NQ_LSS_pipeline_wf_draft_latest.png}
%   \caption[Reduction cascade and association map for N long-slit
%   spectroscopy]{Reduction cascade and association map for long-slit
%     spectroscopy in the N band.  }
%   \label{Fig:NQLssAssomap}
% \end{sidewaysfigure}


%% ---- Table: LM long-slit spectroscopy
\begin{sidewaystable}
  \footnotesize
  \begin{center}
    \caption[Data Processing table for LM/N long-slit spectroscopy]{%
      Data Processing table for LM/N long-slit spectroscopy
      calibration modes}\bigskip
    \label{Tab:LssDatProc}
    \begin{tabular}{|l|l|l|l|l|l|}
      \hline
      Data Type   & Classification & Recipe (Level)	& FITS Keywords & CalibDB & Products\\
    (Templates) & Keywords	 & Processing steps	&		&	  &	\\
    \hline
    \TPL{DARK}	& \CODE{DPR.CATG==CALIB} & \REC{metis_det_dark} & Exposure time	&	& Averaged dark frame\\
    		& \CODE{DPR.TYPE==DARK}  &			&		&	& Bad pixel map\\
    		& \CODE{DPR.TECH==IMAGE}  &			&		&	& \\
    \hline
    \TPL{FLAT}	& \CODE{DPR.CATG==CALIB} & \REC{metis_LM_lss_rsrf} & Exposure time	& dark	& Averaged, normalized flatfield\\
    		& \CODE{DPR.TYPE==FLAT}  &			&		&	& Bad pixel map\\
    		& \CODE{DPR.TECH==SPECTRUM}  &			&		&	& \\
    \hline
    \TPL{SCIENCE} & \CODE{DPR.CATG==SCIENCE} & \REC{metis_LM_lss_wave} & Object name & 	 & Science grade spectrum\\
    		& \CODE{DPR.TYPE==LSS}   &			   & Exposure time & &\\
    		& \CODE{DPR.TECH==SPECTRUM}  &			&		&	& \\
    		& \CODE{PRO.CATG==SPECTRUM}   &  &  & & \\
    \hline
    \TPL{SCIENCE} & \CODE{DPR.CATG==SCIENCE} & \REC{metis_LM_lss_sci} & Object name & 	 & Science grade spectrum\\
    		& \CODE{DPR.TYPE==LSS}   &			   & Exposure time & &\\
    		& \CODE{DPR.TECH==SPECTRUM}  &			&		&	& \\
    		& \CODE{PRO.CATG==SPECTRUM}   &  &  & & \\
    \hline
    \TPL{SCIENCE} & \CODE{DPR.CATG==SCIENCE} & \REC{metis_LM_lss_flux} & Object name & 	 & Science grade spectrum\\
    		& \CODE{DPR.TYPE==LSS}   &			   & Exposure time & &\\
    		& \CODE{DPR.TECH==SPECTRUM}  &			&		&	& \\
    		& \CODE{PRO.CATG==SPECTRUM}   &  &  & & \\
    \hline
    \TPL{SCIENCE} & \CODE{DPR.CATG==SCIENCE} & \REC{metis_LM_lss_tac} & Object name & 	 & Science grade telluric\\
    		& \CODE{DPR.TYPE==LSS}   &			   & Transmission curve & &Absorption corrected spectrum\\
    		& \CODE{DPR.TECH==SPECTRUM}  &			&		&	& \\
    		& \CODE{PRO.CATG==SPECTRUM}   &  &  & & \\
    \hline
%     \hline
% %     \TPL{DARK}	& \CODE{DPR.CATG==CALIB} & \REC{metis_det_dark} & Exposure time	&	& Averaged dark frame\\
% %     		& \CODE{DPR.TYPE==DARK}  &			&		&	& \\
% %     		& \CODE{DPR.TECH==IMAGE}  &			&		&	& \\
% %     \hline
%     \TPL{FLAT}	& \CODE{DPR.CATG==CALIB} & \REC{metis_N_lss_rsrf} & Exposure time	& dark	& Averaged, normalized flatfield\\
%     		& \CODE{DPR.TYPE==FLAT}  &			&		&	& Bad pixel map\\
%     		& \CODE{DPR.TECH==SPECTRUM}  &			&		&	& \\
%     \hline
%      \TPL{SCIENCE} & \CODE{DPR.CATG==SCIENCE} & \REC{metis_N_lss_sci} & Object name & 	 & Science grade scpectrum\\
%     		& \CODE{DPR.TYPE==LSS}   &			   & Exposure time & &\\
%     		& \CODE{DPR.TECH==SPECTRUM}  &			&		&	& \\
%     		& \CODE{PRO.CATG==SPECTRUM}   &  &  & & \\
%     \hline
%       \TPL{SCIENCE} & \CODE{DPR.CATG==SCIENCE} & \REC{metis_N_lss_tac} & Object name & 	 & Science grade telluric\\
%     		& \CODE{DPR.TYPE==LSS}   &			   & Transmission curve & &Absorption corrected spectrum\\
%     		& \CODE{DPR.TECH==SPECTRUM}  &			&		&	& \\
%     		& \CODE{PRO.CATG==SPECTRUM}   &  &  & & \\
%     \hline
    \end{tabular}
  \end{center}
\end{sidewaystable}

%%%
\subsection{LM IFU: integral-field spectroscopy}
\label{ssec:overview_ifu}

In general, the workflow is similar to the \ac{LSS} mode,
except the extensive post-processing stage.
The main difference arises from the need to co-add multiple exposures
to achieve the full resolution, since pixel scales are different:
\begin{itemize}
    \item in the along-slice direction, the sampling is sufficient, ie. above the Nyquist rate;
        at $8.2$ mas per pixel;
    \item in the across-slice direction, the sampling is below the Nyquist rate at
        $20.7$ mas per pixel, and dithering/co-adding of multiple exposures is needed.
\end{itemize}

The ratio of these resolutions shows that at least three exposures shifted by one third of the
pixel size are required. The image is then reconstructed on a square pixel grid of
$8.2 \times 8.2 \text{mas}^2$.

The exposures are taken in two perpendicular field rotations,
so that full resolution is obtained naturally in the along-slice direction
and by dithering in the across-slice direction; in the other sequence of three exposures
these directions are swapped.

The association map is shown in Fig.~\ref{Fig:IfuAssomap}.

\newgeometry{bottom=0.1cm, top=0.1cm}
% This geometry makes the tables/figures fit, but messes up the header a bit.
\begin{landscape}
\begin{figure}[ht]
  \centering
  \resizebox{\linewidth}{!}{%%%%%%%%%%%%%%%%%% BEGIN DOCUMENT %%%%%%%%%%%%%%%%%%%%%%%%%%%%%%%%%%%%%%%%
\sffamily


% ADDING NEW DEFINITIONS -------------------------------------------- start
\definecolor{listingbg}{gray}{0.95}
\definecolor{darkgreen}{rgb}{0.0, 0.7, 0.0}
\definecolor{darkblue} {rgb}{0.0, 0.0, 0.7}
\definecolor{cyan} {rgb}{0.0, 0.4, 0.4}
\definecolor{darkred}  {rgb}{0.7, 0.0, 0.0}
\definecolor{darkorange}{rgb}{1.0, 0.49, 0.0}
\definecolor{violett}{rgb}{255, 0, 255}
\definecolor{turq}{rgb}{0.0, 0.7, 0.8}
\definecolor{fits}{rgb}{0.4, 0.1, 1}


\makeatletter
\lstdefinestyle{RAWstyle}{%
  basicstyle=\ttfamily\color{black}%
  \lst@ifdisplaystyle\scriptsize\fi}

\lstdefinestyle{PARstyle}{%
  basicstyle=\ttfamily\color{black}%
  \lst@ifdisplaystyle\scriptsize\fi}

\lstdefinestyle{DRLstyle}{%
  basicstyle=\ttfamily\color{black}%
  \lst@ifdisplaystyle\scriptsize\fi}

\lstdefinestyle{RECstyle}{%
  basicstyle=\ttfamily\color{black}%
  \lst@ifdisplaystyle\scriptsize\fi}

\lstdefinestyle{QCstyle}{%
  basicstyle=\ttfamily\color{black}%
  \lst@ifdisplaystyle\scriptsize\fi}

\lstdefinestyle{TPLstyle}{%
  basicstyle=\ttfamily\color{black}%
  \lst@ifdisplaystyle\scriptsize\fi}

\lstdefinestyle{PRODstyle}{%
  basicstyle=\ttfamily\color{black}%
  \lst@ifdisplaystyle\scriptsize\fi}

\lstdefinestyle{EXTCALIBstyle}{%
  basicstyle=\ttfamily\color{black}%
  \lst@ifdisplaystyle\scriptsize\fi}

\lstdefinestyle{STATCALIBstyle}{%
  basicstyle=\ttfamily\color{black}%
  \lst@ifdisplaystyle\scriptsize\fi}
\makeatother

%%% This file contains definitions of shapes and nodes used
%%% for a recipe workflow
%%% Author       : Oliver Czoske
%%% Created      : 2021-03-03
%%% Last Changed : 2021-03-03
%%% Changes:
%%%

\usetikzlibrary{
  shapes.misc,
  positioning,
  calc,
  arrows.meta}

%% All connecting lines have an arrow
\tikzset{
  every path/.style={->, >=Latex[open], thick}
}

%% Start and stop buttons (black disks, stop with ring)
%% These are pics, use as
%%         \pic (name) [above of=..] {picname};
\tikzset{
  start/.pic = {
    \node (-m) at (0, 0){};
    \filldraw [fill=black] (0, 0) circle (0.2);
  }
}

\tikzset{
  stop/.pic = {
    \node (-m) at (0, 0){};
    \node (-t) at (0, -0.3){};
    \filldraw [fill=black] (0, 0) circle(0.2);
    \draw[black] (0, 0) circle (0.3);
  }
}


%%%% Various boxes and their colours
%%%% These are nodes, use as
%%%% \node (name) [type, location]  {text};

\definecolor{stepcolor}{RGB}{210,169,188}
\definecolor{rawcolor}{RGB}{235,235,235}
\definecolor{externalcolor}{RGB}{183,255,255}
\definecolor{calibcolor}{RGB}{255,250,216}
\definecolor{calproductcolor}{RGB}{185,184,237}
\definecolor{qcproductcolor}{RGB}{255,201,165}
\definecolor{sciproductcolor}{RGB}{197,219,183}
\definecolor{framecolor}{RGB}{127,13,65}

\tikzset{
  %% template : the template(s) that trigger(s) the recipe
  template/.style={
    rectangle,
    draw=black,
    minimum width=4.0cm,
    minimum height=0.5cm,
    align=center
  },
  %% input : the input files
  input/.style={
    rectangle,
    fill=rawcolor,
    minimum width=4.0cm,
    minimum height=0.75cm,
    text width=3cm,
    align=center
  },
  %% calib : calibration input
  calib/.style={
    rectangle,
    fill=calibcolor,
    minimum width=4.0cm,
    minimum height=0.75cm,
    text width=3cm,
    align=center
  },
  %% external : external input
  external/.style={
    rectangle,
    fill=externalcolor,
    minimum width=4.0cm,
    minimum height=0.75cm,
    text width=3.5cm,
    align=center
  },
  %% params : parameters
  params/.style={
    rectangle,
    draw=red,
    thick,
    minimum width=4.0cm,
    minimum height=0.75cm,
    text width=3cm,
    align=center
  },
  %% redstep : a reduction step
  %%      ("step" is predefined and can't be used)
  redstep/.style={
    rectangle,
    rounded corners=0.2cm,
    fill=stepcolor,   %%% define colour!
    minimum width=4.0cm,
    minimum height=1cm,
    text width=3cm,
    align=center
  },
  %% connection : connection to input or output
  connection/.style={
    circle,
    fill=black,
    minimum size=0.15cm,
    inner sep=0pt
  },
  %% sciproduct : a science product
  sciproduct/.style={
    rectangle,
    fill=sciproductcolor,
    minimum width=4.0cm,
    minimum height=0.75cm,
    text width=3.5cm,
    align=center
  },
  %% calproduct : a calibration product
  calproduct/.style={
    rectangle,
    fill=calproductcolor,
    minimum width=4.0cm,
    minimum height=0.75cm,
    text width=3.5cm,
    align=center
  },
  %% frame : frame around the recipe
  %% This is a path, use as
  %%    \draw [frame] (upper left) rectangle (lower right);
  frame/.style={framecolor, very thick, dashed}
}


%%% Picture: flow chart
\begin{tikzpicture}[on grid=false, node distance=0.8cm]

  \matrix (recipes) [column sep=1mm, row sep=1cm]{

    % Row *_raw

    \node[above] (REClin_raw){\recipebox{\RAW{DETLIN_IFU_RAW}}{\REC{metis_det_lingain}}}; &
% TODO: Put back in once we actually include the persistence recipe in the DRLD
%    \node[above] (RECpers_raw){\recipebox{\RAW{PERSISTENCE}}{\REC{metis_det_persistence}}}; &
    \node[above] (RECpers_raw)[empty]{}; &
    \node[above] (RECdark_raw){\recipebox{\RAW{DARK_IFU_RAW}}{\REC{metis_det_dark}}}; &
    \node[above] (RECgeom_raw){\recipebox{\RAW{IFU_DISTORTION_RAW}}{\REC{metis_ifu_distortion}}}; &
    \node[above] (RECrsrf_raw){\recipebox{\RAW{IFU_RSRF_RAW}}{\REC{metis_ifu_rsrf}}}; &
    \node[above] (RECwcal_raw){\recipebox{\RAW{IFU_WAVE_RAW}}{\REC{metis_ifu_wavecal}}}; &
    \node[above] (RECstdreduce_raw){\recipebox{\RAW{IFU_STD_RAW}}{\REC{metis_ifu_reduce}}}; &
    \node[above] (RECscireduce_raw){\recipebox{\RAW{IFU_SCI_RAW}}{\REC{metis_ifu_reduce}}}; &
%      \recipebox{\RAW{IFU_STD_RAW}}}{\REC{metis_ifu_std_process}}}
%      \recipenotitlebox{\REC{metis_ifu_std_process}}}
%      so actually this is now just the same as telluric
    \node[above] (RECstd_raw){\recipenotitlebox{\REC{metis_ifu_telluric}}}; &
    \node[above] (RECtac_raw){\recipenotitlebox{\REC{metis_ifu_telluric}}}; &
%      \recipebox{\RAW{IFU_SCI_RAW}}}{\REC{metis_ifu_sci_process}}}
%      \recipenotitlebox{\REC{metis_ifu_sci_process}}}
    \node[above] (RECsci1_raw){\recipenotitlebox{\REC{metis_ifu_calibrate}}}; &
    \node[above] (RECsci2_raw){\recipenotitlebox{\REC{metis_ifu_postprocess}}}; \\
%    &
%    \node[above] (adi_raw){%
%      \recipenotitlebox{\REC{metis_ifu_adi_cgrph}}}
%    };
    \node (REClin_DIlin)[statcalfile]{\STATCALIB{LINEARITY_IFU}}; &
    \node (RECpers_DIlin)[empty]{}; &
    \node (RECdark_DIlin)[empty]{}; &
    \node (RECgeom_DIlin)[empty]{}; &
    \node (RECrsrf_DIlin)[empty]{}; &
    \node (RECwcal_DIlin)[empty]{}; &
    \node (RECstdreduce_DIlin)[connection]{}; &
    \node (RECscireduce_DIlin)[connection]{}; &
    \node (RECtac_DIlin)[empty]{}; &
    \node (RECstd_DIlin)[empty]{}; &
    \node (RECsci1_DIlin)[empty]{}; &
    \node (RECsci2_DIlin)[empty]{}; \\

    \node (REClin_DIpers)[empty]{}; &
    \node (RECpers_DIpers)[extcalfile]{\STATCALIB{PERSISTENCE_MAP}}; &
    \node (RECdark_DIpers)[empty]{}; &
    \node (RECgeom_DIpers)[empty]{}; &
    \node (RECrsrf_DIpers)[empty]{}; &
    \node (RECwcal_DIpers)[empty]{}; &
    \node (RECstdreduce_DIpers)[connection]{}; &
    \node (RECscireduce_DIpers)[connection]{}; &
    \node (RECtac_DIpers)[empty]{}; &
    \node (RECstd_DIpers)[empty]{}; &
    \node (RECsci1_DIpers)[empty]{}; &
    \node (RECsci2_DIpers)[empty]{}; \\

    % Row *_dark
    \node (REClin_DIdark)[empty]{}; &
    \node (RECpers_DIdark)[empty]{}; &
    \node (RECdark_DIdark)[calibproduct]{\PROD{MASTER_DARK_IFU}}; &
    \node (RECgeom_DIdark)[connection]{}; &
    \node (RECrsrf_DIdark)[connection]{}; &
    \node (RECwcal_DIdark)[connection]{}; &
    \node (RECstdreduce_DIdark)[connection]{}; &
    \node (RECscireduce_DIdark)[connection]{}; &
    \node (RECtac_DIdark)[empty]{}; &
    \node (RECstd_DIdark)[empty]{}; &
    \node (RECsci1_DIdark)[empty]{}; &
    \node (RECsci2_DIdark)[empty]{}; \\
%    &
%    \node (adi_DIdark)[empty]{};

    % Row *_geom
    \node (REClin_DIgeom)[empty]{}; &
    \node (RECpers_DIgeom)[empty]{}; &
    \node (RECdark_DIgeom)[empty]{}; &
    \node (RECgeom_DIgeom)[statcalfile]{\PROD{IFU_DISTORTION_TABLE}}; &
    \node (RECrsrf_DIgeom)[empty]{}; &
    \node (RECwcal_DIgeom)[connection]{}; &
    \node (RECstdreduce_DIgeom)[connection]{}; &
    \node (RECscireduce_DIgeom)[connection]{}; &
    \node (RECstd_DIgeom)[empty]{}; &
    \node (RECtac_DIgeom)[empty]{}; &
    \node (RECsci1_DIgeom)[empty]{}; &
    \node (RECsci2_DIgeom)[empty]{}; \\
%    &
%    \node (adi_DIgeom)[empty]{};

% Row *_rsrf
    \node (REClin_DIrsrf)[empty]{}; &
    \node (RECpers_DIrsrf)[empty]{}; &
    \node (RECdark_DIrsrf)[empty]{}; &
    \node (RECgeom_DIrsrf)[empty]{}; &
    \node (RECrsrf_DIrsrf)[statcalfile]{\PROD{RSRF_IFU}}; &
    \node (RECwcal_DIrsrf)[empty]{}; &
    \node (RECstdreduce_DIrsrf)[connection]{}; &
    \node (RECscireduce_DIrsrf)[connection]{}; &
    \node (RECstd_DIrsrf)[empty]{}; &
    \node (RECtac_DIrsrf)[empty]{}; &
    \node (RECsci1_DIrsrf)[empty]{}; &
    \node (RECsci2_DIrsrf)[empty]{}; \\
%    & \node (adi_DIrsrf)[empty]{};

% Row *_wcal
    \node (REClin_DIwcal)[empty]{}; &
    \node (RECpers_DIwcal)[empty]{}; &
    \node (RECdark_DIwcal)[empty]{}; &
    \node (RECgeom_DIwcal)[empty]{}; &
    \node (RECrsrf_DIwcal)[empty]{}; &
    \node (RECwcal_DIwcal) [calibproduct]{\PROD{IFU_WAVECAL}}; &
    \node (RECstdreduce_DIwcal)[connection]{}; &
    \node (RECscireduce_DIwcal)[connection]{}; &
    \node (RECstd_DIwcal)[empty]{}; &
    \node (RECsci1_DIwcal)[empty]{}; &
    \node (RECtac_DIwcal)[empty]{}; &
    \node (RECsci2_DIwcal)[empty]{}; \\
%    \node (adi_DIwcal)[empty]{};


    % Row *_scireduced
    \node (REClin_DIscireduced)[empty]{}; &
    \node (RECpers_DIscireduced)[empty]{}; &
    \node (RECdark_DIstdreduced)[empty]{}; &
    \node (RECgeom_DIscireduced)[empty]{}; &
    \node (RECrsrf_DIscireduced)[empty]{}; &
    \node (RECwcal_DIscireduced)[empty]{}; &
    \node (RECstdreduce_DIscireduced)[empty]{}; &
    \node (RECscireduce_DIscireduced)[scienceproduct]{\EXTCALIB{IFU_SCI_REDUCED}};
    \node (RECscireduce_DIscibackground)[scienceproduct,below=0.5cm]{\EXTCALIB{IFU_SCI_BACKGROUND}};
    \node (RECscireduce_DIscireducedcube)[scienceproduct,below=1.25cm]{\EXTCALIB{IFU_SCI_REDUCED_CUBE}};
    \node (RECscireduce_DIscicombined)[scienceproduct,below=2.0cm]{\EXTCALIB{IFU_SCI_COMBINED}};
    \draw [-] (RECscireduce_DIscireduced) -- (RECscireduce_DIscibackground) -- (RECscireduce_DIscireducedcube) -- (RECscireduce_DIscicombined); &
    \node (RECstd_DIscireduced)[empty]{}; &
    \node (RECtac_DIscireduced)[empty]{}; &
    \node (RECsci1_DIscireduced)[empty]{}; &
    \node (RECsci2_DIscireduced)[empty]{}; \\
%    \node (adi_DIscireduced)[empty]{};

    % Row *_stdreduced
    \node (REClin_DIstdreduced)[empty]{}; &
    \node (RECpers_DIstdreduced)[empty]{}; &
    \node (RECdark_DIstdreduced)[empty]{}; &
    \node (RECgeom_DIstdreduced)[empty]{}; &
    \node (RECrsrf_DIstdreduced)[empty]{}; &
    \node (RECwcal_DIstdreduced)[empty]{}; &
    \node (RECstdreduce_DIstdreduced)[scienceproduct]{\PROD{IFU_STD_REDUCED}};
    \node (RECstdreduce_DIstdbackground)[scienceproduct,below=0.5cm]{\PROD{IFU_STD_BACKGROUND}};
    \node (RECstdreduce_DIstdreducedcube)[scienceproduct,below=1.25cm]{\EXTCALIB{IFU_STD_REDUCED_CUBE}};
    \node (RECstdreduce_DIstdcombined)[scienceproduct,below=2.cm]{\EXTCALIB{IFU_STD_COMBINED}};
    \draw [-] (RECstdreduce_DIstdreduced) -- (RECstdreduce_DIstdbackground) -- (RECstdreduce_DIstdreducedcube) -- (RECstdreduce_DIstdcombined); &
    \node (RECscireduce_DIstdreduced)[empty]{}; &
    \node (RECstd_DIstdreduced)[empty]{}; &
    \node (RECtac_DIstdreduced)[empty]{}; &
    \node (RECsci1_DIstdreduced)[empty]{}; &
    \node (RECsci2_DIstdreduced)[empty]{}; \\
    %    &
%    \node (adi_DIstdreduced)[empty]{};

    % Row *_basicstd
    \node (REClin_DIfluxstd)[empty]{}; &
    \node (RECpers_DIfluxstd)[extcalfile]{\EXTCALIB{FLUXSTD_CATALOG}}; &
    \node (RECdark_DIfluxstd)[empty]{}; &
    \node (RECgeom_DIfluxstd)[empty]{}; &
    \node (RECrsrf_DIfluxstd)[empty]{}; &
    \node (RECwcal_DIfluxstd)[empty]{}; &
    \node (RECstdreduce_DIfluxstd)[empty]{}; &
    \node (RECscireduce_DIfluxstd)[empty]{}; &
    \node (RECstd_DIfluxstd)[connection]{}; &
    \node (RECtac_DIfluxstd)[empty]{}; &
    \node (RECsci1_DIfluxstd)[empty]{}; &
    \node (RECsci2_DIfluxstd)[empty]{}; \\
%    &
%    \node (adi_DIfluxstd)[empty]{};

    % Row *_tac
    \node (REClin_DItac)[empty]{}; &
    \node (RECpers_DItac)[empty]{}; &
    \node (RECdark_DItac)[empty]{}; &
    \node (RECgeom_DItac)[empty]{}; &
    \node (RECrsrf_DItac)[empty]{}; &
    \node (RECwcal_DItac)[empty]{}; &
    \node (RECstdreduce_DItac)[empty]{}; &
    \node (RECscireduce_DItac)[empty]{}; &
    \node (RECstd_DItac)[empty]{}; &
    \node (RECtac_DItac)[calibproduct]{\PROD{IFU_TELLURIC}}; &
    \node (RECsci1_DItac)[connection]{}; &
    \node (RECsci2_DItac)[empty]{}; \\
%    &
%    \node (adi_DItac)[empty]{};

    % Row *_fcal
    \node (REClin_DIfcal)[empty]{}; &
    \node (RECpers_DIfcal)[empty]{}; &
    \node (RECdark_DIfcal)[empty]{}; &
    \node (RECgeom_DIfcal)[empty]{}; &
    \node (RECrsrf_DIfcal)[empty]{}; &
    \node (RECwcal_DIfcal)[empty]{}; &
    \node (RECstdreduce_DIfcal)[empty]{}; &
    \node (RECscireduce_DIfcal)[empty]{}; &
    \node (RECstd_DIfcal)[calibproduct]{\PROD{FLUXCAL_TAB}};
    \node (RECstd_DItelluricstd)[calibproduct,below=.5cm]{\PROD{IFU_TELLURIC}};
    \draw [-] (RECstd_DIfcal) -- (RECstd_DItelluricstd); &
    \node (RECtac_DIfcal)[empty]{}; &
    \node (RECsci1_DIfcal)[connection]{};
    % No clue how to get this .70 nicely done
    \node (RECsci1_DItelluricstd)[connection,below=.70cm]{}; &
    \node (RECsci2_DIfcal)[empty]{}; \\
%    &
%    \node (adi_DIfcal)[empty]{};

    % Row *_sci1
    \node (REClin_DIsci1)[empty]{}; &
    \node (RECpers_DIsci1)[empty]{}; &
    \node (RECdark_DIsci1)[empty]{}; &
    \node (RECgeom_DIsci1)[empty]{}; &
    \node (RECrsrf_DIsci1)[empty]{}; &
    \node (RECwcal_DIsci1)[empty]{}; &
    \node (RECstdreduce_DIsci1)[empty]{}; &
    \node (RECscireduce_DIsci1)[empty]{}; &
    \node (RECstd_DIsci1)[empty]{}; &
%    \node (RECtac_DIsci1)[scienceproduct]{\PROD{IFU_SCI_CALIBRATED_TAC}}};
    \node (RECtac_DIsci1)[empty]{}; &
    \node (RECsci1_DIsci1)[scienceproduct]{\PROD{IFU_SCI_CUBE_CALIBRATED}}; &
    \node (RECsci2_DIsci1)[connection]{}; \\
%    &
%    \node (adi_DIsci1)[empty]{};

    % Row *_sci2
    \node (REClin_DIsci2)[empty]{}; &
    \node (RECpers_DIsci2)[empty]{}; &
    \node (RECdark_DIsci2)[empty]{}; &
    \node (RECgeom_DIsci2)[empty]{}; &
    \node (RECrsrf_DIsci2)[empty]{}; &
    \node (RECwcal_DIsci2)[empty]{}; &
    \node (RECstdreduce_DIsci2)[empty]{}; &
    \node (RECscireduce_DIsci2)[empty]{}; &
    \node (RECstd_DIsci2)[empty]{}; &
%    \node (RECtac_DIsci2)[scienceproduct]{\PROD{IFU_SCI_COMBINED_TAC}}};
    \node (RECtac_DIsci2)[empty]{}; &
%    \node (RECsci1_DIsci2)[scienceproduct]{\PROD{IFU_SCI_COMBINED}}};
    \node (RECsci1_DIsci2)[empty]{}; &
    \node (RECsci2_DIsci2) [empty]{}; \\
%    &
%    \node (adi_DIsci2) [connection]{};

    % Row *_adi
    \node (REClin_adi)[empty]{}; &
    \node (RECpers_adi)[empty]{}; &
    \node (RECdark_adi)[empty]{}; &
    \node (RECgeom_adi)[empty]{}; &
    \node (RECrsrf_adi)[empty]{}; &
    \node (RECwcal_adi)[empty]{}; &
    \node (RECstdreduce_adi)[empty]{}; &
    \node (RECscireduce_adi)[empty]{}; &
    \node (RECstd_adi)[empty]{}; &
    \node (RECtac_adi)[empty]{}; &
    \node (RECsci1_adi)[empty]{}; &
    \node (RECsci2_adi)[scienceproduct]{\PROD{IFU_SCI_COADD}}; \\
%    &
%    \node (adi_adi)[scienceproduct]{ADI\_SCI\_COADD};
  };    % end matrix


%  Dashed line separating daily procedure
%  Commented out: right now there is no clear separation
%  \node (t1) at ($(RECrsrf_raw.east)!0.5!(RECdark_raw.west)$){};
%  \node (t2) at ($(RECrsrf_adi.east)!0.5!(RECdark_adi.west)$){} ;
%  \draw [thick,dashed] ([yshift=4ex]t1.north) -- ([yshift=-0ex]t2.south);


  %% Connections
  \draw [arrow] (REClin_raw) -- (REClin_DIlin);
% TODO: Put back in once we actually include the persistence recipe in the DRLD
%  \draw [arrow] (RECpers_raw) -- (RECpers_DIpers);
  \draw [arrow] (RECgeom_raw) -- (RECgeom_DIgeom);
  \draw [arrow] (RECrsrf_raw) -- (RECrsrf_DIrsrf);
  \draw [arrow] (RECwcal_raw) -- (RECwcal_DIwcal);
  \draw [arrow] (RECdark_raw) -- (RECdark_DIdark);
  \draw [arrow] (RECscireduce_raw)  -- (RECscireduce_DIscireduced);
  \draw [arrow] (RECstdreduce_raw)  -- (RECstdreduce_DIstdreduced);

  \draw [match] (REClin_DIlin) --
% TODO: Put back in once we actually include the persistence recipe in the DRLD
%        (RECpers_DIlin) [xshift=-0.15cm] arc [start angle=180, end angle=0, radius=.15cm] --
        (RECdark_DIlin) [xshift=-0.15cm] arc [start angle=180, end angle=0, radius=.15cm] --
        (RECgeom_DIlin) [xshift=-0.15cm] arc [start angle=180, end angle=0, radius=.15cm] --
        (RECrsrf_DIlin) [xshift=-0.15cm] arc [start angle=180, end angle=0, radius=.15cm] --
        (RECwcal_DIlin) [xshift=-0.15cm] arc [start angle=180, end angle=0, radius=.15cm] --
        (RECscireduce_DIlin);

% Persistence
  \draw [match] (RECpers_DIpers) -- 
        (RECdark_DIpers) [xshift=-0.15cm] arc [start angle=180, end angle=0, radius=.15cm] --
        (RECgeom_DIpers) [xshift=-0.15cm] arc [start angle=180, end angle=0, radius=.15cm] --
        (RECrsrf_DIpers) [xshift=-0.15cm] arc [start angle=180, end angle=0, radius=.15cm] --
        (RECwcal_DIpers) [xshift=-0.15cm] arc [start angle=180, end angle=0, radius=.15cm] --
        (RECscireduce_DIpers);
  \draw [match] (RECdark_DIdark) -- (RECscireduce_DIdark);

  \draw [match] (RECrsrf_DIrsrf) -- (RECscireduce_DIrsrf);

  \draw [match] (RECwcal_DIwcal)  -- (RECscireduce_DIwcal);
  \draw [match] (RECgeom_DIgeom) -- (RECrsrf_DIgeom) [xshift=-0.15cm] arc [start angle=180, end angle=0, radius=.15cm] -- (RECscireduce_DIgeom);
  \draw [match] (RECpers_DIfluxstd)   -- (RECstd_DIfluxstd);   % Line from FLUXSTD_CATALOG to further
  \draw [match] (RECstd_DIfcal)  -- (RECsci1_DIfcal);
  \draw [match,dashed] (RECstd_DItelluricstd)  -- (RECsci1_DItelluricstd);
  \draw [match] (RECtac_DItac)  -- (RECsci1_DItac);

% -| means first horizontal, then vertical
  \draw [arrow] (RECstdreduce_DIstdcombined) -| (RECstd_DIfcal);
  \draw [arrow] (RECscireduce_DIscicombined) -| (RECtac_DItac);
  \draw [arrow] (RECscireduce_DIscireduced) -| (RECsci1_DIsci1);
  \draw [arrow] (RECsci1_DIsci1) -| (RECsci2_adi);



  %\draw [very thick,dashed] ($(raw_geometry.north)!0.5!(raw_dark.north)$) -- ++(270:15cm);

  %% Legend
  \matrix (legend) [draw, fill=gray!15, above right, row sep=0.3cm,
    column 1/.style={anchor=base},
    column 2/.style={anchor=base west}]
  at ([yshift=0cm]current bounding box.south west){%
    \node (leg_recipe) [recipe]{ifu\_sci\_process};
    & \node {recipe}; \\
    \node (leg_calproduct) [calibproduct]{MASTER\_DARK};
    & \node{calib.\ product}; \\
    \node (leg_sciproduct)[scienceproduct]{SCI\_REDUCED};
    & \node {science product}; \\
    \node (leg_statcalfile)[statcalfile]{MASTER\_RSRF};
    & \node {static calib.\ file};\\
    \node (leg_calfile)[extcalfile]{FLUXSTD\_CATALOG};
    & \node {external file}; \\

    \draw [arrow,fill=black] (0,0.4) -- (0,-0.3);  %% should be centred relative to column
    & \node {processing step}; \\

    \draw [connection_arrow] (-1, 0.5ex) -- (1,0.5ex) node [connection,yshift=0cm]{};
    & \node {product match}; \\
  };    %% end matrix (legend)

\end{tikzpicture}


% ADDING NEW DEFINITIONS -------------------------------------------- start
\definecolor{listingbg}{gray}{0.95}
\definecolor{darkgreen}{rgb}{0.0, 0.7, 0.0}
\definecolor{darkblue} {rgb}{0.0, 0.0, 0.7}
\definecolor{cyan} {rgb}{0.0, 0.4, 0.4}
\definecolor{darkred}  {rgb}{0.7, 0.0, 0.0}
\definecolor{darkorange}{rgb}{1.0, 0.49, 0.0}
\definecolor{violet}{rgb}{255, 0, 255}
\definecolor{turq}{rgb}{0.0, 0.7, 0.8}
\definecolor{fits}{rgb}{0.4, 0.1, 1}


\makeatletter
\lstdefinestyle{RAWstyle}{%
  basicstyle=\ttfamily\color{fits}%
  \lst@ifdisplaystyle\scriptsize\fi}

\lstdefinestyle{PARstyle}{%
  basicstyle=\ttfamily\color{cyan}%
  \lst@ifdisplaystyle\scriptsize\fi}

\lstdefinestyle{DRLstyle}{%
  basicstyle=\ttfamily\color{violet}%
  \lst@ifdisplaystyle\scriptsize\fi}

\lstdefinestyle{RECstyle}{%
  basicstyle=\ttfamily\color{darkgreen}%
  \lst@ifdisplaystyle\scriptsize\fi}

%% Write QC parameters like this: \QC*{QC_SOMETHING_OR_OTHER}
\lstdefinestyle{QCstyle}{%
  basicstyle=\ttfamily\color{darkblue}%
  \lst@ifdisplaystyle\scriptsize\fi}

%% Write templates like this: \TPL{DARK_LM}
\lstdefinestyle{TPLstyle}{%
  basicstyle=\ttfamily\color{darkred}%
  \lst@ifdisplaystyle\scriptsize\fi}

%% Write products like this: \PROD{SOME_THING}
\lstdefinestyle{PRODstyle}{%
  basicstyle=\ttfamily\color{darkorange}%
  \lst@ifdisplaystyle\scriptsize\fi}

%% external calib files
\lstdefinestyle{EXTCALIBstyle}{%
  basicstyle=\ttfamily\color{Turquoise}%
  \lst@ifdisplaystyle\scriptsize\fi}

% static calib files
\lstdefinestyle{STATCALIBstyle}{%
  basicstyle=\ttfamily\color{teal}%
  \lst@ifdisplaystyle\scriptsize\fi}

% static calib files
\lstdefinestyle{FITSstyle}{%
  basicstyle=\ttfamily\color{black}%
  \lst@ifdisplaystyle\scriptsize\fi}
\makeatother

}
  \caption[Reduction cascade and association map for IFU spectroscopy]{%
    Association map for \ac{IFU} spectroscopy in L- and M-band. The
    figure shows only the primary products created by each recipe; for
    a full list of products refer to the recipe descriptions in
    Sect.~\ref{ssec:IFU_recipes}. The dashed line separates
    calibration tasks that are done at AIT or infrequently during
    operations from tasks done daily.}
  \label{Fig:IfuAssomap}
\end{figure}
\end{landscape}
\restoregeometry



%%%%%%%%%%%%%%%%%%%%%%%%%%%%%%%%%%%%%%%%%%%%%%%%%%%%%%

%%% Local Variables:
%%% TeX-master: "METIS_DRLD"
%%% End:


\subsection{Parallel Observing Modes}
\label{ssec:combinedmodes}

There are three parallel observing modes:

\begin{itemize}
\item Parallel observing mode IMG-LM and IMG-N (Section~\ref{sssec:parallellmnimg})
\item Parallel observing mode LSS-LM and LSS-N (Section~\ref{sssec:parallellmnspec})
\item Parallel observing mode IMG-LM and IFU (Section~\ref{sssec:parallellmnspec})
\end{itemize}

The respective templates corresponding to these modes will produce raw data files that can be processed independently by the relevant workflows.
That is, the templates will trigger two different recipe cascades, and there are no specific recipes for any of the parallel observing modes.

\subsubsection{Parallel observing mode IMG-LM and IMG-N}\label{sssec:parallellmnimg}
There are two specific templates for parallel imaging observations in the LM-band and N-band:
\begin{itemize}
 \item \TPL{METIS_img_lmn_obs_AutoChopNod}
 \item \TPL{METIS_img_lmn_obs_GenericChopNod}
\end{itemize}
These templates produce two kind of raw images that are processed independently in either the LM-band or N-band imaging workflow.
This fulfills \REQ{METIS-7244}.


\subsubsection{Parallel observing mode LSS-LM and LSS-N}\label{sssec:parallellmnspec}
There is one specific template for parallel LSS observations in the LM-band and N-band:
\begin{itemize}
 \item \TPL{METIS_spec_lmn_obs_AutoChopNodOnSlit}
\end{itemize}
These templates produce two kind of raw images that are processed independently in either the LM-band or N-band LSS workflow.
This fulfills \REQ{METIS-7245}.

\subsubsection{Parallel observing mode IMG-LM and IFU}\label{sssec:parallellmifu}
Parallel observing mode IMG-LM and IFU is also needed to perform non-common path pointing and aberration correction in \ac{HCI} modes:
real-time monitoring of non-common path aberrations between the \ac{SCAO} \ac{WFS} and the \ac{HCI} elements cannot be performed with the \ac{IFU} due to slicing and sampling issues.

The IFU pickoff optic is a beamsplitter with a transmission of ~10\% to the LM imager and a reflectivity of ~90\% to the IFU. % https://polarion.astron.nl/polarion/#/project/METIS/workitem?id=METIS-3111
LM-band images can therefore be taken in parallel with the IFU exposures, for any of the IFU templates.
The LM-band images and IFU exposures are processed independently.
This fulfills \REQ{METIS-6072}.


\subsection{Workflows}
The association matrices described in the previous sections will be converted one-to-one into Reflex or \ac{EDPS} workflows.

Any interactivity in the workflows is described with the individual recipes.


\subsection{Matched FITS keywords}

The workflow management system (e.g. \ac{EDPS}) uses the `matched keywords'
to find calibration data when processing data.
That is, the system will compare the FITS keywords of the primary input, to
the FITS headers of the pool of possible calibration files to use, in order
to decide what data to use.

For example, most calibration data has to be taken with the same \FITS{DET.DIT}
and \FITS{DET.NDIT} combination as the science data.
Several calibration products also need to use the same filter, or the same mask, or the same grism as the science data.

All selections are done on equality.
That is, no interpolation between, e.g., \FITS{DET.DIT}, will be done if only an approximate matching data product is found.

The following two tables provide an overview of the matched FITS
keywords. Table~\ref{tab:fitskeywordaliasses} defines several high level keyword
aliases used for convenience when there are several combinations
of instrumental keywords (INS.) which are needed to match with the correct
calibration files. These keywords are used in the Matched Keyword
descriptions in the recipes in Chapter~\ref{sec:pipeline_recipes}.
Table~\ref{tab:fitsmatchedkeywordssummary}
summaries the input data, calibration files, and the FITS
keyworlds needed to match them for all the recipes listed in
Chapter~\ref{sec:pipeline_recipes}. The second defines various high
level aliases used for convenience when there are several combinations
of instrumental keywords which are needed to match with the correct
calibration files.

\begin{table}
    \caption{FITS keyword aliases}
    \label{tab:fitskeywordaliasses}
  \begin{tabular}[c]{|p{3cm}|p{5cm}|p{5cm}|}
      \hline
      \textbf{Alias} & \textbf{Description} & \textbf{FITS keywords} \\
      \hline
\FITS{DRS.FILTER}   & Filter Information (LM or N)  & value of \FITS{INS.OPTI10.NAME} or \FITS{INS.OPTI13.NAME} \\
\FITS{DRS.NDFILTER} & ND Filter information	        & value of \FITS{INS.OPTI11.NAME} or \FITS{INS.OPTI13.NAME} \\
\FITS{DRS.SLIT}     & Slit Information (LM or N)    & value of \FITS{INS.OPTI3.NAME} and \FITS{INS.OPTI12.NAME} or  \FITS{INS.OPTI9.NAME} \\
\FITS{DRS.MASK}     & Mask information for Coronagraphy	& value of \FITS{INS.OPTI1.NAME}, \FITS{INS.OPTI3.NAME}, \FITS{INS.OPTI5.NAME}, \FITS{INS.OPTI9.NAME}, \FITS{INS.OPTI12.NAME} \\
\FITS{DRS.PUPIL}    & Pupil information             & value of \FITS{INS.OPTI15.NAME} or \FITS{INS.OPTI16.NAME}\\
\hline
    \end{tabular}
\end{table}


\newgeometry{bottom=0.1cm, right=0.1cm, left=0.1cm, top=0.1cm}
\begin{landscape}
{
  \begin{longtable}[c]{|p{4.2cm}|p{4.6cm}|p{5.6cm}|p{3.5cm}|p{3.5cm}|}
%  \begin{longtable}[c]{|l|l|l|l|l|}
 \caption{FITS matched keywords summary}
 \label{tab:fitsmatchedkeywordssummary}
 \endfirsthead
 \hline
 \textbf{Recipe} & \textbf{Main Input} & \textbf{Calibration Data} & \textbf{FITS keywords} & \textbf{Aliases} \\
 \hline
    \endhead
 \hline
 \textbf{Recipe} & \textbf{Main Input} & \textbf{Calibration Data} & \textbf{FITS keywords} & \textbf{Aliases} \\
 \hline
\REC{metis_det_lingain} & \RAW{DETLIN_det_RAW}, \RAW{det_WCU_OFF_RAW} &  &  &  \\
\REC{metis_det_dark} & \RAW{DARK_det_RAW} & \PROD{LINEARITY_det}, \PROD{PERSISTENCE_MAP} &  &  \\
\REC{metis_det_persistence} &  &  &  &  \\
\REC{metis_lm_img_flat} & \RAW{LM_FLAT_LAMP_RAW}, \RAW{LM_FLAT_TWILIGHT_RAW} & \PROD{MASTER_DARK_2RG} & \FITS{DET.DIT}, \FITS{DET.NDIT}, \FITS{INS.OPTI10.NAME} & \FITS{DET.DIT}, \FITS{DET.NDIT}, \FITS{DRS.FILTER},  \\
\REC{metis_lm_img_basic_reduce} & \RAW{LM_IMAGE_SCI_RAW} & \PROD{MASTER_DARK_2RG}, \PROD{MASTER_IMG_FLAT_LAMP_LM}, \PROD{MASTER_IMG_FLAT_TWILIGHT_LM} & \FITS{DET.DIT}, \FITS{DET.NDIT}, \FITS{INS.OPTI10.NAME} & \FITS{DET.DIT}, \FITS{DET.NDIT}, \FITS{DRS.FILTER},  \\
\REC{metis_lm_img_background} & \RAW{LM_SCI_BASIC_REDUCED}, \RAW{LM_STD_BASIC_REDUCED} &  & \FITS{INS.OPTI10.NAME} & \FITS{DRS.FILTER},  \\
\REC{metis_lm_img_std_process} & \RAW{LM_STD_BKG_SUBTRACTED} & \PROD{FLUXSTD_CATALOG} & \FITS{INS.OPTI10.NAME} & \FITS{DRS.FILTER},  \\
\REC{metis_lm_img_calibrate} & \RAW{LM_SCI_BKG_SUBTRACTED} & \PROD{FLUXCAL_TAB}, \PROD{LM_DISTORTION_TABLE} & \FITS{INS.OPTI10.NAME} & \FITS{DRS.FILTER},  \\
\REC{metis_lm_img_sci_postprocess} & \RAW{LM_SCI_CALIBRATED} &  & \FITS{INS.OPTI10.NAME} & \FITS{DRS.FILTER},  \\
\REC{metis_lm_img_distortion} & \RAW{LM_DISTORTION_RAW}, \RAW{LM_WCU_OFF_RAW} & \EXTCALIB{PINHOLE_TABLE} & \FITS{INS.OPTI10.NAME} & \FITS{DRS.FILTER},  \\
\REC{metis_n_img_flat} & \RAW{N_FLAT_LAMP_RAW}, \RAW{N_FLAT_TWILIGHT_RAW} & \PROD{MASTER_DARK_GEO} & \FITS{DET.DIT}, \FITS{DET.NDIT}, \FITS{INS.OPTI10.NAME} & \FITS{DET.DIT}, \FITS{DET.NDIT}, \FITS{DRS.FILTER},  \\
\REC{metis_n_img_chopnod} & \RAW{N_IMAGE_SCI_RAW} & \PROD{N_FLAT_LAMP_RAW}, \PROD{N_FLAT_TWILIGHT_RAW} & \FITS{DET.DIT}, \FITS{DET.NDIT}, \FITS{INS.OPTI10.NAME} & \FITS{DET.DIT}, \FITS{DET.NDIT}, \FITS{DRS.FILTER},  \\
\REC{metis_n_img_std_process} & \RAW{N_STD_BKG_SUBTRACTED} & \PROD{FLUXSTD_CATALOG} & \FITS{INS.OPTI13.NAME} & \FITS{DRS.FILTER},  \\
\REC{metis_n_img_calibrate} & \RAW{N_SCI_BKG_SUBTRACTED} & \PROD{FLUXCAL_TAB}, \PROD{N_DISTORTION_TABLE} & \FITS{INS.OPTI13.NAME} & \FITS{DRS.FILTER},  \\
\REC{metis_n_img_restore} & \RAW{N_SCI_CALIBRATED} &  & \FITS{INS.OPTI13.NAME} & \FITS{DRS.FILTER},  \\
\REC{metis_n_img_distortion} & \RAW{N_DISTORTION_RAW}, \RAW{N_WCU_OFF_RAW} & \EXTCALIB{PINHOLE_TABLE} & \FITS{INS.OPTI13.NAME} & \FITS{DRS.FILTER},  \\
\REC{metis_LM_lss_rsrf} & \RAW{LM_LSS_RSRF_RAW} & \PROD{LINEARITY_2RG}, \PROD{PERSISTENCE_MAP}, \PROD{GAIN_MAP_2RG}, \PROD{MASTER_DARK_2RG} & \FITS{DET.DIT}, \FITS{DET.NDIT}, \FITS{INS.OPTI9.NAME} & \FITS{DET.DIT}, \FITS{DET.NDIT}, \FITS{DRS.SLIT},  \\
\REC{metis_LM_lss_trace} & \RAW{LM_LSS_RSRF_PINH_RAW} & \PROD{LINEARITY_2RG}, \PROD{PERSISTENCE_MAP}, \PROD{GAIN_MAP_2RG}, \PROD{MASTER_DARK_2RG}, \PROD{MASTER_LM_LSS_RSRF} & \FITS{DET.DIT}, \FITS{DET.NDIT}, \FITS{INS.OPTI9.NAME} & \FITS{DET.DIT}, \FITS{DET.NDIT}, \FITS{DRS.SLIT},  \\
\REC{metis_LM_lss_wave} & \RAW{LM_LSS_WAVE_RAW} & \PROD{LINEARITY_2RG}, \PROD{PERSISTENCE_MAP}, \PROD{GAIN_MAP_2RG}, \PROD{MASTER_DARK_2RG}, \PROD{MASTER_LM_LSS_RSRF}, \PROD{LM_LSS_TRACE}, \PROD{LASER_TAB} & \FITS{DET.DIT}, \FITS{DET.NDIT}, \FITS{INS.OPTI9.NAME}, \FITS{SEQ.WCU.LASERn} & \FITS{DET.DIT}, \FITS{DET.NDIT}, \FITS{DRS.SLIT}, \FITS{SEQ.WCU.LASERn},  \\
\REC{metis_LM_lss_std} & \RAW{LM_LSS_STD_RAW} & \PROD{LINEARITY_2RG}, \PROD{PERSISTENCE_MAP}, \PROD{GAIN_MAP_2RG}, \PROD{MASTER_DARK_2RG}, \PROD{MASTER_LM_LSS_RSRF}, \PROD{LM_LSS_DIST_SOL}, \PROD{LM_LSS_WAVE_GUESS}, \PROD{AO_PSF_MODEL}, \PROD{ATM_LINE_CAT}, \PROD{LM_ADC_SLITLOSS}, \PROD{LM_SYNTH_TRANS}, \PROD{REF_STD_CAT} & \FITS{DET.DIT}, \FITS{DET.NDIT}, \FITS{INS.OPTI9.NAME} & \FITS{DET.DIT}, \FITS{DET.NDIT}, \FITS{DRS.SLIT},  \\
\REC{metis_LM_lss_sci} & \RAW{LM_LSS_SCI_RAW} & \PROD{LINEARITY_2RG}, \PROD{PERSISTENCE_MAP}, \PROD{GAIN_MAP_2RG}, \PROD{MASTER_DARK_2RG}, \PROD{MASTER_LM_LSS_RSRF}, \PROD{LM_LSS_DIST_SOL}, \PROD{LM_LSS_WAVE_GUESS}, \PROD{AO_PSF_MODEL}, \PROD{ATM_LINE_CAT}, \PROD{LM_ADC_SLITLOSS}, \PROD{STD_TRANSMISSION}, \PROD{MASTER_LM_RESPONSE} & \FITS{DET.DIT}, \FITS{DET.NDIT}, \FITS{INS.OPTI9.NAME} & \FITS{DET.DIT}, \FITS{DET.NDIT}, \FITS{DRS.SLIT},  \\
\REC{metis_LM_lss_mf_model} & \RAW{LM_LSS_SCI_FLUX_1D} & \PROD{LSF_KERNEL}, \PROD{ATM_PROFILE}, \PROD{ATM_LINE_CAT} & \FITS{INS.OPTI9.NAME} & \FITS{DRS.SLIT},  \\
\REC{metis_LM_lss_mf_calctrans} & \RAW{MF_BEST_FIT_TAB} & \PROD{LSF_KERNEL}, \PROD{ATM_PROFILE}, \PROD{ATM_LINE_CAT} & \FITS{INS.OPTI9.NAME} & \FITS{DRS.SLIT},  \\
\REC{metis_LM_lss_mf_correct} & \RAW{LM_LSS_SCI_FLUX_1D}, \RAW{LM_LSS_SYNTH_TRANS} &  & \FITS{INS.OPTI9.NAME} & \FITS{DRS.SLIT},  \\
\REC{metis_N_lss_rsrf} & \RAW{N_LSS_RSRF_RAW} & \PROD{LINEARITY_GEO}, \PROD{PERSISTENCE_MAP}, \PROD{GAIN_MAP_GEO}, \PROD{MASTER_DARK_GEO} & \FITS{DET.DIT}, \FITS{DET.NDIT}, \FITS{INS.OPTI12.NAME} & \FITS{DET.DIT}, \FITS{DET.NDIT}, \FITS{DRS.SLIT},  \\
\REC{metis_N_lss_trace} & \RAW{N_LSS_RSRF_PINH_RAW} & \PROD{LINEARITY_GEO}, \PROD{PERSISTENCE_MAP}, \PROD{GAIN_MAP_GEO}, \PROD{MASTER_DARK_GEO}, \PROD{MASTER_N_LSS_RSRF} & \FITS{DET.DIT}, \FITS{DET.NDIT}, \FITS{INS.OPTI12.NAME} & \FITS{DET.DIT}, \FITS{DET.NDIT}, \FITS{DRS.SLIT},  \\
%\REC{metis_N_lss_wave} & \RAW{N_LSS_WAVE_RAW} & \PROD{LINEARITY_GEO}, \PROD{PERSISTENCE_MAP}, \PROD{GAIN_MAP_GEO}, \PROD{MASTER_DARK_GEO}, \PROD{MASTER_N_LSS_RSRF}, \PROD{N_LSS_TRACE}, \PROD{LASER_TAB} & \FITS{DET.DIT}, \FITS{DET.NDIT}, \FITS{INS.OPTI12.NAME}, \FITS{SEQ.WCU.LASERn} & \FITS{DET.DIT}, \FITS{DET.NDIT}, \FITS{DRS.SLIT}, \FITS{SEQ.WCU.LASERn},  \\
\REC{metis_N_lss_std} & \RAW{N_LSS_STD_RAW} & \PROD{LINEARITY_GEO}, \PROD{PERSISTENCE_MAP}, \PROD{GAIN_MAP_GEO}, \PROD{MASTER_DARK_GEO}, \PROD{MASTER_N_LSS_RSRF}, \PROD{N_LSS_DIST_SOL}, \PROD{N_LSS_WAVE_GUESS}, \PROD{AO_PSF_MODEL}, \PROD{ATM_LINE_CAT}, \PROD{N_ADC_SLITLOSS}, \PROD{N_SYNTH_TRANS}, \PROD{REF_STD_CAT} & \FITS{DET.DIT}, \FITS{DET.NDIT}, \FITS{INS.OPTI12.NAME} & \FITS{DET.DIT}, \FITS{DET.NDIT}, \FITS{DRS.SLIT},  \\
\REC{metis_N_lss_sci} & \RAW{N_LSS_SCI_RAW} & \PROD{LINEARITY_GEO}, \PROD{PERSISTENCE_MAP}, \PROD{GAIN_MAP_GEO}, \PROD{MASTER_DARK_GEO}, \PROD{MASTER_N_LSS_RSRF}, \PROD{N_LSS_DIST_SOL}, \PROD{N_LSS_WAVE_GUESS}, \PROD{AO_PSF_MODEL}, \PROD{ATM_LINE_CAT}, \PROD{N_ADC_SLITLOSS}, \PROD{STD_TRANSMISSION}, \PROD{MASTER_N_RESPONSE} & \FITS{DET.DIT}, \FITS{DET.NDIT}, \FITS{INS.OPTI12.NAME} & \FITS{DET.DIT}, \FITS{DET.NDIT}, \FITS{DRS.SLIT},  \\
\REC{metis_N_lss_mf_model} & \RAW{N_LSS_SCI_FLUX_1D} & \PROD{LSF_KERNEL}, \PROD{ATM_PROFILE}, \PROD{ATM_LINE_CAT} & \FITS{INS.OPTI12.NAME} & \FITS{DRS.SLIT},  \\
\REC{metis_N_lss_mf_calctrans} & \RAW{MF_BEST_FIT_TAB} & \PROD{LSF_KERNEL}, \PROD{ATM_PROFILE}, \PROD{ATM_LINE_CAT} & \FITS{INS.OPTI12.NAME} & \FITS{DRS.SLIT},  \\
\REC{metis_N_lss_mf_correct} & \RAW{N_LSS_SCI_FLUX_1D}, \RAW{N_LSS_SYNTH_TRANS} &  & \FITS{INS.OPTI12.NAME} & \FITS{DRS.SLIT},  \\
\REC{metis_ifu_wavecal} & \RAW{IFU_WAVE_RAW} & \PROD{MASTER_DARK_IFU}, \PROD{IFU_DISTORTION_TABLE} & \FITS{DET.DIT}, \FITS{DET.NDIT}, \FITS{INS.OPTI6.NAME} & \FITS{DET.DIT}, \FITS{DET.NDIT}, \FITS{DRS.IFU},  \\
\REC{metis_ifu_rsrf} & \RAW{IFU_RSRF_RAW} & \PROD{MASTER_DARK_IFU}, \PROD{IFU_WAVECAL} & \FITS{DET.DIT}, \FITS{DET.NDIT}, \FITS{INS.OPTI6.NAME} & \FITS{DET.DIT}, \FITS{DET.NDIT}, \FITS{DRS.IFU},  \\
\REC{metis_ifu_calibrate} & \RAW{IFU_STD_RAW} & \PROD{MASTER_DARK_IFU}, \PROD{RSRF_IFU}, \PROD{IFU_WAVECAL}, \PROD{IFU_DISTORTION_TABLE} & \FITS{DET.DIT}, \FITS{DET.NDIT}, \FITS{INS.OPTI6.NAME} & \FITS{DET.DIT}, \FITS{DET.NDIT}, \FITS{DRS.IFU},  \\
\REC{metis_ifu_telluric} & \RAW{IFU_SCI_COMBINED} & \PROD{LSF_KERNEL}, \PROD{ATM_PROFILE} & \FITS{DET.DIT}, \FITS{DET.NDIT}, \FITS{INS.OPTI6.NAME} & \FITS{DET.DIT}, \FITS{DET.NDIT}, \FITS{DRS.IFU},  \\
\REC{metis_ifu_calibrate} & \RAW{IFU_SCI_REDUCED}, \PROD{FLUXCAL_TAB} &  & \FITS{INS.OPTI6.NAME} & \FITS{DRS.IFU},  \\
\REC{metis_ifu_postprocess} & \RAW{IFU_SCI_REDUCED} &  & \FITS{INS.OPTI6.NAME} & \FITS{DRS.IFU},  \\
\REC{metis_ifu_distortion} & \RAW{IFU_DISTORTION_RAW} & \EXTCALIB{PINHOLE_TABLE} & \FITS{INS.OPTI6.NAME} & \FITS{DRS.IFU},  \\
\REC{metis_img_adi_cgrph} & \RAW{LM_SCI_BASIC_REDUCED}, \RAW{N_SCI_BKG_SUBTRACTED} & \PROD{LM_DISTORTION_TABLE}, \PROD{N_DISTORTION_TABLE}, \PROD{LM_cgrph_SCI_THROUGHPUT}, \PROD{N_cgrph_SCI_THROUGHPUT} & \FITS{INS.OPTI1.NAME}, \FITS{INS.OPTI3.NAME}, \FITS{INS.OPTI5.NAME}, \FITS{INS.OPTI9.NAME}, \FITS{INS.OPTI10.NAME}, \FITS{INS.OPTI13.NAME} & \FITS{DRS.MASK},  \\
\REC{metis_lm_adi_app} & \RAW{LM_SCI_BASIC_REDUCED} & \PROD{LM_DISTORTION_TABLE}, \PROD{LM_OFF_AXIS_PSF_RAW} & \FITS{INS.OPTI5.NAME}, \FITS{INS.OPTI9.NAME}, \FITS{INS.OPTI10.NAME} & \FITS{DRS.MASK},  \\
\REC{metis_ifu_adi_cgrph} & \RAW{IFU_SCI_REDUCED} & \PROD{IFU_DISTORTION_TABLE}, \PROD{IFU_CGRPH_SCI_THROUGHPUT} & \FITS{INS.OPTI6.NAME}, \FITS{INS.OPTI1.NAME}, \FITS{INS.OPTI3.NAME}, \FITS{INS.OPTI5.NAME}, \FITS{INS.OPTI9.NAME}, \FITS{INS.OPTI12.NAME} & \FITS{DRS.MASK},  \\
\REC{metis_pupil_imaging} & \RAW{LM_PUPIL_RAW}, \RAW{N_PUPIL_RAW} &  & \FITS{INS.OPTI15.NAME}, \FITS{INS.OPTI16.NAME}, \FITS{INS.OPTI10.NAME}, \FITS{INS.OPTI13.NAME} & \FITS{DRS.PUPIL},  \\
\REC{metis_img_chophome} & \RAW{LM_CHOPPERHOME_RAW} & \PROD{LINEARITY_2RG}, \PROD{PERSISTENCE_MAP}, \PROD{GAIN_MAP_2RG}, \PROD{MASTER_DARK_2RG}, \PROD{MASTER_IMG_FLAT_LAMP_LM} & \FITS{DET.DIT}, \FITS{DET.NDIT}, \FITS{INS.OPTI10.NAME} & \FITS{DET.DIT}, \FITS{DET.NDIT}, \FITS{DRS.FILTER},  \\
\REC{metis_lm_adc_slitloss} & \RAW{LM_SLITLOSSES_RAW}, \RAW{LM_WCU_OFF_RAW} & \PROD{LINEARITY_2RG}, \PROD{PERSISTENCE_MAP}, \PROD{GAIN_MAP_2RG}, \PROD{MASTER_DARK_2RG}, \PROD{MASTER_IMG_FLAT_LAMP_LM} & \FITS{INS.OPTI10.NAME}, \FITS{INS.OPTI13.NAME} & \FITS{DET.DIT}, \FITS{DET.NDIT}, \FITS{DRS.FILTER},  \\
\REC{metis_n_adc_slitloss} & \RAW{LM_SLITLOSSES_RAW}, \RAW{LM_WCU_OFF_RAW} & \PROD{LINEARITY_GEO}, \PROD{PERSISTENCE_MAP}, \PROD{GAIN_MAP_GEO}, \PROD{MASTER_DARK_GEO}, \PROD{MASTER_IMG_FLAT_LAMP_N} & \FITS{INS.OPTI10.NAME}, \FITS{INS.OPTI13.NAME} & \FITS{DET.DIT}, \FITS{DET.NDIT}, \FITS{DRS.FILTER},  \\
    \hline
    \end{longtable}
  }
\end{landscape}
\restoregeometry

%%% Local Variables:
%%% TeX-master: "METIS_DRLD"
%%% End:
