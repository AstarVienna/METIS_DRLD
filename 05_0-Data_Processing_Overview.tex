\section{Data Processing Overview}
\label{sec:data_processing_overview}

The METIS data reduction system runs in different environments and
serves various purposes.  According to the setting, the following
pipeline levels are distinguished \cite{1618}:

\begin{description}
\item[Quality Control Level 0 (QC0):] The QC0 pipeline runs
  automatically in real time on a dedicated pipeline workstation in
  the instrument control room at the observatory. Its purpose is to
  analyse every FITS file created by the instrument and produce
  quality control parameters that allow assessment of whether the
  observation and instrument performance were within specifications.
  The appropriate reduction recipe is triggered either when a single
  FITS file is delivered to the workstation or when a template is
  finished. The files are classified based on header keywords, grouped
  and associated to the necessary standard calibration files.

\item[Quality Control Level 1 (QC1):] The goal of the QC1 pipeline is
  to produce certified calibration products from calibration
  observations as well as to produce QC parameters that are used to
  check the quality of observations and to monitor observing
  conditions and instrument health. Calibration products and QC
  parameters are ingested into the ESO Science Archive.

\item[Quality Control Level 2 (QC2):] The QC2 pipeline produces
  Science Data Products compliant with \cite{ESO-products_standard} as
  well as QC parameters derived from science exposures. It runs
  offline in an automatic way and uses the best calibration products
  for the night of observation (produced by the QC1 pipeline). Science
  data products and QC parameters are ingested into the ESO Science
  Archive.

\item[Science-Grade Desktop Environment:] The pipeline recipes are
  delivered to the astronomical community to enable users to reduce
  data in an optimal and interactive way. Recipes can be run from the
  command line using the \lstinline{esorex} front-end or in the
  context of a \lstinline{Reflex} workflow. While the desktop recipes
  are identical to those used in the QC2 pipeline, the user can change
  recipe parameters to optimise the reduction. Within the
  \lstinline{Reflex} environment, interactive tools are provided that
  allow the user to assess the quality of individual reduction steps
  and to repeat them with different parameters. The products of this
  pipeline are compliant with \cite{ESO-products_standard}.

\end{description}


The following sections describe the recipes used in the QC1, QC2 and
desktop pipelines. Recipes used in the QC0 environment may need to be
streamlined to allow them to run in real time.

%%%
\subsection{Long-Slit Spectroscopy in L/M- and N-bands}

The purpose of the pipeline is to correct or remove contributions from
the instrument, telescope, and atmosphere and produce science-grade
data products for the L/M- and N-band long-slit spectroscopy
mode. Since the type of the detectors are not yet defined, we assume
the same reduction cascade for both spectral ranges LM and
N. However, to keep flexibility and independence of both branches, we
define different recipes for the time being, although they will be
mostly based on the same algorithms.

Figures~\ref{Fig:LMLssAssomap} and \ref{Fig:NQLssAssomap} show the reduction cascade and the
association map for the recipes handling L/M- and N-band long-slit
spectroscopy data.  Table~\ref{Tab:LssDatProc} contains the data processing table for these
modes. For the time being it is not clear whether a geometric
distortion correction will be needed. We therefore consider to investigate
the geometry of atmospheric lines for this purpose. These lines also will
serve as reference frame for the wavelength calibration.

\begin{sidewaysfigure}[ht]
  \centering
  \includegraphics[width=0.9\textheight]{figures/LM_LSS_pipeline_wf_draft_latest_v0.62.pdf}
  \caption[Reduction cascade and association map for LM long-slit
  spectroscopy]{Reduction cascade and association map for long-slit
    spectroscopy in the LM bands.  }
  \label{Fig:LMLssAssomap}
\end{sidewaysfigure}

% \begin{sidewaysfigure}[ht]
%   \centering
%   \includegraphics[width=0.9\textheight]{figures/NQ_LSS_pipeline_wf_draft_latest.png}
%   \caption[Reduction cascade and association map for N long-slit
%   spectroscopy]{Reduction cascade and association map for long-slit
%     spectroscopy in the N band.  }
%   \label{Fig:NQLssAssomap}
% \end{sidewaysfigure}


%% ---- Table: LM long-slit spectroscopy
\begin{sidewaystable}
  \footnotesize
  \begin{center}
    \caption[Data Processing table for LM/N long-slit spectroscopy]{%
      Data Processing table for LM/N long-slit spectroscopy
      calibration modes}\bigskip
    \label{Tab:LssDatProc}
    \begin{tabular}{|l|l|l|l|l|l|}
      \hline
      Data Type   & Classification & Recipe (Level)	& FITS Keywords & CalibDB & Products\\
    (Templates) & Keywords	 & Processing steps	&		&	  &	\\
    \hline
    \TPL{DARK}	& \CODE{DPR.CATG==CALIB} & \REC{metis_det_dark} & Exposure time	&	& Averaged dark frame\\
    		& \CODE{DPR.TYPE==DARK}  &			&		&	& Bad pixel map\\
    		& \CODE{DPR.TECH==IMAGE}  &			&		&	& \\
    \hline
    \TPL{FLAT}	& \CODE{DPR.CATG==CALIB} & \REC{metis_LM_lss_rsrf} & Exposure time	& dark	& Averaged, normalized flatfield\\
    		& \CODE{DPR.TYPE==FLAT}  &			&		&	& Bad pixel map\\
    		& \CODE{DPR.TECH==SPECTRUM}  &			&		&	& \\
    \hline
    \TPL{SCIENCE} & \CODE{DPR.CATG==SCIENCE} & \REC{metis_LM_lss_wave} & Object name & 	 & Science grade spectrum\\
    		& \CODE{DPR.TYPE==LSS}   &			   & Exposure time & &\\
    		& \CODE{DPR.TECH==SPECTRUM}  &			&		&	& \\
    		& \CODE{PRO.CATG==SPECTRUM}   &  &  & & \\
    \hline
    \TPL{SCIENCE} & \CODE{DPR.CATG==SCIENCE} & \REC{metis_LM_lss_sci} & Object name & 	 & Science grade spectrum\\
    		& \CODE{DPR.TYPE==LSS}   &			   & Exposure time & &\\
    		& \CODE{DPR.TECH==SPECTRUM}  &			&		&	& \\
    		& \CODE{PRO.CATG==SPECTRUM}   &  &  & & \\
    \hline
    \TPL{SCIENCE} & \CODE{DPR.CATG==SCIENCE} & \REC{metis_LM_lss_flux} & Object name & 	 & Science grade spectrum\\
    		& \CODE{DPR.TYPE==LSS}   &			   & Exposure time & &\\
    		& \CODE{DPR.TECH==SPECTRUM}  &			&		&	& \\
    		& \CODE{PRO.CATG==SPECTRUM}   &  &  & & \\
    \hline
    \TPL{SCIENCE} & \CODE{DPR.CATG==SCIENCE} & \REC{metis_LM_lss_tac} & Object name & 	 & Science grade telluric\\
    		& \CODE{DPR.TYPE==LSS}   &			   & Transmission curve & &Absorption corrected spectrum\\
    		& \CODE{DPR.TECH==SPECTRUM}  &			&		&	& \\
    		& \CODE{PRO.CATG==SPECTRUM}   &  &  & & \\
    \hline
%     \hline
% %     \TPL{DARK}	& \CODE{DPR.CATG==CALIB} & \REC{metis_det_dark} & Exposure time	&	& Averaged dark frame\\
% %     		& \CODE{DPR.TYPE==DARK}  &			&		&	& \\
% %     		& \CODE{DPR.TECH==IMAGE}  &			&		&	& \\
% %     \hline
%     \TPL{FLAT}	& \CODE{DPR.CATG==CALIB} & \REC{metis_N_lss_rsrf} & Exposure time	& dark	& Averaged, normalized flatfield\\
%     		& \CODE{DPR.TYPE==FLAT}  &			&		&	& Bad pixel map\\
%     		& \CODE{DPR.TECH==SPECTRUM}  &			&		&	& \\
%     \hline
%      \TPL{SCIENCE} & \CODE{DPR.CATG==SCIENCE} & \REC{metis_N_lss_sci} & Object name & 	 & Science grade scpectrum\\
%     		& \CODE{DPR.TYPE==LSS}   &			   & Exposure time & &\\
%     		& \CODE{DPR.TECH==SPECTRUM}  &			&		&	& \\
%     		& \CODE{PRO.CATG==SPECTRUM}   &  &  & & \\
%     \hline
%       \TPL{SCIENCE} & \CODE{DPR.CATG==SCIENCE} & \REC{metis_N_lss_tac} & Object name & 	 & Science grade telluric\\
%     		& \CODE{DPR.TYPE==LSS}   &			   & Transmission curve & &Absorption corrected spectrum\\
%     		& \CODE{DPR.TECH==SPECTRUM}  &			&		&	& \\
%     		& \CODE{PRO.CATG==SPECTRUM}   &  &  & & \\
%     \hline
    \end{tabular}
  \end{center}
\end{sidewaystable}



%%% Local Variables:
%%% TeX-master: "METIS_DRLD"
%%% End:
