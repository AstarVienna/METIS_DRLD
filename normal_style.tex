
% ADDING NEW DEFINITIONS -------------------------------------------- start
\definecolor{listingbg}{gray}{0.95}
\definecolor{darkgreen}{rgb}{0.0, 0.7, 0.0}
\definecolor{darkblue} {rgb}{0.0, 0.0, 0.7}
\definecolor{cyan} {rgb}{0.0, 0.4, 0.4}
\definecolor{darkred}  {rgb}{0.7, 0.0, 0.0}
\definecolor{darkorange}{rgb}{1.0, 0.49, 0.0}
\definecolor{violett}{rgb}{255, 0, 255}
\definecolor{turq}{rgb}{0.0, 0.7, 0.8}
\definecolor{fits}{rgb}{0.4, 0.1, 1}


\makeatletter
\lstdefinestyle{RAWstyle}{%
  basicstyle=\ttfamily\color{fits}%
  \lst@ifdisplaystyle\scriptsize\fi}
\makeatother

\makeatletter
\lstdefinestyle{PARstyle}{%
  basicstyle=\ttfamily\color{cyan}%
  \lst@ifdisplaystyle\scriptsize\fi}
\makeatother

\makeatletter
\lstdefinestyle{DRLstyle}{%
  basicstyle=\ttfamily\color{violett}%
  \lst@ifdisplaystyle\scriptsize\fi}
\makeatother

\makeatletter
\lstdefinestyle{RECstyle}{%
  basicstyle=\ttfamily\color{darkgreen}%
  \lst@ifdisplaystyle\scriptsize\fi}
\makeatother

%% Write QC parameters like this: \QC{QC_SOMETHING_OR_OTHER}
\makeatletter
\lstdefinestyle{QCstyle}{%
  basicstyle=\ttfamily\color{darkblue}%
  \lst@ifdisplaystyle\scriptsize\fi}
\makeatother

%% Write templates like this: \TPL{DARK_LM}
\makeatletter
\lstdefinestyle{TPLstyle}{%
  basicstyle=\ttfamily\color{darkred}%
  \lst@ifdisplaystyle\scriptsize\fi}
\makeatother

%% Write products like this: \hyperref[dataitem:some_thing]{\PROD{SOME_THING}}
\makeatletter
\lstdefinestyle{PRODstyle}{%
  basicstyle=\ttfamily\color{darkorange}%
  \lst@ifdisplaystyle\scriptsize\fi}
\makeatother

%% external calib files
\makeatletter
\lstdefinestyle{EXTCALIBstyle}{%
  basicstyle=\ttfamily\color{Turquoise}%
  \lst@ifdisplaystyle\scriptsize\fi}
\makeatother

% static calib files
\makeatletter
\lstdefinestyle{STATCALIBstyle}{%
  basicstyle=\ttfamily\color{teal}%
  \lst@ifdisplaystyle\scriptsize\fi}
\makeatother
