\subsection{LSS observing mode}\label{sec:drl_functions_lss}

%---------------------------------------------------------------------
\subsubsection{Order background correction}\label{drl:correctorder}\label{drl:correct_order_bg}
\begin{recipedef}
Name: & \NEWDRL{correct_order_bg} \\
Purpose: & Removal of order background contamination (e.g. stray light)\\
Used in recipes: & \NEWREC{metis_LM_lss_rsrf} \newline
                  \NEWREC{metis_N_lss_rsrf} \newline
                  \NEWREC{metis_LM_lss_wave} \newline
                  \NEWREC{metis_LM_lss_std} \newline
                  \NEWREC{metis_N_lss_std} \newline
                  \NEWREC{metis_LM_lss_sci}\newline
                  \NEWREC{metis_N_lss_sci}\\
%Working remarks: & None \\
%Function Parameters: & None \\
Input: & $n\times$ \texttt{const hdrl\_image * input} \\
%Other inputs: & None \\
%QC outputs: & None\\
%Output FITS files: & None \\
Other outputs: & \texttt{cpl\_error\_code} \\
General description: & Removal of stray light by a 2D-polynomial fits of the order background \\
Mathematical description: & see Section~\ref{ssec:orderbg} \\
Quality assessment: & Through QC parameters \\
Error conditions: & See~\cite{DRLVT}. \\
Unit tests: & See~\cite{DRLVT}. \\
\end{recipedef}


%---------------------------------------------------------------------
\subsubsection{Subtract WCU OFF frame from frame}\label{drl:subtractwcuoffillum}\label{drl:subtract_wcu_off_illum}
\begin{recipedef}\label{rec:subtrwcuoffillum}
Name: & \NEWDRL{subtract_wcu_off_illum} \\
Purpose: & Subtract a WCU-OFF frame \\
Used in recipes: & \NEWREC{metis_LM_lss_rsrf} \\
& \NEWREC{metis_N_lss_rsrf} \\
& \NEWREC{metis_lm_adc_slitloss} \\
& \NEWREC{metis_n_adc_slitloss} \\
%Working remarks: & None \\
%Function Parameters: & None \\
Input: & \texttt{const hdrl\_image * input} \\
Other inputs: & None\\
%QC outputs: & None \\
%Output FITS files: & None \\
Other outputs: & \texttt{cpl\_error\_code} \\
General description: & Subtracts a WCU-OFF frame from an WCU image \\
Mathematical description: &  see~\cite{pis02} and~\cite{pis21}\\
Quality assessment: & Through QC parameters \\
Error conditions: & See~\cite{DRLVT}. \\
Unit tests: & See~\cite{DRLVT}. \\
\end{recipedef}
%---------------------------------------------------------------------
\subsubsection{Calculate blackbody spectrum}\label{drl:calcbb}\label{drl:calc_bb}
\begin{recipedef}\label{rec:calcbb}
Name: & \NEWDRL{calc_bb} \\
Purpose: & Calculate blackbody spectrum \\
Used in recipes: & \NEWREC{metis_LM_lss_rsrf} \\
& \NEWREC{metis_N_lss_rsrf} \\
%Working remarks: & None \\
%Function Parameters: & None \\
Input: & temperature parameter \\
Other inputs: & None\\
%QC outputs: & None \\
%Output FITS files: & None \\
Other outputs: & \texttt{cpl\_error\_code} \\
            & \texttt{const hdrl\_spectrum1D * output}\\
General description: & Calculates a blackbody spectrum for the \ac{RSRF} normalisation\\
Mathematical description: &  usual Planck formula\\
Quality assessment: & Through QC parameters \\
Error conditions: & See~\cite{DRLVT}. \\
Unit tests: & See~\cite{DRLVT}. \\
\end{recipedef}

%---------------------------------------------------------------------
\subsubsection{Normalise RSRF}\label{drl:normrsrf}\label{drl:norm_flat}
\begin{recipedef}\label{drl:normflat}
Name: & \NEWDRL{norm_flat} \\
Purpose: & Creates normalised flats from \NEWPROD{LM_LSS_RSRF_RAW} / \NEWPROD{N_LSS_RSRF_RAW}\\
Used in recipes: & \NEWREC{metis_LM_lss_rsrf} \\
& \NEWREC{metis_N_lss_rsrf} \\
%Working remarks: & None \\
%Function Parameters: & None \\
Input: & \texttt{const hdrl\_image * input} \\
%Other inputs: & None\\
%QC outputs: & None \\
%Output FITS files: & None \\
Other outputs: & \texttt{cpl\_error\_code} \\
General description: & Creates normalised \ac{RSRF} \\
Mathematical description: &  see~\cite{pis02} and~\cite{pis21}\\
Quality assessment: & Through QC parameters \\
Error conditions: & See~\cite{DRLVT}. \\
Unit tests: & See~\cite{DRLVT}. \\
\end{recipedef}

%---------------------------------------------------------------------
\subsubsection{Normalise spectrum}\label{drl:normspec}\label{drl:norm_spec}
\begin{recipedef}\label{drl:norm_tss}
Name: & \NEWDRL{norm_spec} \\
Purpose: & Creates unity-normalised version from an input \ac{STD} spectra \\
Used in recipes: & \NEWREC{metis_LM_lss_std} \\
& \NEWREC{metis_N_lss_std} \\
%Working remarks: & None \\
%Function Parameters: & None \\
Input: & \texttt{const hdrl\_image * input} \\
%Other inputs: & None\\
%QC outputs: & None \\
%Output FITS files: & None \\
Other outputs: & \texttt{cpl\_error\_code} \\
General description: & Creates normalised version of a \ac{STD} input spectrum  by fitting the continuum and subsequent division of the fit\\
Mathematical description: &  Polynomial fit on regions known to contain continuum only; these regions will be defined in Table \NEWSTATCALIB{REF_STD_CAT}\\
Quality assessment: & Through QC parameters \\
Error conditions: & See~\cite{DRLVT}. \\
Unit tests: & See~\cite{DRLVT}. \\
\end{recipedef}

%---------------------------------------------------------------------
%\subsubsection{Detect order traces}\label{drl:tracedetect}
%Used in recipes:\\ 
%\NEWREC{metis_LM_lss_trace} \newline
%\NEWREC{metis_N_lss_trace} \newline
%TBD



%---------------------------------------------------------------------
\subsubsection{Apply RSRF}\label{drl:applyrsrf}\label{drl:apply_rsrf}
\begin{recipedef}
Name: & \NEWDRL{apply_rsrf}\\
Purpose: & applies \NEWPROD{MASTER_LM_LSS_RSRF} to LM-frames\newline
           applies \NEWPROD{MASTER_N_LSS_RSRF} to N-frames\\
Used in recipes: & \NEWREC{metis_LM_lss_trace} \newline
                 \NEWREC{metis_N_lss_trace} \newline
                 \NEWREC{metis_LM_lss_wave} \newline
                 \NEWREC{metis_LM_lss_std} \newline
                 \NEWREC{metis_N_lss_std} \newline
                 \NEWREC{metis_LM_lss_sci} \newline
                 \NEWREC{metis_N_lss_sci} \newline 
                 \NEWREC{metis_lm_adc_slitloss} \newline
                \NEWREC{metis_n_adc_slitloss}\\
%Working remarks: & None \\
%Function Parameters: & None \\
%Input FITS files: & None \\
Inputs: & \texttt{const hdrl\_image * input}\\
%QC outputs: & None \\
%Output FITS files: & None \\
Other outputs: & \texttt{cpl\_error\_code} \\
General description: & Normal flatfield process for correcting pixel-to-pixel sensitivities \\
Mathematical description: & Division by \NEWPROD{MASTER_LM_LSS_RSRF} or  \NEWPROD{MASTER_N_LSS_RSRF}, respectively \\
Quality assessment: & Through QC parameters \\
Error conditions: & See~\cite{DRLVT}. \\
Unit tests: & See~\cite{DRLVT}. \\
\end{recipedef}

%%---------------------------------------------------------------------
%\subsubsection{Wavelength calibration}\label{drl:wavecal}
%\begin{recipedef}\label{drl:wavecal}
%Name: & \NEWDRL{wave_cal} \\
%Purpose: & Computes wavelength calibration \\
%Used in recipes: & \NEWREC{metis_LM_lss_wave} \\
%%Working remarks: & None \\
%%Function Parameters: & None \\
%Input: & \texttt{const hdrl\_image * input} \\https://www.overleaf.com/project/5f1abb4137d7690001f8aeb1
%%Other inputs: & None\\
%%QC outputs: & None \\
%%Output FITS files: & None \\
%Other outputs: & \texttt{cpl\_error\_code} \\
%General description: & Determines wavelength calibration by means of line lamp spectra \\
%Mathematical description: &  see~\cite{pis02} and~\cite{pis21}\\
%Quality assessment: & Through QC parameters \\
%Error conditions: & See~\cite{DRLVT}. \\
%Unit tests: & See~\cite{DRLVT}. \\
%\end{recipedef}

%---------------------------------------------------------------------
%\subsubsection{Detect line peak}\label{drl:linedetect}
%Used in recipes:\\ 
%\NEWREC{metis_LM_lss_wave} \newline
%\NEWREC{metis_LM_lss_std} \newline
%\NEWREC{metis_N_lss_std} \newline
%\NEWREC{metis_LM_lss_sci} \newline
%\NEWREC{metis_N_lss_sci} \newline
%TBD



%---------------------------------------------------------------------
\subsubsection{Determine response}\label{drl:determine_response}
\begin{recipedef}
Name: & \NEWDRL{determine_response}\\
Purpose: & Determination of the spectral response function\\
Used in recipes: &  \NEWREC{metis_LM_lss_std} \newline
                 \NEWREC{metis_N_lss_std} \\
%Working remarks: & None \\
%Function Parameters: & None \\
Input: & \texttt{const hdrl\_spectrum1D * input} \\
Other inputs: & \texttt{const hdrl\_spectrum1D * input}\\
%QC outputs: & None \\
%Output FITS files: & None \\
Other outputs: & \texttt{cpl\_error\_code} \\
General description: & Determination of the instrumental response function to be used for the absolute flux calibration \\
Mathematical description: & see HDRL manual \\
Quality assessment: & Through QC parameters \\
Error conditions: & See~\cite{DRLVT}. \\
Unit tests: & See~\cite{DRLVT}. \\
\end{recipedef}
%---------------------------------------------------------------------
\subsubsection{Remove sky background}\label{drl:remove_sky_background}
\begin{recipedef}\label{rec:removeskybackground}
Name: & \NEWDRL{remove_sky_background} \\
Purpose: & Remove sky background from spectra \\
Used in recipes: & \NEWREC{metis_LM_lss_std} \\
& \NEWREC{metis_LM_lss_sci}\\
& \NEWREC{metis_N_lss_std} \\
& \NEWREC{metis_N_lss_sci}\\
%Working remarks: & None \\
%Function Parameters: & None \\
Input: & \texttt{const hdrl\_image * input} \\
Other inputs: & None\\
%QC outputs: & None \\
%Output FITS files: & None \\
Other outputs: & \texttt{cpl\_error\_code} \\
General description: & Removes the sky emission background (nodding/chop-nodding technique) \\
Mathematical description: &  see~\cite{METIS-operational_concept}, Section 1.4\\
Quality assessment: & Through QC parameters \\
Error conditions: & See~\cite{DRLVT}. \\
Unit tests: & See~\cite{DRLVT}. \\
\end{recipedef}

%---------------------------------------------------------------------
\subsubsection{Slit curvature detection}\label{drl:slit_curvature}
\begin{recipedef}\label{rec:slitcurvature}
Name: & \NEWDRL{slit_curvature} \\
Purpose: & Determines the slit curvature along the orders \\
Used in recipes: & \NEWREC{metis_LM_lss_wave} \\
& \NEWREC{metis_LM_lss_std} \\
& \NEWREC{metis_LM_lss_sci}\\
& \NEWREC{metis_N_lss_std} \\
& \NEWREC{metis_N_lss_sci}\\
%Working remarks: & None \\
%Function Parameters: & None \\
Input: & \texttt{const hdrl\_image * input} \\
Other inputs: & None\\
%QC outputs: & None \\
%Output FITS files: & None \\
Other outputs: & \texttt{cpl\_error\_code} \\
General description: & Determines the tilt and shear of the orders using the sky emission lines \\
Mathematical description: &  see~\cite{pis02} and~\cite{pis21}\\
Quality assessment: & Through QC parameters \\
Error conditions: & See~\cite{DRLVT}. \\
Unit tests: & See~\cite{DRLVT}. \\
\end{recipedef}

%---------------------------------------------------------------------
\subsubsection{Order trace detection}\label{drl:detect_order_trace}
\begin{recipedef}\label{rec:detectordertrace}
Name: & \NEWDRL{detect_order_trace} \\
Purpose: & Determines the order traces on the detectors\\
Used in recipes: & \NEWREC{metis_LM_lss_wave} \\
& \NEWREC{metis_LM_lss_std} \\
& \NEWREC{metis_LM_lss_sci}\\
& \NEWREC{metis_N_lss_std} \\
& \NEWREC{metis_N_lss_sci}\\
%Working remarks: & None \\
%Function Parameters: & None \\
Input: & \texttt{const hdrl\_image * input} \\
Other inputs: & None\\
%QC outputs: & None \\
%Output FITS files: & None \\
Other outputs: & \texttt{cpl\_error\_code} \\
General description: & Determines the order locations on the detectors \\
Mathematical description: &  see~\cite{pis02} and~\cite{pis21}\\
Quality assessment: & Through QC parameters \\
Error conditions: & See~\cite{DRLVT}. \\
Unit tests: & See~\cite{DRLVT}. \\
\end{recipedef}

%---------------------------------------------------------------------
\subsubsection{Object extraction}\label{drl:extract_object}
\begin{recipedef}
Name: & \NEWDRL{extract_object}\\
Purpose: & Extract object spectrum from 2D-spectrum\\
Used in recipes: & \NEWREC{metis_LM_lss_std} \newline
                 \NEWREC{metis_N_lss_std} \newline
                 \NEWREC{metis_LM_lss_sci} \newline
                 \NEWREC{metis_N_lss_sci} \\
%Working remarks: & None \\
%Function Parameters: & None \\
Input: & \texttt{const hdrl\_image * input\newline const hdrl\_image * background\newline hdrl\_image * output}  \\
%Other inputs: & None\\
%QC outputs: & None \\
%Output FITS files: & None \\
Other outputs: & \texttt{cpl\_error\_code} \\
General description: & Routine to extract a 1D spectrum of the target from the 2D spectrum\\
Mathematical description: & See optimal extraction algorithm~\cite{pis02} and~\cite{pis21} \\
Quality assessment: & Through QC parameters \\
Error conditions: & See~\cite{DRLVT}. \\
Unit tests: & See~\cite{DRLVT}. \\
\end{recipedef}

%---------------------------------------------------------------------
\subsubsection{Apply wavelength calibration}\label{drl:apply_wavecal}
\begin{recipedef}
Name: & \NEWDRL{apply_wavecal}\\
Purpose: & applies \NEWPROD{LM_LSS_WAVE_GUESS}\\
Used in recipes: & \NEWREC{metis_LM_lss_std} \newline
                 \NEWREC{metis_LM_lss_sci} \\
%Working remarks: & None \\
%Function Parameters: & None \\
%Input FITS files: & None \\
Inputs: & \texttt{const hdrl\_spectrum1D * input}\\
%QC outputs: & None \\
%Output FITS files: & None \\
Other outputs: & \texttt{cpl\_error\_code} \\
General description: & Applies the first guess wavelength calibration to LM LSS data \\
%Mathematical description: & Applies projection from detector space to $(y, \lambda)$-space\\
Quality assessment: & Through QC parameters \\
Error conditions: & See~\cite{DRLVT}. \\
Unit tests: & See~\cite{DRLVT}. \\
\end{recipedef}

%---------------------------------------------------------------------
\subsubsection{Apply flux calibration}\label{drl:apply_fluxcal}
\begin{recipedef}
Name: & \NEWDRL{apply_fluxcal}\\
Purpose: & applies \NEWPROD{MASTER_LM_RESPONSE} or \NEWPROD{MASTER_N_RESPONSE} to science frames\\
Used in recipes: & \NEWREC{metis_LM_lss_sci} \newline
                 \NEWREC{metis_N_lss_sci} \\
%Working remarks: & None \\
%Function Parameters: & None \\
%Input FITS files: & None \\
Inputs: & \texttt{const hdrl\_spectrum1D * input}\\
%QC outputs: & None \\
%Output FITS files: & None \\
Other outputs: & \texttt{cpl\_error\_code} \\
General description: & Applies the absolute flux calibration to science frames \\
%Mathematical description: & Applies projection from detector space to $(y, \lambda)$-space\\
Quality assessment: & Through QC parameters \\
Error conditions: & See~\cite{DRLVT}. \\
Unit tests: & See~\cite{DRLVT}. \\
\end{recipedef}

%---------------------------------------------------------------------
\subsubsection{Apply telluric correction}\label{drl:apply_tellcorr}
\begin{recipedef}
Name: & \NEWDRL{apply_tellcorr}\\
Purpose: & applies \NEWSTATCALIB{LM_SYNTH_TRANS} and \NEWSTATCALIB{N_SYNTH_TRANS} to corresponding standard star spectrum (\NEWPROD{LM_LSS_STD_1D} and \NEWPROD{N_LSS_STD_1D})\\
Used in recipes: & \NEWREC{metis_LM_lss_sci} \\
                 & \NEWREC{metis_N_lss_sci} \\
%Working remarks: & None \\
%Function Parameters: & None \\
%Input FITS files: & None \\
Inputs: & \texttt{const hdrl\_spectrum1D * input}\\
%QC outputs: & None \\
%Output FITS files: & None \\
Other outputs: & \texttt{cpl\_error\_code} \\
General description: & Applies a simple transmission to \ac{STD} spectra better determine the response curve \\
%Mathematical description: & Applies projection from detector space to $(y, \lambda)$-space\\
Quality assessment: & Through QC parameters \\
Error conditions: & See~\cite{DRLVT}. \\
Unit tests: & See~\cite{DRLVT}. \\
\end{recipedef}

