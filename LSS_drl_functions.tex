\subsection{LSS observing mode}\label{sec:drl_functions_lss}
\textcolor{red}{TBD: Up to now mostly only copied from MICADO and crudely adapted -- check for METIS consistence!!!!}
%---------------------------------------------------------------------
\subsubsection{Interorder background correction}\label{drl:correctinterorder}
\begin{recipedef}
Name: & \hyperref[drl:correctinterorder]{\DRL{correct\_interorder\_bg}} \\
Purpose: & Removal of interorder light (e.g. stray light)\\
Used in recipes: & \hyperref[rec:metisspecflat]{\REC{metis\_LM\_lss\_flat}} \newline
\hyperref[rec:lsslmwave]{\REC{metis\_LM\_lss\_wave}} \newline
\hyperref[rec:metisspecflux]{\REC{metis\_LM\_lss\_flux}} \newline
\hyperref[rec:metisspecsci]{\REC{metis\_LM\_lss\_sci}}\\
%Working remarks: & None \\
%Function Parameters: & None \\
Input: & $n\times$ \texttt{const hdrl\_image * input} \\
%Other inputs: & None \\
%QC outputs: & None\\
%Output FITS files: & None \\
Other outputs: & \texttt{cpl\_error\_code} \\
General description: & Removal of stray light by a 2D-polynomial fits of the interorder background \\
Mathematical description: & see Section~\ref{ssec:interorderbg} \\
Quality assessment: & Through QC parameters \\
Error conditions: & See \cite{DRLVT}. \\
Unit tests: & See \cite{DRLVT}. \\
\end{recipedef}

%---------------------------------------------------------------------
\subsubsection{Slit curvature detection}\label{drl:slitcurvature}
\begin{recipedef}\label{drl:slitcurvature}
Name: & \hyperref[drl:slitcurvature]{\DRL{slit\_curvature}} \\
Purpose: & Determines the slit curvature along the orders \\
Used in recipes: & \hyperref[rec:lsslmwave]{\REC{metis\_LM\_lss\_wave}} \\
%Working remarks: & None \\
%Function Parameters: & None \\
Input: & \texttt{const hdrl\_image * input} \\
Other inputs: & None\\
%QC outputs: & None \\
%Output FITS files: & None \\
Other outputs: & \texttt{cpl\_error\_code} \\
General description: & Determines the tilt and shear of the orders using the sky emission lines \\
Mathematical description: &  see \cite{pis02} and \cite{pis21}\\
Quality assessment: & Through QC parameters \\
Error conditions: & See \cite{DRLVT}. \\
Unit tests: & See \cite{DRLVT}. \\
\end{recipedef}

%---------------------------------------------------------------------
\subsubsection{Create normalised flats}\label{drl:normflat}
\begin{recipedef}\label{drl:normflat}
Name: & \hyperref[drl:normflat]{\DRL{norm\_flat}} \\
Purpose: & Creates normalised flats from \hyperref[dataitem:mastersflat]{\PROD{MASTER\_SFLAT\_IMG}} \\
Used in recipes: & \hyperref[rec:lsslmwave]{\REC{metis\_LM\_lss\_wave}} \\
%Working remarks: & None \\
%Function Parameters: & None \\
Input: & \texttt{const hdrl\_image * input} \\
%Other inputs: & None\\
%QC outputs: & None \\
%Output FITS files: & None \\
Other outputs: & \texttt{cpl\_error\_code} \\
General description: & Creates normalised master flat and determines blaze function \\
Mathematical description: &  see \cite{pis02} and \cite{pis21}\\
Quality assessment: & Through QC parameters \\
Error conditions: & See \cite{DRLVT}. \\
Unit tests: & See \cite{DRLVT}. \\
\end{recipedef}

%%---------------------------------------------------------------------
%\subsubsection{Wavelength calibration}\label{drl:wavecal}
%\begin{recipedef}\label{drl:wavecal}
%Name: & \hyperref[drl:wavecal]{\DRL{wave\_cal}} \\
%Purpose: & Computes wavelength calibration \\
%Used in recipes: & \hyperref[rec:lsslmwave]{\REC{metis\_LM\_lss\_wave}} \\
%%Working remarks: & None \\
%%Function Parameters: & None \\
%Input: & \texttt{const hdrl\_image * input} \\https://www.overleaf.com/project/5f1abb4137d7690001f8aeb1
%%Other inputs: & None\\
%%QC outputs: & None \\
%%Output FITS files: & None \\
%Other outputs: & \texttt{cpl\_error\_code} \\
%General description: & Determines wavelength calibration by means of line lamp spectra \\
%Mathematical description: &  see \cite{pis02} and \cite{pis21}\\
%Quality assessment: & Through QC parameters \\
%Error conditions: & See \cite{DRLVT}. \\
%Unit tests: & See \cite{DRLVT}. \\
%\end{recipedef}

%---------------------------------------------------------------------
\subsubsection{Determine response}\label{drl:determineresponse}
\begin{recipedef}
Name: & \hyperref[drl:determineresponse]{\DRL{determine\_response}}\\
Purpose: & Determination of the spectral response function\\
Used in recipes: & \hyperref[rec:metisspecflat]{\REC{metis\_LM\_lss\_flat}} \newline
\hyperref[rec:metisspecflat]{\REC{metis\_LM\_lss\_flat}}\newline
\hyperref[rec:metisspecflux]{\REC{metis\_LM\_lss\_flux}} \newline
\hyperref[rec:metisspecsci]{\REC{metis\_LM\_lss\_sci}} \\
%Working remarks: & None \\
%Function Parameters: & None \\
Input: & \texttt{const hdrl\_LM\_lsstrum1d * input} \\
Other inputs: & \texttt{const hdrl\_LM\_lsstrum1d * input}\\
%QC outputs: & None \\
%Output FITS files: & None \\
Other outputs: & \texttt{cpl\_error\_code} \\
General description: & Determination of the instrumental response function to be used for the absolute flux calibration \\
Mathematical description: & see HDRL manual \\
Quality assessment: & Through QC parameters \\
Error conditions: & See \cite{DRLVT}. \\
Unit tests: & See \cite{DRLVT}. \\
\end{recipedef}

%---------------------------------------------------------------------
\subsubsection{Object extraction}\label{drl:extractobject}
\begin{recipedef}
Name: & \hyperref[drl:extractobject]{\DRL{extract\_object}}\\
Purpose: & Extract object spectrum from 2D-spectrum\\
Used in recipes: & \hyperref[rec:metisspecflux]{\REC{metis\_LM\_lss\_flux}} \newline
\hyperref[rec:metisspecsci]{\REC{metis\_LM\_lss\_sci}} \\
%Working remarks: & None \\
%Function Parameters: & None \\
Input: & \texttt{const hdrl\_image * input\newline const hdrl\_image * background\newline hdrl\_image * output}  \\
%Other inputs: & None\\
%QC outputs: & None \\
%Output FITS files: & None \\
Other outputs: & \texttt{cpl\_error\_code} \\
General description: & Routine to extract a 1D spectrum of the target from the 2D spectrum\\
Mathematical description: & See optimal extraction algorithm \cite{pis02} and \cite{pis21} \\
Quality assessment: & Through QC parameters \\
Error conditions: & See \cite{DRLVT}. \\
Unit tests: & See \cite{DRLVT}. \\
\end{recipedef}

%---------------------------------------------------------------------
\subsubsection{Apply flatfield}\label{drl:applyflat}
\begin{recipedef}
Name: & \hyperref[drl:applyflat]{\DRL{apply\_flat}}\\
Purpose: & applies \hyperref[dataitem:mastersflat]{\PROD{MASTER\_SFLAT\_IMG}} to frames\\
Used in recipes: & \hyperref[rec:metisspecflux]{\REC{metis\_LM\_lss\_flux}} \newline
\hyperref[rec:metisspecsci]{\REC{metis\_LM\_lss\_sci}} \\
%Working remarks: & None \\
%Function Parameters: & None \\
%Input FITS files: & None \\
Inputs: & \texttt{const hdrl\_image * input}\\
%QC outputs: & None \\
%Output FITS files: & None \\
Other outputs: & \texttt{cpl\_error\_code} \\
General description: & Normal flatfield process for correcting pixel-to-pixel sensitivities \\
Mathematical description: & Division by \hyperref[dataitem:mastersflat]{\PROD{MASTER\_SFLAT\_IMG}} \\
Quality assessment: & Through QC parameters \\
Error conditions: & See \cite{DRLVT}. \\
Unit tests: & See \cite{DRLVT}. \\
\end{recipedef}

%---------------------------------------------------------------------
\subsubsection{Apply flux calibration}\label{drl:applyfluxcal}
\begin{recipedef}
Name: & \hyperref[drl:applyfluxcal]{\DRL{apply\_fluxcal}}\\
Purpose: & applies \hyperref[dataitem:responsespec]{\PROD{RESPONSE\_LM\_lss}} to science frames\\
Used in recipes: & \hyperref[recipe:metisspecsci]{\REC{metis\_LM\_lss\_sci}} \\
%Working remarks: & None \\
%Function Parameters: & None \\
%Input FITS files: & None \\
Inputs: & \texttt{const hdrl\_image * input}\\
%QC outputs: & None \\
%Output FITS files: & None \\
Other outputs: & \texttt{cpl\_error\_code} \\
General description: & Applies the absolute flux calibration to science frames \\
%Mathematical description: & Applies projection from detector space to $(y, \lambda)$-space\\
Quality assessment: & Through QC parameters \\
Error conditions: & See \cite{DRLVT}. \\
Unit tests: & See \cite{DRLVT}. \\
\end{recipedef}