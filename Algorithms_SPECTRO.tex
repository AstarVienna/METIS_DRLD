% This section is dedicated to algorithm which are used in all spectroscopic modes

%-----------------------------------------------------------------------------------------
\subsection{Telluric absorption correction}\label{ssec:tellcorr}
Due to the dense molecular absorption arising from the Earth's atmosphere, nearly every \c{MIR} regime spectrum requires a correction for these telluric features. The required atmospheric transmission curve can be achieved either by specific observations of a telluric standard star (\ac{TSS}, the "classical" way) or by a modelling approach. \\
%------------------------------------------------------------------------------
\subsubsection{Classic \ac{TSS} approach:}
A telluric standard star (\ac{TSS}) spectrum is taken ideally directly before/after the science observations near the science target position (or at least at the same airmass) to probe the same pathway through the Earth's atmosphere. This \ac{TSS}-spectrum is processed in the same way as the science spectrum (except the absolute flux calibration). To remove intrinsic stellar features this spectrum is corrected with a model spectrum of this \ac{TSS}. Finally its continuum is normalised to unity. The resulting normalised spectrum (ideally) only contains the fingerprint of the Earth's atmospheric absorptions and can be used for the telluric correction.\\
There are several sources for model spectra available:
\begin{itemize}
    \item Cohen set (\cite{coh99}): Set of 422 stellar model templates, mainly K and M giants. One of the standard sets in \ac{MIR}.
    \item The SPEX \ac{IRTF} Spectral library\footnote{\url{http://irtfweb.ifa.hawaii.edu/~spex/IRTF_Spectral_Library/}}: Set of observed stellar spectra (F to M-type, some carbon and S-type stars and L and T dwarfs) in the range $0.8...5.0\mu$m mostly at a resolving power of $R\equiv\lambda/\Delta\lambda\sim2,000$.
    \item Phoenix library\footnote{\url{https://phoenix.astro.physik.uni-goettingen.de/}}\cite{phoenix}: Library of synthetic medium and high resolution spectra between $0.5...5.5\mu$m covering a wide stellar parameter space ($2,300\textrm{K}\leq T_\textrm{eff}\leq12,000\textrm{K}$; $0.0\leq\log g\leq+6.0$; $-4.0\leq$[Fe/H]$\leq+1.0$; $0.2\leq$[$\alpha$/Fe]$\leq+1.2$). \\
\end{itemize}
The \ac{METIS} consortium will look into that topic and to assemble a set of appropriate \ac{TSS} stars for each observing mode.
%------------------------------------------------------------------------------
\subsubsection{Modelling synthetic transmission spectra:}
In the last years the modelling method has evolved. It is based on radiative transfer modelling of the Earth's atmosphere. A height model of the Earth's atmosphere containing information of pressure, temperature and the concentration of molecules in combination with a radiative transfer model and a molecular line list is used to determine the transmission of the Earth's atmosphere at the time of observations. The approach by fitting specific molecular absorption features in the science spectra. The best-fit transmission function is finally used for the telluric correction.\\
In the past years the approach of modelling transmission curves of Earth's atmosphere has made significant progress leading to versatile and mature software packages for the telluric correction. One of these packages is \texttt{molecfit}\footnote{\url{http://www.eso.org/sci/software/pipelines/skytools/molecfit}} (\cite{mf1, mf2, molecfit}). This package is optimised for the ESO framework and ESO instruments and is also foreseen to be used for \ac{METIS}.\\
The outcome of the \ac{ELT} working group meeting of 2021-03-15 was that future instrument pipelines should include dedicated recipes based on the telluric correction \texttt{telluriccorr} library \cite{telluriccorr}. This package is based on \texttt{molecfit} and will be provided and maintained by \ac{ESO}. The telluric correction will be performed in three dedicated recipes as post-correction   and closely follow the approach as implemented in the \ac{KMOS} pipeline. The three steps comprise the fit of the telluirc features (e.g. \REC{metis_LM_lss_mf_model} for the LM range), the calculation of the transmission curve (e.g. \REC{metis_LM_lss_mf_calctrans}) and the application of the actual correction (e.g. \REC{metis_LM_lss_mf_correct}). For the determination of the \ac{LSF} we primarily rely on the possibilities as offered by the \texttt{telluricorr}/\texttt{molecfit} package, which is based on a fitting of the \ac{LSF} by a combination of a boxcar, Gaussian or Lorentzian. On basis of the commissioning data we will establish a parameter set providing a good starting point for the fits.\\
We also intend to enable the user to include a dedicated line kernel instead, in case a reliable kernel model can be determined with other (external) tools (e.g. a model-based convolution of an internal \ac{LSF}, the slit-widths and/or an \ac{AO} component). In addition, also external supplementary meteorological data (e.g. provided by an \ac{LHATPRO} radiometer) can be included if the \texttt{telluricorr}/\texttt{molecfit} package offers that possibility.
%------------------------------------------------------------------------------
\subsubsection{Approach for \ac{METIS}:} 
The modelling approach has become the standard way for the telluric correction in several ESO pipelines as it avoids to spend valuable observing time on taking \ac{TSS} spectra. Since the synthetic transmission function is also noise-free, it also conserves the \ac{SNR} of the science spectrum. It is therefore also foreseen as default method for the \ac{METIS} pipeline.\\
Nevertheless, there might be situations where the classical way becomes the better option. For example, in case the science object's continuum is too weak to be used for fitting telluric features. In addition, \texttt{molecfit} relies on a number of fitting parameters, which might not lead to the best minimum. In particular, the quality of the fit is very sensitive to the incorporated \ac{LSF}-Kernel, whereas the \ac{LSF} of a \ac{TSS} spectrum is naturally (almost) identical.\\
We therefore will include three different approaches for the telluric correction in the \ac{METIS} pipeline (cf. Fig.~\ref{Fig:tellcorrmethods}):
\begin{itemize}
    \item "\textit{molecfit-on-science}": This is the usual way of using \texttt{molecfit}, i.e. the science spectrum ("\texttt{1D SCIENCE SPECTRUM}") is used to determine the state of the Earth's atmosphere and to calculate a synthetic transmission (left branch in Fig.~\ref{Fig:tellcorrmethods})
    \item "classical" approach: The transmission of the Earth's atmopshere is determined in the classical way with the help of a \ac{TSS} as described above (right branch in Fig.~\ref{Fig:tellcorrmethods})
    \item "\textit{molecfit-on-star}": This is a combination of the both methods in the sense that \texttt{molecfit} is applied to \ac{TSS} observations, and the resulting synthetic transmission spectrum is used for the telluric correction of the science target  (middle branch in Fig.~\ref{Fig:tellcorrmethods})
\end{itemize}
\begin{figure}[ht]
  \centering
  \includegraphics[width=0.9\textwidth]{figures/tell_corr_methods.pdf}
  \caption{Methods for the telluric correction to be included in the \ac{METIS} pipeline.}
  \label{Fig:tellcorrmethods}
\end{figure}
%------------------------------------------------------------------------------
\subsubsection{Other algorithms included in the telluric correction approach}\label{ssec:otheralgstellcorr}
\paragraph{Normalisation of stellar continua\newline}\label{ssec:spec_normalisation}
The normalisation of spectra can be a tricky issue especially for cool stars with plenty intrinsic spectral features. A general approach is therefore not possible. However, we can restrict our routines to stars, which are (a) well-known and (b) model spectra are available. We therefore use the following approach: 
\begin{itemize}
    \item For each star we determine spectral regions, which are known to belong to the continuum, i.e. do not contain absorption/emission features neither intrinsic nor atmospheric.
    \item These spectral regions are then fit by a polynomial. The degree of the polynomial will be determined when the set of \ac{TSS} is established.
    \item The \ac{TSS} spectrum is divided by this polynomial.
\end{itemize} 

\paragraph{Airmass correction\newline}\label{ssec:airmass_coo}
see approach in latest mf release TBWritten

\paragraph{Quality control parameters\newline}\label{ssec_tellcorr_qc_params}
see approach in latest mf release TBWritten see sect~\ref{ssec:spec_normalisation}